%%%%%%
% BOOK 10
%%%%%%
\cleardoublepage 
\pdfbookmark[0]{Book 10}{book10}
\addcontentsline{toc}{chapter}{Book 10}
\pagestyle{plain}
\begin{center}
{\Huge ELEMENTS BOOK 10}\\
\spa\spa\spa
{\huge\it Incommensurable Magnitudes\symbolfootnote[2]{The theory of incommensurable magntidues set out in this
book is generally attributed to Theaetetus of Athens. In the footnotes
throughout this book, $k$, $k'$, {\em etc.}\ stand for distinct ratios
of positive integers.}}
\end{center}\newpage

%%%%%%%
% Definitions
%%%%%%%
\pdfbookmark[1]{Definitions I}{def10.1}
\pagestyle{fancy}
\cfoot{\gr{\thepage}}
\chead{\large ELEMENTS BOOK 10}
\begin{Parallel}{}{} 
\ParallelLText{
\begin{center}
\large{\gr{<'Oroi}.}
\end{center}\vspace*{-7pt}

\ggn{1}.~\gr{S'ummetra meg'ejh l'egetai t`a t~w| a>ut~w| metrw|
metro'umena, >as'ummetra d'e, <~wn mhd`en >end'eqetai koin`on
m'etron gen'esjai.}

\ggn{2}.~\gr{E>uje~iai dun'amei s'ummetro'i e>isin, <'otan t`a >ap>
a>ut~wn tetr'agwna t~w| a>ut~w| qwr'iw| metr~htai, >as'ummetroi
d'e, <'otan to~ic >ap> a>ut~wn tetrag'wnoic mhd`en >end'eqhtai qwr'ion
koin`on m'etron gen'esjai.}

\ggn{3}.~\gr{To'utwn <upokeim'enwn de'iknutai, <'oti t~h| proteje'ish|
e>uje'ia| <up'arqousin e>uje~iai pl'hjei >'apeiroi s'ummetro'i
te ka`i >as'ummetroi a<i m`en m'hkei m'onon, a<i d`e ka`i dun'amei.
kale'isjw o>~un <h m`en proteje~isa e>uje~ia <rht'h, ka`i
a<i ta'uth| s'ummetroi e>'ite m'hkei ka`i dun'amei e>'ite
dun'amei m'onon <rhta'i, a<i d`e ta'uth| >as'ummetroi >'alogoi kale'isjwsan.}

\ggn{4}.~\gr{Ka`i t`o m`en >ap`o t~hc proteje'ishc e>uje'iac tetr'agw\-non <rht'on,
ka`i t`a to'utw| s'ummetra <rht'a, t`a d`e to'utw| >as'ummetra >'aloga
kale'isjw, ka`i a<i dun'amenai a>ut`a >'alogoi, e>i m`en tetr'agwna
e>'ih, a>uta`i a<i pleura'i, e>i d`e <'eter'a tina e>uj'ugramma, a<i
>'isa a>uto~ic tetr'agwna >anagr'afousai.}}

\ParallelRText{
\begin{center}
{\large Definitions}
\end{center}

1.~Those magnitudes   measured by the
same measure are said (to be) commensurable, but (those)
of which no (magnitude) admits to be  a common measure (are said to be)
incommensurable.$^\dag$

2.~(Two) straight-lines are commensurable in square$^\ddag$ when the squares on
them are measured by the same area, but (are) incommensurable (in square)
when no area admits to be a common measure of the squares on them.$^\S$

3.~These things being assumed, it is proved that there
exist an infinite multitude of straight-lines commensurable and
incommensurable with an assigned straight-line---those  (incommensurable) in length only, and
those also (commensurable or incommensurable) in square.$^\P$ 
Therefore, let
the assigned straight-line be called rational. And (let) the (straight-lines) commensurable
with it, either in length and square, or in square only, (also be called) rational. But let the (straight-lines) incommensurable with it be called
irrational.$^\ast$ 

4.~And let the square on the assigned straight-line be called  rational. And (let areas) commensurable with it (also be called) rational. But  (let areas) incommensurable with it (be called) irrational, and (let) their square-roots$^\$$ (also be called) irrational---the sides themselves, if the (areas)
are squares, and the (straight-lines) describing squares equal to them, if the (areas) are some other rectilinear (figure).$^\|$}
\end{Parallel}



\vspace{7pt}{\footnotesize\noindent $^\dag$ In other words, two magnitudes $\alpha$ and
$\beta$ are commensurable if $\alpha:\beta::1:k$, and incommensurable otherwise.\\[0.5ex]
$^\ddag$ Literally, ``in power''.\\[0.5ex]
$^\S$  In other words, two straight-lines of length $\alpha$ and $\beta$ are commensurable in square if $\alpha : \beta::1:k^{1/2}$,  and incommensurable in square otherwise. Likewise, the straight-lines
are commensurable in length if $\alpha:\beta::1:k$, and incommensurable in length otherwise.\\[0.5ex]
$^\P$ To be more exact, straight-lines can either be commensurable in square only, incommensurable in length only, or commenusrable/incommensurable in both
length and square, with an assigned straight-line.\\[0.5ex]
$^\ast$ Let the length of the assigned straight-line be unity. Then rational
straight-lines have lengths expressible as $k$ or $k^{1/2}$, depending on whether the lengths are commensurable in length, or in square only, respectively,
with unity.  All other straight-lines are irrational.\\[0.5ex]
$^\$$ The square-root of an area is the length of the side of an equal area square.\\[0.5ex]
$^\|$ The area of the  square 
on the assigned straight-line is unity. Rational areas are expressible as $k$.  All other areas are irrational. Thus, squares whose
sides are of rational length have  rational areas, and {\em vice versa}.}

%%%%%%
% Prop 10.1
%%%%%%
\pdfbookmark[1]{Proposition 10.1}{pdf10.1}
\begin{Parallel}{}{} 
\ParallelLText{
\begin{center}
{\large \ggn{1}.}
\end{center}\vspace*{-7pt}

\gr{D'uo megej~wn >an'iswn >ekkeim'enwn, >e`an >ap`o to~u me'izonoc >afairej~h| me~izon >`h t`o <'hmisu ka`i to~u kataleipom'enou me~izon >`h
t`o <'hmisu, ka`i to~uto >ae`i g'ignhtai, leifj'hseta'i ti m'egejoc, <`o >'estai
>'elasson to~u >ekkeim'enou >el'assonoc meg'ejouc.}

\gr{>'Estw d'uo meg'ejh >'anisa t`a AB, G, <~wn me~izon t`o AB; l'egw, <'oti,
>ean >ap`o to~u AB >afairej~h| me~izon >`h t`o <'hmisu
ka`i to~u kataleipom'enou me~izon >`h t`o <'hmisu, ka`i to~uto >ae`i
g'ignhtai, leifj'hseta'i ti m'egejoc, <`o >'estai >'elasson to~u G meg'ejouc.}\\~\\~\\

\epsfysize=1.1in
\centerline{\epsffile{Book10/fig001g.eps}}

\gr{T`o G g`ar pollaplasiaz'omenon >'estai pot`e to~u AB me~izon. pepollaplasi'asjw, ka`i >'estw t`o DE to~u m`en G pollapl'asion, to~u d`e AB
me~izon, ka`i dih|r'hsjw t`o DE e>ic t`a t~w| G >'isa t`a DZ, ZH, HE,
ka`i >afh|r'hsjw >ap`o m`en to~u AB me~izon >`h t`o <'hmisu t`o BJ,
>ap`o d`e to~u AJ me~izon >`h t`o <'hmisu t`o JK, ka`i to~uto
>ae`i gign'esjw, <'ewc >`an a<i >en t~w| AB diair'eseic >isoplhje~ic
g'enwntai ta~ic >en t~w| DE diair'esesin.}

\gr{>'Estwsan o>~un a<i AK, KJ, JB diair'eseic >isoplhje~ic o>~usai ta~ic
DZ, ZH, HE; ka`i >epe`i me~iz'on >esti t`o DE to~u AB, ka`i >af'h|rhtai
>ap`o m`en to~u DE >'elasson to~u <hm'isewc t`o EH, >ap`o d`e to~u AB
me~izon >`h t`o <'hmisu t`o BJ, loip`on >'ara t`o HD loipo~u to~u JA
me~iz'on >estin. ka`i >epe`i me~iz'on >esti t`o HD to~u JA, ka`i
>af'h|rhtai to~u m`en HD <'hmisu t`o HZ, to~u d`e JA me~izon >`h
t`o <'hmisu t`o JK, loip`on >'ara t`o DZ loipo~u to~u AK me~iz'on >estin.
>'ison d`e t`o DZ t~w| G; ka`i t`o G >'ara to~u AK me~iz'on >estin.
>'elasson >'ara t`o AK to~u G.}

\gr{Katale'ipetai >'ara >ap`o to~u AB meg'ejouc t`o AK m'egejoc >'elasson >`on
to~u >ekkeim'enou >el'assonoc meg'ejouc to~u G; <'oper >'edei de~ixai.}\,---\,\gr{<omo'iwc d`e deiqj'hsetai, k>`an <hm'ish >~h| t`a >afairo'umena.}}

\ParallelRText{
\begin{center}
{\large Proposition 1}$^\dag$
\end{center}

If, from
the greater of two unequal magnitudes (which are) laid out,  (a part) greater than half is subtracted, and (if from) the remainder (a part) greater than
half (is subtracted), and (if) this happens continually, then some magnitude will (eventually) be left which
will be less than the  lesser laid out magnitude.

Let $AB$ and $C$ be two unequal magnitudes, of which (let) $AB$ (be) the greater. I say that if (a part) greater than half is subtracted  from $AB$,
and (if a part) greater than half (is subtracted) from the remainder, and (if) this happens
continually, then some magnitude will (eventually) be left which will be less
than the magnitude $C$. 

\epsfysize=1.1in
\centerline{\epsffile{Book10/fig001e.eps}}

For $C$, when multiplied (by some number), will sometimes be greater than $AB$ [Def. 5.4]. Let it have
been (so) multiplied. And let $DE$ be (both) a multiple of $C$, and greater
than $AB$. And let $DE$ have been divided into the (divisions) $DF$, $FG$,
$GE$, equal to $C$. And   let $BH$, (which is) greater than half, have been subtracted from $AB$. And  (let) $HK$, (which is) greater than half, (have been subtracted) from $AH$. And let this happen continually, until the divisions in
$AB$ become equal in number to the divisions in $DE$.

Therefore, let the divisions (in $AB$) be $AK$, $KH$, $HB$, being equal in number to $DF$, $FG$, $GE$.  And since $DE$ is greater than $AB$, and $EG$, (which is) less than half, has been subtracted from $DE$, and $BH$, (which is) greater
than half, from $AB$, the remainder $GD$ is thus greater than the remainder
$HA$. And since $GD$ is greater than $HA$, and the half $GF$ has been
subtracted from $GD$, and $HK$, (which is) greater than half, from $HA$,
the remainder $DF$ is thus greater than the remainder $AK$. And $DF$
(is) equal to $C$. $C$ is thus also greater than $AK$.
Thus, $AK$ (is) less than $C$.

Thus, the magnitude $AK$,  which is less
than the lesser laid out magnitude $C$, is left over from the magnitude $AB$. (Which is) the very thing it was required to show. --- (The theorem) can similarly be proved even if  the
(parts) subtracted are halves.}
\end{Parallel}


\vspace{7pt}{\footnotesize\noindent $^\dag$ This theorem is the basis of the so-called {\em method of exhaustion}, and is generally attributed to Eudoxus of Cnidus.}

%%%%%%
% Prop 10.2
%%%%%%
\pdfbookmark[1]{Proposition 10.2}{pdf10.2}
\begin{Parallel}{}{} 
\ParallelLText{
\begin{center}
{\large \ggn{2}.}
\end{center}\vspace*{-7pt}

\gr{>E`an d'uo megej~wn [>ekkeim'enwn] >an'iswn >anjufairoum'enou >ae`i
to~u >el'assonoc >ap`o to~u me'izonoc t`o kataleip'omenon mhd'epote
katametr~h| t`o pr`o <eauto~u, >as'ummetra >'estai t`a meg'ejh.}

\gr{D'uo g`ar megej~wn >'ontwn >an'iswn t~wn AB, GD ka`i >el'assonoc
to~u AB >anjufairoum'enou >ae`i to~u >el'assonoc >ap`o to~u me'izonoc t`o
perileip'omenon mhd'epote katametre'itw t`o pr`o <eauto~u; l'egw, <'oti
>as'ummetr'a >esti t`a AB, GD meg'ejh.}\\

\epsfysize=0.9in
\centerline{\epsffile{Book10/fig002g.eps}}

\gr{E>i g'ar >esti s'ummetra, metr'hsei ti a>ut`a m'egejoc. metre'itw, e>i dunat'on,
ka`i >'estw t`o E; ka`i t`o m`en AB t`o ZD katametro~un leip'etw <eauto~u
>'elasson t`o GZ, t`o d`e GZ t`o BH katametro~un leip'etw <eauto~u
>'elasson t`o AH, ka`i to~uto >ae`i gin'esjw, <'ewc o<~u leifj~h| ti
m'egejoc, <'o >estin >'elasson to~u E. gegon'etw, ka`i lele'ifjw t`o AH
>'elasson to~u E. >epe`i o>~un t`o E t`o AB metre~i, >all`a t`o AB t`o
DZ metre~i, ka`i t`o E >'ara t`o ZD metr'hsei. metre~i d`e ka`i <'olon
t`o GD; ka`i loip`on >'ara t`o GZ metr'hsei. >all`a t`o GZ t`o BH metre~i;
ka`i t`o E >'ara t`o BH metre~i. metre~i d`e ka`i <'olon t`o AB; ka`i loip`on
>'ara t`o AH metr'hsei, t`o me~izon t`o >'elasson; <'oper >est`in >ad'unaton.
o>uk >'ara t`a AB, GD meg'ejh metr'hsei ti m'egejoc; >as'ummetra >'ara
>est`i t`a AB, GD meg'ejh.}

\gr{>E`an >'ara d'uo megej~wn >an'iswn, ka`i t`a <ex~hc.}}

\ParallelRText{
\begin{center}
{\large Proposition 2}
\end{center}

If the remainder of two unequal magnitudes (which are) [laid out] never measures the (magnitude) before it, (when) the lesser (magnitude is) continually subtracted in turn from the greater, then the (original) magnitudes will be incommensurable.

For, $AB$ and $CD$ being two unequal magnitudes, and $AB$ (being) the lesser, let the remainder
never measure the (magnitude) before it,
(when) the lesser (magnitude is) continually subtracted in turn from the greater. I say that
the magnitudes $AB$ and $CD$ are incommensurable.

\epsfysize=0.9in
\centerline{\epsffile{Book10/fig002e.eps}}

For if they are commensurable then some magnitude will measure them (both). If possible, let it
(so) measure (them),  and let it be $E$. And let $AB$ leave $CF$
less than itself (in) measuring $FD$, and let $CF$ leave $AG$ less than
itself (in) measuring $BG$, and let this happen continually, until  some magnitude which is less than  $E$ is left. Let (this) have occurred,$^\dag$ and
let $AG$, (which is) less than $E$, have been left. Therefore, since
$E$ measures $AB$, but $AB$ measures $DF$, $E$ will thus also measure $FD$. And it also measures the whole (of) $CD$. Thus, it will
also measure the remainder  $CF$. But, $CF$ measures $BG$. Thus,
$E$ also measures $BG$. And it also measures the whole (of) $AB$. 
Thus, it will also measure the remainder $AG$, the greater (measuring)
the lesser. The very thing is impossible. Thus, some magnitude cannot
measure (both) the magnitudes $AB$ and $CD$. Thus, the
magnitudes $AB$ and $CD$ are incommensurable [Def. 10.1].

Thus, if \ldots of two unequal magnitudes, and so on \ldots.}
\end{Parallel}


\vspace{7pt}{\footnotesize\noindent$^\dag$ The fact that this will eventually occur is guaranteed by
Prop.~10.1.}

%%%%%%
% Prop 10.3
%%%%%%
\pdfbookmark[1]{Proposition 10.3}{pdf10.3}
\begin{Parallel}{}{} 
\ParallelLText{
\begin{center}
{\large \ggn{3}.}
\end{center}\vspace*{-7pt}

\gr{D'uo megej~wn summ'etrwn doj'entwn t`o m'egiston a>ut~wn koin`on
m'etron e<ure~in.}

\epsfysize=1.1in
\centerline{\epsffile{Book10/fig003g.eps}}

\gr{>'Estw t`a doj'enta d'uo meg'ejh s'ummetra t`a AB, GD, <~wn
>'elasson t`o AB; de~i d`h t~wn AB, GD t`o m'egiston koin`on m'etron
e<ure~in.}

\gr{T`o AB g`ar m'egejoc >'htoi metre~i t`o GD >`h o>'u. e>i
m`en o>~un metre~i, metre~i d`e ka`i <eaut'o, t`o AB >'ara t~wn
AB, GD koin`on m'etron >est'in; ka`i faner'on, <'oti ka`i
m'egiston. me~izon g`ar to~u AB meg'ejouc t`o AB o>u
metr'hsei.}

\gr{M`h metre'itw d`h t`o AB t`o GD. ka`i >anjufairoum'enou >ae`i to~u >el'assonoc >ap`o to~u  me'izonoc, t`o perileip'omenon metr'hsei
pot`e t`o pr`o <eauto~u di`a t`o m`h e>~inai >as'ummetra t`a
AB, GD; ka`i t`o m`en AB t`o ED katametro~un leip'etw <eauto~u
>'elasson t`o EG, t`o d`e EG t`o ZB katametro~un leip'etw <eauto~u
>'elasson t`o AZ, t`o d`e AZ t`o GE metre'itw.}

\gr{>Epe`i o>~un t`o AZ t`o GE metre~i, >all`a t`o GE t`o ZB metre~i, ka`i
t`o AZ >'ara t`o ZB metr'hsei. metre~i d`e ka`i <eaut'o; ka`i <'olon >'ara
t`o AB metr'hsei t`o AZ. >all`a t`o AB t`o DE metre~i; ka`i t`o AZ >'ara
t`o ED metr'hsei. metre~i d`e ka`i t`o GE; ka'i <'olon >'ara t`o GD
metre~i; t`o AZ >'ara t~wn AB, GD koin`on m'etron >est'in. l'egw d'h,
<'oti ka`i m'egiston. e>i g`ar m'h, >'estai ti m'egejoc me~izon to~u AZ,
<`o
metr'hsei t`a AB, GD. >'estw t`o H. >epe`i o>~un t`o H t`o AB
metre~i, >all`a t`o AB t`o ED metre~i, ka`i t`o H >'ara t`o ED metr'hsei. metre~i d`e ka`i <'olon t`o GD; ka`i loip`on >'ara t`o GE metr'hsei
t`o H. >all`a t`o GE t`o ZB metre~i; ka`i t`o H >'ara
t`o ZB metr'hsei. metre~i d`e ka`i <'olon t`o AB, ka`i loip`on t`o AZ
metr'hsei, t`o me~izon t`o >'elasson; <'oper >est`in >ad'unaton. o>uk
>'ara me~iz'on ti m'egejoc to~u AZ t`a AB, GD metr'hsei; t`o AZ >'ara
t~wn AB, GD t`o m'egiston koin`on m'etron >est'in.}

\gr{D'uo >'ara megej~wn summ'etrwn doj'entwn t~wn AB, GD t`o m'egiston
koin`on m'etron h<'urhtai; <'oper
>'edei de~ixai.}\\~\\~\\~\\~\\~\\~\\~\\~\\~\\

\begin{center}
{\large \gr{P'orisma}.}
\end{center}\vspace*{-7pt}

\gr{>Ek d`h to'utou faner'on, <'oti, >e`an m'egejoc d'uo meg'ejh metr~h|, ka`i
t`o m'egiston a>ut~wn koin`on m'etron metr'hsei.}}

\ParallelRText{
\begin{center}
{\large Proposition 3}
\end{center}

To find the greatest common measure of two
given commensurable magnitudes.

\epsfysize=1.1in
\centerline{\epsffile{Book10/fig003e.eps}}

Let $AB$ and $CD$ be the two given magnitudes, of which (let) $AB$ (be)
the lesser. So, it is required to find the greatest common measure of
$AB$ and $CD$.

For the magnitude $AB$ either measures, or (does) not (measure), $CD$. Therefore, if
it measures ($CD$), and (since) it also measures itself, $AB$ is thus
a common measure of $AB$ and $CD$. And (it is) clear that (it is) also
(the) greatest. For a (magnitude) greater than magnitude $AB$ cannot
measure $AB$.

So let $AB$ not measure $CD$. And continually subtracting in turn the lesser 
(magnitude) from the greater, the remaining (magnitude) will (at) some time measure the (magnitude)
before it, on account of $AB$ and $CD$ not being incommensurable [Prop. 10.2]. And let $AB$ leave $EC$ less than
itself (in) measuring $ED$, and let $EC$ leave $AF$ less than itself (in)
measuring $FB$, and let $AF$ measure $CE$.

Therefore, since $AF$ measures $CE$, but $CE$ measures $FB$, $AF$
will thus also measure $FB$. And it also measures itself. Thus, $AF$
will also measure the whole (of) $AB$. But, $AB$ measures $DE$. Thus, $AF$
will also measure $ED$. And it also measures $CE$. Thus, it also measures the whole of $CD$. Thus, $AF$ is a common measure of $AB$ and $CD$.
So I say that (it is) also (the) greatest (common measure). For, if not,
there will be some magnitude, greater than $AF$, which will measure
(both) $AB$ and $CD$. Let it be $G$. Therefore, since $G$ measures $AB$,
but $AB$ measures $ED$, $G$ will thus also measure $ED$. And
it also measures the whole of $CD$. Thus, $G$ will also measure the remainder $CE$.
But $CE$ measures $FB$. Thus, $G$ will also measure $FB$. And
it also measures the whole (of) $AB$. And (so) it will measure the
remainder $AF$, the greater (measuring) the lesser. The very thing is impossible. Thus, some magnitude greater than $AF$ cannot measure
(both) $AB$ and $CD$. Thus, $AF$ is the greatest common measure of
$AB$ and $CD$.

Thus, the greatest common measure of two given commensurable magnitudes, $AB$ and $CD$, has been found. (Which is) the very thing
it was required to show.\\

\begin{center}
{\large Corollary}
\end{center}\vspace*{-7pt}

So (it is) clear, from this, that if a magnitude measures two magnitudes
then it will also measure their greatest common measure.}
\end{Parallel}

%%%%%%
% Prop 10.4
%%%%%%
\pdfbookmark[1]{Proposition 10.4}{pdf10.4}
\begin{Parallel}{}{} 
\ParallelLText{
\begin{center}
{\large \ggn{4}.}
\end{center}\vspace*{-7pt}

\gr{Tri~wn megej~wn summ'etrwn doj'entwn t`o m'egiston a>ut~wn
koin`on m'etron e<ure~in.}

\epsfysize=1.2in
\centerline{\epsffile{Book10/fig004g.eps}}

\gr{>'Estw t`a doj'enta tr'ia meg'ejh s'ummetra t`a A, B, G; de~i d`h t~wn
A, B, G t`o m'egiston koin`on m'etron e<ure~in.}

\gr{E>il'hfjw g`ar d'uo t~wn A, B t`o m'egiston koin`on m'etron, ka`i >'estw
t`o D; t`o d`h D t`o G >'htoi metre~i >`h o>'u [metre~i]. metre'itw pr'oteron.
>epe`i o>~un t`o D t`o G metre~i, metre~i d`e ka`i t`a A, B, t`o D >'ara
t`a A, B, G metre~i; t`o D >'ara t~wn A, B, G koin`on m'etron >est'in.
ka`i faner'on, <'oti ka`i m'egiston; me~izon g`ar to~u D meg'ejouc t`a A, B
o>u metre~i.}

\gr{M`h metre'itw d`h t`o D t`o G. l'egw pr~wton, <'oti
s'ummetr'a >esti t`a G, D. >epe`i g`ar s'ummetr'a >esti t`a A, B, G, metr'hsei ti 
a>ut`a m'egejoc, <`o dhlad`h ka`i t`a A, B metr'hsei; <'wste ka`i t`o t~wn
A, B m'egiston koin`on m'etron t`o D metr'hsei. metre~i d`e ka`i t`o G;
<'wste t`o e>irhm'enon m'egejoc metr'hsei t`a G, D; s'ummetra >'ara 
>est`i t`a G, D. e>il'hfjw o>~un a>ut~wn t`o m'egiston koin`on m'etron,
ka`i >'estw t`o E. >epe`i o>~un t`o E t`o D metre~i, >all`a t`o D t`a A, B
metre~i, ka`i t`o E >'ara t`a A, B metr'hsei. metre~i d`e ka`i t`o G.
t`o E >'ara t`a A, B, G metre~i; t`o E >'ara t~wn A, B, G koin'on >esti
m'etron. l'egw d'h, <'oti ka`i m'egiston. e>i g`ar dunat'on, >'estw ti to~u
E me~izon m'egejoc t`o Z, ka`i metre'itw t`a A, B, G. ka`i >epe`i t`o
Z t`a A, B, G metre~i, ka`i t`a A, B >'ara metr'hsei ka`i t`o t~wn A, B
m'egiston koin`on m'etron metr'hsei. t`o d`e t~wn A, B m'egiston koin`on
m'etron >est`i t`o D; t`o Z >'ara t`o D metre~i. metre~i d`e ka`i t`o G; t`o Z
>'ara t`a G, D metre~i; ka`i t`o t~wn G, D >'ara m'egiston koin`on
m'etron metr'hsei t`o Z. >'esti d`e t`o E; t`o Z >'ara t`o E metr'hsei,
t`o me~izon t`o >'elasson; <'oper >est`in >ad'unaton. o>uk >'ara me~iz'on
ti to~u E meg'ejouc [m'egejoc] t`a A, B, G metre~i; t`o E >'ara t~wn A, B, G t`o m'egiston koin`on m'etron >est'in, >e`an m`h metr~h| t`o D t`o G, >e`an
d`e metr~h|, a>ut`o t`o D.}

\gr{Tri~wn >'ara megej~wn summ'etrwn doj'entwn t`o m'egiston koin`on
m'etron h<'urhtai [<'oper >'edei de~ixai].}\\~\\~\\~\\~\\~\\~\\~\\~\\~\\~\\~\\~\\

\begin{center}
{\large \gr{P'orisma}.}
\end{center}\vspace*{-7pt}

\gr{>Ek d`h to'utou faner'on, <'oti, >e`an m'egejoc tr'ia meg'ejh metr~h|, ka`i
t`o m'egiston a>ut~wn koin`on m'etron metr'hsei.}

\gr{<Omo'iwc d`h ka`i >ep`i plei'onwn t`o m'egiston koin`on
m'etron lhfj'hsetai, ka`i t`o p'orisma proqwr'hsei. <'oper >'edei de~ixai.}}

\ParallelRText{
\begin{center}
{\large Proposition 4}
\end{center}

To find the greatest common measure of three
given commensurable magnitudes.

\epsfysize=1.2in
\centerline{\epsffile{Book10/fig004e.eps}}

Let $A$, $B$, $C$ be the three given commensurable magnitudes. So
it is required to find the greatest common measure of $A$, $B$, $C$.

For let the greatest common measure of the two (magnitudes)
$A$ and $B$ have been taken [Prop. 10.3],
and let it be $D$. So $D$ either measures, or [does] not
[measure], $C$. Let it, first of all, measure ($C$). Therefore, since
$D$ measures $C$, and it also measures $A$ and $B$, $D$ thus
measures $A$, $B$, $C$. Thus, $D$ is a common measure of $A$, $B$, $C$.
And (it is) clear that (it is) also (the) greatest (common measure). For
no magnitude larger than $D$ measures (both) $A$ and $B$.

So let $D$ not measure $C$. I say, first, that $C$ and $D$ are
commensurable. For if $A$, $B$, $C$ are commensurable then some
magnitude will measure them which will clearly also measure $A$ and $B$.
Hence, it will also measure $D$, the greatest common measure of $A$ and $B$ [Prop. 10.3~corr.]. And it also measures
$C$. Hence, the aforementioned magnitude will measure (both)
$C$ and $D$. Thus, $C$ and $D$ are commensurable [Def. 10.1]. Therefore, let their
greatest common measure have been taken [Prop. 10.3], and let it be $E$. Therefore, since
$E$ measures $D$, but $D$ measures (both) $A$ and $B$, $E$ will
thus also measure $A$ and $B$. And it also measures $C$. Thus,
$E$ measures $A$, $B$, $C$. Thus, $E$ is a common measure of $A$, $B$, $C$.  So I say that (it is) also (the) greatest (common measure). For, if
possible, let $F$ be some magnitude greater than $E$, and let it
measure $A$, $B$, $C$. And since $F$ measures $A$, $B$, $C$, it will
thus also measure $A$ and $B$, and  will (thus) measure the greatest common measure of $A$ and $B$ [Prop. 10.3~corr.].  And
$D$ is the greatest common measure of $A$ and $B$.  Thus, $F$ measures $D$. And it also measures $C$. Thus, $F$ measures (both) $C$ and $D$.
Thus, $F$ will also measure the greatest common measure of $C$ and $D$  [Prop. 10.3~corr.].
And it is $E$. Thus, $F$ will measure $E$, the greater (measuring)
the lesser. The very thing is impossible. Thus, some [magnitude]
greater than the magnitude $E$ cannot measure $A$, $B$, $C$. Thus,
if $D$ does not
measure $C$ then
$E$ is the greatest common measure of $A$, $B$, $C$.  And if it does measure ($C$) then $D$ itself (is the
greatest common measure).

Thus, the greatest common measure  of three given
commensurable magnitudes has been found.  [(Which is) the very thing it was required to
show.]\\

\begin{center}
{\large Corollary}
\end{center}\vspace*{-7pt}

So (it is) clear, from this, that if a magnitude measures three
magnitudes then it will also measure their greatest common measure.

So, similarly, the greatest common measure of more (magnitudes) can also be
taken, and the (above) corollary will go forward. (Which is) the very thing
it was required to show.}
\end{Parallel}~\\

%%%%%%
% Prop 10.5
%%%%%%
\pdfbookmark[1]{Proposition 10.5}{pdf10.5}
\begin{Parallel}{}{} 
\ParallelLText{
\begin{center}
{\large \ggn{5}.}
\end{center}\vspace*{-7pt}

\gr{T`a s'ummetra meg'ejh pr`oc >'allhla l'ogon >'eqei, <`on >arijm`oc pr`oc
>arijm'on.}

\epsfysize=0.8in
\centerline{\epsffile{Book10/fig005g.eps}}

\gr{>'Estw s'ummetra meg'ejh t`a A, B; l'egw, <'oti t`o A pr`oc t`o B l'ogon >'eqei,
<`on >arijm`oc pr`oc >arijm'on.}

\gr{>Epe`i g`ar s'ummetr'a >esti t`a A, B, metr'hsei ti a>ut`a m'egejoc. metre'itw,
ka`i >'estw t`o G. ka`i <os'akic t`o G t`o A metre~i, tosa~utai mon'adec
>'estwsan >en t~w| D, <os'akic d`e t`o G t`o B metre~i, tosa~utai mon'adec
>'estwsan >en t~w| E.}

\gr{>Epe`i o>~un t`o G t`o A metre~i kat`a t`ac >en t~w| D mon'adac, 
metre~i d`e ka`i <h mon`ac t`on D kat`a t`ac >en a>ut~w| mon'adac,
>is'akic
>'ara <h mon`ac t`on D metre~i >arijm`on ka`i t`o G m'egejoc t`o A; >'estin
>'ara <wc t`o G pr`oc t`o A, o<'utwc <h mon`ac pr`oc t`on D; >an'apalin
>'ara, <wc t`o A pr`oc t`o G, o<'utwc <o D pr`oc t`hn mon'ada. p'alin
>epe`i t`o G t`o B metre~i kat`a t`ac >en t~w| E mon'adac, metre~i
d`e ka`i <h mon`ac t`on E kat`a t`ac >en a>ut~w| mon'adac, >is'akic
>'ara <h mon`ac t`on E metre~i ka`i t`o G t`o B; >'estin >'ara <wc t`o G
pr`oc t`o B, o<'utwc <h mon`ac pr`oc t`on E. >ede'iqjh d`e ka`i <wc t`o
A pr`oc t`o G, <o D pr`oc t`hn mon'ada; di> >'isou >'ara >est`in <wc t`o
A pr`oc t`o B, o<'utwc <o D >arijm`oc pr`oc t`on E.}

\gr{T`a >'ara s'ummetra meg'ejh t`a A, B pr`oc >'allhla l'ogon >'eqei,
<`on >arijm`oc <o D pr`oc >arijm`on t`on E; <'oper >'edei de~ixai.}}

\ParallelRText{
\begin{center}
{\large Proposition 5}
\end{center}

Commensurable magnitudes have to one another the
ratio which (some) number (has) to (some) number.

\epsfysize=0.8in
\centerline{\epsffile{Book10/fig005e.eps}}

Let $A$ and $B$ be commensurable magnitudes. I say that $A$ has to $B$
the ratio which (some) number (has) to (some) number.

For if $A$ and $B$ are commensurable (magnitudes) then some magnitude
will measure them. Let it (so) measure (them), and let it be $C$. And
as many times as $C$ measures $A$, so many units let there be in
$D$. And as many times as $C$ measures $B$, so many units let there be
in $E$.

Therefore, since $C$ measures $A$ according to the units in $D$, and
a unit also measures $D$ according to the units in it,  a unit thus measures the
number $D$ as many times as the magnitude $C$ (measures) $A$. 
Thus, as $C$ is to $A$, so a unit (is) to $D$ [Def. 7.20].$^\dag$ Thus, inversely, as $A$ (is) to $C$, so
$D$ (is) to a unit [Prop. 5.7~corr.]. Again, since
$C$ measures $B$ according to the units in $E$, and a unit also measures $E$ according to the units in it,   a unit thus measures $E$ the
same number of times that $C$ (measures) $B$. Thus, as $C$ is to $B$, so
a unit (is) to $E$ [Def. 7.20]. And it was also
shown that as $A$ (is) to $C$, so $D$ (is) to a unit. Thus, via equality,
as $A$ is to $B$, so the number $D$ (is) to the (number) $E$ [Prop. 5.22].

Thus, the commensurable magnitudes $A$ and $B$ have to one another
the ratio which the number $D$ (has) to the number $E$. (Which is) the
very thing it was required to show.}
\end{Parallel}


\vspace{7pt}{\footnotesize\noindent$^\dag$ There is a slight logical gap here, since Def.~7.20  applies to four numbers, rather than two number and two magnitudes.}

%%%%%%
% Prop 10.6
%%%%%%
\pdfbookmark[1]{Proposition 10.6}{pdf10.6}
\begin{Parallel}{}{} 
\ParallelLText{
\begin{center}
{\large \ggn{6}.}
\end{center}\vspace*{-7pt}

\gr{>E`an d'uo meg'ejh pr`oc >'allhla l'ogon >'eqh|, <`on >arijm`oc pr`oc
>arijm'on, s'ummetra >'estai t`a meg'ejh.}\\

\epsfysize=0.8in
\centerline{\epsffile{Book10/fig006g.eps}}

\gr{D'uo g`ar meg'ejh t`a A, B pr`oc >'allhla l'ogon >eq'etw, <`on >arijm`oc
<o D pr`oc >arijm`on t`on E; l'egw, <'oti s'ummetr'a >esti t`a A, B
meg'ejh.}

\gr{<'Osai g'ar e>isin >en t~w| D mon'adec, e>ic tosa~uta >'isa
dih|r'hsjw t`o A, ka`i <en`i a>ut~wn >'ison >'estw t`o G; <'osai
d'e e>isin >en t~w| E mon'adec, >ek toso'utwn megej~wn >'iswn
t~w| G sugke'isjw t`o Z.}

\gr{>Epe`i o>~un, <'osai e>is`in >en t~w| D mon'adec, tosa~ut'a e>isi
ka`i >en t~w| A meg'ejh >'isa t~w| G, <`o >'ara m'eroc >est`in <h mon`ac
to~u D, t`o a>ut`o m'eroc >est`i ka`i t`o G to~u A; >'estin >'ara <wc t`o G
pr`oc t`o A, o<'utwc <h mon`ac pr`oc t`on D. metre~i d`e <h
mon`ac t`on D >arijm'on; metre~i >'ara ka`i t`o G t`o A. ka`i >epe'i >estin
<wc t`o G pr`oc t`o A, o<'utwc <h mon`ac pr`oc t`on D [>arijm'on],
>an'apalin >'ara <wc t`o A pr`oc t`o G, o<'utwc <o D >arijm`oc pr`oc
t`hn mon'ada. p'alin >epe'i, <'osai e>is`in >en t~w| E mon'adec,
tosa~ut'a e>isi ka`i >en t~w| Z >'isa t~w| G, >'estin >'ara <wc
t`o G pr`oc t`o Z, o<'utwc <h mon`ac pr`oc t`on E [>arijm'on]. >ede'iqjh
d`e ka`i <wc t`o A pr`oc t`o G, o<'utwc <o D pr`oc t`hn mon'ada; di>
>'isou >'ara >est`in <wc t`o A pr`oc t`o Z, o<'utwc <o D pr`oc t`on
E. >all> <wc <o D pr`oc t`on E, o<'utwc >est`i t`o A pr`oc t`o B; ka`i
<wc >'ara t`o A pr`oc t`o B, o<'utwc ka`i pr`oc t`o Z. t`o A >'ara pr`oc
<ek'ateron t~wn B, Z t`on a>ut`on >'eqei l'ogon; >'ison >'ara >est`i
t`o B t~w| Z. metre~i d`e t`o G t`o Z; metre~i >'ara ka`i t`o B. >all`a
m`hn ka`i t`o A; t`o G >'ara t`a A, B metre~i. s'ummetron >'ara >est`i
t`o A t~w| B.}

\gr{>E`an >'ara d'uo meg'ejh pr`oc >'allhla, ka`i t`a <ex~hc.}\\~\\~\\~\\~\\

\begin{center}
{\large \gr{P'orisma}.}
\end{center}\vspace*{-7pt}

\gr{>Ek d`h to'utou faner'on, <'oti, >e`an >~wsi d'uo >arijmo'i, <wc o<i
D, E, ka`i e>uje~ia, <wc <h A, d'unat'on >esti poi~hsai <wc <o D >arijm`oc
pr`oc t`on E >arijm'on, o<'utwc t`hn e>uje~ian pr`oc e>uje~ian. >e`an d`e ka`i
t~wn 
A, Z m'esh >an'alogon lhfj~h|, <wc <h B, >'estai <wc <h A pr`oc t`hn Z,
o<'utwc t`o >ap`o t~hc A pr`oc t`o >ap`o t~hc B, tout'estin <wc <h
pr'wth pr`oc t`hn tr'ithn, o<'utwc t`o >ap`o t~hc pr'wthc pr`oc t`o >ap`o
t~hc deut'erac t`o <'omoion ka`i <omo'iwc >anagraf'omenon. >all> <wc
<h A pr`oc t`hn Z, o<'utwc >est`in <o D >arijmoc pr`oc t`on E >arijm'on;
g'egonen >'ara ka`i <wc <o 
D >arijm`oc pr`oc t`on E >arijm'on, o<'utwc
t`o >ap`o t~hc A e>uje'iac pr`oc t`o >ap`o t~hc B e>uje'iac; <'oper
>'edei de~ixai.}}

\ParallelRText{
\begin{center}
{\large Proposition 6}
\end{center}

If two magnitudes have to one another the ratio
which (some) number (has) to (some) number then the magnitudes
will be commensurable.

\epsfysize=0.8in
\centerline{\epsffile{Book10/fig006e.eps}}

For let the two magnitudes $A$ and $B$ have to one another the ratio which
the number $D$ (has) to the number $E$. I say that the magnitudes $A$ and $B$ are commensurable.

For, as many units as there are in $D$, let $A$ have been divided into so many
equal (divisions). And let $C$ be equal to one of them. And as many
units as there are in $E$, let $F$ be the sum  of so many magnitudes equal to
$C$.

Therefore, since as many units as there are in $D$, so many magnitudes equal to $C$ are also in $A$, therefore whichever part a unit is of $D$, $C$ is also
the same part of $A$. Thus, as $C$ is to $A$, so a unit (is) to $D$ [Def. 7.20]. And a unit measures the number $D$.
Thus, $C$ also measures $A$. And since as $C$ is to $A$, so a unit (is) to
the [number] $D$, thus, inversely, as $A$ (is) to $C$, so the number $D$
(is) to a unit [Prop. 5.7~corr.]. Again, since as many 
units as there are in $E$,  so many (magnitudes) equal to $C$ are also in $F$,
thus as $C$ is to $F$, so a unit (is) to the [number] $E$ [Def. 7.20]. And it was also shown that as $A$ (is) to $C$, so $D$ (is) to a unit. Thus, via equality, as $A$ is to $F$, so $D$ (is)
to $E$ [Prop. 5.22]. But, as $D$ (is) to $E$, 
so $A$ is to $B$. And thus as $A$ (is) to $B$, so (it) also is to $F$ [Prop. 5.11]. Thus,
$A$ has the same ratio to each of $B$ and $F$. Thus, $B$ is equal to
$F$ [Prop. 5.9]. And $C$ measures $F$. Thus,
it also measures $B$. But, in fact, (it) also (measures) $A$. Thus, $C$ measures (both) $A$ and $B$. Thus, $A$ is commensurable with $B$ [Def. 10.1].

Thus,  if two magnitudes \ldots to one another, and so on \ldots.\\

\begin{center}
{\large Corollary}
\end{center}\vspace*{-7pt}

So it is clear, from this, that if there are two numbers, like $D$ and $E$, and
a straight-line, like $A$, then it is possible to contrive that as the number $D$ (is)
to the number $E$, so the straight-line (is) to (another) straight-line ({\em i.e.,} $F$). And if the mean proportion, (say) $B$,  is taken of $A$ and $F$, 
then as $A$ is to $F$, so the (square) on $A$ (will be) to the (square)
on $B$. That is to say, as the first (is) to the third, so the
(figure) on the first (is) to the similar, and similarly described, (figure) on the second [Prop. 6.19~corr.]. But, as $A$ (is) to
$F$, so the number $D$ is to the number $E$. Thus, it has also been contrived
that as the number $D$ (is) to the number $E$, so the (figure) on the
straight-line $A$ (is) to the (similar figure) on the straight-line $B$. (Which is)
the very thing it was required to show.}
\end{Parallel}

%%%%%%
% Prop 10.7
%%%%%%
\pdfbookmark[1]{Proposition 10.7}{pdf10.7}
\begin{Parallel}{}{} 
\ParallelLText{
\begin{center}
{\large\ggn{7}.}
\end{center}\vspace*{-7pt}

\gr{T`a >as'ummetra meg'ejh pr`oc >'allhla l'ogon o>uk >'eqei, <`on >arijm`oc
pr`oc >arijm'on.}

\gr{>'Estw >as'ummetra meg'ejh t`a A, B; l'egw, <'oti t`o A pr`oc t`o B
l'ogon o>uk >'eqei, <`on >arijm`oc pr`oc >arijm'on.}\\~\\

\epsfysize=0.5in
\centerline{\epsffile{Book10/fig007g.eps}}

\gr{E>i g`ar >'eqei t`o A pr`oc t`o B l'ogon, <`on >arijm`oc pr`oc >arijm'on,
s'ummetron >'estai t`o A t~w| B. o>uk >'esti d'e; o>uk >'ara t`o A pr`oc
t`o B l'ogon >'eqei, <`on >arijm`oc pr`oc >arijm'on.}

\gr{T`a >'ara >as'ummetra meg'ejh pr`oc >'allhla l'ogon o>uk >'eqei, ka`i
t`a <ex~hc.}}

\ParallelRText{
\begin{center}
{\large Proposition 7}
\end{center}

Incommensurable magnitudes do not
have to one another the ratio which (some) number (has) to (some)
number.

Let $A$ and $B$ be incommensurable magnitudes. I say that $A$ does not
have to $B$ the ratio which (some) number (has) to (some) number.

\epsfysize=0.5in
\centerline{\epsffile{Book10/fig007e.eps}}

For if $A$ has to $B$ the ratio which (some) number (has) to (some)
number then $A$ will be commensurable with $B$ [Prop. 10.6]. But it is not. Thus, $A$ does not
have to $B$ the ratio which (some) number (has) to (some) number.

Thus, incommensurable numbers do not have to one another, and
so on \ldots.}
\end{Parallel}

%%%%%%
% Prop 10.8
%%%%%%
\pdfbookmark[1]{Proposition 10.8}{pdf10.8}
\begin{Parallel}{}{} 
\ParallelLText{
\begin{center}
{\large \ggn{8}.}
\end{center}\vspace*{-7pt}

\gr{>E`an d'uo meg'ejh pr`oc >'allhla l'ogon m`h >'eqh|, <`on >arijm`oc pr`oc
>arijm'on, >as'ummetra >'estai t`a meg'ejh.}\\

\epsfysize=0.5in
\centerline{\epsffile{Book10/fig007g.eps}}

\gr{D'uo g`ar meg'ejh t`a A, B pr`oc >'allhla l'ogon m`h >eq'etw, <`on 
>arijm`oc pr`oc >arijm'on; l'egw, <'oti >as'ummetr'a >esti t`a A, B
meg'ejh.}

\gr{E>i g`ar >'estai s'ummetra, t`o A pr`oc t`o B l'ogon <'exei, <`on >arijm`oc
pr`oc >arijm'on. o>uk >'eqei d'e. >as'ummetra >'ara >est`i t`a A, B meg'ejh.}

\gr{>E`an >'ara d'uo meg'ejh pr`oc >'allhla, ka`i t`a <ex~hc.}}

\ParallelRText{
\begin{center}
{\large Proposition 8}
\end{center}

If two magnitudes do not have to one another the
ratio which (some) number (has) to (some) number then the magnitudes
will be incommensurable.

\epsfysize=0.5in
\centerline{\epsffile{Book10/fig007e.eps}}

For let the two magnitudes $A$ and $B$ not have to one another the ratio
which (some) number (has) to (some) number. I say that the magnitudes
$A$ and $B$ are incommensurable.

For if they are commensurable, $A$ will have to $B$ the ratio which (some)
number (has) to (some) number [Prop. 10.5]. But it does not have (such a ratio).
Thus, the magnitudes $A$ and $B$ are incommensurable.

Thus, if two magnitudes \ldots to one another, and so on \ldots.}
\end{Parallel}

%%%%%%
% Prop 10.9
%%%%%%
\pdfbookmark[1]{Proposition 10.9}{pdf10.9}
\begin{Parallel}{}{} 
\ParallelLText{
\begin{center}
{\large \ggn{9}.}
\end{center}\vspace*{-7pt}

\gr{T`a >ap`o t~wn m'hkei summ'etrwn e>ujei~wn tetr'agwna pr`oc >'allhla
l'ogon >'eqei, <`on tetr'agwnoc >arijm`oc pr`oc tetr'agwnon >arijm'on;
ka`i t`a tetr'agwna t`a pr`oc >'allhla l'ogon >'eqonta, <`on tetr'agwnoc >arijm`oc pr`oc tetr'agwnon >arijm'on, ka`i t`ac pleur`ac <'exei m'hkei
summ'etrouc. t`a d`e >ap`o t~wn m'hkei >asumm'etrwn e>ujei~wn
tetr'agwna pr`oc >'allhla l'ogon o>uk >'eqei,  <'onper
tetr'agwnoc >arijm`oc pr`oc tetr'agwnon >arijm'on; ka`i t`a tetr'agwna
t`a pr`oc >'allhla l'ogon m`h >'eqonta, <`on tetr'agwnoc >arijm`oc
pr`oc tetr'agwnon >arijm'on, o>ud`e t`ac pleur`ac <'exei m'hkei
summ'etrouc.}\\~\\~\\

\epsfysize=0.5in
\centerline{\epsffile{Book10/fig009g.eps}}

\gr{>'Estwsan g`ar a<i A, B m'hkei s'ummetroi; l'egw, <'oti t`o >ap`o t~hc
A tetr'agwnon pr`oc t`o >ap`o t~hc B tetr'agwnon l'ogon >'eqei,
<`on tetr'agwnoc >arijm`oc pr`oc tetr'agwnon >arijm'on.}

\gr{>Epe`i g`ar s'ummetr'oc >estin <h A t~h| B m'hkei, <h A >'ara pr`oc t`hn
B l'ogon >'eqei, <`on >arijm`oc pr`oc >arijm'on. >eq'etw, <`on <o G pr`oc
t`on D. >epe`i o>~un >estin <wc <h A pr`oc t`hn B, o<'utwc <o G
pr`oc t`on D, >all`a to~u m`en t~hc A pr`oc t`hn B l'ogou diplas'iwn
>est`in <o to~u >ap`o t~hc A tetrag'wnou pr`oc t`o >ap`o t~hc B tetr'agwnon;
t`a g`ar <'omoia sq'hmata >en diplas'ioni l'ogw| >est`i t~wn
<omol'ogwn pleur~wn; to~u d`e to~u G [>arijmo~u] pr`oc t`on D
[>arijm`on] l'ogou diplas'iwn >est`in <o to~u >ap`o to~u G
tetrag'wnou pr`oc t`on >ap`o to~u D tetr'agwnon; d'uo g`ar tetrag'wnwn
>arijm~wn e<~ic m'esoc >an'alog'on >estin >arijm'oc, ka'i <o tetr'agwnoc
pr`oc t`on tetr'agwnon [>arijm`on] diplas'iona l'ogon >'eqei, >'hper
<h pleur`a pr`oc t`hn pleur'an; >'estin >'ara ka`i <wc t`o >ap`o t~hc
A tetr'agwnon pr`oc t`o >ap`o t~hc B tetr'agwnon, o<'utwc <o 
>ap`o to~u G tetr'agwnoc [>arijm`oc] pr`oc t`on >ap`o to~u D [>arijmo~u]
tetr'agwnon [>arijm'on].}

\gr{>All`a d`h >'estw <wc t`o >ap`o t~hc A tetr'agwnon pr`oc t`o >ap`o
t~hc B, o<'utwc <o >ap`o to~u G tetr'agwnoc pr`oc t`on >ap`o to~u
D [tetr'agwnon]; l'egw, <'oti s'ummetr'oc >estin <h A t~h| B m'hkei.}

\gr{>Epe`i g'ar >estin <wc t`o >ap`o t~hc A tetr'agwnon pr`oc t`o >ap`o
t~hc B [tetr'agwnon], o<'utwc <o >ap`o to~u G tetr'agwnoc pr`oc
t`on >ap`o to~u D [tetr'agwnon], >all> <o m`en to~u >ap`o t~hc A
tetrag'wnou pr`oc t`o >ap`o t~hc B [tetr'agwnon] l'ogoc
diplas'iwn >est`i to~u t~hc A pr`oc t`hn B l'ogou, <o d`e to~u >ap`o to~u G [>arijmo~u]
tetrag'wnou [>arijmo~u] pr`oc t`on >ap`o to~u D [>arijmo~u]
tetr'agwnon [>arijm`on] l'ogoc diplas'iwn >est`i to~u to~u G
[>arijmo~u] pr`oc t`on D [>arijm`on] l'ogou, >'estin >'ara ka`i <wc
<h A pr`oc t`hn B, o<'utwc <o G [>arijm`oc] pr`oc t`on D [>arijm'on]. <h A >'ara pr`oc t`hn B l'ogon >'eqei, <`on >arijm`oc <o G
pr`oc >arijm`on t`on D; s'ummetroc >'ara >est`in <h A t~h| B m'hkei.}

\gr{>All`a d`h >as'ummetroc >'estw <h A t~h| B m'hkei; l'egw, <'oti t`o
>ap`o t~hc A tetr'agwnon pr`oc t`o >ap`o t~hc B [tetr'agwnon] l'ogon
o>uk >'eqei, <`on tetr'agwnoc >arijm`oc pr`oc tetr'agwnon >arijm'on.}

\gr{E>i g`ar >'eqei t`o >ap`o t~hc A tetr'agwnon pr`oc t`o >ap`o t~hc B
[tetr'agwnon] l'ogon, <`on tetr'agwnoc >arijm`oc pr`oc tetr'agwnon
>arijm'on, s'ummetroc >'estai <h A t~h| B. o>uk >'esti d'e; o>uk  >'ara t`o
>ap`o t~hc A tetr'agwnon pr`oc t`o >ap`o t~hc B [tetr'agwnon] l'ogon
>'eqei, <`on tetr'agwnoc >arijm`oc pr`oc tetr'agwnon >arijm'on.}

\gr{P'alin d`h t`o >ap`o t~hc A tetr'agwnon pr`oc t`o >ap`o t~hc B
[tetr'agwnon] l'ogon m`h >eq'etw, <`on tetr'agwnoc >arijm`oc pr`oc
tetr'agwnon >arijm'on; l'egw, <'oti >as'ummetr'oc >estin <h A t~h| B
m'hkei.}

\gr{E>i g'ar >esti s'ummetroc <h A t~h| B, <'exei t`o >ap`o t~hc A pr`oc
t`o >ap`o t~hc B l'ogon, <`on tetr'agwnoc >arijm`oc pr`oc
tetr'agwnon >arijm'on. o>uk >'eqei d'e; o>uk >'ara s'ummetr'oc >estin
<h A t~h| B m'hkei.}

\gr{T`a >'ara >ap`o t~wn m'hkei summ'etrwn, ka`i t`a <ex~hc.}\\~\\~\\~\\

\begin{center}
{\large \gr{P'orisma}.}
\end{center}

\gr{Ka`i faner`on >ek t~wn dedeigm'enwn >'estai, <'oti a<i m'hkei s'ummetroi
p'antwc ka`i dun'amei, a<i d`e dun'amei o>u p'antwc ka`i m'hkei.}}

\ParallelRText{
\begin{center}
{\large Proposition 9}
\end{center}

Squares on straight-lines (which are) commensurable in length have to one another the ratio which (some) square number (has) to (some)
square number. And squares having to one another the ratio which (some)
square number (has) to (some) square number will also have sides (which are)
commensurable in length. But squares on straight-lines (which are) incommensurable
in length do not have to one another the ratio which (some) square
number (has) to (some) square number. And squares not having
to one another the ratio which (some) square number (has) to (some)
square number will not have sides (which are) commensurable in length either.

\epsfysize=0.5in
\centerline{\epsffile{Book10/fig009e.eps}}

For let $A$ and $B$ be (straight-lines which are) commensurable in length.
I say that the square on $A$ has to the square on $B$ the ratio
which (some) square number (has) to (some) square number.

For since $A$ is commensurable in length with $B$, $A$ thus
has to $B$ the ratio which (some) number (has) to (some) number [Prop. 10.5]. Let it have (that) which $C$ (has) to $D$.
Therefore, since as $A$ is to $B$, so $C$ (is) to $D$. But the (ratio)
of the square on $A$ to the square on $B$ is the square of the
ratio of $A$ to $B$. For similar figures are in the squared ratio of (their) corresponding sides [Prop. 6.20~corr.].
And the (ratio) of the square on $C$ to the square on $D$ is the
square of the ratio of the [number] $C$ to the [number] $D$.
For there exits one number in mean proportion to two square numbers,
and (one) square (number) has to the (other) square [number] a squared ratio
with respect to (that) the side (of the former has) to the side (of the latter)
[Prop. 8.11]. And, thus,  as the square on $A$
is to the square on $B$, so the square  [number] on the (number) $C$ (is) to
the square [number] on the [number] $D$.$^\dag$

And so let the square on $A$ be to the (square) on $B$ as the
square (number) on $C$ (is) to the [square] (number) on $D$. I say
that $A$ is commensurable in length  with $B$.

For since as the square on $A$ is to the [square] on $B$, so the
square (number) on $C$ (is) to the [square] (number) on $D$. But,
the ratio of the square on $A$ to the (square) on $B$ is the square
of the (ratio) of $A$ to $B$ [Prop. 6.20 corr.].
And the (ratio) of the square [number] on the [number] $C$ to
the square [number] on the [number] $D$ is the square of the ratio
of the [number] $C$ to the [number] $D$ [Prop. 8.11].  Thus, as $A$ is to $B$, so the
[number] $C$ also (is) to the [number] $D$.
$A$, thus,  has to $B$ the ratio
which the number $C$ has to the number $D$. Thus, $A$ is
commensurable in length with $B$ [Prop. 10.6].$^\ddag$

And so let $A$ be incommensurable in length with $B$. I say that
 the square on $A$ does not have to the [square] on $B$ the ratio
which (some) square number (has) to (some) square number.

For if the square on $A$ has to the [square] on $B$ the ratio which
(some) square number (has) to (some) square number then $A$
will be commensurable (in length) with $B$. But it is not. Thus, the square on $A$
does not have to the [square] on the $B$ the ratio which (some)
square number (has) to (some) square number.

So, again, let the square on $A$ not have to the [square] on $B$
the ratio which (some) square number (has) to (some) square number.
I say that $A$ is incommensurable in length with $B$.

For if $A$ is commensurable (in length) with $B$ then  the 
(square) on $A$ will have to the (square) on $B$ the ratio which (some) square
number (has) to (some) square number. But it does not have (such a ratio). 
Thus, $A$ is not commensurable in length with $B$.

Thus, (squares) on (straight-lines which are) commensurable in length,
and so on \ldots.\\

\begin{center}
{\large Corollary}
\end{center}\vspace*{-7pt}

And it will be clear, from (what) has been demonstrated, that (straight-lines)
commensurable in length (are) always also (commensurable) in square, 
but (straight-lines commensurable) in square (are) not always also (commensurable)
in length.}
\end{Parallel}


\vspace{7pt}{\footnotesize\noindent$^\dag$ There is an unstated assumption here that if $\alpha:\beta::\gamma:\delta$ then $\alpha^2:\beta^2::\gamma^2:\delta^2$.\\[0.5ex]
$^\ddag$ There is an unstated
assumption here that if $\alpha^2:\beta^2::\gamma^2:\delta^2$ then $\alpha:\beta::\gamma:\delta$.}

%%%%%%
% Prop 10.10
%%%%%%
\pdfbookmark[1]{Proposition 10.10}{pdf10.10}
\begin{Parallel}{}{} 
\ParallelLText{
\begin{center}
{\large \ggn{10}.}
\end{center}\vspace*{-7pt}

\gr{T~h| proteje'ish| e>uje'ia| proseure~in d'uo e>uje'iac >asumm'et\-rouc,
t`hn m`en m'hkei m'onon, t`hn d`e ka`i dun'amei.}\\

\epsfysize=1.5in
\centerline{\epsffile{Book10/fig010g.eps}}

\gr{>'Estw <h proteje~isa e>uje~ia <h A; de~i d`h t~h| A proseure~in
d'uo e>uje'iac >asumm'etrouc, t`hn m`en m'hkei m'onon, t`hn d`e
ka`i dun'amei.}

\gr{>Ekke'isjwsan g`ar d'uo arijmo`i o<i B, G pr`oc >all'hlouc l'ogon
m`h >'eqontec, <`on tetr'agwnoc >arijm`oc pr`oc tetr'agwnon
>arijm'on, tout'esti m`h <'omoioi >ep'ipedoi, ka`i gegon'etw
<wc <o B pr`oc t`on G, o<'utwc t`o >ap`o t~hc A tetr'agwnon pr`oc
t`o >ap`o t~hc D tetr'agwnon; >em'ajomen g'ar; s'ummetron >'ara t`o
>ap`o t~hc A t~w| >ap`o t~hc D. ka`i >epe`i <o B pr`oc t`on G l'ogon
o>uk >'eqei, <`on tetr'agwnoc >arijm`oc pr`oc tetr'agwnon >arijm'on,
o>ud> >'ara t`o >ap`o t~hc A pr`oc t`o >ap`o t~hc D l'ogon >'eqei,
<`on tetr'agwnoc >arijm`oc pr`oc tetr'agwnon >arijm'on; >as'ummetroc
 >'ara >est`in
<h A t~h| D m'hkei. e>il'hfjw t~wn A, D m'esh >an'alogon <h E; >'estin
>'ara <wc <h A pr`oc t`hn D, o<'utwc t`o >ap`o t~hc A tetr'agwnon
pr`oc t`o >ap`o t~hc E. >as'ummetroc d'e >estin <h A t~h| D m'hkei;
>as'ummetron >'ara >est`i ka`i t`o >ap`o t~hc A tetr'agwnon
t~w| >ap`o t~hc E tetrag'wnw|; >as'ummetroc >'ara >est`in <h A t~h|
E dun'amei.}

\gr{T`h| >'ara proteje'ish| e>uje'ia| t~h| A prose'urhntai d'uo e>uje~iai
>as'ummetroi a<i D, E, m'hkei m`en m'onon <h D,
dun'amei d`e ka`i m'hkei dhlad`h <h E [<'oper >'edei de~ixai].}}

\ParallelRText{
\begin{center}
{\large Proposition 10}$^\dag$
\end{center}

To find two straight-lines incommensurable
with a given straight-line, the one (incommensurable) in length only,
the other also (incommensurable) in square.

\epsfysize=1.5in
\centerline{\epsffile{Book10/fig010e.eps}}

Let $A$ be the given straight-line. So it is required to find two straight-lines
incommensurable with $A$, the one (incommensurable) in length only,
the other also (incommensurable) in square.

For let two numbers, $B$ and $C$,  not having to one another
the ratio which (some) square number (has) to (some) square number---that
is to say, not (being) similar plane (numbers)---have been taken. And let it be contrived
that as $B$ (is) to $C$, so the square on $A$ (is) to the square on $D$.
For we learned (how to do this) [Prop. 10.6 corr.].
Thus, the (square) on $A$ (is) commensurable with the (square) on $D$
[Prop. 10.6]. And since $B$ does not have to $C$ the
ratio which (some) square number (has) to (some) square number, the
(square) on $A$ thus does not have to the (square) on $D$ the ratio which (some) square number (has) to (some) square number either. Thus, $A$ is
incommensurable in length with $D$ [Prop. 10.9].
Let the (straight-line) $E$ (which is) in mean proportion to $A$ and $D$ have
been taken [Prop. 6.13]. Thus, as $A$ is to $D$, so the square on $A$ (is) to the (square)
on $E$ [Def. 5.9]. And $A$ is incommensurable in length with $D$. Thus, the square on $A$ is also incommensurble
with the square on $E$ [Prop. 10.11]. 
Thus, $A$ is incommensurable in square with $E$.

Thus, two straight-lines, $D$ and $E$, (which are) incommensurable with the given
straight-line $A$, have been found, the one, $D$, (incommensurable)
in length only, the other, $E$, (incommensurable) in square, and, clearly, also in length. [(Which is) the very thing it was required to show.]}
\end{Parallel}


\vspace{7pt}{\footnotesize\noindent$^\dag$ This whole proposition is regarded by Heiberg as an interpolation into the original text.}

%%%%%%
% Prop 10.11
%%%%%%
\pdfbookmark[1]{Proposition 10.11}{pdf10.11}
\begin{Parallel}{}{} 
\ParallelLText{
\begin{center}
{\large \ggn{11}.}
\end{center}\vspace*{-7pt}

\gr{>E`an t'essara meg'ejh >an'alogon >~h|, t`o d`e pr~wton t~w| deut'erw|
s'ummetron >~h|, ka`i t`o tr'iton t~w| tet'artw| s'ummetron >'estai; k>`an
t`o pr~wton t~w| deut'erw| >as'ummetron >~h|, ka`i t`o tr'iton t~w|
tet'artw| >as'ummetron >'estai.}\\

\epsfysize=0.5in
\centerline{\epsffile{Book10/fig011g.eps}}

\gr{>'Estwsan t'essara meg'ejh >an'alogon t`a A, B, G, D, <wc t`o A
pr`oc t`o B, o<'utwc t`o G pr`oc t`o D, t`o A d`e t~w| B s'ummetron
>'estw; l'egw,  <'oti ka`i t`o G t~w| D s'ummetron >'estai.}

\gr{>Epe`i g`ar s'ummetr'on >esti t`o A t~w| B, t`o A >'ara pr`oc t`o B
l'ogon >'eqei, <`on >arijm`oc pr`oc >arijm'on. ka'i >estin <wc t`o A
pr`oc t`o B, o<'utwc t`o G pr`oc t`o D; ka`i t`o G >'ara pr`oc t`o D
l'ogon >'eqei, <`on >arijm`oc pr`oc >arijm'on; s'ummetron >'ara
>est`i t`o G t~w| D.}

\gr{>All`a d`h t`o A t~w| B >as'ummetron >'estw; l'egw, <'oti ka`i t`o
G t~w| D >as'ummetron >'estai. >epe`i g`ar >as'ummetr'on >esti
t`o A t~w| B, t`o A >'ara pr`oc t`o B l'ogon o>uk >'eqei, <`on >arijm`oc
pr`oc >arijm'on. ka'i >estin <wc t`o A pr`oc t`o B, o<'utwc t`o G pr`oc t`o D;
o>ud`e t`o G >'ara pr`oc t`o D l'ogon >'eqei, <`on >arijm`oc pr`oc
>arijm'on; >as'ummetron >'ara >est`i t`o G t~w| D.}

\gr{>E`an >'ara t'essara meg'ejh, ka`i t`a <ex~hc.}}

\ParallelRText{
\begin{center}
{\large Proposition 11}
\end{center}

If four magnitudes are proportional, and the
first is commensurable with the second, then the third will also be
commensurable with the fourth. And if the first is incommensurable with
the second, then the third will also be incommensurable with the fourth.

\epsfysize=0.5in
\centerline{\epsffile{Book10/fig011e.eps}}

Let $A$, $B$, $C$, $D$ be four proportional magnitudes, (such that) as
$A$ (is) to $B$, so $C$ (is) to $D$. And let $A$ be commensurable
with $B$. I say that $C$ will also be commensurable with $D$.

For since $A$ is commensurable with $B$, $A$ thus has to $B$ the ratio
which (some) number (has) to (some) number [Prop. 10.5]. And as $A$ is to $B$, so $C$ (is) to $D$. Thus, $C$ also has to $D$ the ratio which (some) number (has) to
(some) number. Thus, $C$ is commensurable with $D$ [Prop. 10.6].

And so let $A$ be incommensurable with $B$. I say that $C$ will also
be incommensurable with $D$. For since $A$ is incommensurable with
$B$, $A$ thus does not have to $B$ the ratio which  (some) number (has) to
(some) number [Prop. 10.7]. And as $A$ is to $B$, so $C$ (is) to $D$. Thus, $C$ does not have to $D$ the ratio which (some) number (has) to
(some) number either. Thus, $C$ is incommensurable with $D$ [Prop. 10.8].

Thus, if four magnitudes, and so on \ldots.}
\end{Parallel}

%%%%%%
% Prop 10.12
%%%%%%
\pdfbookmark[1]{Proposition 10.12}{pdf10.12}
\begin{Parallel}{}{} 
\ParallelLText{
\begin{center}
{\large \ggn{12}.}
\end{center}\vspace*{-7pt}

\gr{T`a t~w| a>ut~w| meg'ejei s'ummetra ka`i >all'hloic >est`i s'ummetra.}

\gr{<Ek'ateron g`ar t~wn A, B t~w| G >'estw s'ummetron. l'egw, <'oti
ka`i t`o A t~w| B >esti s'ummetron.}

\gr{>Epe`i g`ar s'ummetr'on >esti t`o A t~w| G, t`o A >'ara pr`oc t`o G l'ogon
>'eqei, <`on >arijm`oc pr`oc >arijm'on. >eq'etw, <`on <o D pr`oc t`on E.
p'alin, >epe`i s'ummetr'on >esti t`o G t~w| B, t`o G >'ara pr`oc t`o B l'ogon
>'eqei, <`on >arijm`oc pr`oc >arijm'on. >eq'etw, <`on <o Z pr`oc t`on H.
ka`i l'ogwn doj'entwn <oposwno~un to~u te, <`on >'eqei <o D pr`oc t`on
E, ka`i <o Z pr`oc t`on H e>il'hfjwsan >arijmo`i <ex~hc >en to~ic doje~isi
l'ogoic o<i J, K, L; <'wste e>~inai <wc m`en t`on D pr`oc t`on E, o<'utwc
t`on J pr`oc t`on K, <wc d`e t`on Z pr`oc t`on H, o<'utwc  t`on K pr`oc t`on L.}\\~\\~\\

\epsfysize=1.in
\centerline{\epsffile{Book10/fig012g.eps}}

\gr{>Epe`i o>~un >estin <wc t`o A pr`oc t`o G, o<'utwc <o D pr`oc t`on E,
>all> <wc <o D pr`oc t`on E, o<'utwc <o J pr`oc t`on K, >'estin >'ara
ka`i <wc t`o A pr`oc t`o G, o<'utwc <o J pr`oc t`on K. p'alin, >epe'i >estin
<wc t`o G pr`oc t`o B, o<'utwc <o Z pr`oc t`on H, >all> <wc <o Z pr`oc
t`on H, [o<'utwc] <o K pr`oc t`on L, ka`i <wc >'ara t`o G pr`oc
t`o B, o<'utwc <o K pr`oc t`on L. >'esti d`e ka`i <wc t`o A pr`oc t`o
G, o<'utwc <o J pr`oc t`on K; di> >'isou >'ara >est`in <wc t`o A pr`oc
t`o B, o<'utwc <o J pr`oc t`on L. t`o A >'ara pr`oc t`o B l'ogon
>'eqei, <`on >arijm`oc <o J pr`oc >arijm`on t`on L;
s'ummetron >'ara >est`i t`o A t~w| B.}

\gr{T`a >'ara t~w| a>ut~w| meg'ejei s'ummetra ka`i >all'hloic >est`i
s'ummetra; <'oper >'edei de~ixai.}}

\ParallelRText{
\begin{center}
{\large Proposition 12}
\end{center}

(Magnitudes) commensurable with the same magnitude are also commensurable with one another.

For let $A$ and $B$ each be commensurable with $C$. I say that
$A$ is also commensurable with $B$.

For since $A$ is commensurable with $C$, $A$ thus has to $C$ the
ratio which (some) number (has) to (some) number [Prop. 10.5]. Let it have (the ratio) which $D$ (has)
to $E$. Again, since $C$ is commensurable with $B$, $C$ thus has to $B$
the ratio which (some) number (has) to (some) number [Prop. 10.5].  Let it have (the ratio) which $F$ (has) to $G$. And for any multitude whatsoever of given ratios---(namely,) those
which $D$ has to $E$, and $F$ to $G$---let the numbers $H$, $K$, $L$
(which are) continuously (proportional) in the(se) given ratios have been taken [Prop. 8.4]. Hence, as $D$ is to $E$, so $H$ (is) to $K$, and as $F$ (is) to $G$, so $K$ (is) to $L$.

\epsfysize=1.in
\centerline{\epsffile{Book10/fig012e.eps}}

Therefore, since as $A$ is to $C$, so $D$ (is) to $E$, but as $D$ (is) to 
$E$, so $H$ (is) to $K$, thus also as $A$ is to $C$, so $H$ (is) to $K$
[Prop. 5.11]. Again, since as $C$ is to $B$, 
so $F$ (is) to $G$, but as $F$ (is) to $G$, [so] $K$ (is) to $L$, 
thus also as $C$ (is) to $B$, so $K$ (is) to $L$ [Prop. 5.11]. And also as $A$ is to $C$, so $H$ (is) to $K$. Thus, via equality, as $A$ is to $B$, so $H$ (is) to $L$ [Prop. 5.22]. Thus, $A$ has to $B$ the ratio
which the number $H$ (has) to the number $L$. Thus, $A$ is commensurable
with $B$ [Prop. 10.6].

Thus, (magnitudes) commensurable with the same magnitude are also commensurable with one another. (Which is) the very thing it was required
to show.}
\end{Parallel}

%%%%%%
% Prop 10.13
%%%%%%
\pdfbookmark[1]{Proposition 10.13}{pdf10.13}
\begin{Parallel}{}{} 
\ParallelLText{
\begin{center}
{\large\ggn{13}.}
\end{center}\vspace*{-7pt}

\gr{>E`an >~h| d'uo meg'ejh s'ummetra, t`o d`e <'eteron a>ut~wn meg'ejei
tin`i >as'ummetron >~h|, ka`i t`o loip`on t~w| a>ut~w|
>as'ummetr\-on >'estai.}\\

\epsfysize=0.8in
\centerline{\epsffile{Book10/fig013g.eps}}

\gr{>'Estw d'uo meg'ejh s'ummetra t`a A, B, t`o d`e <'eteron a>ut~wn
t`o A >'allw| tin`i t~w| G >as'ummetron >'estw; l'egw, <'oti ka`i t`o
loip`on t`o B t~w| G >as'ummetr'on >estin.}

\gr{E>i g'ar >esti s'ummetron t`o B t~w| G, >all`a ka`i t`o A
t~w| B s'ummetr'on >estin, ka`i t`o A >'ara t~w| G s'ummetr'on
>estin. >all`a ka`i >as'ummetron; <'oper >ad'unaton. o>uk >'ara
s'ummetr'on >esti t`o B t~w| G; >as'ummetron >'ara.}

\gr{>E`an >'ara >~h| d'uo meg'ejh s'ummetra, ka`i t`a <ex~hc.}}

\ParallelRText{
\begin{center}
{\large Proposition 13}
\end{center}

If two magnitudes are commensurable, and
one of them is incommensurable  with some magnitude, then the remaining (magnitude)
will also be incommensurable with it.

\epsfysize=0.8in
\centerline{\epsffile{Book10/fig013e.eps}}

Let $A$ and $B$ be two commensurable magnitudes, and let one
of them, $A$, be incommensurable with some other (magnitude),
$C$. I say that the remaining (magnitude), $B$, is also incommensurable
with $C$.

For if $B$ is commensurable with $C$, but $A$ is also commensurable
with $B$, $A$ is thus also commensurable with $C$ [Prop. 10.12]. But, (it is) also incommensurable (with
$C$). The very thing (is) impossible. Thus, $B$ is not commensurable
with $C$. Thus, (it is) incommensurable.

Thus, if two magnitudes are commensurable, and so on \ldots.}
\end{Parallel}

%%%%%%
% Prop 10.13a
%%%%%%
\begin{Parallel}{}{} 
\ParallelLText{
\begin{center}
{\large \gr{L~hmma}.}
\end{center}\vspace*{-7pt}

\gr{D'uo dojeis~wn e>ujei~wn >an'iswn e<ure~in, t'ini me~izon d'unatai
<h me'izwn t~hc >el'assonoc.}\\

\epsfysize=1.2in
\centerline{\epsffile{Book10/fig013ag.eps}}

\gr{>'Estwsan a<i doje~isai d'uo >'anisoi e>uje~iai a<i AB, G, <~wn me'izwn
>'estw <h AB; de~i d`h e<ure~in, t'ini me~izon d'unatai <h AB t~hc G.}

\gr{Gegr'afjw >ep`i t~hc AB <hmik'uklion t`o ADB, ka`i
e>ic a>ut`o >enhrm'osjw t~h| G >'ish <h AD, ka`i >epeze'uqjw <h DB.
faner`on d'h, <'oti >orj'h >estin <h <up`o ADB gwn'ia, ka`i <'oti
<h AB t~hc AD, tout'esti t~hc G, me~izon d'unatai t~h| DB.}

\gr{<Omo'iwc d`e ka`i d'uo dojeis~wn e>ujei~wn <h dunam'enh a>ut`ac e<ur'isketai o<'utwc.}

\gr{>'Estwsan a<i doje~isai d'uo e>uje~iai a<i AD, DB, ka`i d'eon >'estw
e<ure~in t`hn dunam'enhn a>ut'ac. ke'isjwsan g'ar, <'wste >orj`hn
gwn'ian peri'eqein t`hn <up`o AD, DB, ka`i >epeze'uqjw <h AB; faner`on
p'alin, <'oti <h t`ac AD, DB dunam'enh >est`in <h AB; <'oper
>'edei de~ixai.}}

\ParallelRText{
\begin{center}
{\large Lemma}
\end{center}

For two given unequal straight-lines, to find by (the square on) which (straight-line)
the square on the greater (straight-line is) larger   than (the square on) the lesser.$^\dag$

\epsfysize=1.2in
\centerline{\epsffile{Book10/fig013ae.eps}}

Let $AB$ and $C$ be the two given unequal straight-lines, and let $AB$
be the greater of them. So it is required to find by (the square on) which (straight-line)  the square on $AB$ (is) greater  than (the square on) $C$.

Let the semi-circle $ADB$ have been described on $AB$.  And let $AD$,
equal to $C$, have been inserted into it [Prop. 4.1]. 
And let $DB$ have been joined. So (it is) clear that the angle $ADB$
is  a right-angle [Prop. 3.31], and that the square on $AB$
(is) greater than (the square on) $AD$---that is to say, (the square on) $C$---by (the square on) $DB$ [Prop. 1.47].

And, similarly, the square-root of (the sum of the squares on) two given straight-lines
is also found likeso.

Let $AD$ and $DB$ be the two given straight-lines. And let it
be necessary to find the square-root of (the sum of the squares on) them. For let
them have been laid down such as to encompass a right-angle---(namely), that (angle encompassed) by
$AD$ and $DB$. And let $AB$ have been joined. (It is) again clear that $AB$ is the square-root of   (the sum of the squares on) $AD$ and $DB$ [Prop. 1.47]. (Which is) the very thing it was required to show.}
\end{Parallel}


\vspace{7pt}{\footnotesize\noindent$^\dag$ That is, if $\alpha$ and $\beta$ are the lengths of two given straight-lines, with $\alpha$ being greater
than $\beta$, to find a straight-line of length $\gamma$ such that $\alpha^2=\beta^2+\gamma^2$. Similarly, we can also find $\gamma$
such that $\gamma^2=\alpha^2+\beta^2$.}

%%%%%%
% Prop 10.14
%%%%%%
\pdfbookmark[1]{Proposition 10.14}{pdf10.14}
\begin{Parallel}{}{} 
\ParallelLText{
\begin{center}
{\large \ggn{14}.}
\end{center}\vspace*{-7pt}

\gr{>E`an t'essarec e>uje~iai >an'alogon >~wsin, d'unhtai d`e <h pr'wth
t~hc deut'erac me~izon t~w| >ap`o summ'etrou <eaut~h| [m'hkei],
ka`i <h tr'ith t~hc tet'arthc me~izon dun'hsetai t~w| >ap`o summ'etrou
<eaut~h| [m'hkei]. ka`i >e`an <h pr'wth t~hc deut'erac me~izon d'unhtai
t~w| >ap`o >asumm'etrou <eaut~h| [m'hkei], ka`i <h tr'ith t~hc tet'arthc
me~izon dun'hsetai t~w| >ap`o >asumm'etrou <eaut~h| [m'hkei].}

\gr{>'Estwsan t'essarec e>uje~iai >an'alogon a<i A, B, G, D, <wc <h A
pr`oc t`hn B, o<'utwc <h G pr`oc t`hn D, ka`i <h A m`en t~hc B me~izon
dun'asjw t~w| >ap`o t~hc E, <h d`e G t~hc D me~izon dun'asjw
t~w| >ap`o t~hc Z; l'egw, <'oti, e>'ite s'ummetr'oc >estin <h A t~h| E,
s'ummetr'oc >esti ka`i <h G t~h| Z, e>'ite >as'ummetr'oc >estin <h A
t~h| E, >as'ummetr'oc >esti ka`i <o G t~h| Z.}\\~\\~\\~\\~\\~\\~\\~\\~\\

\epsfysize=2.in
\centerline{\epsffile{Book10/fig014g.eps}}

\gr{>Epe`i g'ar >estin <wc <h A pr`oc t`hn B, o<'utwc <h G pr`oc t`hn
D, >'estin >'ara ka`i <wc t`o >ap`o t~hc A pr`oc t`o >ap`o t~hc B,
o<'utwc t`o >ap`o t~hc G pr`oc t`o >ap`o t~hc D. >all`a t~w| m`en
>ap`o t~hc A >'isa >est`i t`a >ap`o t~wn E, B, t~w| d`e >ap`o t~hc G
>'isa >est`i t`a >ap`o t~wn D, Z. >'estin >'ara <wc t`a >ap`o t~wn E, B
pr`oc t`o >ap`o t~hc B, o<'utwc t`a >ap`o t~wn D, Z pr`oc t`o >ap`o
t~hc D; diel'onti >'ara >est`in <wc t`o >ap`o t~hc E pr`oc t`o >ap`o
t~hc B, o<'utwc t`o >ap`o t~hc Z pr`oc t`o >ap`o t~hc D; >'estin >'ara
ka`i <wc <h E pr`oc t`hn B, o<'utwc <h Z pr`oc t`hn D; >an'apalin
>'ara >est`in <wc <h B pr`oc t`hn E, o<'utwc <h D pr`oc t`hn Z. >'esti
d`e ka`i <wc <h A pr`oc t`hn B, o<'utwc <h G pr`oc t`hn D; di> >'isou
>'ara >est`in <wc <h A pr`oc t`hn E, o<'utwc <h G pr`oc t`hn Z.
e>'ite o>~un s'ummetr'oc >estin <h A t~h| E, summetr'oc >esti
ka`i <h G t~h| Z, e>'ite >as'ummetr'oc >estin <h A t~h| E, >as'ummetr'oc
>esti ka`i <h G t~h| Z.}

\gr{>E`an >'ara, ka`i t`a <ex~hc.}}

\ParallelRText{
\begin{center}
{\large Proposition 14}
\end{center}

If four straight-lines are proportional, and the
square on the first is greater  than (the square on) the second  by the (square) on (some straight-line) commensurable [in length] with the first, then the
square on the  third will also
be greater  than (the square on) the fourth by the (square) on (some straight-line) commensurable [in length]
with the third. And if the
square on the first is greater  than (the  square on) the second  by the (square) on (some straight-line) incommensurable [in length] with the first, then the square on the third will also
be greater than (the  square on) the fourth by the (square) on (some straight-line) incommensurable [in length]
with the third. 

Let $A$, $B$, $C$, $D$ be four proportional straight-lines, (such that)
as $A$ (is) to $B$, so $C$ (is) to $D$. And let the square on $A$ be greater 
than (the square on) $B$ by the (square) on $E$, and let the square on $C$ be greater
than (the square on) $D$ by the (square) on $F$. I say that 
$A$ is either commensurable (in length) with $E$, and $C$ is also commensurable with $F$, or $A$ is incommensurable (in length) with $E$, and $C$ is also incommensurable with $F$.

\epsfysize=2.in
\centerline{\epsffile{Book10/fig014e.eps}}

For since as $A$ is to $B$, so $C$ (is) to $D$, thus as the (square) on $A$
is to the (square) on $B$, so the (square) on $C$  (is) to the (square) on $D$
[Prop. 6.22]. But the (sum of the squares) on $E$ and $B$ is equal to the (square) on $A$, and the (sum of the squares) on 
$D$ and $F$ is equal to the (square) on $C$.
Thus, as the (sum of the squares) on $E$ and $B$ is to the (square) on
$B$, so the (sum of the squares) on $D$ and $F$ (is) to the (square)
on $D$.
 Thus, via separation, as the (square) on $E$ is to the (square) on $B$, so the (square) on $F$ (is) to the (square)
on $D$ [Prop. 5.17]. Thus, also, as $E$ is to $B$,
so $F$ (is) to $D$ [Prop. 6.22]. 
 Thus, inversely,  as $B$ is to $E$, so $D$ (is) to $F$ [Prop. 5.7~corr.]. But, as $A$ is to $B$, so 
 $C$ also (is) to $D$. Thus, via equality, as $A$ is to $E$, so $C$ (is) to $F$
 [Prop. 5.22]. Therefore, $A$ is either commensurable (in length)
 with $E$, and $C$ is also commensurable with $F$, or $A$ is
 incommensurable (in length) with $E$, and $C$ is also incommensurable with $F$ [Prop. 10.11].
 
 Thus, if, and so on \ldots.}
\end{Parallel}

%%%%%%
% Prop 10.15
%%%%%%
\pdfbookmark[1]{Proposition 10.15}{pdf10.15}
\begin{Parallel}{}{} 
\ParallelLText{
\begin{center}
{\large \ggn{15}.}
\end{center}\vspace*{-7pt}

\gr{>E`an d'uo meg'ejh s'ummetra suntej~h|, ka`i t`o <'olon <ekat'erw| a>ut~wn
s'ummetron >'estai; k>`an t`o <'olon <en`i a>ut~wn s'ummetron >~h|, ka`i
t`a >ex >arq~hc meg'ejh s'ummetra >'estai.}

\gr{Sugke'isjw g`ar d'uo meg'ejh s'ummetra t`a AB, BG;  l'egw, <'oti ka`i
<'olon t`o AG <ekat'erw| t~wn AB, BG >esti s'ummetron.}\\~\\~\\

\epsfysize=0.6in
\centerline{\epsffile{Book10/fig015g.eps}}

\gr{>Epe`i g`ar s'ummetr'a >esti t`a AB, BG, metr'hsei ti a>ut`a m'egejoc.
metre'itw, ka`i >'estw t`o D. >epe`i o>~un t`o D t`a AB, BG metre~i,
ka`i <'olon t`o AG metr'hsei. metre~i d`e ka`i t`a AB, BG. t`o D >'ara
t`a AB, BG, AG metre~i; s'ummetron >'ara >est`i t`o AG <ekat'erw|
t~wn AB, BG.}

\gr{>All`a d`h t`o AG >'estw s'ummetron t~w| AB; l'egw d'h, <'oti ka`i t`a
AB, BG s'ummetr'a >estin.}

\gr{>Epe`i g`ar s'ummetr'a >esti t`a AG, AB, metr'hsei ti a>ut`a m'egejoc.
metre'itw, ka`i >'estw t`o D. >epe`i o>~un t`o D t`a GA, AB
metre~i, ka`i loip`on >'ara t`o BG metr'hsei. metre~i d`e ka`i t`o AB;
t`o D >'ara t`a AB, BG metr'hsei; s'ummetra >'ara >est`i t`a AB, BG.}

\gr{>E`an >'ara d'uo meg'ejh, ka`i t`a <ex~hc.}}

\ParallelRText{
\begin{center}
{\large Proposition 15}
\end{center}

If two commensurable magnitudes are
added together then the whole will also be commensurable
with each of them. And if the whole is commensurable with one of
them then the original magnitudes will also be commensurable (with one another).

For let the two commensurable magnitudes $AB$ and $BC$ be laid
down together. I say that the whole $AC$ is also commensurable with
each of $AB$ and $BC$.

\epsfysize=0.6in
\centerline{\epsffile{Book10/fig015e.eps}}

For since $AB$ and $BC$ are commensurable, some  magnitude will
measure them. Let it (so) measure (them), and let it be $D$. Therefore,
since $D$ measures (both) $AB$ and $BC$, it will also measure the
whole $AC$. And it also measures $AB$ and $BC$. Thus, $D$
measures  $AB$, $BC$, and $AC$. Thus, $AC$ is commensurable
with each of $AB$ and $BC$ [Def. 10.1].

And so let $AC$ be commensurable with $AB$. I say that $AB$ and
$BC$ are also commensurable.

For since $AC$ and $AB$ are commensurable, some  magnitude
will measure them. Let it (so) measure (them), and let it be $D$. Therefore,
since $D$ measures (both) $CA$ and $AB$, it will thus also
measure the remainder $BC$. And it also measures $AB$. Thus, $D$ will
measure (both) $AB$ and $BC$. Thus, $AB$ and $BC$ are commensurable [Def. 10.1].

Thus, if two magnitudes, and so on \ldots.}
\end{Parallel}

%%%%%%
% Prop 10.16
%%%%%%
\pdfbookmark[1]{Proposition 10.16}{pdf10.16}
\begin{Parallel}{}{} 
\ParallelLText{
\begin{center}
{\large \ggn{16}.}
\end{center}\vspace*{-7pt}

\gr{>E`an d'uo meg'ejh >as'ummetra suntej~h|, ka`i t`o <'olon <ekat'erw|
a>ut~wn >as'ummetron >'estai; k>`an t`o <'olon <en`i a>ut~wn
>as'ummetron >~h|, ka`i t`a >ex >arq~hc meg'ejh >as'ummetra >'estai.}\\~\\

\epsfysize=0.6in
\centerline{\epsffile{Book10/fig015g.eps}}

\gr{Sugke'isjw g`ar d'uo meg'ejh >as'ummetra t`a AB, BG; l'egw, <'oti
ka`i <'olon t`o AG <ekat'erw| t~wn AB, BG >as'ummetr'on >estin.}

\gr{E>i g`ar m'h >estin >as'ummetra t`a GA, AB, metr'hsei ti [a>ut`a]
m'egejoc. metre'itw, e>i dunat'on, ka`i >'estw t`o D. >epe`i o>~un
t`o D t`a GA, AB metre~i, ka`i loip`on >'ara t`o BG metr'hsei. metre~i
d`e ka`i t`o AB; t`o D >'ara t`a AB, BG metre~i. s'ummetra >'ara >est`i
t`a AB, BG; <up'ekeinto d`e ka`i >as'ummetra; <'oper >est`in >ad'unaton.
o>uk >'ara t`a GA, AB metr'hsei ti m'egejoc; >as'ummetra >'ara
>est`i t`a GA, AB. <omo'iwc d`h de'ixomen,  <'oti ka`i t`a AG, GB
>as'ummetr'a >estin. t`o AG >'ara <ekat'erw| t~wn AB, BG >as'ummetr'on
>estin.}

\gr{>All`a d`h t`o AG <en`i t~wn AB, BG >as'ummetron >'estw. >'estw d`h
pr'oteron t~w| AB; l'egw, <'oti ka`i t`a AB, BG >as'ummetr'a >estin.
e>i g`ar >'estai s'ummetra, metr'hsei ti a>ut`a m'egejoc. metre'itw,
ka`i >'estw t`o D. >epe`i o>~un t`o D t`a AB, BG metre~i, ka`i
<'olon >'ara t`o AG metr'hsei. metre~i d`e ka`i t`o AB; t`o D >'ara
t`a GA, AB metre~i. s'ummetra >'ara >est`i t`a GA, AB; <up'ekeito
d`e ka`i >as'ummetra; <'oper >est`in >ad'unaton. o>uk >'ara t`a AB, BG
metr'hsei ti m'egejoc; >as'ummetra >'ara >est`i t`a AB, BG.}

\gr{>E`an >'ara d'uo meg'ejh, ka`i t`a <ex~hc.}}

\ParallelRText{
\begin{center}
{\large Proposition 16}
\end{center}

If two incommensurable magnitudes are
added together then the whole will also be incommensurable with
each of them. And if the whole is incommensurable with one of them then
the original magnitudes will also be incommensurable (with one another).

\epsfysize=0.6in
\centerline{\epsffile{Book10/fig015e.eps}}

For let the two incommensurable magnitudes $AB$ and $BC$ be
laid down together. I say that that the whole $AC$ is also
incommensurable with each of $AB$ and $BC$.

For if $CA$ and $AB$ are not incommensurable then some magnitude
will measure [them]. If possible, let it (so) measure (them), and let it
be $D$. Therefore, since $D$ measures (both) $CA$ and $AB$, it will
thus also measure the remainder $BC$. And it also measures $AB$.
Thus, $D$ measures (both) $AB$ and $BC$. Thus, $AB$ and $BC$ are
commensurable [Def. 10.1]. But they were also assumed (to be) incommensurable.
The very thing is impossible. Thus, some magnitude cannot
measure (both) $CA$ and $AB$. Thus, $CA$ and $AB$ are incommensurable [Def. 10.1]. So, similarly, we can show that $AC$ and
$CB$ are also incommensurable. Thus, $AC$ is incommensurable with each
of $AB$ and $BC$.

And so let $AC$ be incommensurable with one of $AB$ and $BC$. So
let it, first of all, be incommensurable with $AB$. I say that $AB$ and
$BC$ are also incommensurable. For if they are commensurable then
some magnitude will measure them. Let it (so) measure (them), and let it
be $D$. Therefore, since $D$ measures (both) $AB$ and $BC$, 
it will thus also measure the whole  $AC$. And it also measures $AB$. 
Thus, $D$ measures (both) $CA$ and $AB$. Thus, $CA$ and $AB$
are commensurable  [Def. 10.1]. But they
were also assumed (to be) incommensurable. The very thing is
impossible. Thus, some magnitude cannot measure (both) $AB$ and
$BC$. Thus, $AB$ and $BC$ are incommensurable  [Def. 10.1].

Thus, if two\ldots magnitudes, and so on \ldots.}
\end{Parallel}

%%%%%%
% Prop 10.16a
%%%%%%
\begin{Parallel}{}{} 
\ParallelLText{
\begin{center}
{\large \gr{L~hmma}.}
\end{center}\vspace*{-7pt}

\gr{>E`an par'a tina e>uje~ian parablhj~h| parallhl'ogram\-mon >elle~ipon
e>'idei tetrag'wnw|, t`o parablhj`en >'ison >est`i t~w| <up`o t~wn
>ek t~hc parabol'hc genom'enwn tmhm'atwn t~hc e>uje'iac.}\\~\\

\epsfysize=1.3in
\centerline{\epsffile{Book10/fig016ag.eps}}

\gr{Par`a g`ar e>uje~ian t`hn AB parabebl'hsjw parallhl'ogrammon t`o
AD >elle~ipon e>'idei tetrag'wnw| t~w| DB;
l'egw, <'oti >'ison >est`i t`o AD t~w| <up`o t~wn AG, GB.}

\gr{Ka'i >estin a>ut'ojen faner'on; >epe`i g`ar tetr'agwn'on >esti t`o DB,
>'ish >est`in <h DG t~h| GB, ka'i >esti t`o AD t`o <up`o t~wn
AG, GD, tout'esti t`o <up`o t~wn AG, GB.}

\gr{>E`an >'ara par'a tina e>uje~ian, ka`i t`a <ex~hc.}}

\ParallelRText{
\begin{center}
{\large Lemma}
\end{center}

If a parallelogram,$^\dag$
 falling short by a square figure, is applied to some straight-line then the applied (parallelogram) is equal (in area) to the (rectangle contained) by the
pieces of the straight-line created via the application (of the parallelogram).

\epsfysize=1.3in
\centerline{\epsffile{Book10/fig016ae.eps}}

For let the parallelogram $AD$, falling short by the square figure $DB$, have been applied to the straight-line $AB$. I say that $AD$ is equal to
the (rectangle contained) by $AC$ and $CB$.

And it is immediately obvious. For since $DB$ is a square, $DC$ is
equal to $CB$. And $AD$ is the (rectangle contained) by $AC$ and
$CD$---that is to say, by $AC$ and $CB$.

Thus, if \ldots to some straight-line, and so on \ldots.}
\end{Parallel}


\vspace{7pt}{\footnotesize\noindent$^\dag$  Note that this lemma only applies to rectangular parallelograms.}

%%%%%%
% Prop 10.17
%%%%%%
\pdfbookmark[1]{Proposition 10.17}{pdf10.17}
\begin{Parallel}{}{} 
\ParallelLText{
\begin{center}
{\large \ggn{17}.}
\end{center}\vspace*{-7pt}

\gr{>E`an >~wsi d'uo e>uje~iai >'anisoi, t~w| d`e tetr'atw| m'erei to~u >ap`o
t~hc >el'assonoc >'ison par`a t`hn me'izona parablhj~h| >elle~ipon
e>'idei tetrag'wnw| ka`i e>ic s'ummetra a>ut`hn diair~h| m'hkei, <h
me'izwn t~hc >el'assonoc me~izon dun'hsetai t~w| >ap`o summ'etou
<eaut~h| [m'hkei]. ka`i >e`an <h me'izwn t~hc >el'assonoc me~izon
d'unhtai t~w| >ap`o summ'etrou <eaut~h| [m'hkei], t~w| d`e tetr'artw| to~u
>ap`o t~hc >el'assonoc >'ison par`a t`hn me'izona parablhj~h| >elle~ipon
e>'idei tetrag'wnw|, e>ic s'ummetra a>ut`hn diaire~i m'hkei.}

\gr{>'Estwsan d'uo e>uje~iai >'anisoi a<i A, BG, <~wn me'izwn
<h BG, t~w| d`e tetr'artw| m'erei to~u >ap`o >el'assonoc
t~hc A, tout'esti t~w| >ap`o t~hc <hmise'iac t~hc A, >'ison par`a
t`hn BG parabebl'hsjw >elle~ipon e>'idei tetrag'wnw|, ka`i >'estw
t`o <up`o t~wn BD, DG, s'ummetroc d`e >'estw <h BD t~h| DG
m'hkei; l'egw, <`oti <h BG t~hc A me~izon d'unatai t~w| >ap`o
summ'etrou <eaut~h|.}\\~\\~\\~\\~\\~\\~\\~\\~\\

\epsfysize=1.1in
\centerline{\epsffile{Book10/fig017g.eps}}

\gr{Tetm'hsjw g`ar <h BG d'iqa kat`a t`o E shme~ion, ka`i ke'isjw t~h| DE
>'ish <h EZ. loip`h >'ara <h DG >'ish >est`i t~h| BZ. ka`i >epe`i e>uje~ia
<h BG t'etmhtai e>ic m`en >'isa kat`a t`o E, e>ic d`e >'anisa kat`a t`o D,
t`o >'ara <up`o BD, DG pereiq'omenon >orjog'wnion met`a to~u >ap`o
t~hc ED tetrag'wnou >'ison >est`i t~w| >ap`o t~hc EG tetrag'wnw|;
ka`i t`a tetrapl'asia; t`o >'ara tetr'akic <up`o t~wn BD, DG met`a to~u
tetraplas'iou to~u >ap`o t~hc DE >'ison >est`i t~w| tetr'akic >ap`o
t~hc EG tetrag'wnw|.
>all`a t~w| m`en tetraplas'iw| to~u <up`o t~wn BD, DG >'ison >est`i
t`o >ap`o t~hc A tetr'agwnon, t~w| d`e tetraplas'iw| to~u >ap`o t~hc
DE >'ison >est`i t`o >ap`o t~hc DZ tetr'agwnon; diplas'iwn g'ar
>estin <h DZ t~hc DE. t~w| d`e tetraplas'iw| to~u >ap`o t~hc EG >'ison
>est`i t`o >ap`o t~hc BG tetr'agwnon; diplas'iwn g'ar >esti p'alin
<h BG t~hc GE. t`a >'ara >ap`o t~wn A, DZ tetr'agwna >'isa >est`i
t~w| >ap`o t~hc BG tetr'agwnw|; <'wste t`o >ap`o t~hc BG to~u >ap`o
t~hc A me~iz'on >esti t~w| >ap`o t~hc DZ; <h BG >'ara t~hc A me~izon
d'unatai t~h| DZ. deikt'eon, <'oti ka`i s'ummetr'oc >estin <h BG t~h|
DZ. >epe`i g`ar s'ummetr'oc >estin <h BD t~h|
DG m'hkei, s'ummetroc >'ara >est`i ka`i <h BG t~h| GD m'hkei. >all`a
<h GD ta~ic GD, BZ >esti s'ummetroc m'hkei; >'ish g'ar
>estin <h GD t~h| BZ. ka`i <h BG >'ara s'ummetr'oc >esti ta~ic BZ, GD
m'hkei; <'wste ka`i loip~h| t~h| ZD s'ummetr'oc >estin <h BG m'hkei;
<h BG >'ara t~hc A me~izon d'unatai t~w| >ap`o
summ'etrou <eaut~h|.}

\gr{>All`a d`h <h BG t~hc A me~izon dun'asjw t~w| >ap`o summ'etrou
<eaut~h|, t~w| d`e tetr'atrw| to~u >ap`o t~hc A >'ison par`a t`hn BG
parabebl'hsjw >elle~ipon e>'idei tetrag'wnw|, ka`i >'estw t`o <up`o
t~wn BD, DG. deikt'eon, <'oti s'ummetr'oc >estin <h BD t~h| DG
m'hkei.}

\gr{T~wn g`ar a>ut~wn kataskeuasj'entwn <omo'iwc de'ixomen, <'oti
<h BG t~hc A me~izon d'unatai t~w| >ap`o t~hc ZD.
d'unatai d`e <h BG t~hc A me~izon t~w| >ap`o summ'etrou <eaut~h|.
s'ummetroc >'ara >est`in <h BG t~h| ZD m'hkei;
<'wste ka`i loip~h| sunamfot'erw| t~h| BZ, DG s'ummetr'oc
>estin <h BG m'hkei. >all`a sunamf'oteroc <h BZ, DG s'ummetr'oc
>esti t~h| DG [m'hkei]. <'wste ka`i <h BG t~h| GD s'ummetr'oc
>esti m'hkei; ka`i diel'onti >'ara <h BD t~h| DG >esti s'ummetroc
m'hkei.}

\gr{>E`an >'ara >~wsi d'uo e>uje~iai >'anisoi, ka`i t`a <ex~hc.}}

\ParallelRText{
\begin{center}
{\large Proposition 17}$^\dag$
\end{center}

If there are two unequal straight-lines,
and a (rectangle) equal to the fourth part of the (square) on the
lesser, falling short by a square figure, is applied to the greater, and
divides it into (parts which are) commensurable in length, then
 the  square on the greater  will be  larger than (the  square on) the lesser  by the (square)
on (some straight-line) commensurable [in length] with the greater.
And if the square on the greater is larger than (the square on) the lesser  by the (square)
on (some straight-line) commensurable [in length] with the greater, and
a (rectangle) equal to the fourth (part) of the (square) on the lesser,
falling short by a square figure, is applied to the greater, then it divides it into (parts
which are) commensurable in length.

Let $A$ and $BC$ be two unequal straight-lines, of which (let) $BC$ (be) the
greater. And let a (rectangle) equal to the fourth part of the (square) on the
lesser, $A$---that is, (equal) to the (square) on half of $A$---falling short by a
square figure, have been applied to $BC$. And let it be the (rectangle contained) by $BD$ and $DC$ [see previous lemma]. And let $BD$ be commensurable in length with
$DC$. I say that that the square on $BC$ is greater than the (square on) $A$ by (the square on some straight-line) commensurable (in length) with ($BC$).

\epsfysize=1.1in
\centerline{\epsffile{Book10/fig017e.eps}}

For let $BC$ have been cut in  half at the point $E$ [Prop. 1.10]. And let $EF$ be made equal to
$DE$ [Prop. 1.3]. Thus, the remainder $DC$
is equal to $BF$. And since the straight-line $BC$ has been cut into equal
(pieces) at $E$, and into unequal (pieces) at $D$, the rectangle contained by
$BD$ and $DC$, plus the square on $ED$, is thus equal to the square on $EC$ [Prop. 2.5]. (The same) also (for) the quadruples.  
Thus, four times the (rectangle contained) by $BD$ and $DC$, plus
the quadruple of the (square) on $DE$, is equal to four times the square on 
$EC$. But, the square on $A$ is equal to the quadruple of the (rectangle
contained) by $BD$ and $DC$, and the square on $DF$ is equal to
the quadruple of the (square) on $DE$. For $DF$ is double $DE$. 
And the square on  $BC$ is equal to the quadruple of the (square) on $EC$.
For, again, $BC$ is double $CE$. Thus, the (sum of the) squares on $A$ and
$DF$ is equal to the square on $BC$. Hence, the (square) on $BC$
is greater than the (square) on $A$ by the (square) on $DF$. Thus,
$BC$ is greater in square than $A$ by $DF$. It must also be shown that $BC$
is commensurable (in length) with $DF$. For since $BD$ is commensurable in length
with $DC$, $BC$ is thus also commensurable in length with $CD$ [Prop. 10.15]. 
But, $CD$ is commensurable in length with $CD$ plus $BF$. For $CD$ is equal
to $BF$ [Prop. 10.6].  Thus, $BC$ is also commensurable in length with $BF$ plus $CD$
[Prop. 10.12]. Hence, $BC$ is also commensurable
in length with the remainder $FD$ [Prop. 10.15]. Thus, the square on $BC$ is
greater than (the square on) $A$ by the (square) on (some straight-line)
commensurable (in length) with ($BC$).

And so let the square on $BC$ be greater than the (square on) $A$ by the
(square) on (some straight-line) commensurable (in length) with ($BC$). And let
a (rectangle) equal to the fourth (part) of the (square) on $A$, falling
short by a square figure, have been applied to $BC$. And let it be
the (rectangle contained) by $BD$ and $DC$. It must be shown that $BD$
is commensurable in length with $DC$.

For, similarly, by the same construction, we can  show that the
square on $BC$ is greater than the (square on) $A$ by the (square) on $FD$.
And the square on $BC$ is greater than the (square on) $A$ by the
(square) on (some straight-line) commensurable (in length) with ($BC$).  Thus, $BC$ is
commensurable in length with $FD$. Hence, $BC$ is also commensurable in length with the remaining sum of $BF$ and $DC$ [Prop. 10.15]. But, the sum of $BF$ and $DC$
is commensurable [in length] with $DC$ [Prop. 10.6]. Hence, $BC$ is also
commensurable in length with $CD$ [Prop. 10.12]. 
Thus, via separation, $BD$ is also commensurable in length with $DC$ [Prop. 10.15].

Thus,  if there are two unequal straight-lines, and so on \ldots.}
\end{Parallel}


\vspace{7pt}{\footnotesize\noindent$^\dag$ This proposition states that if $\alpha\,x-x^2=\beta^2/4$ (where $\alpha=BC$, $x=DC$,
and $\beta=A$) then $\alpha$ and $\sqrt{\alpha^2-\beta^2}$ are commensurable when $\alpha-x$ are $x$ are commensurable, and {\em vice versa}.}

%%%%%%
% Prop 10.18
%%%%%%
\pdfbookmark[1]{Proposition 10.18}{pdf10.18}
\begin{Parallel}{}{} 
\ParallelLText{
\begin{center}
{\large\ggn{18}.}
\end{center}\vspace*{-7pt}

\gr{>E`an >~wsi d'uo e>uje~iai >'anisoi, t~w| d`e tet'artw| m'erei to~u >ap`o
t~hc >el'assonoc >'ison par`a t`hn me'izona parablhj~h| >elle~ipon e>'idei
tetrag'wnw|, ka`i e>ic >asummetra a>ut`hn diair~h| [m'hkei], <h me'izwn
t~hc >el'assonoc me~izon dun'hsetai t~w| >ap`o >asumm'etrou <eaut~h|.
ka`i >e`an <h me'izwn t~hc >el'assonoc me~izon d'unhtai t~w| >ap`o
>asumm'etrou <eaut~h|, t~w| d`e tetr'artw| to~u >ap`o t~hc >el'assonoc
>'ison par`a t`hn me'izona parablhj~h| >elle~ipon e>'idei tetrag'wnw|,
e>ic >as'ummetra a>ut`hn diaire~i [m'hkei].}

\gr{>'Estwsan d'uo e>uje~iai >'anisoi a<i A, BG, <~wn me'izwn <h BG, t~w|
d`e tet'artw| [m'erei] to~u >ap`o t~hc >el'assonoc t~hc A >'ison par`a
t`hn BG parabebl'hsjw >elle~ipon e>'idei tetrag'wnw|, ka`i >'estw
t`o <up`o t~wn BDG, >as'ummetroc d`e >'estw <h BD t~h| DG m'hkei;
l'egw, <'oti <h BG t~hc A me~izon d'unatai t~w| >ap`o >asumm'etrou
<eaut~h|.}\\~\\~\\~\\~\\~\\~\\~\\~\\

\epsfysize=0.8in
\centerline{\epsffile{Book10/fig018g.eps}}

\gr{T~wn g`ar a>ut~wn kataskeuasj'entwn t~w| pr'oteron <omo'iwc de'ixomen,
<'oti <h BG t~hc A me~izon d'unatai t~w| >ap`o t~hc ZD. deikt'eon
[o>~un], <'oti >as'ummetr'oc >estin <h BG t~h| DZ m'hkei. >epe`i g`ar
>as'ummetr'oc >estin <h  BD t~h| DG m'hkei, >as'ummetroc >'ara >est`i ka`i
<h BG t~h| GD m'hkei. >all`a <h DG s'ummetr'oc >esti sunamfot'eraic
ta~ic BZ, DG; ka`i <h BG >'ara >as'ummetr'oc >esti sunamfot'eraic
ta~ic BZ, DG. <'wste ka`i loip~h| t~h| ZD >as'ummetr'oc  >'estin <h BG
m'hkei. ka`i <h BG t~hc A me~izon d'unatai t~w| >ap`o t~hc ZD; <h BG
>'ara t~hc A me~izon d'unatai t~w| >ap`o >asumm'etrou <eaut~h|.}

\gr{Dun'asjw d`h p'alin <h BG t~hc A me~izon t~w| >ap`o >asumm'etr\-ou
<eaut~h|, t~w| d`e tet'artw| to~u >ap`o t~hc A >'ison par`a t`hn
BG parabebl'hsjw >elle~ipon e>'idei tetrag'wnw|, ka`i >'estw t`o <up`o t~wn
BD, DG. deikt'eon, <'oti >as'ummetr'oc >estin <h BD t~h| DG m'hkei.}

\gr{T~wn g`ar a>ut~wn kataskeuasj'entwn <omo'iwc de'ixomen, <'oti
<h BG t~hc A me~izon d'unatai t~w| >ap`o t~hc ZD. >all`a <h BG
t~hc A me~izon d'unatai t~w| >ap`o >asumm'etrou <eaut~h|. >as'ummetroc
>'ara >est`in <h BG t~h| ZD m'hkei; <'wste ka`i loip~h| sunamfot'erw|
t~h| BZ, DG >as'ummetr'oc >estin <h BG. >all`a sunamf'oteroc <h BZ,
DG t~h| DG s'ummetr'oc >esti m'hkei; ka`i <h BG >'ara t~h| DG
>as'ummetr'oc >esti m'hkei; <'wste ka`i diel'onti <h BD t~h| DG
>as'ummetr'oc >esti m'hkei.}

\gr{>E`an >'ara >~wsi d'uo e>uje~iai, ka`i t`a <ex~hc.}}

\ParallelRText{
\begin{center}
{\large Proposition 18}$^\dag$
\end{center}

If there are two unequal straight-lines,
and a (rectangle) equal to the fourth part of the (square) on the
lesser, falling short by a square figure, is applied to the greater, and
divides it into (parts which are) incommensurable [in length], then
 the  square on the greater  will be  larger than the  (square on the) lesser  by the (square)
on (some straight-line) incommensurable  (in length) with the greater.
And if the square on the greater is larger than the (square on the) lesser  by the (square)
on (some straight-line) incommensurable  (in length) with the greater, and
a (rectangle) equal to the fourth (part) of the (square) on the lesser,
falling short by a square figure, is applied to the greater, then it divides it into (parts
which are) incommensurable [in length].

Let $A$ and $BC$ be two unequal straight-lines, of which (let) $BC$ (be) the
greater. And let a (rectangle) equal to the fourth [part] of the (square) on the
lesser, $A$, falling short by a
square figure, have been applied to $BC$. And let it be the (rectangle contained) by $BDC$. And let $BD$ be incommensurable in length with
$DC$. I say that that the square on $BC$ is greater than the (square on) $A$ by the (square) on (some straight-line) incommensurable (in length) with ($BC$).

\epsfysize=0.8in
\centerline{\epsffile{Book10/fig018e.eps}}

For, similarly, by the same construction as before, we can show that
the square on $BC$ is greater than the (square on) $A$ by the (square)
on $FD$. [Therefore] it must be shown that $BC$ is incommensurable
in length
with $DF$. For since $BD$ is incommensurable in length with $DC$, $BC$
is thus also incommensurable in length with $CD$ [Prop. 10.16]. But, $DC$ is commensurable (in length) with the sum of 
$BF$ and $DC$ [Prop. 10.6]. And, thus,
$BC$ is incommensurable (in  length) with the sum of $BF$ and $DC$ [Prop. 10.13]. Hence, $BC$ is also incommensurable in length with the remainder $FD$ [Prop. 10.16]. And the square on $BC$ is greater than the (square on) $A$ by the (square) on $FD$. Thus, the square on $BC$
is greater than the (square on) $A$ by the (square) on (some straight-line) incommensurable
(in length) 
with ($BC$).

So, again, let the square on $BC$ be greater than the (square on) $A$ by
the (square) on (some straight-line) incommensurable (in length) with ($BC$). And let a (rectangle) equal to the fourth [part] of the (square) on  $A$, falling short by a
square figure, have been applied to $BC$.  And let it be the (rectangle contained) by $BD$ and $DC$. It must be shown that $BD$ is incommensurable in length with $DC$.

For, similarly, by the same construction, we can show that the square on $BC$ is greater than the (square) on $A$ by the (square) on $FD$. But,
the square on $BC$ is greater than the (square) on $A$ by the (square) on (some straight-line)
incommensurable (in length) with ($BC$). Thus, $BC$ is incommensurable in length with
$FD$. Hence, $BC$ is also incommensurable (in length) with the remaining sum of $BF$ and
$DC$ [Prop. 10.16]. But, the sum of
$BF$ and $DC$ is commensurable in length with $DC$ [Prop. 10.6]. Thus, $BC$ is also incommensurable
in length with $DC$ [Prop. 10.13]. Hence, via
separation, $BD$ is
also incommensurable in length with $DC$ [Prop. 10.16].

Thus,  if there are two \ldots straight-lines, and so on \ldots.}
\end{Parallel}


\vspace{7pt}{\footnotesize\noindent$^\dag$ This proposition states that if $\alpha\,x-x^2=\beta^2/4$ (where $\alpha=BC$, $x=DC$,
and $\beta=A$) then $\alpha$ and $\sqrt{\alpha^2-\beta^2}$ are incommensurable when $\alpha-x$ are $x$ are incommensurable, and {\em vice versa}.}

%%%%%%
% Prop 10.19
%%%%%%
\pdfbookmark[1]{Proposition 10.19}{pdf10.19}
\begin{Parallel}{}{} 
\ParallelLText{
\begin{center}
{\large \ggn{19}.}
\end{center}\vspace*{-7pt}

\gr{T`o <up`o <rht~wn m'hkei summ'etrwn  e>ujei~wn perieq'omenon >orjog'wnion <rht'on >estin.}

\gr{<Up`o g`ar <rht~wn m'hkei summ'etrwn e>ujei~wn t~wn
AB, BG >orjog'wnion perieq'esjw t`o AG; l'egw, <'oti <rht'on >esti
t`o AG.}

\epsfysize=2in
\centerline{\epsffile{Book10/fig019g.eps}}

\gr{>Anagegr'afjw g`ar >ap`o t~hc AB tetr'agwnon t`o AD;
<rht`on >'ara >est`i t`o AD. ka`i >epe`i s'ummetr'oc >estin <h
AB t~h| BG m'hkei, >'ish d'e >estin <h AB t~h| BD, s'ummetroc
>'ara >est`in <h BD t~h| BG m'hkei. ka'i >estin <wc <h BD pr`oc t`hn
 BG, o<'utwc t`o DA pr`oc t`o AG. s'ummetron
>'ara >est`i t`o DA t~w| AG. <rht`on d`e t`o DA; <rht`on >'ara >est`i
ka`i t`o AG.}

\gr{T`o >'ara <up`o <rht~wn m'hkei summ'etrwn, ka`i t`a <ex~hc.}}

\ParallelRText{
\begin{center}
{\large Proposition 19}
\end{center}

The rectangle contained by rational straight-lines
(which are) commensurable in length is rational.

For let the rectangle $AC$ have been enclosed by the rational straight-lines
$AB$ and $BC$ (which are) commensurable in length. I say that
$AC$ is rational.

\epsfysize=2in
\centerline{\epsffile{Book10/fig019e.eps}}

For let the square $AD$ have been described on $AB$.  $AD$ is thus rational [Def. 10.4]. And since $AB$ is commensurable in length with $BC$, and $AB$ is equal to $BD$, $BD$ is thus
commensurable in  length with $BC$.  And as $BD$ is to
$BC$, so $DA$ (is) to $AC$ [Prop. 6.1]. Thus,
$DA$ is commensurable with $AC$ [Prop. 10.11]. 
And $DA$ (is) rational. Thus, $AC$ is also rational [Def. 10.4].
Thus,  the \ldots by rational straight-lines
\ldots commensurable, and so on \ldots.}
\end{Parallel}

%%%%%%
% Prop 10.20
%%%%%%
\pdfbookmark[1]{Proposition 10.20}{pdf10.20}
\begin{Parallel}{}{} 
\ParallelLText{
\begin{center}
{\large \ggn{20}.}
\end{center}\vspace*{-7pt}

\gr{>E`an <rht`on par`a <rht`hn parablhj~h|, pl'atoc poie~i <rht`hn ka`i s'ummetron t~h|, par> <`hn par'akeitai, m'hkei.}\\~\\

\epsfysize=2.5in
\centerline{\epsffile{Book10/fig020g.eps}}

\gr{<Rht`on g`ar t`o AG par`a <rht`hn t`hn AB parabebl'hsjw pl'atoc poio~un t`hn BG; l'egw, <'oti
<rht'h >estin <h BG ka`i s'ummetroc t~h| BA m'hkei.}

\gr{>Anagegr'afjw g`ar >ap`o t~hc AB tetr'agwnon t`o AD;
<rht`on >'ara >est`i t`o AD. <rht`on d`e ka`i t`o AG; s'ummetron >'ara
>est`i t`o DA t~w| AG. ka'i >estin <wc t`o DA pr`oc t`o AG, o<'utwc
<h DB pr`oc t`hn BG. s'ummetroc >'ara >est`i ka`i <h DB t~h| BG;
>'ish d`e <h DB t~h| BA; s'ummetroc >'ara ka`i <h
AB t~h| BG.
<rht`h d'e >estin <h AB; <rht`h >'ara >est`i ka`i <h BG ka`i
s'ummetroc t~h| AB m'hkei.}

\gr{>E`an >'ara <rht`on par`a <rht`hn parablhj~h|, ka`i t`a <ex~hc.}}

\ParallelRText{
\begin{center}
{\large Proposition 20}
\end{center}

If a rational (area) is applied to a rational (straight-line) then it produces as breadth  a (straight-line which is) rational, and commensurable in length
with the (straight-line) to which it is applied.

\epsfysize=2.5in
\centerline{\epsffile{Book10/fig020e.eps}}

For let the rational (area) $AC$ have been applied to the rational (straight-line) $AB$, producing the (straight-line) $BC$ as breadth. I say that
$BC$ is rational, and commensurable in length with $BA$.

For let the square $AD$ have been described on $AB$. $AD$ is thus rational
[Def. 10.4]. And $AC$ (is) also rational. $DA$
is thus commensurable  with $AC$. And as $DA$ is to $AC$, so $DB$ (is)
to $BC$ [Prop. 6.1]. Thus, $DB$ is also
commensurable (in length) with $BC$ [Prop. 10.11]. And $DB$ (is) equal to $BA$. Thus, $AB$ (is)
also commensurable (in length) with $BC$. And $AB$ is rational. Thus, $BC$ is
also rational, and commensurable in length with $AB$ [Def. 10.3]. 

Thus, if  a rational (area) is applied to a rational (straight-line), and so
on \ldots.}
\end{Parallel}

%%%%%%
% Prop 10.21
%%%%%%
\pdfbookmark[1]{Proposition 10.21}{pdf10.21}
\begin{Parallel}{}{} 
\ParallelLText{
\begin{center}
{\large \ggn{21}.}
\end{center}\vspace*{-7pt}

\gr{T`o <up`o <rht~wn dun'amei m'onon summ'etrwn e>ujei~wn perieq'omenon
>orjog'wnion >'alog'on >estin, ka`i <h dunam'enh a>ut`o >'alog'oc >estin,
kale'isjw d`e m'esh.}

\epsfysize=2.5in
\centerline{\epsffile{Book10/fig020g.eps}}

\gr{<Up`o g`ar <rht~wn dun'amei m'onon summ'etrwn e>ujei~wn t~wn
AB, BG >orjog'wnion perieq'esjw t`o AG; l'egw, <'oti >'alog'on >esti
t`o AG, ka`i <h dunam'enh a>ut`o >'alog'oc >estin, kale'isjw d`e m'esh.}

\gr{>Anagegr'afjw g`ar >ap`o t~hc AB tetr'agwnon t`o AD; <rht`on >'ara
>est`i t`o AD. ka`i >epe`i >as'ummetr'oc >estin <h AB t~h| BG m'hkei;
dun'amei g`ar m'onon <up'okeintai s'ummetroi; >'ish d`e <h AB t~h|
BD, >as'ummetroc >'ara >est`i ka`i <h DB t~h| BG m'hkei. ka'i >estin
<wc <h DB pr`oc t`hn BG, o<'utwc t`o AD pr`oc t`o AG; >as'ummetron
>'ara [>est`i] t`o DA t~w| AG. <rht`on d`e t`o DA; >'alogon >'ara >est`i
t`o AG; <'wste ka`i <h dunam'enh t`o AG [tout'estin <h >'ison a>ut~w|
tetr'agwnon dunam'enh] >'alog'oc >estin, kale'isjw de m'esh; <'oper >'edei
de~ixai.}}

\ParallelRText{
\begin{center}
{\large Proposition 21}
\end{center}

The rectangle contained by rational straight-lines (which are) commensurable in square only is irrational, and its square-root  is irrational---let it be called medial.$^\dag$

\epsfysize=2.5in
\centerline{\epsffile{Book10/fig020e.eps}}

For let the rectangle $AC$ be contained by the rational straight-lines $AB$ and $BC$  (which are) commensurable in square only. I say that $AC$ is irrational, and
its square-root is irrational---let it be called
medial.

For let the square $AD$ have been described on  $AB$. $AD$ is thus
rational [Def. 10.4]. And since $AB$ is incommensurable in length with $BC$. For they were assumed to be commensurable in square only. And $AB$ (is) equal to $BD$. $DB$ is thus
also incommensurable in length with $BC$. And as $DB$ is to $BC$, so
$AD$ (is) to $AC$ [Prop. 6.1]. Thus, $DA$ [is]
incommensurable with $AC$ [Prop. 10.11]. And $DA$ (is) rational. Thus, $AC$ is irrational [Def. 10.4]. Hence, its square-root [that is to say, the square-root of the square equal to it] is also irrational [Def. 10.4]---let it be called medial. (Which is) the
very thing it was required to show.}
\end{Parallel}


\vspace{7pt}{\footnotesize\noindent$^\dag$ 
Thus, a medial straight-line has a length expressible as $k^{1/4}$.}
\newpage

%%%%%%
% Prop 10.21a
%%%%%%
\begin{Parallel}{}{} 
\ParallelLText{
\begin{center}
{\large \gr{L~hmma}.}
\end{center}\vspace*{-7pt}

\gr{>E`an >~wsi d'uo e>uje~iai, >'estin <wc <h pr'wth pr`oc t`hn
deut'eran, o<'utwc t`o >ap`o t~hc pr'wthc pr`oc t`o <up`o
t~wn d'uo e>ujei~wn.}

\epsfysize=1.1in
\centerline{\epsffile{Book10/fig021ag.eps}}

\gr{>'Estwsan d'uo e>uje~iai a<i ZE, EH. l'egw, <'oti >est`in <wc <h ZE
pr`oc t`hn EH, o<'utwc t`o >ap`o t~hc ZE pr`oc t`o <up`o t~wn
ZE, EH.}

\gr{>Anagegr'afjw g`ar >ap`o t~hc ZE tetr'agwnon t`o DZ, ka`i sumpeplhr'wsjw
t`o HD. >epe`i o>~un >estin <wc <h ZE pr`oc t`hn EH, o<'utwc
t`o ZD pr`oc t`o DH, ka'i >esti t`o m`en ZD t`o >ap`o t~hc ZE, t`o d`e
DH t`o <up`o t~wn DE,  EH, tout'esti t`o <up`o t~wn ZE, EH, >'estin
>'ara <wc <h ZE pr`oc t`hn EH, o<'utwc t`o >ap`o t~hc ZE pr`oc
t`o <up`o t~wn ZE, EH. <omo'iwc d`e ka`i <wc t`o <up`o
t~wn HE, EZ pr`oc t`o >ap`o t~hc EZ, tout'estin <wc t`o HD pr`oc t`o ZD,
o<'utwc <h HE pr`oc t`hn EZ; <'oper >'edei de~ixai.}}

\ParallelRText{
\begin{center}
{\large Lemma}
\end{center}

If there are two straight-lines then as the first is to the second, so the
(square) on the first (is) to the (rectangle contained) by the two straight-lines.

\epsfysize=1.1in
\centerline{\epsffile{Book10/fig021ae.eps}}

Let $FE$ and $EG$ be two straight-lines. I say that as $FE$ is to $EG$,
so the (square) on $FE$ (is) to the (rectangle contained) by $FE$ and $EG$.

For let the square $DF$ have been described on $FE$. And let $GD$ have
been completed. Therefore, since as $FE$ is to $EG$, so
$FD$ (is) to $DG$ [Prop. 6.1], and
$FD$ is the (square) on  $FE$, and $DG$ the (rectangle contained) by $DE$ and $EG$---that is to say,
the (rectangle contained) by $FE$ and $EG$---thus as $FE$ is to $EG$,
so the (square) on $FE$ (is) to the (rectangle contained) by $FE$ and $EG$. 
And also, similarly, as the (rectangle contained) by $GE$ and $EF$ is to the
(square on) $EF$---that is to say, as $GD$ (is) to $FD$---so $GE$
(is) to $EF$. (Which is) the very thing it was required to show.}
\end{Parallel}

%%%%%%
% Prop 10.22
%%%%%%
\pdfbookmark[1]{Proposition 10.22}{pdf10.22}
\begin{Parallel}{}{} 
\ParallelLText{
\begin{center}
{\large \ggn{22}.}
\end{center}\vspace*{-7pt}

\gr{T`o >ap`o m'eshc par`a <rht`hn paraball'omenon pl'atoc poie~i <rht`hn ka`i
>as'ummetron t~h|, par> <`hn par'akeitai, m'hkei.}\\~\\

\epsfysize=2in
\centerline{\epsffile{Book10/fig022g.eps}}

\gr{>'Estw m'esh m`en <h A, <rht`h d`e <h GB, ka`i t~w| >ap`o
t~hc A >'ison par`a t`hn BG parabebl'hsjw qwr'ion >orjog'wnion t`o BD
pl'atoc poio~un t`hn GD; l'egw, <'oti <rht'h >estin <h GD ka`i
>as'ummetroc t~h| GB m'hkei.}

\gr{>Epe`i g`ar m'esh >est`in <h A, d'unatai qwr'ion perieq'omenon <up`o
<rht~wn dun'amei m'onon summ'etrwn. dun'asjw
t`o HZ. d'unatai d`e ka`i t`o BD; >'ison >'ara >est`i t`o BD t~w| HZ.
>'esti d`e a>ut~w| ka`i >isog'wnion; t~wn d`e >'iswn te ka`i >isogwn'iwn
parallhlogr'ammwn >antipep'onjasin a<i pleura`i a<i per`i t`ac
>'isac  gwn'iac; >an'alogon >'ara >est`in <wc <h BG pr`oc t`hn
EH, o<'utwc <h EZ pr`oc t`hn GD. >'estin >'ara ka`i <wc t`o >ap`o
t~hc BG pr`oc t`o >ap`o t~hc EH, o<'utwc t`o >ap`o
 t~hc EZ pr`oc t`o >ap`o t~hc GD. s'ummetron d'e >esti t`o >ap`o
t~hc GB t~w| >ap`o t~hc EH; <rht`h g'ar >estin <ekat'era a>ut~wn;
s'ummetron >'ara >est`i ka`i t`o >ap`o t~hc EZ t~w| >ap`o
t~hc GD. <rht`on d'e >esti t`o >ap`o t~hc EZ; <rht`on >'ara >est`i
ka`i t`o >ap`o t~hc GD; <rht`h >'ara >est`in <h GD. ka`i >epe`i
>as'ummetr'oc >estin <h EZ t~h| EH m'hkei; dun'amei
g`ar m'onon e>is`i s'ummetroi; <wc d`e <h EZ pr`oc t`hn EH,
o<'utwc t`o >ap`o t~hc EZ pr`oc t`o <up`o t~wn ZE, EH, >as'ummetron
>'ara [>est`i] t`o >ap`o t~hc EZ t~w| <up`o t~wn ZE, EH. >all`a t~w|
m`en >ap`o t~hc EZ s'ummetr'on >esti t`o >ap`o t~hc GD; <rhta`i
g'ar e>isi dun'amei; t~w| d`e <up`o t~wn ZE, EH s'ummetr'on >esti
t`o <up`o t~wn DG, GB; >'isa g'ar >esti t~w| >ap`o t~hc A; >as'ummetron
>'ara >est`i ka`i t`o >ap`o t~hc GD t~w| <up`o t~wn DG, GB. <wc d`e
t`o >ap`o t~hc GD pr`oc t`o <up`o t~wn DG, GB, o<'utwc
>est`in <h DG pr`oc t`hn GB; >as'ummetroc >'ara >est`in
<h DG t~h| GB m'hkei. <rht`h >'ara >est`in <h GD ka`i >as'ummetroc
t~h| GB m'hkei; <'oper >'edei de~ixai.}}

\ParallelRText{
\begin{center}
{\large Proposition 22}
\end{center}

The square on a medial (straight-line), being
applied to a rational (straight-line), produces as breadth a (straight-line which
is) rational, and incommensurable in length with the (straight-line) to which it is applied.

\epsfysize=2in
\centerline{\epsffile{Book10/fig022e.eps}}

Let $A$ be a medial (straight-line), and $CB$ a rational (straight-line),
and let the rectangular area $BD$, equal to the (square) on $A$, have
been applied to $BC$, producing $CD$ as breadth. I say that
$CD$ is rational, and incommensurable in  length with $CB$.

For since $A$ is medial, the square on it is equal to a (rectangular) area
contained by rational (straight-lines which are) commensurable in square
only [Prop. 10.21]. Let the square on ($A$) be equal
to $GF$. And the square on ($A$) is also equal to $BD$. Thus, $BD$
is equal to $GF$. And ($BD$) is also equiangular with ($GF$). And for equal and
equiangular parallelograms, the sides about the equal angles are
reciprocally proportional [Prop. 6.14]. 
Thus, proportionally,  as $BC$ is to $EG$, so $EF$ (is) to $CD$.
And, also, as the (square) on $BC$ is to the (square) on $EG$, so
the (square) on $EF$ (is) to the (square) on $CD$ [Prop. 6.22]. 
And the (square) on $CB$ is commensurable with the (square) on $EG$.
For they are each rational. Thus, the (square) on $EF$ is also commensurable
with the (square) on $CD$ [Prop. 10.11]. And
the (square) on $EF$ is rational. Thus, the (square) on $CD$ is also rational
[Def. 10.4]. Thus, $CD$ is rational.
And since $EF$ is incommensurable in length with
$EG$. For they are commensurable in square only. And as $EF$ (is) to $EG$,
so the (square) on $EF$ (is) to the (rectangle contained) by $FE$ and $EG$ 
[see previous lemma]. The (square) on $EF$ [is] thus incommensurable with the (rectangle contained) by $FE$ and $EG$ [Prop. 10.11]. But, the (square) on $CD$
is commensurable with the (square) on $EF$. For they are rational
in square.
And the (rectangle contained) by $DC$ and $CB$ is commensurable
with the (rectangle contained) by $FE$ and $EG$. For they are (both)
equal to the (square) on $A$. Thus, the (square) on $CD$ is also
incommensurable with the (rectangle contained) by $DC$ and $CB$
[Prop. 10.13]. And as the (square) on $CD$
(is) to the (rectangle contained) by $DC$ and $CB$, so $DC$ is to $CB$ [see
previous lemma]. Thus, $DC$ is incommensurable in length with $CB$
[Prop. 10.11]. Thus, $CD$ is rational, and
incommensurable in length with $CB$. (Which is) the very thing it was required to show.}
\end{Parallel}


\vspace{7pt}{\footnotesize\noindent$^\dag$  Literally, ``rational''.}

%%%%%%%
% Prop 10.23
%%%%%%%
\pdfbookmark[1]{Proposition 10.23}{pdf10.23}
\begin{Parallel}{}{} 
\ParallelLText{
\begin{center}
{\large \ggn{23}.}
\end{center}\vspace*{-7pt}

\gr{<H  t~h| m'esh| s'ummetroc m'esh >est'in.}

\gr{>'Estw m'esh <h A, ka`i t~h| A s'ummetroc >'estw <h B; l'egw, <'oti ka`i
<h B m'esh >est'in.}

\gr{>Ekke'isjw g`ar <rht`h <h GD, ka`i t~w| m`en >ap`o t~hc
A >'ison par`a t`hn GD parabebl'hsjw qwr'ion >orjog'wnion t`o GE pl'atoc
poio~un t`hn ED; <rht`h >'ara >est`in <h ED ka`i >as'ummetroc t`h|
GD m'hkei. t~w| d`e >ap`o t~hc B >'ison par`a t`hn GD parabebl'hsjw
qwr'ion >orjog'wnion t`o GZ pl'atoc poio~un t`hn DZ. >epe`i o>~un
s'ummetr'oc >estin <h A t~h| B, s'ummetr'on >esti ka`i t`o >ap`o t~hc A t~w|
>ap`o t~hc B. >all`a t~w| m`en >ap`o t~hc A >'ison >est`i t`o EG, t~w|
d`e >ap`o t~hc B >'ison >est`i t`o GZ; s'ummetron >'ara >est`i t`o EG t~w| GZ. ka'i >estin <wc t`o EG pr`oc t`o GZ, o<'utwc <h ED pr`oc t`hn
DZ; s'ummetroc >'ara >est`in <h ED t~h| DZ m'hkei. <rht`h d'e >estin
<h ED ka`i >as'ummetroc t~h| DG m'hkei; <rht`h >'ara >est`i ka`i <h DZ
ka`i >as'ummetroc t~h| DG m'hkei; a<i GD, DZ >'ara <rhta'i e>isi
dun'amei m'onon s'ummetroi. <h d`e t`o <up`o <rht~wn dun'amei
m'onon summ'etrwn dunam'enh m'esh >est'in. <h >'ara t`o <up`o t~wn
GD, DZ dunam'enh m'esh >est'in; ka`i d'unatai t`o <up`o t~wn GD, DZ
<h B; m'esh >'ara >est`in <h B.}\\~\\~\\~\\~\\~\\~\\~\\

\epsfysize=2.1in
\centerline{\epsffile{Book10/fig023g.eps}}

\begin{center}~\\
{\large \gr{P'orisma}.}
\end{center}\vspace*{-7pt}

\gr{>Ek d`h to'utou faner'on, <'oti t`o t~w| m'esw| qwr'iw| s'ummetr\-on m'eson
>est'in.}}

\ParallelRText{
\begin{center}
{\large Proposition 23}
\end{center}

A (straight-line) commensurable with
a medial (straight-line) is medial.

Let $A$ be a medial (straight-line), and let $B$ be commensurable with $A$.
I say that $B$ is also a medial (staight-line).

Let the rational (straight-line) $CD$ be set out, and let the rectangular
area $CE$, equal to the (square) on $A$, have been applied to $CD$,
producing $ED$ as width. $ED$ is thus rational, and incommensurable
in length with $CD$ [Prop. 10.22]. And
let the rectangular area $CF$, equal to the (square) on $B$, have been
applied to $CD$, producing $DF$ as width.
Therefore, since $A$ is commensurable with $B$, the (square) on $A$
is also commensurable with the (square) on $B$.  But, $EC$ is equal
to the (square) on $A$, and $CF$ is equal to the (square) on $B$. Thus,
$EC$ is commensurable with $CF$. And as $EC$ is to $CF$, so $ED$ (is)
to $DF$ [Prop. 6.1]. Thus, $ED$ is commensurable
in  length with $DF$ [Prop. 10.11]. And $ED$
is rational, and incommensurable in length with $CD$.  $DF$ is thus
 also rational [Def. 10.3], and incommensurable in length with $DC$  [Prop. 10.13]. Thus,
 $CD$ and $DF$ are rational, and commensurable in square only.
 And the square-root of a (rectangle contained) by rational (straight-lines which are)
 commensurable in square only is medial [Prop. 10.21]. Thus, the square-root of the
 (rectangle contained) by $CD$ and $DF$ is medial.
 And the square on $B$ is equal to
 the (rectangle contained) by $CD$ and $DF$. Thus, $B$ is a medial (straight-line).
 
\epsfysize=2.1in
\centerline{\epsffile{Book10/fig023e.eps}}
 
\begin{center}~\\
 {\large Corollary}
 \end{center}\vspace*{-7pt}
 
 And (it is) clear, from this, that an (area) commensurable with a medial 
 area$^\dag$ is medial.}
\end{Parallel}


\vspace{7pt}{\footnotesize\noindent$^\dag$ A medial area is equal to the square on some medial straight-line. Hence, a medial area is expressible as $k^{1/2}$.}
 
%%%%%%
% Prop 10.24
%%%%%%
\pdfbookmark[1]{Proposition 10.24}{pdf10.24}
\begin{Parallel}{}{} 
\ParallelLText{
\begin{center}
{\large \ggn{24}.}
\end{center}\vspace*{-7pt}

\gr{T`o <up`o m'eswn m'hkei summ'etrwn e>ujei~wn  perieq'omenon
>orjog'wnion m'eson
>est'in.}

\gr{<Up`o g`ar m'eswn m'hkei summ'etrwn e>ujei~wn t~wn AB, BG
perieq'esjw >orjog'wnion t`o AG; l'egw, <'oti t`o AG m'eson >est'in.}

\gr{>Anagegr'afjw g`ar >ap`o t~hc AB tetr'agwnon t`o AD; m'eson >'ara
>est`i t`o AD. ka`i >epe`i s'ummetr'oc >estin <h AB t~h| BG
m'hkei, >'ish d`e <h AB t~h| BD, s'ummetroc >'ara >est`i ka`i <h DB t~h|
BG m'hkei; <'wste ka`i t`o DA t~w| AG s'ummetr'on >estin. m'eson
d`e t`o DA; m'eson >'ara ka`i t`o AG; <'oper >'edei de~ixai.}\\~\\~\\

\epsfysize=2.2in
\centerline{\epsffile{Book10/fig024g.eps}}
}

\ParallelRText{
\begin{center}
{\large Proposition 24}
\end{center}

A rectangle contained by medial straight-lines
(which are) commensurable in length is medial.

For let the rectangle $AC$ be contained by the medial straight-lines
$AB$ and $BC$ (which are) commensurable in length. I say that $AC$
is medial.


For let the square $AD$ have been described on $AB$. $AD$ is thus medial [see previous footnote]. And since $AB$ is commensurable in length
with $BC$, and $AB$ (is) equal to $BD$, $DB$ is thus also commensurable
in length with $BC$. Hence, $DA$ is also commensurable with $AC$
[Props.~6.1, 10.11].
And $DA$ (is) medial. Thus, $AC$ (is) also medial [Prop. 10.23~corr.]. (Which is) the very thing it
was required to show.

\epsfysize=2.2in
\centerline{\epsffile{Book10/fig024e.eps}}}
\end{Parallel}

%%%%%%
% Prop 10.25
%%%%%%
\pdfbookmark[1]{Proposition 10.25}{pdf10.25}
\begin{Parallel}{}{} 
\ParallelLText{
\begin{center}
{\large \ggn{25}.}
\end{center}\vspace*{-7pt}

\gr{T`o <up`o m'eswn dun'amei m'onon  summ'etrwn e>ujei~wn perieq'omenon
>orjog'wnion >'htoi <rht`on >`h m'eson >est'in.}\\

\epsfysize=1.8in
\centerline{\epsffile{Book10/fig025g.eps}}

\gr{<Up`o g`ar m'eswn dun'amei m'onon summ'etrwn e>ujei~wn t~wn AB, BG
>orjog'wnion perieq'esjw t`o AG; l'egw, <'oti t`o AG >'htoi <rht`on
>`h m'eson >est'in.}

\gr{>Anagegr'afjw g`ar >ap`o t~wn AB, BG tetr'agwna t`a AD, BE; m'eson
>'ara >est`in <ek'ateron t~wn AD, BE. ka`i >ekke'isjw <rht`h <h ZH, ka`i
t~w| m`en AD >'ison par`a t`hn ZH parabebl'hsjw >orjog'wnion parallhl'ogrammon t`o HJ pl'atoc poio~un t`hn ZJ, t~w| d`e AG >'ison
par`a t`hn JM parabebl'hsjw >orjog'wnion parallhl'ogrammon t`o MK pl'atoc
poio~un t`hn JK, ka`i >'eti t~w| BE >'ison <omo'iwc par`a t`hn
KN parabebl'hsjw t`o NL pl'atoc poio~un t`hn KL; >ep> e>uje'iac
>'ara e>is`in a<i ZJ, JK, KL. >epe`i o>~un m'eson >est`in <ek'ateron
t~wn AD, BE, ka'i >estin >'ison t`o m`en AD t~w| HJ, t`o d`e BE t~w|
NL, m'eson >'ara ka`i <ek'ateron t~wn HJ, NL. ka`i par`a <rht`hn
t`hn ZH par'akeitai; <rht`h >'ara >est`in <ekat'era t~wn ZJ, KL ka`i
>as'ummetroc t~h| ZH m'hkei. ka`i >epe`i s'ummetr'on >esti t`o AD
t~w| BE, s'ummetron >'ara >est`i ka`i t`o HJ t~w| NL. ka'i >estin
<wc t`o HJ pr`oc t`o NL, o<'utwc <h ZJ pr`oc t`hn KL;
s'ummetroc >'ara >est`in <h ZJ t~h| KL m'hkei. a<i ZJ, KL
>'ara <rhta'i e>isi m'hkei s'ummetroi; <rht`on >'ara >est`i t`o <up`o
t~wn ZJ, KL. ka`i >epe`i >'ish >est`in <h m`en DB t~h| BA, <h d`e
XB t~h| BG, >'estin >'ara <wc <h DB pr`oc t`hn BG, o<'utwc <h AB
pr`oc t`hn BX. >all> <wc m`en <h DB pr`oc t`hn BG, o<'utwc
t`o DA pr`oc t`o AG; <wc d`e <h AB pr`oc t`hn BX, o<'utwc 
t`o AG pr`oc t`o GX; >'estin >'ara <wc t`o DA pr`oc t`o AG, o<'utwc
t`o AG pr`oc t`o GX. >'ison d'e >esti t`o m`en AD t~w| HJ, t`o d`e
AG t~w| MK, t`o d`e GX t~w| NL; >'estin >'ara <wc t`o HJ pr`oc t`o MK,
o<'utwc t`o MK pr`oc t`o NL; >'estin >'ara ka`i <wc
<h ZJ pr`oc t`hn JK, o<'utwc <h JK pr`oc t`hn KL; t`o >'ara <up`o
t~wn ZJ, KL >'ison >est`i t~w| >ap`o t~hc JK. <rht`on d`e t`o <up`o
t~wn ZJ, KL; <rht`on >'ara >est`i ka`i t`o >ap`o t~hc JK; <rht`h >'ara >est`in
<h JK. ka`i e>i m`en s'ummetr'oc >esti t~h| ZH m'hkei, <rht'on >esti
t`o JN; e>i d`e >as'ummetr'oc >esti t~h| ZH m'hkei, a<i KJ, JM <rhta'i
e>isi dun'amei m'onon s'ummetroi; m'eson >'ara t`o JN. t`o JN
>'ara >'htoi <rht`on >`h m'eson >est'in. >'ison d`e t`o JN t~w| AG; t`o
AG >'ara >'htoi <rht`on >`h m'eson >est'in.}

\gr{T`o >'ara <up`o m'eswn dun'amei m'onon summ'etrwn, ka`i t`a ex~hc.}}

\ParallelRText{
\begin{center}
{\large Proposition 25}
\end{center}

The rectangle contained by medial
straight-lines (which are) commensurable in square only is either
rational or medial.

\epsfysize=1.8in
\centerline{\epsffile{Book10/fig025e.eps}}

For let the rectangle $AC$ be contained by the medial straight-lines
$AB$ and $BC$ (which are) commensurable in square only. I say that
$AC$ is either rational or medial.

For let the squares $AD$ and $BE$ have been described on (the straight-lines) $AB$ and
$BC$ (respectively). $AD$ and $BE$ are thus each medial. And let
the rational (straight-line) $FG$ be laid out. And let the rectangular parallelogram $GH$, equal to $AD$, have been applied to $FG$, producing
$FH$ as breadth. And let the rectangular parallelogram $MK$, equal to $AC$, have been applied to $HM$, producing $HK$ as breadth. And,
finally, let $NL$, equal to $BE$, have similarly been applied to $KN$,
producing $KL$ as breadth. Thus, $FH$, $HK$, and $KL$ are in
a straight-line.  Therefore, since $AD$ and $BE$ are each medial, and
$AD$ is equal to $GH$, and $BE$ to $NL$, $GH$ and $NL$ (are)
thus each also medial. And they are applied to the rational (straight-line)
$FG$. $FH$ and $KL$ are thus each rational, and incommensurable in length
with $FG$ [Prop. 10.22]. And since
$AD$ is commensurable with $BE$, $GH$ is thus also commensurable
with $NL$. And as $GH$ is to $NL$, so $FH$ (is) to $KL$ [Prop. 6.1]. Thus, $FH$ is commensurable
in length with $KL$ [Prop. 10.11]. Thus, 
$FH$ and $KL$ are rational (straight-lines which are) commensurable in length. Thus, the (rectangle contained) by $FH$ and $KL$ is rational
[Prop. 10.19]. And since $DB$ is equal to $BA$,
and $OB$ to $BC$, thus as $DB$ is to $BC$, so $AB$ (is) to $BO$.
But, as $DB$ (is) to $BC$, so $DA$ (is) to $AC$ [Props.~6.1]. 
And as $AB$ (is) to $BO$, so $AC$ (is)
to $CO$ [Prop. 6.1]. Thus, as $DA$ is to $AC$, so $AC$ (is)
to $CO$. And $AD$ is equal to $GH$,
and $AC$ to $MK$, and $CO$ to $NL$. Thus, as $GH$ is to $MK$,
so $MK$ (is) to $NL$. Thus, also, as $FH$ is to $HK$, so $HK$ (is) to
$KL$ [Props.~6.1, 5.11]. Thus, the (rectangle contained) by
$FH$ and $KL$ is equal to the (square) on $HK$ [Prop. 6.17]. And the (rectangle contained) by
$FH$ and $KL$ (is) rational. Thus, the (square) on $HK$ is also
rational. Thus, $HK$ is rational. And if it is commensurable in length with
$FG$ then $HN$ is rational [Prop. 10.19]. 
And if it is incommensurable in length with $FG$ then $KH$ and $HM$
are rational (straight-lines which are) commensurable in square only: thus,
$HN$ is medial [Prop. 10.21]. Thus, $HN$ is
either rational or medial. And $HN$ (is) equal to $AC$. Thus, $AC$
is either rational or medial.

Thus, the \ldots by medial
straight-lines (which are) commensurable in square only, and so on \ldots.}
\end{Parallel}

%%%%%%
% Prop 10.26
%%%%%%
\pdfbookmark[1]{Proposition 10.26}{pdf10.26}
\begin{Parallel}{}{} 
\ParallelLText{
\begin{center}
{\large \ggn{26}.}
\end{center}\vspace*{-7pt}

\gr{M'eson m'esou o>uq <uper'eqei <rht~w|.}\\

\epsfysize=1.8in
\centerline{\epsffile{Book10/fig026g.eps}}

\gr{E>i g`ar dunat'on, m'eson t`o AB m'esou to~u AG <upereq'etw
<rht~w| t~w| DB, ka`i >ekke'isjw <rht`h <h EZ, ka`i t~w| AB
>'ison par`a t`hn EZ parabebl'hsjw parallhl'ogrammon >orjog'wnion
t`o ZJ pl'atoc poio~un t`hn EJ, t~w| d`e AG >'ison >afh|r'hsjw t`o ZH;
loip`on >'ara t`o BD loip~w| t~w| KJ >estin >'ison. <rht`on d'e >esti
t`o DB; <rht`on >'ara >est`i ka`i t`o KJ. >epe`i o>~un m'eson >est`in
<ek'ateron t~wn AB, AG, ka'i >esti t`o m`en AB t~w| ZJ >'ison,
t`o d`e AG t~w| ZH, m'eson >'ara ka`i <ek'ateron t~wn ZJ, ZH. ka`i
par`a <rht`hn t`hn EZ par'akeitai; <rht`h >'ara >est`in <ekat'era
t~wn JE, EH ka`i >as'ummetroc t~h| EZ m'hkei. ka`i >epe`i <rht'on
>esti t`o DB ka'i >estin >'ison t~w| KJ, <rht`on >'ara >est`i ka`i t`o
KJ. ka`i par`a <rht`hn t`hn EZ par'akeitai; <rht`h >'ara >est`in <h HJ
ka`i s'ummetroc t~h| EZ m'hkei. >all'a ka`i <h EH <rht'h >esti ka`i
>as'ummetroc t~h| EZ m'hkei; >as'ummetroc >'ara >est`in <h EH
t~h| HJ m'hkei. ka'i >estin <wc <h EH pr`oc t`hn HJ, o<'utwc t`o
>ap`o t~hc EH pr`oc t`o <up`o t~wn EH, HJ; >as'ummetron
>'ara >est`i t`o >ap`o t~hc EH t~w| <up`o t~wn EH, HJ. >all`a
t~w| m`en >ap`o t~hc EH s'ummetr'a >esti t`a >ap`o t~wn EH, HJ 
tetr'agwna; <rht`a g`ar >amf'otera;
t~w| d`e <up`o t~wn EH, HJ s'ummetr'on >esti t`o d`ic <up`o t~wn
EH, HJ; dipl'asion g'ar >estin a>uto~u; >as'ummetra >'ara >est`i
t`a >ap`o t~wn EH, HJ t~w| d`ic <up`o t~wn EH, HJ; ka`i sunamf'otera >'ara
t'a te >ap`o t~wn EH, HJ ka`i t`o d`ic <up`o t~wn EH, HJ, <'oper
>est`i t`o >ap`o t~hc EJ, >as'ummetr'on >esti to~ic >ap`o t~wn EH, HJ.
<rht`a d`e t`a >ap`o t~wn EH, HJ; >'alogon >'ara t`o >ap`o
t~hc EJ. >'alogoc >'ara >est`in <h EJ. >all`a ka`i <rhr'h;
<'oper >est`in >ad'unaton.}

\gr{M'eson >'ara m'esou o>uq <uper'eqei <rht~w|; <'oper >'edei
de~ixai.}}

\ParallelRText{
\begin{center}
{\large Proposition 26}
\end{center}

A medial (area) does not exceed a medial
(area) by a rational (area).$^\dag$

\epsfysize=1.8in
\centerline{\epsffile{Book10/fig026e.eps}}

For, if possible, let the medial (area) $AB$ exceed the medial (area) $AC$
by the rational (area) $DB$. And let the rational (straight-line)
$EF$ be laid down. And let the rectangular parallelogram $FH$, equal to $AB$, have been
applied to to $EF$, producing $EH$ as breadth. And let $FG$, equal to
$AC$, have been cut off (from $FH$). Thus, the remainder $BD$
is equal to the remainder $KH$. And $DB$ is rational. Thus, $KH$
is also rational. Therefore, since $AB$ and $AC$ are each medial, and
$AB$ is equal to $FH$, and $AC$ to $FG$, $FH$ and $FG$ are thus each also medial. And they are applied to the rational (straight-line)
$EF$. Thus, $HE$ and $EG$ are each rational, and incommensurable
in length with $EF$ [Prop. 10.22]. 
And since $DB$ is rational, and is equal to $KH$, $KH$ is thus  also
rational. And ($KH$) is  applied to the rational (straight-line)
$EF$. $GH$ is thus rational, and commensurable in length with
$EF$ [Prop. 10.20]. But, $EG$ is also
rational, and incommensurable in length with $EF$. Thus, $EG$ is
incommensurable in length with $GH$ [Prop. 10.13]. And as $EG$ is to $GH$, so the
(square) on $EG$ (is) to the (rectangle contained) by $EG$ and
$GH$ [Prop. 10.13~lem.]. Thus, the (square)
on $EG$ is incommensurable with the (rectangle contained) by 
$EG$ and $GH$ [Prop. 10.11]. 
But, the (sum of the) squares on $EG$ and $GH$ is commensurable with
the (square) on $EG$. For ($EG$ and $GH$ are) both rational.
And twice the (rectangle contained) by $EG$ and $GH$ is commensurable
with the (rectangle contained) by $EG$ and $GH$ [Prop. 10.6]. For (the former) is double the latter.
Thus, the (sum of the squares) on $EG$ and $GH$ is incommensurable
with twice the (rectangle contained) by $EG$ and $GH$
[Prop. 10.13]. 
And thus the sum of the (squares) on $EG$ and $GH$ plus twice the (rectangle contained) by
$EG$ and $GH$, that is the (square) on $EH$ [Prop. 2.4], is incommensurable with the (sum
of the squares) on $EG$ and $GH$ [Prop. 10.16].
And the (sum of the squares) on $EG$ and $GH$ (is) rational. 
Thus, the (square) on $EH$ is irrational [Def. 10.4].
Thus, $EH$ is irrational  [Def. 10.4]. But,
(it is) also rational. The very thing is impossible.

Thus, a medial (area) does not exceed a medial
(area) by a rational (area). (Which is) the very thing it was required to show.}
\end{Parallel}


\vspace{7pt}{\footnotesize\noindent$^\dag$ In other words, $\sqrt{k}-\sqrt{k'}\neq k''$.}

%%%%%%
% Prop 10.27
%%%%%%
\pdfbookmark[1]{Proposition 10.27}{pdf10.27}
\begin{Parallel}{}{} 
\ParallelLText{
\begin{center}
{\large \ggn{27}.}
\end{center}\vspace*{-7pt}

\gr{M'esac e<ure~in dun'amei m'onon summ'etrouc <rht`on perieqo'usac.}

\epsfysize=1.6in
\centerline{\epsffile{Book10/fig027g.eps}}

\gr{>Ekke'isjwsan d'uo <rhta`i dun'amei m'onon s'ummetroi a<i
A, B, ka`i e>il'hfjw t~wn A, B m'esh >an'alogon <h G, ka`i gegon'etw
<wc <h A pr`oc t`hn B, o<'utwc <h G pr`oc t`hn D.}

\gr{Ka`i >epe`i a<i A, B <rhta'i e>isi dun'amei m'onon s'ummetroi, t`o
>'ara <up`o t~wn A, B, tout'esti t`o >ap`o t~hc G, m'eson >est'in.
m'esh >'ara <h G. ka`i >epe'i >estin <wc <h A pr`oc t`hn B,
[o<'utwc] <h G pr`oc t`hn D, a<i d`e A, B dun'amei m'onon [e>is`i]
s'ummetroi, ka`i a<i G, D >'ara dun'amei m'onon e>is`i s'ummetroi. ka'i
>esti m'esh <h G; m'esh >'ara ka`i <h D. a<i G, D >'ara m'esai e>is`i
dun'amei m'onon s'ummetroi. l'egw, <'oti ka`i <rht`on peri'eqousin.
>epe`i g'ar >estin <wc <h A pr`oc t`hn B, o<'utwc <h G pr`oc
t`hn D, >enall`ax >'ara >est`in <wc <h A pr`oc t`hn G, <h B
pr`oc t`hn D. >all> <wc <h A pr`oc t`hn G, <h G pr`oc t`hn B; ka`i <wc
>'ara <h G pr`oc t`hn B, o<'utwc <h B pr`oc t`hn D; t`o >'ara
<up`o t~wn G, D >'ison >est`i t~w| >ap`o t~hc B. <rht`on d`e t`o
>ap`o t~hc B; <rht`on >'ara [>est`i] ka`i t`o <up`o t~wn
G, D.}

\gr{E<'urhntai >'ara m'esai dun'amei m'onon s'ummetroi <rht`on peri'eqousai;
<'oper >'edei de~ixai.}}

\ParallelRText{
\begin{center}
{\large Proposition 27}
\end{center}

To find (two) medial (straight-lines), containing a rational (area), (which are) commensurable in square only.

\epsfysize=1.6in
\centerline{\epsffile{Book10/fig027e.eps}}

Let the two rational (straight-lines) $A$ and $B$, (which are) commensurable
in square only, be laid down. And let $C$---the mean proportional (straight-line) to
$A$ and $B$---have been taken [Prop. 6.13]. 
And let it be contrived that as $A$ (is) to $B$, so $C$ (is) to $D$
[Prop. 6.12].

And since the rational (straight-lines) $A$ and $B$ are commensurable in square only,  the (rectangle contained) by $A$ and $B$---that is to
say, the (square) on $C$ [Prop. 6.17]---is thus medial
[Prop~10.21]. Thus, $C$ is medial
[Prop. 10.21]. And since as $A$ is to $B$, [so]
$C$ (is) to $D$, and $A$ and $B$ [are] commensurable in square only, 
$C$ and $D$ are thus also commensurable in square only [Prop. 10.11]. And $C$ is medial.
Thus, $D$ is also medial [Prop. 10.23]. 
Thus, $C$ and $D$ are medial (straight-lines which are) commensurable
in square only. I say that they also contain a rational (area).
For since as $A$ is to $B$, so $C$ (is) to $D$, thus, alternately,
as $A$ is to $C$, so $B$ (is) to $D$ [Prop. 5.16]. 
But, as $A$ (is) to $C$, (so) $C$ (is) to $B$. 
And thus as $C$ (is) to $B$, so $B$ (is) to $D$ [Prop. 5.11].
 Thus, the (rectangle contained) by $C$ and $D$ is equal to the (square) on $B$ [Prop. 6.17]. And the (square) on $B$ (is) rational. Thus, the (rectangle contained) by $C$ and $D$ [is] also rational.
 
Thus, (two) medial (straight-lines, $C$ and $D$), containing a rational (area), (which are) commensurable in square only, have been found.$^\dag$ (Which is) the very thing it was required to show.}
\end{Parallel}


\vspace{7pt}{\footnotesize\noindent$^\dag$ $C$ and $D$ have lengths $k^{1/4}$ and $k^{3/4}$ times that of $A$, respectively, where the length of $B$ is $k^{1/2}$ times that of $A$.}

%%%%%%
% Prop 10.28
%%%%%%
\pdfbookmark[1]{Proposition 10.28}{pdf10.28}
\begin{Parallel}{}{} 
\ParallelLText{
\begin{center}
{\large\ggn{28}.}
\end{center}\vspace*{-7pt}

\gr{M'esac e<ure~in dun'amei m'onon summ'etrouc m'eson peirieqo'usac.}

\epsfysize=0.7in
\centerline{\epsffile{Book10/fig028g.eps}}

\gr{>Ekke'isjwsan [tre~ic] <rhta`i dun'amei m'onon s'ummetroi a<i A, B, G,
ka`i e>il'hfjw t~wn A, B m'esh >an'alogon <h D, ka`i gegon'etw <wc <h
B pr`oc t`hn G, <h D pr`oc t`hn E.}

\gr{>Epe`i a<i A, B <rhta'i e>isi dun'amei m'onon s'ummetroi, t`o >'ara
<up`o t~wn A, B, tout'esti t`o >ap`o t~hc D, m'eson >est`in. m'esh >'ara
<h D. ka`i >epe`i a<i B, G dun'amei m'onon e>is`i s'ummetroi, ka'i
>estin <wc <h B pr`oc t`hn G,  <h D pr`oc t`hn E, ka`i a<i D, E >'ara
dun'amei m'onon e>is`i s'ummetroi. m'esh d`e <h D; m'esh >'ara ka`i
<h E; a<i D, E >'ara m'esai e>is`i dun'amei m'onon s'ummetroi. l'egw
d'h, <'oti ka`i m'eson peri'eqousin. >epe`i g'ar >estin <wc <h B
pr`oc t`hn G, <h D pr`oc t`hn E, >enall`ax >'ara <wc <h B pr`oc
t`hn D, <h G pr`oc t`hn E. <wc d`e <h B pr`oc t`hn D, <h D pr`oc
t`hn A; ka`i <wc <'ara <h D pr`oc t`hn A, <h G pr`oc t`hn E;
t`o >'ara <up`o t~wn A, G >'ison >est`i t~w| <up`o t~wn D, E.
m'eson d`e t`o <up`o t~wn A, G; m'eson >'ara ka`i t`o <up`o t~wn
D, E.}

\gr{>E<'urhntai >'ara m'esai dun'amei m'onon s'ummetroi m'eson peri'eqousai;
<'oper >'edei de~ixai.}}

\ParallelRText{
\begin{center}
{\large Proposition 28}
\end{center}

To find (two) medial (straight-lines),
containing a medial (area), (which are)
commensurable in square only.

\epsfysize=0.7in
\centerline{\epsffile{Book10/fig028e.eps}}

Let the [three] rational (straight-lines) $A$, $B$, and $C$, (which are)
commensurable in square only, be laid down. And let, $D$,  the mean proportional
(straight-line) to $A$ and $B$, have been taken [Prop. 6.13]. And let it be contrived that as
$B$ (is) to $C$, (so) $D$ (is) to $E$ [Prop. 6.12].

Since the rational (straight-lines) $A$ and $B$ are commensurable in square only, the (rectangle contained) by $A$ and $B$---that is to say, the
(square) on $D$ [Prop. 6.17]---is medial
[Prop. 10.21]. Thus, $D$ (is) medial [Prop. 10.21]. And
since $B$ and $C$ are commensurable in square only, and as $B$ is to $C$, (so) $D$ (is) to $E$, $D$ and $E$ are thus commensurable in square
only [Prop. 10.11]. And $D$ (is) medial.
$E$ (is) thus also medial [Prop. 10.23].
Thus, $D$ and $E$ are medial (straight-lines which are) commensurable
in square only. So, I say that they also enclose a medial (area). For since
as $B$ is to $C$, (so) $D$ (is) to $E$, thus, alternately, as $B$ (is) to $D$,
(so) $C$ (is) to $E$ [Prop. 5.16]. And as
$B$ (is) to $D$, (so) $D$ (is) to $A$. And thus as $D$ (is) to $A$, (so)
$C$ (is) to $E$. Thus, the (rectangle contained) by $A$ and $C$ is equal
to the (rectangle contained) by $D$ and $E$ [Prop. 6.16]. And the (rectangle contained) by 
$A$ and $C$ is medial [Prop. 10.21]. Thus, the (rectangle contained) by $D$ and $E$ (is)
also medial.

Thus, (two) medial (straight-lines, $D$ and $E$), containing a medial (area),
(which are) commensurable in square only, have been found. (Which is)
the very thing it was required to show.}
\end{Parallel}


\vspace{7pt}{\footnotesize\noindent$^\dag$ $D$ and $E$ have lengths $k^{1/4}$ and ${k'}^{1/2}/{k}^{1/4}$ times that of $A$, respectively, where the lengths of $B$ and $C$ are
$k^{1/2}$ and ${k'}^{1/2}$ times that of $A$, respectively.}

%%%%%%
% Prop 10.28a
%%%%%%
\begin{Parallel}{}{} 
\ParallelLText{
\begin{center}{\large \gr{L~hmma \ggn{1}.}}
\end{center}\vspace*{-7pt}

\gr{E<urein d'uo tetrag'wnouc >arijmo'uc, <'wste ka`i t`on sugke'imenon
>ex a>ut~wn e>~inai tetr'agwnon.}

\epsfysize=0.3in
\centerline{\epsffile{Book10/fig028ag.eps}}

\gr{>Ekke'isjwsan d'uo >arijmo`i o<i AB, BG, >'estwsan d`e >'htoi >'artioi
>`h peritto'i. ka`i >epe`i, >e'an te >ap`o >art'iou >'artioc >afairej~h|, >e'an
te >ap`o perisso~u periss'oc, <o loip`oc >'arti'oc >estin, <o loip`oc
>'ara <o AG >'arti'oc >estin. tetm'hsjw <o AG d'iqa kat`a t`o D. >'estwsan
d`e ka`i o<i AB, BG >'htoi <'omoioi >ep'ipedoi >`h tetr'agwnoi,
o<`i ka`i a>uto`i <'omoio'i e>isin >ep'ipedoi;
 <o >'ara >ek t~wn
AB, BG met`a to~u >ap`o [to~u] GD tetrag'wnou >'isoc >est`i t~w|
>ap`o to~u BD tetrag'wnw|. ka'i >esti tetr'agwnoc <o >ek
t~wn AB, BG, >epeid'hper >ede'iqjh, <'oti, >e`an d'uo <'omoioi
>ep'ipedoi pollaplasi'asantec >all'hlouc poi~wsi tina, <o
gen'omenoc tetr'agwn'oc >estin. e<'urhntai >'ara d'uo tetr'agwnoi >arijmo`i
<'o te >ek t~wn AB, BG ka`i <o >ap`o to~u GD, o<`i suntej'entec
poio~usi t`on >ap`o to~u BD tetr'agwnon.}

\gr{Ka`i faner'on, <'oti e<'urhntai p'alin d'uo tetr'agwnoi
<'o te >ap`o to~u BD ka`i <o >ap`o to~u GD, <'wste t`hn <uperoq`hn
a>ut~wn t`on <up`o AB, BG e>~inai tetr'agwnon, <'otan
o<i AB, BG <'omoioi >~wsin >ep'ipedoi. <'otan d`e m`h >~wsin
<'omoioi >ep'ipedoi, e<'urhntai d'uo tetr'agwnoi <'o te >ap`o
to~u BD ka`i <o >ap`o to~u DG, <~wn <h <uperoq`h <o <up`o
t~wn AB, BG o>uk >'esti tetr'agwnoc; <'oper >'edei de~ixai.}}

\ParallelRText{
\begin{center}
{\large Lemma I}
\end{center}

To find two square numbers such that
the sum of them is also  square.

\epsfysize=0.3in
\centerline{\epsffile{Book10/fig028ae.eps}}

Let the two numbers $AB$ and $BC$ be laid down. And let them
be either (both)  even or (both) odd. And since, if an even (number) is subtracted
from an even (number), or if an odd (number is subtracted) from an odd
(number), then the remainder is even [Props.~9.24, 9.26], the remainder $AC$ is thus even. Let
$AC$ have been cut in half at $D$. And let $AB$ and $BC$ also be
either similar plane (numbers), or square (numbers)---which are themselves
also similar plane (numbers). Thus, the (number created) from (multiplying)  $AB$ and
$BC$, plus the square on $CD$, is equal to the square on $BD$ [Prop. 2.6].  And the (number created) from (multiplying) $AB$ and $BC$ is square---inasmuch as it was shown that
if two similar plane (numbers) make some (number) by multiplying
one another then the (number so) created is square [Prop. 9.1]. Thus, two square numbers have been
found---(namely,) the (number created) from (multiplying) $AB$ and $BC$,
and the (square) on $CD$---which, (when) added (together),
make the square on $BD$.

And (it is) clear that two square (numbers) have again been found---(namely,) the (square) on $BD$, and the (square) on $CD$---such that their difference---(namely,) the (rectangle) contained by $AB$ and $BC$---is  square
whenever $AB$ and $BC$ are similar plane (numbers). But, when they are
not similar plane numbers, two square (numbers) have been found---(namely,) the (square) on $BD$, and the (square) on $DC$---between which the difference---(namely,) the (rectangle) contained by $AB$ and $BC$---is
not  square. (Which is) the very thing it was required to show.}
\end{Parallel}

%%%%%%%
% Prop 10.28b
%%%%%%%
\begin{Parallel}{}{} 
\ParallelLText{
\begin{center}
{\large \gr{L~hmma \ggn{2}.}}
\end{center}\vspace*{-7pt}

\gr{E<ure~in d'uo tetrag'wnouc >arijmo'uc, <'wste t`on >ex a>ut~wn sugke'imenon m`h e>~inai tetr'agwnon.}

\epsfysize=0.4in
\centerline{\epsffile{Book10/fig028bg.eps}}

\gr{>'Estw g`ar <o >ek t~wn AB, BG, <wc >'efamen, tetr'agwnoc, ka`i
>'artioc <o GA, ka`i tetm'hsjw <o GA d'iqa t~w| D. faner`on d'h,
<'oti <o >ek t~wn AB, BG tetr'agwnoc met`a to~u >ap`o [to~u]
GD tetrag'wnou >'isoc >est`i t~w| >ap`o [to~u]
BD tetrag'wnw|. >afh|r'hsjw mon`ac <h DE; <o >'ara >ek t~wn AB, BG
met`a to~u >ap`o [to~u] GE >el'asswn >est`i to~u >ap`o [to~u]
BD tetrag'wnou. l'egw o>~un, <'oti <o >ek t~wn AB, BG tetr'agwnoc met`a
to~u >ap`o [to~u] GE o>uk >'estai tetr'agwnoc.}

\gr{E>i g`ar >'estai tetr'agwnoc, >'htoi >'isoc >est`i t~w| >ap`o [to~u]
BE >`h >el'asswn to~u >ap`o [to~u] BE, o>uk'eti d`e ka`i me'izwn, <'ina
m`h tmhj~h| <h mon'ac. >'estw, e>i dunat'on, pr'oteron <o >ek t~wn
AB, BG met`a to~u >ap`o GE >'isoc t~w| >ap`o BE, ka`i >'estw t~hc DE mon'adoc diplas'iwn <o HA. >epe`i o>~un <'oloc <o AG <'olou
to~u GD >esti diplas'iwn, <~wn <o AH to~u DE >esti diplas'iwn,
ka`i loip`oc >'ara <o HG loipo~u to~u EG >esti diplas'iwn;
d'iqa >'ara t'etmhtai <o HG t~w| E. <o >'ara >ek t~wn HB, BG
met`a to~u >ap`o GE >'isoc >est`i t~w| >ap`o BE tetrag'wnw|. 
>all`a ka`i
<o >ek t~wn AB, BG met`a to~u >ap`o GE >'isoc <up'okeitai
t~w| >ap`o [to~u] BE tetrag'wnw|; <o >'ara >ek t~wn HB, BG
met`a to~u >ap`o GE >'isoc >est`i 
t~w| >ek t~wn AB, BG met`a
to~u >ap`o GE. ka`i koino~u >afairej'entoc to~u >ap`o GE sun'agetai
<o AB >'isoc t~w| HB; <'oper >'atopon. o>uk >'ara <o >ek t~wn AB, BG
met`a to~u >ap`o [to~u] GE >'isoc >est`i t~w| >ap`o BE. l'egw d'h,
<'oti o>ud`e >el'asswn to~u >ap`o BE. e>i g`ar dunat'on, >'estw
t~w| >ap`o BZ >'isoc, ka`i to~u DZ diplas'iwn <o JA. ka`i sunaqj'hsetai
p'alin diplas'iwn <o JG to~u GZ; <'wste
ka`i t`on GJ d'iqa tetm~hsjai kat`a t`o Z, ka`i di`a to~uto
t`on >ek t~wn JB, BG met`a to~u >ap`o ZG >'ison g'inesjai
t~w| >ap`o BZ. <up'okeitai d`e ka`i <o >ek t~wn AB, BG
met`a to~u >ap`o GE >'isoc t~w| >ap`o BZ. <'wste ka`i <o >ek
t~wn JB, BG met`a to~u >ap`o GZ >'isoc >'estai t~w| >ek
t~wn AB, BG met`a to~u >ap`o GE; <'oper >'atopon. o>uk
>'ara <o >ek t~wn AB, BG met`a to~u >ap`o GE >'isoc
>est`i [t~w|] >el'assoni to~u >ap`o BE. >ede'iqjh d'e, <'oti o>ud`e
[a>ut~w|] t~w| >ap`o BE. o>uk >'ara <o >ek t~wn AB, BG met`a
to~u >ap`o GE tetr'agwn'oc >estin. <'oper >'edei de~ixai.}}

\ParallelRText{
\begin{center}
{\large Lemma II}
\end{center}

To find two square numbers such that
the sum of them is not square.

\epsfysize=0.4in
\centerline{\epsffile{Book10/fig028be.eps}}

For let the (number created) from (multiplying) $AB$ and $BC$, as we said, be square.  And (let) $CA$ (be) even. And let $CA$ have been
cut in half at $D$. So it is clear that the square (number created) from
(multiplying) $AB$ and $BC$, plus the square on $CD$, is equal to the
square on $BD$ [see previous lemma]. Let the unit $DE$ have been subtracted (from $BD$).
Thus, the (number created) from (multiplying) $AB$ and $BC$, plus
the (square) on $CE$, is less than the square on $BD$. I say, therefore,
that the square (number created) from (multiplying) $AB$ and $BC$, plus the
(square) on $CE$, is not square.

For if it is square, it is either equal to the (square) on $BE$, or
less than the (square) on $BE$, but cannot    any more be greater  (than the square on $BE$), lest the unit be divided. First of all, if possible, let the
(number created) from (multiplying) $AB$ and $BC$, plus the (square) on
$CE$, be equal to the (square) on $BE$. And let $GA$ be double the unit
$DE$. Therefore, since the whole of $AC$ is double the whole
of $CD$, of which $AG$ is double $DE$, the remainder $GC$
is thus double the remainder $EC$. Thus, $GC$ has been cut in half at $E$.
Thus, the (number created) from (multiplying) $GB$ and $BC$,
plus the (square) on $CE$, is equal to the square on $BE$ [Prop. 2.6]. But, the (number created) from (multiplying) $AB$ and $BC$, plus the (square) on $CE$, was also assumed
 (to be) equal to the square on $BE$. Thus, the (number created) from
 (multiplying) $GB$ and $BC$, plus the (square) on $CE$, is equal to
 the (number created) from (multiplying)  $AB$ and $BC$, plus the
 (square) on $CE$. And subtracting the (square) on $CE$ from both, $AB$ is  inferred (to be) equal to $GB$. The very thing is absurd. Thus, the (number created) from (multiplying) $AB$ and $BC$, plus the (square) on
$CE$, is not equal to the (square) on $BE$. So I say that (it is) not
less than the (square) on $BE$ either. For, if possible, let it be equal to
the (square) on $BF$. And (let) $HA$ (be) double $DF$. And it
can again be inferred that $HC$ (is) double $CF$. Hence, $CH$ has
also been cut in half at $F$. And, on account of this, the (number created)
from (multiplying) $HB$ and $BC$, plus the (square) on $FC$, becomes
equal to the (square) on $BF$ [Prop. 2.6]. 
And the (number created) from (multiplying) $AB$ and $BC$, plus the
(square) on $CE$, was also
assumed (to be) equal to the (square) on $BF$.
Hence,
the (number created) from (multiplying) $HB$ and $BC$, plus the
(square) on $CF$, will also be equal to the
(number created) from (multiplying) $AB$ and $BC$, plus the (square) on $CE$. The very thing is
absurd. Thus, the (number created) from (multiplying) $AB$ and $BC$,
plus the (square) on $CE$, is not equal to less than the
(square) on $BE$. And it was shown that  (is it) not equal to the (square)
on $BE$ either. Thus, the (number created) from (multiplying)
$AB$ and $BC$, plus the  square on $CE$, is not square. (Which is)
the very thing it was required to show.}
\end{Parallel}

%%%%%%
% Prop 10.29
%%%%%%
\pdfbookmark[1]{Proposition 10.29}{pdf10.29}
\begin{Parallel}{}{} 
\ParallelLText{
\begin{center}
{\large \ggn{29}.}
\end{center}\vspace*{-7pt}

\gr{E<ure~in d'uo <rht`ac dun'amei m'onon summ'etrouc, <'wste t`hn me'izona
t~hc >el'assonoc me~izon d'unasjai t~w| >ap`o summ'etrou <eaut~h|
m'hkei.}\\~\\

\epsfysize=1.7in
\centerline{\epsffile{Book10/fig029g.eps}}

\gr{>Ekke'isjw g'ar tic <rht`h <h AB ka`i d'uo tetr'agwnoi >arijmo`i
o<i GD, DE, <'wste t`hn <uperoq`hn a>ut~wn t`on GE m`h e>~inai
tetr'agwnon, ka`i gegr'afjw >ep`i t~hc AB <hmik'uklion t`o AZB,
ka`i pepoi'hsjw <wc <o DG pr`oc t`on GE, o<'utwc t`o >ap`o t~hc BA
tetr'agwnon pr`oc t`o >ap`o t~hc AZ tetr'agwnon, ka`i >epeze'uqjw <h ZB.}

\gr{>Epe`i [o>~un] >estin <wc t`o >ap`o t~hc BA pr`oc t`o >ap`o t~hc AZ,
o<'utwc <o DG pr`oc t`on GE, t`o >ap`o t~hc BA >'ara pr`oc t`o
>ap`o t~hc AZ l'ogon >'eqei, <'on >arijm`oc <o DG pr`oc >arijm`on
t`on GE; s'ummetron >'ara >est`i t`o >ap`o t~hc BA t~w| >ap`o t~hc
AZ. <rht`on d`e t`o >ap`o t~hc AB; <rht`on >'ara ka`i t`o >ap`o
t~hc AZ; <rht`h >'ara ka`i <h AZ. ka`i >epe`i <o DG pr`oc t`on GE
l'ogon o>uk >'eqei, <`on tetr'agwnoc >arijm`oc pr`oc tetr'agwnon
>arijm'on,  o>ud`e t`o >ap`o t~hc BA >'ara pr`oc t`o >ap`o t~hc
AZ l'ogon >'eqei, <`on tetr'agwnoc >arijm`oc pr`oc tetr'agwnon
>arijm'on; >as'ummetroc >'ara >est`in <h AB t~h| AZ m'hkei; a<i
BA, AZ >'ara <rhta'i e>isi dun'amei m'onon s'ummetroi. ka`i >epe'i
[>estin] <wc <o DG pr`oc t`on GE, o<'utwc t`o >ap`o t~hc BA pr`oc
t`o >ap`o t~hc AZ, >anastr'eyanti >'ara <wc <o GD pr`oc t`on
DE, o<'utwc t`o >ap`o t~hc AB pr`oc t`o >ap`o t~hc BZ. <o d`e
GD pr`oc t`on DE l'ogon >'eqei, <`on tetr'agwnoc >arijm`oc
pr`oc tetr'agwnon >arijm'on; ka`i t`o >ap`o t~hc AB >'ara pr`oc t`o
>ap`o t~hc BZ l'ogon >'eqei, <`on tetr'agwnoc >arijm`oc pr`oc tetr'agwnon
>arijm'on;
s'ummetroc >'ara >est`in <h AB
t~h| BZ m'hkei. ka'i >esti t`o >ap`o t~hc AB >'ison to~ic >ap`o t~wn AZ, ZB; <h AB >'ara t~hc AZ me~izon d'unatai t~h|
BZ summ'etrw| <eaut~h|.}

\gr{E<'urhntai >'ara d'uo <rhta`i dun'amei m'onon s'ummetroi a<i BA, AZ,
<'wste t`hn me~izona t`hn AB t~hc >el'assonoc t~hc AZ me~izon d'unasjai
t~w| >ap`o t~hc BZ summ'etrou <eaut~h| m'hkei; <'oper >'edei
de~ixai.}}

\ParallelRText{
\begin{center}
{\large Proposition 29}
\end{center}

To  find two rational (straight-lines which are)
commensurable in square only, such that the square on the greater is larger
than the (square on the) lesser by the (square) on (some straight-line which is) commensurable in length with the greater.

\epsfysize=1.7in
\centerline{\epsffile{Book10/fig029e.eps}}

For let some rational (straight-line) $AB$ be laid down, and two
square numbers, $CD$ and $DE$, such that the difference between them, $CE$, is not square [Prop. 10.28 lem.~I]. 
And let the semi-circle $AFB$ have been drawn on $AB$. And let
it be contrived that as $DC$ (is) to $CE$, so the square on $BA$
(is) to the square on $AF$ [Prop. 10.6~corr.].
And let $FB$ have been joined.

\mbox{[}Therefore,] since as the (square) on $BA$ is to the (square) on $AF$,
so $DC$ (is) to $CE$, the (square) on $BA$ thus has to the (square)
on $AF$ the ratio which the number $DC$ (has) to the number $CE$.
Thus, the (square) on $BA$ is commensurable with the (square) on $AF$
[Prop. 10.6]. And the (square) on $AB$ (is)
rational [Def. 10.4]. Thus, the (square) on $AF$
(is) also rational. Thus, $AF$ (is) also rational. 
And since $DC$ does not have to $CE$ the ratio which
(some) square number (has) to (some) square number,  the (square)
on $BA$ thus does not have to the (square) on $AF$ the ratio which (some)
square number has to (some) square number either. Thus, $AB$ is incommensurable in length with $AF$ [Prop. 10.9].
Thus, the rational (straight-lines) $BA$
and $AF$ are commensurable in square only. And since as $DC$ [is] to $CE$, so the
(square) on $BA$ (is) to the (square) on $AF$, thus, via conversion,
as $CD$ (is) to $DE$, so the (square) on $AB$ (is) to the (square) on
$BF$ [Props.~5.19~corr., 3.31, 1.47]. And $CD$
has to $DE$ the ratio which (some) square number (has) to (some)
square number. Thus, the (square) on $AB$ also has to the (square) on
$BF$ the ratio which (some) square number has to (some) square number.
$AB$ is thus commensurable in length with $BF$ [Prop. 10.9]. And the (square) on $AB$
is equal to the (sum of the squares) on $AF$ and $FB$ [Prop. 1.47]. Thus, the
square on $AB$ is greater than (the square on) $AF$ by (the square on)
$BF$, (which is) commensurable (in length) with ($AB$).

Thus, two rational (straight-lines), $BA$ and $AF$, commensurable in square only, have been found such that the square on the greater, $AB$, is larger
than (the square on) the lesser, $AF$, by the (square) on $BF$, (which is)
commensurable in length with ($AB$).$^\dag$ (Which is) the very thing it
was required to show.}
\end{Parallel}


\vspace{7pt}{\footnotesize\noindent$^\dag$ $BA$ and $AF$
have lengths $1$ and $\sqrt{1-k^2}$ times that of $AB$, respectively, where $k=\sqrt{DE/CD}$.}

%%%%%%
% Prop 10.30
%%%%%%
\pdfbookmark[1]{Proposition 10.30}{pdf10.30}
\begin{Parallel}{}{} 
\ParallelLText{
\begin{center}
{\large \ggn{30}.}
\end{center}\vspace*{-7pt}

\gr{E<ure~in d'uo <rht`ac dun'amei m'onon summ'etrouc, <'wste t`hn
me'izona t~hc >el'assonoc me~izon d'unasjai t~w| >ap`o >asumm'etrou
<eaut~h| m'hkei.}\\~\\

\epsfysize=1.7in
\centerline{\epsffile{Book10/fig030g.eps}}

\gr{>Ekke'isjw <rht`h <h AB ka`i d'uo tetr'agwnoi >arijmo`i
o<i GE, ED, <'wste t`on sugke'imenon >ex a>ut~wn t`on GD m`h
e>~inai tetr'agwnon, ka`i gegr'afjw >ep`i t~hc AB <hmik'uklion
t`o AZB, ka`i pepoi'hsjw <wc <o DG pr`oc t`on GE, o<'utwc
t`o >ap`o t~hc BA pr`oc t`o >ap`o t~hc AZ, ka`i >epeze'uqjw <h ZB.}

\gr{<Omo'iwc d`h de'ixomen t~w| pr`o to'utou, <'oti a<i BA, AZ <rhta'i
e>isi dun'amei m'onon s'ummetroi. ka`i >epe'i >estin <wc <o
DG pr`oc t`on GE, o<'utwc t`o >ap`o t~hc BA pr`oc t`o >ap`o t~hc
AZ, >anastr'eyanti >'ara <wc <o GD pr`oc t`on DE, o<'utwc t`o >ap`o
t~hc AB pr`oc t`o >ap`o t~hc BZ. <o d`e GD pr`oc t`on DE l'ogon
o>uk >'eqei, <`on tetr'agwnoc >arijm`oc pr`oc tetr'agwnon >arijm'on;
o>ud> >'ara t`o >ap`o t~hc AB pr`oc t`o >ap`o t~hc BZ l'ogon
>'eqei, <`on tetr'agwnoc >arijm`oc pr`oc tetr'agwnon >arijm'on;
>as'ummetroc >'ara >est`in <h AB t~h| BZ m'hkei. ka`i d'unatai
<h AB t~hc AZ me~izon t~w| >ap`o t~hc ZB >asumm'etrou
<eaut~h|.}

\gr{A<i AB, AZ >'ara <rhta'i e>isi dun'amei m'onon s'ummetroi, ka`i
<h AB t~hc AZ me~izon d'unatai t~w| >ap`o t~hc ZB
>asumm'etrou <eaut~h| m'hkei; <'oper >'edei de~ixai.}}

\ParallelRText{
\begin{center}
{\large Proposition 30}
\end{center}

To find two rational (straight-lines which are)
commensurable in square only, such that the square on the greater
is larger than the (the square on)  lesser by the (square) on (some
straight-line which is) incommensurable in length with the greater.

\epsfysize=1.7in
\centerline{\epsffile{Book10/fig030e.eps}}

Let the rational (straight-line) $AB$ be laid out, and the two square
numbers, $CE$ and $ED$, such that the sum of them, $CD$,
is not square [Prop. 10.28~lem.~II].
And let the semi-circle $AFB$ have been drawn on $AB$. And let it
be contrived that as $DC$ (is) to $CE$, so the (square) on
$BA$ (is) to the (square) on $AF$ [Prop. 10.6~corr].
And let $FB$ have been joined.

So, similarly to the (proposition) before this, we can show that $BA$ and
$AF$ are rational (straight-lines which are) commensurable in square only.
And since as $DC$ is to $CE$, so the (square) on $BA$ (is) to the (square)
on $AF$, thus, via conversion, as $CD$ (is) to $DE$, so the (square) on
$AB$ (is) to the (square) on $BF$ [Props.~5.19~corr., 3.31, 1.47]. And
$CD$ does not have to $DE$ the ratio which (some) square number
(has) to (some) square number. Thus, the (square) on $AB$
does not have to the (square) on $BF$ the ratio which (some)
square number has to (some) square number either. Thus, $AB$ is
incommensurable in length with $BF$ [Prop. 10.9].
And the square on $AB$ is greater than the (square on) $AF$ by the
(square) on $FB$  [Prop. 1.47], (which is) incommensurable (in length) with ($AB$).

Thus, $AB$ and $AF$ are rational (straight-lines which are) commensurable
in square only, and the square on $AB$ is greater than (the square on) $AF$
by the (square) on $FB$, (which is) incommensurable (in length) with
 $(AB)$.$^\dag$ (Which is) the very thing it was required to show.}
\end{Parallel}


\vspace{7pt}{\footnotesize\noindent$^\dag$ $AB$ and $AF$ have lengths $1$ and $1/\sqrt{1+k^2}$
times that of $AB$, respectively, where $k=\sqrt{DE/CE}$.}

%%%%%%
% Prop 10.31
%%%%%%
\pdfbookmark[1]{Proposition 10.31}{pdf10.31}
\begin{Parallel}{}{} 
\ParallelLText{
\begin{center}
{\large \ggn{31}.}
\end{center}\vspace*{-7pt}

\gr{E<ure~in d'uo m'esac dun'amei m'onon summ'etrouc <rht`on perieqo'usac,
<'wste t`hn me'izona t~hc >el'assonoc me~izon d'unasjai t~w| >ap`o
summ'etrou <eaut~h| m'hkei.}\\~\\

\epsfysize=1.6in
\centerline{\epsffile{Book10/fig031g.eps}}

\gr{>Ekke'isjwsan d'uo <rhta`i dun'amei m'onon s'ummetroi a<i A, B,
<'wste t`hn A me'izona o>~usan t~hc >el'assonoc t~hc B me~izon
d'unasjai t~w| >ap`o summ'etrou <eaut~h| m'hkei. ka`i t~w| <up`o
t~wn A, B >'ison >'estw t`o >ap`o t~hc G. m'eson d`e t`o <up`o
t~wn A, B; m'eson >'ara ka`i t`o >ap`o t~hc G; m'esh >'ara ka`i <h G.
t~w| d`e >ap`o t~hc B >'ison >'estw t`o <up`o t~wn G, D; <rht`on
d`e t`o >ap`o t~hc B; <rht`on >'ara ka`i t`o <up`o t~wn G, D. ka`i
>epe'i >estin <wc <h A pr`oc t`hn B, o<'utwc
t`o <up`o t~wn A, B pr`oc t`o >ap`o t~hc B, >all`a t~w| m`en
<up`o t~wn A, B >'ison >est`i t`o >ap`o t~hc G, t~w| d`e >ap`o
t~hc B >'ison t`o <up`o t~wn G, D, <wc >'ara <h A pr`oc t`hn B,
o<'utwc t`o >ap`o t~hc G pr`oc t`o <up`o t~wn G, D. <wc d`e
t`o >ap`o t~hc G pr`oc t`o <up`o t~wn G, D, o<'utwc <h G pr`oc t`hn
D; ka`i <wc >'ara <h A pr`oc t`hn B, o<'utwc <h G pr`oc t`hn D.
 s'ummetroc d`e <h A t~h| B dun'amei m'onon; s'ummetroc
>'ara ka`i <h G t~h| D dun'amei m'onon. ka'i >esti m'esh <h G;
m'esh >'ara ka`i <h D. ka`i >epe'i >estin <wc <h A pr`oc t`hn
B, <h G pr`oc t`hn D, <h d`e A t~hc B me~izon d'unatai t~w|
>ap`o summ'etrou <eaut~h|, ka`i <h G >'ara t~hc D me~izon
d'unatai t~w| >ap`o summ'etrou <eaut~h|.}

\gr{E<'urhntai >'ara d'uo m'esai dun'amei m'onon s'ummetroi
a<i G, D <rht`on peri'eqousai, ka`i <h G t~hc D me~izon dun'atai
t~w| >ap`o summ'etrou <eaut~h| m'hkei.}

\gr{<Omo'iwc d`h deiqj'hsetai ka`i t~w| >ap`o >asumm'etrou,
<'otan <h A t~hc B me~izon d'unhtai t~w| >ap`o >asumm'etrou
<eaut~h|.}}

\ParallelRText{
\begin{center}
{\large Proposition 31}
\end{center}

To find two medial (straight-lines),
commensurable in square only, (and) containing a rational (area), such that
the square on the greater is larger than the (square on the) lesser by the
(square) on (some straight-line) commensurable in length with the greater.

\epsfysize=1.6in
\centerline{\epsffile{Book10/fig031e.eps}}

Let two rational (straight-lines), $A$ and $B$, commensurable in square only, 
be laid out, such that the square on the greater $A$ is larger than the (square on
the) lesser $B$ by the (square) on (some straight-line) commensurable
in length with ($A$) [Prop. 10.29]. And let the
(square) on $C$ be equal to the (rectangle contained) by $A$ and $B$.
And the (rectangle contained by) $A$ and $B$ (is) medial [Prop. 10.21]. Thus, the (square) on $C$ (is)
also medial. Thus, $C$ (is) also medial [Prop. 10.21]. And let the (rectangle contained) by $C$ and $D$
be equal to the (square) on $B$. And the (square)
on $B$ (is) rational. Thus, the (rectangle contained) by $C$ and $D$ (is)
also rational. And since as $A$ is to $B$, so the (rectangle contained) by
$A$ and $B$ (is) to the (square) on $B$ [Prop. 10.21~lem.], but the (square) on $C$
is equal to the (rectangle contained) by $A$ and $B$,  
and the (rectangle contained) by $C$ and $D$ to the (square) on
$B$, thus as $A$ (is) to $B$,
so the (square) on $C$ (is) to the (rectangle contained) by $C$ and $D$.
And as the (square) on $C$ (is) to the (rectangle contained) by $C$ and $D$,
so $C$ (is) to $D$ [Prop. 10.21~lem.].
And thus as $A$ (is) to $B$, so $C$ (is) to $D$. And $A$ is commensurable
in square only with $B$. Thus, $C$ (is) also commensurable in square
only with $D$ [Prop. 10.11]. And $C$ is medial.
Thus, $D$ (is) also medial [Prop. 10.23]. 
And since as $A$ is to $B$, (so) $C$ (is) to $D$, and the square on $A$
is greater than (the square on) $B$ by the (square) on (some straight-line)
commensurable (in length) with ($A$), the square on $C$
is thus also greater than (the square on) $D$ by the (square) on (some straight-line)
commensurable (in length) with ($C$) [Prop. 10.14]. 

Thus, two medial (straight-lines), $C$ and $D$, 
commensurable in square only, (and) containing a rational (area), have been found. And the square on $C$ is greater than (the square on) $D$ by the
(square) on (some straight-line) commensurable in length with ($C$).$^\dag$

So, similarly, (the proposition) can also be demonstrated for  (some straight-line) incommensurable
(in length with $C$), provided that the square on $A$ is greater than (the square on $B$) by the (square) on (some straight-line) incommensurable (in length) with ($A$) [Prop. 10.30].$^\ddag$}
\end{Parallel}


\vspace{7pt}{\footnotesize\noindent$^\dag$ $C$ and $D$ have lengths $(1-k^2)^{1/4}$ and $(1-k^2)^{3/4}$ times that of $A$, respectively, where $k$ is defined in the
footnote to Prop.~10.29.\\[0.5ex]
$^\ddag$ $C$ and $D$ would have lengths $1/(1+k^2)^{1/4}$ and $1/(1+k^2)^{3/4}$ times that of $A$, respectively, where $k$ is defined in the footnote to  Prop.~10.30.}

%%%%%%
% Prop 10.32
%%%%%%
\pdfbookmark[1]{Proposition 10.32}{pdf10.32}
\begin{Parallel}{}{} 
\ParallelLText{
\begin{center}
{\large \ggn{32}.}
\end{center}\vspace*{-7pt}

\gr{E<ure~in d'uo m'esac dun'amei m'onon summ'etrouc m'eson perieqo'usac,
<'wste t`hn me'izona t~hc >el'assonoc me~izon d'unasjai t~w| >ap`o
summ'etrou <eaut~h|.}\\~\\

\epsfysize=0.75in
\centerline{\epsffile{Book10/fig032g.eps}}

\gr{>Ekke'isjwsan tre~ic <rhta`i dun'amei m'onon s'ummetroi a<i A, B, G,
<'wste t`hn A t~hc G me~izon d'unasjai t~w| >ap`o summ'etrou
<eaut~h|, ka`i t~w| m`en <up`o t~wn A, B >'ison >'estw t`o >ap`o
t`hc D. m'eson >'ara t`o >ap`o t~hc D; ka`i <h D >'ara m'esh >est'in.
t~w| d`e <up`o t~wn B, G >'ison >'estw t`o <up`o t~wn D, E. ka`i
>epe'i >estin <wc t`o <up`o t~wn A, B pr`oc t`o <up`o t~wn B, G, o<'utwc
<h A pr`oc t`hn G, >all`a t~w| m`en <up`o t~wn A, B >'ison >est`i t`o
>ap`o t~hc D, t~w| d`e <up`o t~wn B, G >'ison t`o <up`o t~wn D, E, >'estin
>'ara <wc <h A pr`oc t`hn G, o<'utwc t`o >ap`o t~hc D pr`oc
t`o <up`o t~wn D, E. <wc d`e t`o >ap`o t~hc D pr`oc t`o <up`o
t~wn D, E, o<'utwc <h D pr`oc t`hn E; ka`i <wc >'ara <h A pr`oc t`hn
G, o<'utwc <h D pr`oc t`hn E. s'ummetroc d`e <h A t~h| G dun'amei [m'onon].
s'ummetroc >'ara ka`i <h D t~h| E dun'amei m'onon. m'esh d`e <h D;
m'esh >'ara ka`i <h E. ka`i >epe'i >estin <wc <h A pr`oc t`hn G, <h D pr`oc
t`hn E, <h d`e A t~hc G me~izon d'unatai t~w| >ap`o summ'etrou
<eaut~h|, ka`i <h D >'ara t~hc E me~izon dun'hsetai t~w| >ap`o
summ'etrou <eaut~h|. l'egw d'h, <'oti ka`i m'eson >est`i t`o <up`o
t~wn D, E. >epe`i g`ar >'ison >est`i t`o <up`o t~wn
B, G t~w| <up`o t~wn D, E, m'eson d`e t`o <up`o t~wn B, G [a<i g`ar
B, G <rhta'i e>isi dun'amei m'onon s'ummetroi], m'eson >'ara ka`i
t`o <up`o t~wn D, E.}

\gr{E<'urhntai >'ara d'uo m'esai dun'amei m'onon s'ummetroi a<i D, E
m'eson peri'eqousai, <'wste t`hn me'izona t~hc >el'assonoc
me~izon d'unasjai t~w| >ap`o summ'etrou <eaut~h|.}

\gr{<Omo'iwc d`h p'alin dieqj'hsetai ka`i t~w| >ap`o >asumm'etr\-ou, <'otan
<h A t~hc G me~izon d'unhtai t~w| >ap`o >asumm'etrou <eauth|.}}

\ParallelRText{
\begin{center}
{\large Proposition 32}
\end{center}

To find two medial (straight-lines), commensurable in square only, (and) containing a medial (area), such that
the square on the greater is larger than the (square on the) lesser by the
(square) on (some straight-line) commensurable (in length) with the greater.

\epsfysize=0.75in
\centerline{\epsffile{Book10/fig032e.eps}}

Let three rational (straight-lines), $A$, $B$ and $C$, commensurable
in square only, be laid out such that the square on $A$ is greater
than (the square on $C$) by the (square) on (some straight-line)
commensurable (in length) with ($A$) [Prop. 10.29].
And let the (square) on $D$ be equal to the (rectangle contained) by $A$ and
$B$. Thus, the (square) on $D$ (is) medial. Thus, $D$ is also
medial [Prop. 10.21].  And let the 
(rectangle contained) by $D$ and $E$ be equal to the (rectangle contained)
by $B$ and $C$. And since as the (rectangle contained) by $A$ and $B$
is to the (rectangle contained) by $B$ and $C$, so $A$ (is) to $C$
[Prop. 10.21~lem.], but the (square) on $D$
is equal to the (rectangle contained) by $A$ and $B$, and the
(rectangle contained) by $D$ and $E$ to the (rectangle contained) by
$B$ and $C$, thus as $A$ is to $C$, so the (square) on $D$ (is) to
the (rectangle contained) by $D$ and $E$. And as the (square) on $D$
(is) to the (rectangle contained) by $D$ and $E$, so $D$ (is) to $E$ [Prop. 10.21~lem.]. And thus as
$A$ (is) to $C$, so $D$ (is) to $E$. And $A$ (is) commensurable in square
[only] with $C$. Thus, $D$ (is) also commensurable in square only
with $E$ [Prop. 10.11]. And $D$ (is) medial.
Thus, $E$ (is) also medial [Prop. 10.23]. And
since as $A$ is to $C$, (so) $D$ (is) to $E$, and the square on $A$
is greater than (the square on) $C$ by the (square)
on (some straight-line) commensurable (in length) with ($A$),  the square on $D$
will thus also be greater than (the square on) $E$ by the (square) on (some straight-line)
commensurable (in length) with ($D$) [Prop. 10.14].
So, I also say that the (rectangle contained) by $D$ and $E$ is medial.
For since the (rectangle contained) by $B$ and $C$ is equal to the
(rectangle contained) by $D$ and $E$, and the (rectangle contained)
by $B$ and $C$ (is) medial [for $B$ and $C$ are rational
(straight-lines which are) commensurable in square only] [Prop. 10.21], the (rectangle contained) by $D$ and
$E$ (is) thus also medial.

Thus, two medial (straight-lines), $D$ and $E$, commensurable in square only, (and) containing a medial (area), have been found such that
the square on the greater is larger than the (square on the) lesser by the
(square) on (some straight-line) commensurable (in length) with the greater.$^\dag$.

So, similarly, (the proposition) can again also be demonstrated for
(some straight-line) incommensurable (in length with the greater), provided that
the square on $A$ is greater than (the square on) $C$ by the (square)
on (some straight-line) incommensurable (in length) with ($A$) [Prop. 10.30].$^\ddag$}
\end{Parallel}


\vspace{7pt}{\footnotesize\noindent $^\dag$ $D$ and $E$ have lengths ${k'}^{1/4}$ and ${k'}^{1/4}\sqrt{1-k^2}$ times that of $A$, respectively, where the length of $B$ is ${k'}^{1/2}$ times that of $A$, and $k$ is defined in the footnote to
Prop.~10.29.\\[0.5ex]
$^\ddag$ $D$ and $E$ would have lengths ${k'}^{1/4}$ and ${k'}^{1/4}/\sqrt{1+k^2}$ times that of $A$, respectively, where the length of $B$ is ${k'}^{1/2}$ times that of $A$, and $k$ is defined in the footnote to
Prop.~10.30.}

%%%%%%
% Prop 10.32a
%%%%%%
\begin{Parallel}{}{} 
\ParallelLText{
\begin{center}
{\large \gr{L~hmma}.}
\end{center}\vspace*{-7pt}

\gr{>'Estw tr'igwnon >orjog'wnion t`o ABG >orj`hn >'eqon t`hn A, ka`i
>'hqjw k'ajetoc <h AD; l'egw, <'oti t`o m`en <up`o t~wn
GBD >'ison >est`i t~w| >ap`o t~hc BA, t`o d`e <up`o t~wn BGA
>'ison t~w| >ap`o t~hc GA, ka`i t`o <up`o t~wn BD, DG >'ison t~w|
>ap`o t~hc AD, ka`i >'eti t`o <up`o t~wn BG, AD >'ison [>est`i]
t~w| <up`o t~wn BA, AG.}

\gr{Ka`i pr~wton, <'oti t`o <up`o t~wn GBD >'ison [>est`i] t~w|
>ap`o t~hc BA.}\\~\\~\\~\\

\epsfysize=2in
\centerline{\epsffile{Book10/fig032ag.eps}}

\gr{>Epe`i g`ar >en >orjogwn'iw| trig'wnw| >ap`o t~hc >orj~hc gwn'iac
>ep`i t`hn b'asin k'ajetoc >~hktai <h AD, t`a ABD, ADG >'ara tr'igwna
<'omoi'a >esti t~w| te <'olw| t~w| ABG ka`i >all'hloic. ka`i >epe`i
<'omoi'on >esti t`o ABG tr'igwnon t~w| ABD trig'wnw|, >'estin
>'ara <wc <h GB pr`oc t`hn BA, o<'utwc <h BA pr`oc t`hn BD;
t`o >'ara <up`o t~wn GBD >'ison >est`i t~w| >ap`o t~hc AB.}

\gr{Di`a t`a a>ut`a d`h ka`i t`o <up`o t~wn BGD >'ison >est`i t~w|
>ap`o t~hc AG.}

\gr{Ka`i >epe'i, >e`an >en >orjogwn'iw| trig'wnw| >ap`o t~hc >orj~hc
gwn'iac >ep`i t`hn b'asin k'ajetoc >aqj~h|, <h >aqje~isa t~wn t~hc
b'asewc tmhm'atwn m'esh >an'alog'on >estin, >'estin >'ara <wc <h
BA pr`oc t`hn DA, o<'utwc <h AD pr`oc t`hn DG; t`o >'ara <up`o
t~wn BD, DG >'ison >est`i t~w| >ap`o t~hc DA.}

\gr{L'egw, <'oti ka`i t`o <up`o t~wn BG, AD >'ison >est`i t~w| <up`o
t~wn BA, AG. >epe`i g`ar, <wc >'efamen, <'omoi'on >esti
t`o ABG t~w| ABD, >'estin >'ara <wc <h BG pr`oc t`hn GA,
o<'utwc <h BA pr`oc t`hn AD. t`o >'ara <up`o t~wn BG, AD >'ison
>est`i t~w| <up`o t~wn BA, AG; <'oper >'edei de~ixai.}}

\ParallelRText{
\begin{center}
{\large Lemma}
\end{center}

Let $ABC$ be a right-angled triangle having
the (angle) $A$ a right-angle. And let the perpendicular $AD$ have been
drawn. I say that the (rectangle contained) by $CBD$ is equal to the
(square) on $BA$, and the (rectangle contained) by $BCD$ (is)
equal to the (square) on $CA$, and the (rectangle contained) by
$BD$ and $DC$ (is) equal to the (square) on $AD$, and, further, the
(rectangle contained) by $BC$ and $AD$ [is] equal to the (rectangle
contained) by $BA$ and $AC$.

And, first of all, (let us prove) that the (rectangle contained) by
$CBD$ [is] equal to the (square) on $BA$.

\epsfysize=2in
\centerline{\epsffile{Book10/fig032ae.eps}}

For since $AD$ has been drawn from the right-angle  in a right-angled triangle,
perpendicular to the base, $ABD$ and $ADC$ are thus
triangles (which are) similar to the whole, $ABC$, and to one another
[Prop. 6.8]. And since triangle $ABC$ is
similar to triangle $ABD$, thus as $CB$ is to $BA$, so $BA$ (is)
to $BD$ [Prop. 6.4]. Thus, the
(rectangle contained) by $CBD$ is equal to the (square) on $AB$
[Prop. 6.17].

So, for the same (reasons), the (rectangle contained) by $BCD$
is also equal to the (square) on $AC$.

And since if a (straight-line) is drawn from the right-angle in a right-angled
triangle,  perpendicular to the base, the (straight-line so) drawn
is the mean proportional to the pieces of the base [Prop. 6.8~corr.], thus as $BD$ is to $DA$, so $AD$ (is) to $DC$. Thus, the (rectangle contained) by
$BD$ and $DC$ is equal to the (square) on $DA$ [Prop. 6.17]. 

I also say that the (rectangle contained) by $BC$ and
$AD$ is equal to the (rectangle contained) by $BA$ and $AC$.
For since,  as we said, $ABC$ is similar to $ABD$, thus as $BC$
is to $CA$, so $BA$ (is) to $AD$ [Prop. 6.4].
Thus, the (rectangle contained) by $BC$ and $AD$
is equal to the (rectangle contained) by $BA$ and $AC$ [Prop. 6.16].
(Which is) the very thing it was required to show.}
\end{Parallel}

%%%%%%
% Prop 10.33
%%%%%%
\pdfbookmark[1]{Proposition 10.33}{pdf10.33}
\begin{Parallel}{}{} 
\ParallelLText{
\begin{center}
{\large \ggn{33}.}
\end{center}\vspace*{-7pt}

\gr{E<ure~in d'uo e>uje'iac dun'amei >asumm'etrouc poio'usac t`o m`en sugke'imenon >ek t~wn >ap> a>ut~wn tetrag'wnwn <rht'on, t`o
d> <up> a>ut~wn m'eson.}

\gr{>Ekke'isjwsan d'uo <rhta`i dun'amei m'onon s'ummetroi a<i
AB, BG, <'wste t`hn me'izona t`hn AB t~hc >el'assonoc
t~hc BG me'izon d'unasjai t~w| >ap`o >asumm'etrou <eaut~h|,
ka`i tetm'hsjw <h BG d'iqa kat`a t`o D, ka`i t~w| >af>
<opot'erac t~wn BD, DG >'ison par`a t`hn AB parabebl'hsjw
parallhl'ogrammon >elle~ipon e>'idei tetrag'wnw|, ka`i >'estw t`o <up`o
t~wn AEB, ka`i gegr'afjw >ep`i t~hc AB hmik'uklion t`o AZB, ka`i
>'hqjw t~h| AB pr`oc >orj`ac <h EZ, ka`i >epeze'uqjwsan a<i AZ, ZB.}\\~\\~\\~\\~\\

\epsfysize=1.in
\centerline{\epsffile{Book10/fig033g.eps}}

\gr{Ka`i >epe`i [d'uo] e>uje~iai >'aniso'i e>isin a<i AB, BG, ka`i <h AB
t~hc BG me~izon d'unatai t~w| >ap`o >asumm'etrou <eaut~h|, t~w|
d`e tet'artw| to~u >ap`o t~hc BG, tout'esti t~w| >ap`o t~hc <hmise'iac
a>ut~hc, >'ison par`a t`hn AB parab'eblhtai parallhl'ogrammon
>elle~ipon e>'idei tetrag'wnw| ka`i poie~i t`o <up`o t~wn AEB,
>as'ummetroc >'ara >est`in <h AE t~h| EB. ka'i >estin <wc <h AE pr`oc
EB, o<'utwc t`o <up`o t~wn BA, AE pr`oc t`o <up`o t~wn AB, BE, >'ison
d`e t`o m`en <up`o t~wn BA, AE t~w| >ap`o t~hc AZ, t`o d`e <up`o
t~wn AB, BE t~w| <ap`o t~hc
 BZ; >as'ummetron
>'ara >est`i t`o >ap`o t~hc AZ t~w| >ap`o t~hc ZB; a<i AZ, ZB
>'ara dun'amei e>is`in >as'ummetroi. ka`i >epe`i <h AB <rht'h >estin,
<rht`on >'ara >est`i ka`i t`o >ap`o t~hc AB; <'wste ka`i t`o sugke'imenon
>ek t~wn >ap`o t~wn AZ, ZB <rht'on >estin. ka`i >epe`i p'alin t`o
<up`o t~wn AE, EB >'ison >est`i t~w| >ap`o t~hc EZ, <up'okeitai
d`e t`o <up`o t~wn AE, EB ka`i t~w| >ap`o t~hc BD >'ison, >'ish
>'ara >est`in <h ZE t~h| BD; dipl~h >'ara <h BG  t`hc ZE; <'wste
ka`i t`o <up`o t~wn AB, BG s'ummetr'on >esti t~w| <up`o t~wn
AB, EZ. m'eson d`e t`o <up`o t~wn AB, BG; m'eson >'ara ka`i t`o <up`o
t~wn AB, EZ. >'ison d`e t`o <up`o t~wn AB, EZ t~w| <up`o t~wn
AZ, ZB; m'eson >'ara ka`i t`o <up`o t~wn AZ, ZB. >ede'iqjh
d`e ka`i <rht`on t`o
 sugke'imenon >ek t~wn >ap>
a>ut~wn tetrag'wnwn.}

\gr{E<'urhntai >'ara d'uo e>uje~iai dun'amei >as'ummetroi a<i AZ, ZB poio~usai t`o m`en sugke'imenon >ek t~wn >ap> a>ut~wn tetrag'wnwn <rht'on,
t`o d`e <up> a>ut~wn m'eson; <'oper >'edei de~ixai.}}

\ParallelRText{
\begin{center}
{\large Proposition 33}
\end{center}

To find two straight-lines (which are) incommensurable
in square, making the sum of the squares on them rational, and the (rectangle contained) by them medial.

Let the two rational (straight-lines) $AB$ and $BC$, (which are) commensurable in square only, be laid out such that the square on the greater, $AB$, is larger
than (the square on) the lesser, $BC$, by the (square) on (some straight-line which is) incommensurable (in length)
with ($AB$) [Prop. 10.30]. And let $BC$ have been
cut in half at $D$. And let a parallelogram equal to the (square) on
either of $BD$ or $DC$, (and) falling short by a square figure, have
been applied to $AB$ [Prop. 6.28], and
let it be the (rectangle contained) by $AEB$. And let the semi-circle
$AFB$ have been drawn on $AB$. And let $EF$ have been
drawn at right-angles to $AB$. And let $AF$ and $FB$ have been joined.

\epsfysize=1.in
\centerline{\epsffile{Book10/fig033e.eps}}

And since $AB$ and $BC$ are [two] unequal straight-lines, and the square
on $AB$ is greater than (the square on) $BC$ by the (square) on (some
straight-line which is) incommensurable (in length) with ($AB$). And a parallelogram,
equal to one quarter of the (square) on $BC$---that is to say, (equal) to the
(square) on half of it---(and) falling short by a square figure, has
been applied to $AB$, and  makes the (rectangle contained) by $AEB$.
$AE$ is thus incommensurable (in length) with $EB$ [Prop. 10.18]. And as $AE$ is to $EB$, so the
(rectangle contained) by $BA$ and $AE$ (is) to the (rectangle contained) by
$AB$ and $BE$.
 And the (rectangle contained) by $BA$ and $AE$ (is) equal to the (square) on $AF$, and the (rectangle contained) by $AB$ and $BE$ to the (square) on
$BF$ [Prop. 10.32~lem.]. The (square) on
$AF$ is thus incommensurable with the (square) on $FB$ [Prop. 10.11]. Thus, $AF$ and $FB$ are
incommensurable in square. And since $AB$ is rational,
the (square) on $AB$ is also rational. Hence, the sum of the
(squares) on $AF$ and $FB$ is also rational [Prop. 1.47]. And, again, since the (rectangle
contained) by $AE$ and $EB$ is equal to the (square) on $EF$, and
the (rectangle contained) by $AE$ and $EB$ 
was assumed (to be) equal to the (square) on $BD$, $FE$ is thus equal to $BD$. Thus, $BC$
is double $FE$. And hence the (rectangle contained) by $AB$ and $BC$ is commensurable with the (rectangle contained) by $AB$ and $EF$ [Prop. 10.6]. And the (rectangle contained) by
$AB$ and $BC$ (is) medial [Prop. 10.21]. 
Thus, the (rectangle contained) by $AB$ and $EF$ (is) also medial
[Prop. 10.23~corr.].  And the (rectangle contained)
by $AB$ and $EF$ (is) equal to the (rectangle contained) by $AF$ and $FB$
[Prop. 10.32~lem.]. Thus, the (rectangle
contained) by $AF$ and $FB$ (is) also medial. And the
sum of the squares on them was also shown (to be) rational.

Thus, the two straight-lines, $AF$ and $FB$, (which are)
incommensurable in  square, have been found, making the
sum of the squares on them rational, and the (rectangle contained)
by them medial. (Which is) the very thing it was required to show.}
\end{Parallel}


\vspace{7pt}{\footnotesize\noindent$^\dag$ $AF$ and $FB$ have lengths
$\sqrt{[1+k/(1+k^2)^{1/2}]/2}$ and $\sqrt{[1-k/(1+k^2)^{1/2}]/2}$
times that of $AB$, respectively, where $k$ is defined in the footnote
to Prop.~10.30.}

%%%%%%
% Prop 10.34
%%%%%%
\pdfbookmark[1]{Proposition 10.34}{pdf10.34}
\begin{Parallel}{}{} 
\ParallelLText{
\begin{center}
{\large \ggn{34}.}
\end{center}\vspace*{-7pt}

\gr{E<ure~in d'uo e>uje'iac dun'amei >asumm'etrouc poio'usac t`o m`en
sugke'imenon >ek t~wn >ap> a>ut~wn tetrag'wnwn m'eson, t`o d>
<up> a>ut~wn <rht'on.}

\epsfysize=1.in
\centerline{\epsffile{Book10/fig034g.eps}}

\gr{>Ekke'isjwsan d'uo m'esai dun'amei m'onon s'ummetroi a<i AB, BG <rht`on
peri'eqousai t`o <up> a>ut~wn, <'wste t`hn AB t~hc BG me~izon d'unasjai
t~w| >ap`o >asumm'etrou <eaut~h|, ka`i gegr'afjw >ep`i t~hc AB t`o
ADB <hmik'uklion, ka`i tetm'hsjw <h BG d'iqa kat`a t`o E, ka`i
parabebl'hsjw par`a t`hn AB t~w| >ap`o t~hc BE >'ison
parallhl'ogrammon >elle~ipon e>'idei tetrag'wnw| t`o <up`o t~wn AZB;
>as'ummetroc >'ara [>est`in] <h AZ t~h| ZB m'hkei. ka`i >'hqjw >ap`o
to~u Z t~h| AB pr`oc >orj`ac <h ZD, ka`i >epeze'uqjwsan a<i AD, DB.}

\gr{>Epe`i >as'ummetr'oc >estin <h AZ t~h| ZB, >as'ummetron
>'ara >est`i ka`i t`o <up`o t~wn BA, AZ t~w| <up`o t~wn
AB, BZ. >'ison d`e t`o m`en <up`o t~wn BA, AZ t~w| >ap`o t~hc
AD, t`o d`e <up`o t~wn AB, BZ t~w| >ap`o t~hc DB; >as'ummetron
>'ara >est`i ka`i t`o >ap`o t~hc AD t~w| >ap`o t~hc DB.
ka`i >epe`i m'eson >est`i t`o >ap`o t~hc AB, m'eson >'ara ka`i t`o
sugke'imenon >ek t~wn >ap`o t~wn AD, DB. ka`i >epe`i dipl~h
>estin <h BG t~hc DZ, dipl'asion >'ara ka`i t`o
<up`o t~wn AB, BG to~u <up`o t~wn AB, ZD. <rht`on d`e t`o <up`o
t~wn AB, BG; <rht`on >'ara ka`i t`o <up`o t~wn AB, ZD. t`o d`e
<up`o t~wn AB, ZD >'ison t~w| <up`o
t~wn AD, DB; <'wste ka`i t`o <up`o t~wn AD, DB <rht'on
>estin.}

\gr{E<'urhntai >'ara d'uo e>uje~iai dun'amei >as'ummetroi a<i
AD, DB poio~usai t`o [m`en] sugke'imenon >ek t~wn >ap>
a>ut~wn tetrag'wnwn m'eson, t`o d> <up> a>ut~wn <rht'on;
<'oper >'edei de~ixai.}}

\ParallelRText{
\begin{center}
{\large Proposition 34}
\end{center}

To find two straight-lines (which are) incommensurable in square, making the sum of the squares on them medial,
and the (rectangle contained) by them rational.

\epsfysize=1.in
\centerline{\epsffile{Book10/fig034e.eps}}

Let the two medial (straight-lines) $AB$ and $BC$,
(which are) commensurable in square only, be laid out having the
(rectangle contained) by them rational, (and) such that the square on
$AB$ is greater than (the square on) $BC$ by the (square) on (some straight-line)
incommensurable (in length) with ($AB$) [Prop. 10.31].
And let the semi-circle $ADB$ have been drawn on $AB$.
And let $BC$ have been cut in half at $E$. And let a (rectangular) parallelogram
equal to the (square) on $BE$, (and) falling short by a square figure,
have been applied to $AB$, (and let it be) the (rectangle
contained by) $AFB$ [Prop. 6.28].
Thus, $AF$ [is] incommensurable in length with $FB$
[Prop. 10.18]. And let 
$FD$ have been drawn from $F$ at right-angles to $AB$.
And let $AD$ and $DB$ have been joined.

Since $AF$ is incommensurable (in length) with $FB$, the (rectangle contained) by
$BA$ and $AF$ is thus also incommensurable with the (rectangle contained)
by $AB$ and $BF$ [Prop. 10.11].  And the
(rectangle contained) by $BA$ and $AF$ (is) equal to the (square) on $AD$, and the (rectangle contained) by $AB$ and $BF$ to the (square) on $DB$
[Prop. 10.32~lem.]. Thus, the (square) on
$AD$ is also incommensurable with the (square) on $DB$. And
since the (square) on $AB$ is medial, the sum of the (squares) on
$AD$ and $DB$ (is) thus also medial [Props.~3.31, 1.47]. And since $BC$ is double $DF$ [see previous proposition], the (rectangle
contained) by $AB$ and $BC$ (is) thus also double the (rectangle contained)
by $AB$ and $FD$. And the (rectangle contained) by $AB$ and $BC$ (is)
rational. Thus, the (rectangle contained) by $AB$ and $FD$
(is) also rational [Prop. 10.6, Def.~10.4]. And the (rectangle contained) by
$AB$ and $FD$ (is) equal to the (rectangle contained) by $AD$ and $DB$
[Prop. 10.32~lem.]. And hence the (rectangle
contained) by $AD$ and $DB$ is rational.

Thus, two straight-lines, $AD$ and $DB$, (which are) incommensurable
in square, have been found, making the sum of the squares on them medial,
and the (rectangle contained) by them rational.$^\dag$
(Which is) the very thing
it was required to show.}
\end{Parallel}


\vspace{7pt}{\footnotesize\noindent$^\dag$ $AD$ and $DB$ have lengths $\sqrt{[(1+k^2)^{1/2}+k]/[2\,(1+k^2)]}$
and  $\sqrt{[(1+k^2)^{1/2}-k]/[2\,(1+k^2)]}$ times that of $AB$, respectively, where $k$ is defined in the footnote
to Prop.~10.29.}

%%%%%%
% Prop 10.35
%%%%%%
\pdfbookmark[1]{Proposition 10.35}{pdf10.35}
\begin{Parallel}{}{} 
\ParallelLText{
\begin{center}
{\large \ggn{35}.}
\end{center}\vspace*{-7pt}

\gr{E<ure~in d'uo e>uje'iac dun'amei >asumm'etrouc poio'usac t'o te sugke'imenon >ek t~wn >ap> a>ut~wn tetrag'wnwn m'eson ka`i
t`o <up> a>ut~wn m'eson ka`i >'eti >as'ummetron t~w| sugkeim'enw|
>ek t~wn >ap> a>ut~wn tetrag'wnw|.}\\

\epsfysize=1.in
\centerline{\epsffile{Book10/fig034g.eps}}

\gr{>Ekke'isjwsan d'uo m'esai dun'amei m'onon s'ummetroi a<i AB, BG
m'eson peri'eqousai, <'wste t`hn AB t~hc BG me~izon d'unasjai t~w|
>ap`o >asumm'etrou <eaut~h|, ka`i gegr'afjw >ep`i t~hc AB
<hmik'uklion t`o ADB, ka`i t`a loip`a gegon'etw to~ic >ep'anw
<omo'iwc.}

\gr{Ka`i >epe`i >as'ummetr'oc >estin <h AZ t~h| ZB m'hkei, >as'ummetr\-'oc
>esti ka`i <h AD t~h| DB dun'amei. ka`i >epe`i m'eson >est`i
t`o >ap`o t~hc AB, m'eson >'ara ka`i t`o sugke'imenon >ek t~wn
>ap`o t~wn AD, DB. ka`i >epe`i t`o <up`o t~wn AZ, ZB >'ison
>est`i t~w| >af> <ekat'erac t~wn BE, DZ, >'ish >'ara
>est`in <h BE t~h| DZ; dipl~h >'ara <h BG t~hc ZD; <'wste
ka`i t`o <up`o t~wn AB, BG dipl'asi'on >esti to~u <up`o t~wn
AB, ZD. m'eson d`e t`o <up`o t~wn AB, BG; m'eson >'ara ka`i t`o
<up`o t~wn AB, ZD. ka'i >estin >'ison t~w| <up`o t~wn AD, DB;
m'eson >'ara ka`i t`o <up`o t~wn AD, DB. ka`i >epe`i >as'ummetr'oc
>estin <h AB t~h| BG m'hkei, s'ummetroc d`e <h GB t~h| BE,
>as'ummetroc >'ara ka`i <h AB t~h| BE m'hkei; <'wste ka`i t`o >ap`o
t~hc AB t~w| <up`o t~wn AB, BE >as'ummetr'on >estin. >all`a
t~w| m`en >ap`o t~hc AB >'isa >est`i t`a >ap`o t~wn AD, DB, t~w|
d`e <up`o t~wn AB, BE >'ison >est`i t`o <up`o t~wn AB,
ZD, tout'esti t`o <up`o t~wn AD, DB; >as'ummetron >'ara
>est`i t`o sugke'imenon >ek t~wn >ap`o t~wn AD, DB
t~w| <up`o t~wn AD, DB.}

\gr{E<'urhntai >'ara d'uo e>uje~iai a<i AD, DB dun'amei >as'ummetroi
poio~usai t'o te sugke'imenon >ek t~wn >ap> a>ut~wn m'eson
ka`i t`o <up> a>ut~wn m'eson ka`i >'eti >as'ummetron t~w|
sugkeim'enw| >ek t~wn >ap> a>ut~wn tetrag'wnwn;
<'oper >'edei de~ixai.}}

\ParallelRText{
\begin{center}
{\large Proposition 35}
\end{center}

To find two straight-lines (which are) incommensurable in square, making the sum of the squares on them medial,
and the (rectangle contained) by them medial, and, moreover,
incommensurable  with the sum of the squares on them.

\epsfysize=1.in
\centerline{\epsffile{Book10/fig034e.eps}}

Let the two medial (straight-lines) $AB$ and $BC$, (which are) commensurable in square only, be laid out containing a medial (area), such
that the square on $AB$ is greater than (the square on) $BC$ by the
(square) on (some straight-line) incommensurable (in length) with ($AB$) [Prop. 10.32].  And let the semi-circle
$ADB$ have been drawn on $AB$. And let the remainder (of the figure)
be generated similarly to the above (proposition).

And since $AF$ is incommensurable in length with $FB$ [Prop. 10.18], $AD$
is also incommensurable in square with $DB$ [Prop. 10.11]. And since the (square) on $AB$ is
medial, the sum of the (squares) on $AD$ and $DB$ (is) thus also
medial [Props.~3.31, 1.47]. And since the (rectangle
contained) by $AF$ and $FB$ is equal to the (square) on each of $BE$ and
$DF$, $BE$ is thus equal to $DF$. Thus, $BC$ (is) double $FD$. And
hence the (rectangle contained) by $AB$ and $BC$ is double the
(rectangle) contained by $AB$ and $FD$. And the (rectangle contained)
by $AB$ and $BC$ (is) medial. Thus, the (rectangle contained) by $AB$
and $FD$ (is) also medial. And it is equal to the (rectangle contained) 
by $AD$ and $DB$ [Prop. 10.32~lem.]. Thus, the
(rectangle contained) by $AD$ and $DB$ (is) also medial.
And since $AB$ is incommensurable in length with $BC$, and $CB$
(is) commensurable (in length) with $BE$, $AB$ (is) thus also incommensurable
in length with $BE$ [Prop. 10.13]. And hence the (square) on $AB$ is also incommensurable with the (rectangle contained) by $AB$ and $BE$ [Prop. 10.11]. But the (sum of the squares) on $AD$ and
$DB$ is equal to the (square) on $AB$ [Prop. 1.47].
And the (rectangle contained) by $AB$ and $FD$---that is to say, the (rectangle contained) by $AD$ and $DB$---is equal to the
(rectangle contained) by $AB$ and $BE$. Thus, the sum of the (squares) on
$AD$ and $DB$ is incommensurable with the (rectangle contained) by
$AD$ and $DB$.

Thus, two straight-lines, $AD$ and $DB$, (which are) incommensurable
in square, have been found, making the sum of the (squares) on them
medial, and the (rectangle contained) by them medial, and, moreover, 
incommensurable  with the sum of the squares on them.$^\dag$
 (Which is) the
very thing it was required to show.}
\end{Parallel}


\vspace{7pt}{\footnotesize\noindent$^\dag$ $AD$ and $DB$ have lengths
${k'}^{1/4}\sqrt{[1+k/(1+k^2)^{1/2}]/2}$ and  ${k'}^{1/4}\sqrt{[1-k/(1+k^2)^{1/2}]/2}$ times that of $AB$, respectively, where $k$ and $k'$
are defined in the footnote to Prop.~10.32.}

%%%%%%
% Prop 10.36
%%%%%%
\pdfbookmark[1]{Proposition 10.36}{pdf10.36}
\begin{Parallel}{}{} 
\ParallelLText{
\begin{center}
{\large \ggn{36}.}
\end{center}\vspace*{-7pt}

\gr{>E`an d'uo <rhta`i dun'amei m'onon s'ummetroi suntej~wsin, <h <'olh
>'alog'oc >estin, kale'isjw d`e >ek d'uo >onom'atwn.}\\~\\

\epsfysize=0.3in
\centerline{\epsffile{Book10/fig036g.eps}}

\gr{Sugke'isjwsan g`ar d'uo <rhta`i dun'amei m'onon s'ummet\-roi a<i
AB, BG; l'egw, <'oti <'olh <h AG >'alog'oc >estin.}

\gr{>Epe`i g`ar >as'ummetr'oc >estin <h AB t~h| BG m'hkei; dun'amei
g`ar m'onon e>is`i s'ummetroi; <wc d`e <h AB pr`oc t`hn BG, o<'utwc
t`o <up`o t~wn ABG pr`oc t`o >ap`o t~hc BG, >as'ummetron >'ara
>est`i t`o <up`o t~wn AB, BG t~w| >ap`o t~hc BG. >all`a t~w| m`en
<up`o t~wn AB, BG s'ummetr'on >esti t`o d`ic <up`o t~wn AB, BG,
t~w| d`e >ap`o t~hc BG s'ummetr'a >esti t`a >ap`o t~wn AB, BG;
a<i g`ar AB, BG <rhta'i e>isi dun'amei m'onon s'ummetroi; >as'ummetron
>'ara >est`i t`o d`ic <up`o t~wn AB, BG to~ic
>ap`o t~wn AB, BG. ka`i sunj'enti t`o d`ic <up`o t~wn AB, BG
met`a t~wn >ap`o t~wn AB, BG, tout'esti t`o >ap`o t~hc AG,
>as'ummetr'on >esti t~w| sugkeim'enw| >ek t~wn >ap`o t~wn AB, BG;
<rht`on d`e t`o sugke'imenon >ek t~wn >ap`o t~wn AB, BG;
>'alogon >'ara [>est`i] t`o >ap`o t~hc AG; <'wste ka`i <h AG >'alog'oc
>estin, kale'isjw d`e >ek d'uo >onom'atwn; <'oper >'edei de~ixai.}}

\ParallelRText{
\begin{center}
{\large Proposition 36}
\end{center}

If two rational (straight-lines which are) commensurable
in square only are added together then the whole (straight-line) is irrational---let it
be called a binomial (straight-line).$^\dag$

\epsfysize=0.3in
\centerline{\epsffile{Book10/fig036e.eps}}

For let the two rational (straight-lines), $AB$ and $BC$, (which are)
commensurable in square only, be laid down together. I say that the
whole (straight-line), $AC$, is irrational.
For since $AB$ is incommensurable in length with $BC$---for they
are commensurable in square only---and as $AB$ (is) to $BC$, so
the (rectangle contained) by $ABC$ (is) to the (square) on $BC$, the (rectangle contained)
by $AB$ and $BC$ is thus incommensurable with the
(square) on $BC$ [Prop. 10.11]. But,
twice the (rectangle contained) by $AB$ and $BC$ is commensurable
with the (rectangle contained) by $AB$ and $BC$ [Prop. 10.6]. And (the sum of) the (squares) on
$AB$ and $BC$
 is commensurable with the (square) on $BC$---for the rational (straight-lines) $AB$ and $BC$ are commensurable in square only [Prop. 10.15].
Thus, twice the (rectangle contained) by $AB$ and $BC$ is incommensurable
with (the sum of) the (squares) on $AB$ and $BC$ [Prop. 10.13]. And, via composition, twice
the (rectangle contained) by $AB$ and $BC$, plus (the sum of) the (squares)
on $AB$ and $BC$---that is to say, the (square) on $AC$ [Prop. 2.4]---is incommensurable with the
sum of the (squares) on $AB$ and $BC$ [Prop. 10.16]. And the sum of the (squares) on $AB$ and $BC$ (is) rational. Thus, the (square) on $AC$ [is]
irrational [Def. 10.4]. Hence, $AC$
is also irrational [Def. 10.4]---let it be called
a binomial (straight-line).$^\ddag$  (Which is) the very thing it was required to show.}
\end{Parallel}


\vspace{7pt}{\footnotesize\noindent$^\dag$ Literally, ``from two names".\\[0.5ex]
$^\ddag$ Thus, a binomial straight-line has a length
expressible as  $1 +k^{1/2}$ [or, more generally, $\rho\,(1+k^{1/2})$, where $\rho$ is rational---the same proviso applies to the definitions in the following propositions]. The binomial and the corresponding apotome, whose length is expressible as
$1-k^{1/2}$ (see Prop.~10.73), are
the positive roots of the quartic $x^4-2\,(1+k)\,x^2+ (1-k)^2 = 0$.}

%%%%%%
% Prop 10.37
%%%%%%
\pdfbookmark[1]{Proposition 10.37}{pdf10.37}
\begin{Parallel}{}{} 
\ParallelLText{
\begin{center}
{\large \ggn{37}.}
\end{center}\vspace*{-7pt}

\gr{>E`an d'uo m'esai dun'amei m'onon s'ummetroi suntej~wsi <rht`on peri'eqousai, <h <'olh >'alog'oc >estin, kale'isjw d`e >ek d'uo m'eswn
pr'wth.}\\

\epsfysize=0.3in
\centerline{\epsffile{Book10/fig036g.eps}}

\gr{Sugke'isjwsan g`ar d'uo m'esai dun'amei m'onon s'ummet\-roi a<i AB, BG
<rht`on peri'eqousai; l'egw, <'oti <'olh <h AG >'alog'oc >estin.}

\gr{>Epe`i g`ar >as'ummetr'oc >estin <h AB t~h| BG m'hkei, ka`i t`a >ap`o
t~wn AB, BG >'ara >as'ummetr'a >esti t~w| d`ic <up`o t~wn AB, BG;
ka`i sunj'enti t`a >ap`o t~wn AB, BG met`a to~u d`ic <up`o t~wn AB, BG,
<'oper >est`i t`o >ap`o t~hc AG, >as'ummetr'on >esti t~w| <up`o t~wn
AB, BG.
<rht`on d`e t`o <up`o t~wn AB, BG; <up'okeintai g`ar a<i AB, BG
<rht`on peri'eqousai; >'alogon >'ara t`o >ap`o t~hc AG;
>'alogoc >'ara <h AG, kale'isjw d`e >ek d'uo m'eswn pr'wth; <'oper
>'edei de~ixai.}}

\ParallelRText{
\begin{center}
{\large Proposition 37}
\end{center}

If two medial (straight-lines), commensurable in square only, which contain a rational (area), are added together then
the whole (straight-line) is irrational---let it be called a first bimedial (straight-line).$^\dag$

\epsfysize=0.3in
\centerline{\epsffile{Book10/fig036e.eps}}

For let the  two medial (straight-lines), $AB$ and $BC$, 
commensurable in square only, (and) containing a rational (area), be laid down together. I say that the
whole (straight-line), $AC$, is irrational.

For since $AB$ is incommensurable in length with $BC$, (the sum of)
the (squares) on $AB$ and $BC$ is thus also incommensurable with twice
the (rectangle contained) by $AB$ and $BC$ [see previous proposition].
And, via composition, (the sum of) the (squares)  on $AB$ and $BC$,
plus twice the (rectangle contained) by $AB$ and $BC$---that is, 
the (square) on $AC$ [Prop. 2.4]---is incommensurable with the (rectangle contained) by $AB$ and $BC$
[Prop. 10.16]. And the (rectangle contained) by
$AB$ and $BC$ (is) rational---for $AB$ and $BC$ were assumed to
enclose a rational (area). Thus, the (square) on $AC$ (is) irrational. Thus, $AC$ (is) irrational [Def. 10.4]---let it be called
a first bimedial (straight-line).$^\ddag$
 (Which is) the very thing it was required to show.}
\end{Parallel}


\vspace{7pt}{\footnotesize\noindent$^\dag$ Literally, ``first from two medials''.\\[0.5ex]
$^\ddag$ Thus, a first bimedial straight-line
has a length expressible as $k^{1/4}+k^{3/4}$. The first bimedial and the corresponding first apotome of a medial, whose length is expressible as
$k^{1/4}-k^{3/4}$ (see Prop.~10.74), are
the positive roots of the quartic $x^4-2\,\sqrt{k}\,(1+k)\,x^2+ k\,(1-k)^2 = 0$.}

%%%%%%
% Prop 10.38
%%%%%%
\pdfbookmark[1]{Proposition 10.38}{pdf10.38}
\begin{Parallel}{}{} 
\ParallelLText{
\begin{center}
{\large \ggn{38}.}
\end{center}\vspace*{-7pt}

\gr{>E`an d'uo m'esai dun'amei m'onon s'ummetroi suntej~wsi m'eson peri'eqousai, <h <'olh >'alog'oc >estin, kale'isjw d`e >ek d'uo m'eswn
duet'era.}\\

\epsfysize=1.7in
\centerline{\epsffile{Book10/fig038g.eps}}

\gr{Sugke'isjwsan g`ar d'uo m'esai dun'amei m'onon s'ummet\-roi a<i AB, BG
m'eson peri'eqousai; l'egw, <'oti >'alog'oc >estin <h AG.}

\gr{>Ekke'isjw g`ar <rht`h <h DE, ka`i t~w| >ap`o t~hc AG >'ison
par`a t`hn DE parabebl'hsjw t`o DZ pl'atoc poio~un t`hn DH. ka`i
>epe`i t`o >ap`o t~hc AG >'ison >est`i to~ic te >ap`o t~wn AB, BG
ka`i t~w| d`ic <up`o t~wn AB, BG, parabebl'hsjw d`h to~ic >ap`o t~wn
AB, BG par`a t`hn DE >'ison t`o EJ; loip`on >'ara t`o JZ >'ison
>est`i t~w| d`ic <up`o t~wn AB, BG. ka`i >epe`i m'esh >est`in <ekat'era
t~wn AB, BG, m'esa >'ara >est`i ka`i t`a >ap`o t~wn AB, BG. m'eson
d`e <up'okeitai ka`i t`o d`ic <up`o t~wn AB, BG. ka'i >esti to~ic
m`en >ap`o t~wn AB, BG >'ison t`o EJ, t~w| d`e d`ic <up`o t~wn
AB, BG >'ison t`o ZJ; m'eson >'ara <ek'ateron t~wn EJ, JZ. ka`i par`a
<rht`hn t`hn DE par'akeitai; <rht`h >'ara >est`in <ekat'era t~wn
DJ, JH ka`i >as'ummetroc t~h| DE m'hkei. >epe`i o>~un >as'ummetr'oc
>estin <h AB t~h| BG m'hkei, ka'i >estin <wc <h AB pr`oc t`hn BG, o<'utwc
t`o >ap`o t~hc AB pr`oc t`o <up`o t~wn AB, BG, >as'ummetron >'ara >est`i
t`o >ap`o t~hc AB t~w| <up`o t~wn AB, BG. >all`a t~w| m`en >ap`o t~hc
AB s'ummetr'on >esti t`o sugke'imenon >ek t~wn >ap`o t~wn AB, BG
tetrag'wnwn, t~w| d`e <up`o t~wn AB, BG s'ummetr'on >esti t`o d`ic
<up`o t~wn AB, BG. >as'ummetron >'ara >est`i t`o sugke'imenon
>ek t~wn >ap`o t~wn AB, BG t~w| d`ic <up`o t~wn AB, BG. >all`a
to~ic m`en >ap`o t~wn AB, BG >'ison >est`i t`o EJ, t~w| d`e d`ic <up`o
t~wn AB, BG >'ison
>est`i t`o JZ. >as'ummetron >'ara >est`i t`o EJ t~w| JZ; <'wste ka`i <h DJ
t~h| JH >estin >as'ummetroc m'hkei. a<i DJ, JH >'ara <rhta'i e>isi
dun'amei m'onon s'ummetroi. <'wste <h DH >'alog'oc >estin. <rht`h d`e <h DE; t`o d`e <up`o >al'ogou ka`i <rht~hc perieq'omenon >orjog'wnion
>'alog'on >estin; >'alogon >'ara >est`i t`o DZ qwr'ion, ka`i <h dunam'enh
[a>ut`o] >'alog'oc >estin. d'unatai d`e t`o DZ <h AG; >'alogoc >'ara
>est`in <h AG, kale'isjw d`e >ek d'uo m'eswn deut'era. <'oper
>'edei de~ixai.}}

\ParallelRText{
\begin{center}
{\large Proposition 38}
\end{center}
If two medial (straight-lines), commensurable in
square only, which contain a medial (area), are added together then the
whole (straight-line) is irrational---let it be called a second bimedial
(straight-line).

\epsfysize=1.7in
\centerline{\epsffile{Book10/fig038e.eps}}

For let the  two medial (straight-lines), $AB$ and $BC$, 
commensurable in square only, (and) containing a medial (area), be laid down together [Prop. 10.28]. I say that $AC$ is irrational.

For let the rational (straight-line) $DE$ be laid down, and let (the rectangle) $DF$, equal
to the (square) on $AC$, have been applied to $DE$, making $DG$ as breadth
[Prop. 1.44]. And since the
(square) on $AC$ is equal to (the sum of) the (squares) on $AB$ and $BC$,
plus twice the (rectangle contained) by $AB$ and $BC$ [Prop. 2.4], so let (the rectangle) $EH$, equal to
(the sum of) the squares on $AB$ and $BC$, have been applied
to $DE$. The remainder $HF$ is thus equal to twice the (rectangle contained)
by $AB$ and $BC$. And since $AB$ and $BC$ are each medial, (the
sum of) the squares on $AB$ and $BC$ is thus also medial.$^\ddag$ And twice the (rectangle contained) by $AB$ and $BC$ was also assumed (to be) medial.
And $EH$ is equal to (the sum of) the squares on $AB$ and $BC$, and
$FH$ (is) equal to twice the (rectangle contained) by $AB$ and $BC$. 
Thus, $EH$ and $HF$ (are) each medial. And they were applied to the
rational (straight-line) $DE$. Thus, $DH$ and $HG$ are
each rational, and incommensurable in length with $DE$ [Prop. 10.22]. Therefore, since $AB$ is incommensurable in length with $BC$, and as $AB$ is to $BC$, so the
(square) on $AB$ (is) to the (rectangle contained) by $AB$ and $BC$ [Prop. 10.21~lem.], the (square) on $AB$
is thus incommensurable with the (rectangle contained) by $AB$ and $BC$
[Prop. 10.11]. But, the sum of the squares on $AB$ and $BC$ is commensurable with the (square) on $AB$ [Prop. 10.15], and twice the (rectangle contained)
by $AB$ and $BC$ is commensurable with the (rectangle contained) by $AB$ and $BC$ [Prop. 10.6]. Thus, the sum of
the (squares) on $AB$ and $BC$ is incommensurable with twice the (rectangle
contained) by $AB$ and $BC$ [Prop. 10.13].
But, $EH$ is equal to (the sum of) the squares on $AB$ and $BC$,
and $HF$ is equal to twice the (rectangle) contained by $AB$ and $BC$. Thus, $EH$ is incommensurable with $HF$. Hence, $DH$ is also
incommensurable in length with $HG$ [Props.~6.1, 10.11]. Thus, $DH$ and 
$HG$ are rational (straight-lines which are) commensurable in square only.
Hence, $DG$ is irrational [Prop. 10.36].
And  $DE$ (is) rational. And the rectangle contained by irrational and
rational (straight-lines) is irrational [Prop. 10.20].
The area $DF$ is thus irrational, and (so) the square-root [of it] is irrational
[Def. 10.4]. And  $AC$ is the square-root of $DF$. $AC$ is thus irrational---let it be called a second bimedial (straight-line).$^\S$ (Which is) the very thing it was required to show.}
\end{Parallel}


\vspace{7pt}{\footnotesize\noindent$^\dag$ Literally, 
``second from two medials''.\\[0.5ex]
$^\ddag$ Since, by hypothesis, the squares on $AB$ and $BC$ are commensurable---see Props.~10.15, 10.23.\\[0.5ex]
$^\S$ Thus, a second bimedial straight-line
has a length expressible as $k^{1/4}+{k'}^{1/2}/k^{1/4}$. The second bimedial and the corresponding second apotome of a medial, whose length is expressible as
$k^{1/4}-{k'}^{1/2}/k^{1/4}$ (see Prop.~10.75), are
the positive roots of the quartic $x^4-2\,[(k+k')/\sqrt{k}\,]\,x^2+ [(k-k')^2/k] = 0$.}

%%%%%%
% Prop 10.39
%%%%%%
\pdfbookmark[1]{Proposition 10.39}{pdf10.39}
\begin{Parallel}{}{} 
\ParallelLText{
\begin{center}
{\large \ggn{39}.}
\end{center}\vspace*{-7pt}

\gr{>E`an d'uo e>uje~iai dun'amei >as'ummetroi suntej~wsi poio~us\-ai t`o
m`en sugke'imenon >ek t~wn >ap> a>ut~wn tetrag'wnwn <rht'on, t`o
d> <up> a>ut~wn m'eson, <h <'olh e>uje~ia >'alog'oc >estin, kale'isjw
d`e me'izwn.}\\

\epsfysize=0.3in
\centerline{\epsffile{Book10/fig036g.eps}}

\gr{Sugke'isjwsan g`ar d'uo e>uje~iai dun'amei >as'ummetroi a<i AB, BG
poio~usai t`a proke'imena; l'egw, <'oti >'alog'oc >estin <h AG.}

\gr{>Epe`i g`ar t`o <up`o t~wn AB, BG m'eson >est'in, ka`i t`o d`ic [>'ara]
<up`o t~wn AB, BG m'eson >est'in. t`o d`e sugke'imenon >ek t~wn
>ap`o t~wn AB, BG <rht'on; >as'ummetron >'ara >est`i t`o d`ic <up`o
t~wn AB, BG t~w| sugkeim'enw| >ek t~wn >ap`o t~wn AB, BG;
 <'wste ka`i t`a >ap`o t~wn AB, BG met`a to~u d`ic <up`o
t~wn AB, BG, <'oper >est`i t`o >ap`o t~hc AG, >as'ummetr'on >esti
t~w| sugkeim'enw| >ek t~wn >ap`o t~wn AB, BG [<rht`on d`e t`o sugme'imenon >ek t~wn >ap`o t~wn AB, BG]; >'alogon >'ara >est`i
t`o >ap`o t~hc AG. <'wste ka`i <h AG >'alog'oc >estin, kale'isjw
d`e me'izwn. <'oper >'edei de~ixai.}}

\ParallelRText{
\begin{center}
{\large Proposition 39}
\end{center}

If two straight-lines (which are) incommensurable
in square, making the sum of the squares on them
rational, and the (rectangle contained) by them medial,  are added together then the whole
straight-line is irrational---let it be called a major (straight-line).

\epsfysize=0.3in
\centerline{\epsffile{Book10/fig036e.eps}}

For let the two straight-lines, $AB$ and $BC$,  incommensurable
in square, and fulfilling  the  prescribed (conditions), be laid down together [Prop. 10.33]. I say that $AC$ is irrational.

For since the (rectangle contained) by $AB$ and $BC$ is medial, twice
the (rectangle contained) by $AB$ and $BC$ is [thus] also medial [Props.~10.6, 10.23~corr.].
And the sum of the (squares) on $AB$ and $BC$ (is) rational. Thus, twice
the (rectangle contained) by $AB$ and $BC$ is incommensurable with the
sum of the (squares) on $AB$ and $BC$ [Def. 10.4].
Hence, (the sum of) the squares on $AB$ and $BC$, plus twice the
(rectangle contained) by $AB$ and $BC$---that is, the (square) on $AC$ [Prop. 2.4]---is also
incommensurable with the sum of the (squares) on $AB$ and $BC$
[Prop. 10.16] [and the
sum of the (squares) on $AB$ and $BC$ (is) rational]. Thus, the (square) on $AC$ is irrational. Hence, $AC$ is also irrational [Def. 10.4]---let it be called a major (straight-line).$^\dag$ (Which is) the
very thing it was required to show.}
\end{Parallel}


\vspace{7pt}{\footnotesize\noindent$^\dag$ Thus, a major straight-line
has a length expressible as $\sqrt{[1+k/(1+k^2)^{1/2}]/2} + \sqrt{[1-k/(1+k^2)^{1/2}]/2}$. The major and the corresponding minor, whose length is expressible as $\sqrt{[1+k/(1+k^2)^{1/2}]/2} - \sqrt{[1-k/(1+k^2)^{1/2}]/2}$ (see Prop.~10.76), are
the positive roots of the quartic $x^4-2\,x^2+ k^2/(1+k^2) = 0$.}

%%%%%%
% Prop 10.40
%%%%%%
\pdfbookmark[1]{Proposition 10.40}{pdf10.40}
\begin{Parallel}{}{} 
\ParallelLText{
\begin{center}
{\large\ggn{40}.}
\end{center}\vspace*{-7pt}

\gr{>E`an d'uo e>uje~iai dun'amei >as'ummetroi suntej~wsi poio~us\-ai t`o m`en
sugke'imenon >ek t~wn >ap> a>ut~wn tetrag'wnwn m'eson, t`o d> <up>
a>ut~wn <rht'on, <h <'olh e>uje~ia >'alog'oc >estin, kale'isjw d`e
<rht`on ka`i m'eson dunam'enh.}\\~\\

\epsfysize=0.3in
\centerline{\epsffile{Book10/fig036g.eps}}

\gr{Sugke'isjwsan g`ar d'uo e>uje~iai dun'amei >as'ummetroi a<i AB, BG
poio~usai t`a proke'imena; l'egw, <'oti >'alog'oc >estin <h AG.}

\gr{>Epe`i g`ar t`o sugke'imenon >ek t~wn >ap`o t~wn AB, BG m'eson
>est'in, t`o d`e d`ic <up`o t~wn AB, BG <rht'on, >as'ummetron
>'ara >est`i t`o sugke'imenon >ek t~wn >ap`o t~wn AB, BG t~w| d`ic
<up`o t~wn AB, BG; <'wste ka`i t`o <ap`o t~hc AG >as'ummetr'on >esti
t~w| d`ic <up`o t~wn AB, BG. <rht`on d`e t`o d`ic <up`o t~wn AB, BG;
>'alogon >'ara t`o >ap`o t~hc AG. >'alogoc >'ara <h AG, kale'isjw d`e <rht`on
ka`i m'eson dunam'enh. <'oper >'edei de~ixai.}}

\ParallelRText{
\begin{center}
{\large Proposition 40}
\end{center}

If two straight-lines (which are) incommensurable
in square, making the sum of the squares on them medial, 
and the (rectangle contained) by them rational, are added together then the whole straight-line
is irrational---let it be called the square-root of a rational plus a medial (area).

\epsfysize=0.3in
\centerline{\epsffile{Book10/fig036e.eps}}

For let the two straight-lines, $AB$ and $BC$, incommensurable
in square, (and) fulfilling the prescribed (conditions), be laid down together [Prop. 10.34]. I say that $AC$ is irrational.

For since the sum of the (squares) on $AB$ and $BC$ is medial,
and twice the (rectangle contained) by $AB$ and $BC$ (is) rational, 
the sum of the (squares) on $AB$ and $BC$ is thus incommensurable
with twice the (rectangle contained) by $AB$ and $BC$. Hence, the (square)
on $AC$ is also incommensurable with twice the (rectangle contained)
by $AB$ and $BC$ [Prop. 10.16]. And twice the
(rectangle contained) by $AB$ and $BC$ (is) rational. The (square) on
$AC$ (is) thus irrational. Thus, $AC$ (is) irrational [Def. 10.4]---let it be called
the square-root of a rational plus a medial (area).$^\dag$
(Which is) the very thing it
was required to show.}
\end{Parallel}

% note: hard line breaks in this footnote so that the formula's won't overflow into the margin


\vspace{7pt}{\footnotesize\noindent$^\dag$ Thus, the square-root
of a rational plus a medial (area)
has a length expressible as $\sqrt{[(1+k^2)^{1/2}+k]/[2\,(1+k^2)]}$ \\
$+\sqrt{[(1+k^2)^{1/2}-k]/[2\,(1+k^2)]}$. 
This  and the corresponding irrational with a minus sign, whose length is expressible as\\
 $\sqrt{[(1+k^2)^{1/2}+k]/[2\,(1+k^2)]}-\sqrt{[(1+k^2)^{1/2}-k]/[2\,(1+k^2)]}$
 (see Prop.~10.77), are the positive roots of the quartic $x^4-(2/\sqrt{1+k^2})\,x^2+ k^2/(1+k^2)^2 = 0$.} 
 
%%%%%%
% Prop 10.41
%%%%%%
\pdfbookmark[1]{Proposition 10.41}{pdf10.41}
\begin{Parallel}{}{} 
\ParallelLText{
\begin{center}
{\large\ggn{41}.}
\end{center}\vspace*{-7pt}

\gr{>E`an d'uo e>uje~iai dun'amei >as'ummetroi suntej~wsi poio~us\-ai t'o
te sugke'imenon >ek t~wn >ap> a>ut~wn tetrag'wnwn m'eson ka`i t`o
<up> a>ut~wn m'eson ka`i >'eti >as'ummetron t~w| sugkeim'enw|
>ek t~wn >ap> a>ut~wn tetrag'wnwn, <h <'olh e>uje~ia >'alog'oc
>estin, kale'isjw d`e d'uo m'esa dunam'enh.}\\~\\

\epsfysize=2.2in
\centerline{\epsffile{Book10/fig041g.eps}}

\gr{Sugke'isjwsan g`ar d'uo e>uje~iai dun'amei >as'ummetroi
a<i AB, BG poio~usai t`a proke'imena; l'egw, <'oti <h AG >'alog'oc
>estin.}

\gr{>Ekke'isjw <rht`h <h DE, ka`i parabebl'hsjw par`a t`hn DE to~ic m`en >ap`o
t~wn AB, BG >'ison t`o DZ, t~w| d`e d`ic <up`o t~wn AB, BG >'ison
t`o HJ; <'olon >'ara t`o DJ >'ison >est`i t~w| >ap`o t~hc AG tetrag'wnw|.
ka`i >epe`i m'eson >est`i t`o sugke'imenon >ek t~wn >ap`o t~wn
AB, BG, ka'i >estin >'ison t~w| DZ, m'eson >'ara >est`i ka`i t`o DZ.
ka`i par`a <rht`hn t`hn DE par'akeitai; <rht`h >'ara >est`in <h
DH ka`i >as'ummetroc t~h| DE m'hkei. di`a t`a a>ut`a d`h
ka`i <h HK <rht'h >esti ka`i >as'ummetroc t~h| HZ, tout'esti t~h| DE, m'hkei.
ka`i >epe`i >as'ummetr'a >esti t`a >ap`o t~wn AB, BG t~w| d`ic
<up`o t~wn AB, BG, >as'ummetr'on >esti t`o DZ t~w| HJ; <'wste ka`i <h
DH t~h| HK >as'ummetr'oc >estin. ka`i e>isi <rhta'i; a<i DH, HK
>'ara <rhta'i e>isi dun'amei m'onon s'ummetroi;
>'alogoc >'ara >est`in <h DK <h kaloum'enh >ek d'uo >onom'atwn.
<rht`h d`e <h DE; >'alogon >'ara >est`i t`o DJ ka`i <h dunam'enh
a>ut`o >'alog'oc >estin. d'unatai d`e t`o JD <h AG; >'alogoc >'ara
>est`in <h AG, kale'isjw d`e d'uo m'esa dunam'enh. <'oper >'edei de~ixai.}}

\ParallelRText{
\begin{center}
{\large Proposition 41}
\end{center}

If two straight-lines (which are) incommensurable
in square, making the sum of the squares
on them medial, and the (rectangle contained) by them medial, and,
moreover, incommensurable with the sum of the squares on them, are added together then
the whole straight-line is irrational---let it be called the square-root
of (the sum of) two medial (areas).

\epsfysize=2.2in
\centerline{\epsffile{Book10/fig041e.eps}}

For let the two straight-lines, $AB$ and $BC$,  incommensurable
in square,  (and) fulfilling the prescribed (conditions), be laid down together
[Prop. 10.35]. I say that $AC$ is irrational.

Let the rational (straight-line) $DE$ be laid out, and let (the rectangle) $DF$,
equal to (the sum of) the (squares) on $AB$ and $BC$, and (the rectangle) $GH$, equal to twice the (rectangle contained) by
$AB$ and $BC$, have been applied
to $DE$. Thus, the whole of $DH$ is equal to the square on $AC$
[Prop. 2.4]. And since the sum of the (squares)
on $AB$ and $BC$ is medial, and is equal to $DF$, $DF$ is thus also
medial. And it is applied to the rational (straight-line) $DE$. 
Thus, $DG$ is rational, and incommensurable in length with $DE$ [Prop. 10.22]. So, for the same (reasons), $GK$
is also rational, and incommensurable in length with $GF$---that is
to say, $DE$. And since (the sum of) the (squares) on $AB$ and $BC$
is incommensurable with twice the (rectangle contained) by $AB$ and
$BC$, $DF$ is incommensurable with $GH$.
Hence, $DG$ is also incommensurable (in length)
with $GK$ [Props.~6.1, 10.11]. And they are rational. Thus, $DG$ and
$GK$ are rational (straight-lines which are) commensurable in square only. Thus, $DK$ is irrational, and that (straight-line which is) called binomial 
[Prop. 10.36]. And $DE$ (is) rational.
Thus, $DH$ is irrational, and its square-root is irrational [Def. 10.4]. And $AC$ (is) the square-root of $HD$. Thus,
$AC$ is irrational---let it be called the square-root of (the sum of) two medial
(areas).$^\dag$ (Which is) the very thing it was required to show.}
\end{Parallel}


\vspace{7pt}{\footnotesize\noindent$^\dag$ Thus, the square-root
of (the sum of) two medial (areas)
has a length expressible as ${k'}^{1/4}\left(\sqrt{[1+k/(1+k^2)^{1/2}]/2}
+\sqrt{[1-k/(1+k^2)^{1/2}]/2}\right)$. This  and the corresponding irrational with a minus sign, whose length is expressible as ${k'}^{1/4}\left(\sqrt{[1+k/(1+k^2)^{1/2}]/2}
-\sqrt{[1-k/(1+k^2)^{1/2}]/2}\right)$
 (see Prop.~10.78), are
the positive roots of the quartic $x^4-2\,{k'}^{1/2}\,x^2+ k'\, k^2/(1+k^2)= 0$.} 

%%%%%%
% Prop 10.41a
%%%%%%
\begin{Parallel}{}{} 
\ParallelLText{
\begin{center}
{\large \gr{L~hmma}.}
\end{center}\vspace*{-7pt}

\gr{<'Oti d`e a<i e>irhm'enai >'alogoi monaq~wc diairo~untai e>ic
t`ac e>uje'iac, >ex <~wn s'ugkeintai poious~wn t`a proke'imena
e>'idh, de'ixomen >'hdh proekj'emenoi lhmm'ation toio~uton;}\\~\\

\epsfysize=0.3in
\centerline{\epsffile{Book10/fig041ag.eps}}

\gr{>Ekke'isjw e>uje~ia <h AB ka`i tetm'hsjw <h <'olh e>ic >'anisa
kaj> <ek'ateron t~wn G, D, <upoke'isjw d`e me'izwn <h AG t~hc DB;
l'egw, <'oti t`a >ap`o t~wn AG, GB me'izon'a >esti t~wn >ap`o
t~wn AD, DB.}

\gr{Tetm'hsjw g`ar <h AB d'iqa kat`a t`o E. ka`i >epe`i me'izwn >est`in
<h AG t~hc DB, koin`h >afh|r'hsjw <h DG; loip`h >'ara <h AD loip~hc
t~hc GB me'izwn >est'in. >'ish d`e <h AE t~h| EB; >el'attwn
>'ara <h DE t~hc EG; t`a G, D >'ara shme~ia o>uk >'ison
>ap'eqousi t~hc diqotom'iac. ka`i >epe`i t`o <up`o t~wn AG, GB
met`a to~u >ap`o t~hc EG >'ison >est`i t~w| >ap`o t~hc EB, >all`a
m`hn ka`i t`o <up`o t~wn AD, DB met`a to~u >ap`o DE >'ison
>est`i t~w| >ap`o t~hc EB, t`o >'ara <up`o t~wn
AG, GB met`a to~u >ap`o t~hc EG >'ison >est`i t~w| <up`o t~wn AD, DB met`a to~u >ap`o t~hc DE; <~wn t`o >ap`o t~hc DE
>'elass'on >esti to~u >ap`o t~hc EG; ka`i loip`on >'ara t`o <up`o
t~wn AG, GB >'elass'on >esti to~u <up`o t~wn AD, DB. <'wste
ka`i t`o d`ic <up`o t~wn AG, GB >'elass'on >esti to~u d`ic <up`o t~wn
AD, DB. ka`i loip`on >'ara t`o sugke'imenon >ek t~wn >ap`o t~wn
AG, GB me~iz'on >esti to~u sugkeim'enou >ek t~wn >ap`o t~wn
AD, DB. <'oper >'edei de~ixai.}}

\ParallelRText{
\begin{center}
{\large Lemma}
\end{center}

We will now demonstrate that the aforementioned irrational (straight-lines)
 are uniquely divided into the straight-lines  of which they are the sum,
 and which produce the prescribed types, 
 (after) setting forth the following
 lemma.
 
 \epsfysize=0.3in
\centerline{\epsffile{Book10/fig041ae.eps}}
 
 Let the straight-line $AB$ be laid out, and let the whole (straight-line) have been cut
 into unequal parts at each of the (points) $C$ and $D$. And let $AC$ be assumed (to be) greater than $DB$. I say that (the sum of) the (squares) on
 $AC$ and $CB$ is greater than (the sum of) the (squares) on $AD$ and $DB$.
 
 For let $AB$ have been cut in half at $E$. And since $AC$ is greater than
 $DB$, let $DC$ have been subtracted from both. Thus, the remainder
 $AD$ is greater than the remainder $CB$. And $AE$ (is) equal to $EB$.
 Thus, $DE$ (is) less than $EC$. Thus, points $C$ and $D$ are not
 equally far from the point of bisection. And since the (rectangle
 contained) by $AC$ and $CB$, plus the (square) on $EC$,
 is equal to the (square) on $EB$ [Prop. 2.5], 
 but, moreover, the (rectangle contained) by $AD$ and $DB$, plus the
 (square) on $DE$, is also equal to the (square) on $EB$ [Prop. 2.5], the (rectangle contained) by $AC$ and
 $CB$, plus the (square) on $EC$, is thus equal to the (rectangle contained)
 by $AD$ and $DB$, plus the (square) on $DE$. And, of these, the
 (square) on $DE$ is less than the (square) on $EC$. And, thus, the remaining
 (rectangle contained) by $AC$ and $CB$ is  less than
 the (rectangle contained) by $AD$ and $DB$. And, hence, twice
 the (rectangle contained) by $AC$ and $CB$ is less than twice the
 (rectangle contained) by $AD$ and $DB$. And thus the remaining
 sum of the (squares) on $AC$ and $CB$ is greater than the sum of the
 (squares) on $AD$ and $DB$.$^\dag$  (Which is) the very thing it was required to show.}
\end{Parallel}


\vspace{7pt}{\footnotesize\noindent$^\dag$ Since, $AC^{\,2}+CB^{\,2}+2\,AC\,CB=
 AD^{\,2}+DB^{\,2}
+2\,AD\,DB = AB^{\,2}$.} 

%%%%%%
% Prop 10.42
%%%%%%
\pdfbookmark[1]{Proposition 10.42}{pdf10.42}
\begin{Parallel}{}{} 
\ParallelLText{
\begin{center}
{\large \ggn{42}.}
\end{center}\vspace*{-7pt}

\gr{<H >ek d'uo >onom'atwn kat`a <`en m'onon shme~ion diaire~itai e>ic
t`a >on'omata.}

\epsfysize=0.3in
\centerline{\epsffile{Book10/fig042g.eps}}

\gr{>'Estw >ek d'uo >onom'atwn <h AB dih|rhm'enh e>ic t`a >on'omata
kat`a t`o G; a<i AG, GB >'ara <rhta'i e>isi dun'amei m'onon s'ummetroi.
l'egw, <'oti <h AB kat> >'allo shme~ion o>u diaire~itai e>ic d'uo
<rht`ac dun'amei m'onon summ'etrouc.}

\gr{E>i g`ar dunat'on, dih|r'hsjw ka`i kat`a t`o D, <'wste ka`i t`ac AD,
DB <rht`ac e>~inai dun'amei m'onon summ'etrouc. faner`on d'h,
<'oti <h AG t~h| DB o>uk >'estin <h a>ut'h. e>i g`ar dunat'on,
>'estw. >'estai d`h ka`i <h AD t~h| GB <h a>ut'h; ka`i >'estai
<wc <h AG pr`oc t`hn GB, o<'utwc <h BD pr`oc t`hn DA,
ka`i >'estai <h AB kat`a t`o a>ut`o t~h| kat`a t`o G diair'esei diaireje~isa
ka`i kat`a t`o D; <'oper o>uq <up'okeitai. o>uk >'ara <h AG t~h|
DB >estin <h a>ut'h. di`a d`h to~uto ka`i t`a G, D shme~ia
o>uk >'ison >ap'eqousi t~hc diqotom'iac. <~w| >'ara diaf'erei t`a
>ap`o t~wn AG, GB t~wn >ap`o t~wn AD, DB, to'utw| diaf'erei ka`i
t`o d`ic <up`o t~wn AD, DB to~u d`ic <up`o t~wn AG, GB
di`a t`o ka`i t`a >ap`o t~wn AG, GB met`a to~u d`ic <up`o t~wn
AG, GB ka`i t`a >ap`o t~wn AD, DB met`a to~u d`ic <up`o t~wn
AD, DB >'isa e>~inai t~w| >ap`o t~hc AB. >all`a t`a >ap`o
t~wn AG, GB t~wn >ap`o t~wn AD, DB diaf'erei <rht~w|;
<rht`a g`ar >amf'otera; ka`i t`o d`ic >'ara <up`o t~wn AD, DB
to~u d`ic <up`o t~wn AG, GB diaf'erei <rht~w| m'esa
>'onta; <'oper >'atopon; m'eson g`ar m'esou o>uq <uper'eqei
<rht~w|.}

\gr{O>uq >'ara <h >ek d'uo >onom'atwn kat> >'allo ka`i >'allo
shme~ion diaire~itai; kaj> <`en >'ara m'onon; <'oper
>'edei de~ixai.}}

\ParallelRText{
\begin{center}
{\large Proposition 42}
\end{center}
A binomial (straight-line) can be divided into
its (component) terms at one
point only.$^\dag$

\epsfysize=0.3in
\centerline{\epsffile{Book10/fig042e.eps}}

Let $AB$ be a binomial (straight-line) which has been divided into its
(component) terms at $C$. $AC$ and $CB$ are thus
rational (straight-lines which are) commensurable in square only [Prop. 10.36]. I say that $AB$ cannot be divided
at another point
into two rational (straight-lines which are) commensurable in square only.

For, if possible, let it also have been divided at $D$, such that $AD$ and
$DB$ are also rational (straight-lines which are) commensurable in
square only. So, (it is) clear that $AC$ is not the same as $DB$.
For, if possible, let it be (the same). So, $AD$ will also be the same
as $CB$. And as $AC$ will be to $CB$, so $BD$ (will be) to $DA$.
And $AB$ will (thus) also be divided at $D$ in the same (manner) as the division at $C$. The very opposite was assumed. Thus, $AC$ is not the
same as $DB$. So, on account of this,  points $C$ and $D$ are
not equally far from the point of bisection. Thus, by whatever (amount the sum of) the (squares) on $AC$ and $CB$ differs from (the sum of) the
(squares) on $AD$ and $DB$, twice the (rectangle contained)
by $AD$ and $DB$ also differs from twice the (rectangle contained)
by $AC$ and $CB$ by this (same amount)---on account of both (the sum of) the (squares) on $AC$ and
$CB$, plus twice the (rectangle contained) by $AC$ and $CB$, and
(the sum of) the (squares) on $AD$ and $DB$, plus twice the
(rectangle contained) by $AD$ and $DB$,  being equal to the (square) on $AB$ [Prop. 2.4]. But, (the sum of) the
(squares) on  $AC$ and $CB$ differs from (the sum of) the (squares)
on $AD$ and $DB$ by a rational (area). For (they are) both rational (areas).
Thus, twice the (rectangle contained) by $AD$ and $DB$ also differs from
twice the (rectangle contained) by $AC$ and $CB$ by a
rational (area, despite both) being medial (areas) [Prop. 10.21]. The very thing is absurd. For a
medial (area) cannot exceed a medial (area) by a rational (area) [Prop. 10.26].

Thus, a binomial (straight-line) cannot be divided (into its component terms) at different points.
Thus, (it can be so divided) at one point only. (Which is) the very thing it
was required to show.}
\end{Parallel}


\vspace{7pt}{\footnotesize\noindent$^\dag$ In other words, $k + {k'}^{1/2} = k'' + {k'''}^{1/2}$
has only one solution: {\em i.e.}, $k''=k$ and ${k'''}=k'$. Likewise,
$k^{1/2} + {k'}^{1/2} = {k''}^{1/2}+ {k'''}^{1/2}$ has only one solution:
{\em i.e.}, $k''=k$ and $k'''=k'$ (or, equivalently, $k''=k'$ and $k'''=k$).}

%%%%%%
% Prop 10.43
%%%%%%
\pdfbookmark[1]{Proposition 10.43}{pdf10.43}
\begin{Parallel}{}{} 
\ParallelLText{
\begin{center}
{\large \ggn{43}.}
\end{center}\vspace*{-7pt}

\gr{<H >ek d'uo m'eswn pr'wth kaj> <`en m'onon shme~ion diaire~itai.}\\

\epsfysize=0.3in
\centerline{\epsffile{Book10/fig042g.eps}}

\gr{>'Estw >ek d'uo m'eswn pr'wth <h AB dih|rhm'enh kat`a t`o G, <'wste
t`ac AG, GB m'esac e>~inai dun'amei m'onon summ'etrouc <rht`on
perieqo'usac; l'egw, <'oti <h AB kat> >'allo shme~ion o>u
diaire~itai.}

\gr{E>i g`ar dunat'on dih|r'hsjw ka`i kat`a t`o D, <'wste ka`i t`ac AD, DB m'esac
e>~inai dun'amei m'onon summ'etrouc <rht`on perieqo'usac. >epe`i
o>~un, <~w| diaf'erei t`o d`ic <up`o t~wn AD, DB to~u d`ic
<up`o t~wn AG, GB, to'utw| diaf'erei t`a >ap`o t~wn AG, GB t~wn >ap`o
t~wn AD, DB, <rht~w| d`e diaf'erei t`o d`ic <up`o t~wn AD, DB
to~u d`ic <up`o t~wn AG, GB; <rht`a g`ar >amf'otera; <rht~w|
>'ara diaf'erei ka`i t`a >ap`o t~wn AG, GB t~wn >ap`o t~wn AD, DB m'esa
>'onta; <'oper >'atopon.}

\gr{O>uk >'ara <h >ek d'uo m'eswn pr'wth kat> >'allo ka`i >'allo
shme~ion diaire~itai e>ic t`a >on'omata; kaj> <`en >'ara
m'onon; <'oper >'edei de~ixai.}}

\ParallelRText{
\begin{center}
{\large Proposition 43}
\end{center}

A first bimedial (straight-line) can be divided (into its component terms) at
one point only.$^\dag$

\epsfysize=0.3in
\centerline{\epsffile{Book10/fig042e.eps}}

Let $AB$ be a first bimedial (straight-line) which has been divided at $C$, such
that $AC$ and $CB$ are medial (straight-lines), commensurable
in square only, (and) containing a rational (area) [Prop. 10.37].  I say that $AB$ cannot be (so) divided
at another point.

For, if possible, let it also have been divided at $D$, such that $AD$ and
$DB$ are also medial (straight-lines), commensurable in square only, (and)
containing a rational (area). Since, therefore, by whatever (amount) twice the
(rectangle contained) by $AD$ and $DB$ differs from twice the (rectangle
contained) by $AC$ and $CB$, (the sum of) the (squares)
on $AC$ and $CB$ differs from (the sum of) the (squares) on $AD$ and
$DB$ by this (same amount) [Prop. 10.41~lem.]. And twice the (rectangle
contained) by  $AD$ and $DB$ differs from twice the (rectangle contained)
by $AC$ and $CB$ by a rational (area). For (they are) both
rational (areas). (The sum of) the (squares) on $AC$ and $CB$
thus differs from (the sum of) the (squares) on $AD$ and $DB$ by a
rational (area, despite both) being medial (areas). The very thing is
absurd [Prop. 10.26].

Thus,  a first bimedial (straight-line) cannot be divided into its (component)
terms at different points. Thus, (it can be so divided) at one
point only.
(Which is) the very thing it was required to show.}
\end{Parallel}


\vspace{7pt}{\footnotesize\noindent$^\dag$ In other words, $k^{1/4} + k^{3/4} = {k'}^{1/4} + {k'}^{3/4}$
has only one solution: {\em i.e.}, $k'=k$.} 


%%%%%%
% Prop 10.44
%%%%%%
\pdfbookmark[1]{Proposition 10.44}{pdf10.44}
\begin{Parallel}{}{} 
\ParallelLText{
\begin{center}
{\large \ggn{44}.}
\end{center}\vspace*{-7pt}

\gr{<H >ek d'uo m'eswn deut'era kaj> <`en m'onon shme~ion diaire~itai.}

\gr{>'Estw >ek d'uo m'eswn deut'era <h AB dih|rhm'enh kat`a t`o G, <'wste
t`ac AG, GB m'esac e>~inai dun'amei m'onon summ'etrouc m'eson
perieqo'usac; faner`on d'h, <'oti t`o G o>uk >'esti kat`a t~hc diqotom'iac,
<'oti o>uk e>is`i m'hkei s'ummetroi. l'egw, <'oti <h AB kat> >'allo
shme~ion o>u diaire~itai.}\\~\\

\epsfysize=1.5in
\centerline{\epsffile{Book10/fig044g.eps}}

\gr{E>i g`ar dunat'on, dih|r'hsjw ka`i kat`a t`o D, <'wste t`hn AG t~h|
DB m`h e>~inai t`hn a>ut'hn, >all`a me'izona kaj> <up'ojesin t`hn AG;
d~hlon d'h, <'oti ka`i t`a >ap`o t~wn AD, DB, <wc >ep'anw >ede'ixamen,
>el'assona t~wn >ap`o t~wn AG, GB; ka`i t`ac AD, DB m'esac e>~inai
dun'amei m'onon summ'etrouc m'eson perieqo'usac. ka`i >ekke'isjw
<rht`h <h EZ, ka`i t~w| m`en >ap`o t~hc AB >'ison par`a t`hn EZ
parallhl'ogrammon >orjog'wnion parabebl'hsjw t`o EK, to~ic
d`e >ap`o t~wn AG, GB >'ison >afh|r'hsjw t`o EH; loip`on >'ara t`o JK
>'ison >est`i t~w| d`ic <up`o t~wn AG, GB. p'alin d`h to~ic >ap`o
t~wn AD, DB, <'aper >el'assona >ede'iqjh t~wn >ap`o t~wn AG, GB, >'ison
>afh|r'hsjw t`o EL; ka`i loip`on >'ara t`o MK >'ison t~w| d`ic <up`o t~wn
AD, DB. ka`i >epe`i m'esa >est`i t`a >ap`o t~wn AG, GB, m'eson >'ara [ka`i]
t`o EH. ka`i par`a <rht`hn t`hn EZ par'akeitai; <rht`h >'ara >est`in
<h EJ ka`i >as'ummetroc t~h| EZ m'hkei. di`a t`a a>ut`a d`h  ka`i <h JN
<rht'h >esti ka`i >as'ummetroc t~h| EZ m'hkei. ka`i >epe`i a<i
AG, GB m'esai e>is`i dun'amei m'onon s'ummetroi, >as'ummetroc >'ara
>est`in <h AG t~h| GB m'hkei. <wc d`e <h AG pr`oc t`hn GB, o<'utwc
t`o >ap`o t~hc AG pr`oc t`o <up`o t~wn AG, GB; >as'ummetron
>'ara >est`i t`o >ap`o t~hc AG t~w| <up`o t~wn AG, GB. >all`a
t~w| m`en >ap`o t~hc AG s'ummetr'a >esti t`a >ap`o t~wn AG, GB; dun'amei
g'ar e>isi s'ummetroi a<i AG, GB. t~w| d`e <up`o t~wn AG, GB s'ummetr'on
>esti t`o d`ic <up`o t~wn AG, GB. ka`i t`a >ap`o t~wn AG, GB >'ara >as'ummetr'a >esti t~w| d`ic <up`o t~wn AG, GB. >all`a to~ic m`en >ap`o
t~wn AG, GB >'ison >est`i t`o EH, t~w| d`e d`ic <up`o t~wn AG, GB
>'ison t`o JK; >as'ummetron >'ara >est`i t`o EH t~w| JK; <'wste ka`i <h
EJ t~h| JN >as'ummetr'oc >esti m'hkei. ka'i e>isi <rhta'i; a<i EJ, JN
>'ara <rhta'i e>isi dun'amei m'onon s'ummetroi.
>e`an d`e d'uo <rhta`i dun'amei m'onon s'ummetroi
 suntej~wsin, <h <'olh 
>'alog'oc >estin <h kaloum'enh >ek d'uo >onom'atwn;
<h EN >'ara >ek d'uo >onom'atwn >est`i dih|rhm'enh kat`a t`o
J. kat`a t`a a>ut`a d`h deiqj'hsontai ka`i a<i EM, MN
<rhta`i dun'amei m'onon s'ummetroi; ka`i >'estai <h EN >ek d'uo
>onom'atwn kat> >'allo ka`i >'allo dih|rhm'enh t'o te J ka`i t`o M, ka`i
o>uk >'estin <h EJ t~h| MN <h a>ut'h, <'oti
t`a >ap`o t~wn AG, GB me'izon'a >esti t~wn >ap`o t~wn AD, DB.
>all`a t`a >ap`o t~wn AD, DB me'izon'a >esti to~u d`ic <up`o
 AD, DB; poll~w| >'ara ka`i t`a >ap`o t~wn AG, GB,
 tout'esti t`o EH, me~iz'on >esti to~u d`ic <up`o t~wn AD, DB, tout'esti
 to~u 
  MK; <'wste ka`i <h EJ t~hc MN
me'izwn >est'in. <h >'ara EJ t~h| MN o>uk >'estin <h a>ut'h;
<'oper >'edei de~ixai.}}

\ParallelRText{
\begin{center}
{\large Proposition 44}
\end{center}

A second bimedial (straight-line)  can be divided
(into its component terms)
at one point only.$^\dag$

Let $AB$ be a second bimedial (straight-line) which has  been divided at $C$, so that $AC$ and $BC$ are medial (straight-lines), commensurable in square
only, (and) containing a medial (area) [Prop. 10.38].
So, (it is) clear that $C$ is not 
(located) at the point of bisection, since ($AC$ and $BC$)
are not commensurable in length. I say that $AB$ cannot be (so) divided at another point.

\epsfysize=1.5in
\centerline{\epsffile{Book10/fig044e.eps}}

For, if possible, let it also have  been (so) divided at $D$, so that $AC$ is not the
same as $DB$, but $AC$ (is), by hypothesis, greater. So, (it is)  clear that
(the sum of) the (squares) on $AD$ and $DB$ is also less than (the sum of)
the (squares) on $AC$ and $CB$, as we showed above [Prop. 10.41~lem.].  And $AD$ and $DB$ are medial (straight-lines), commensurable in square only, (and)
containing a medial (area). And let the rational (straight-line) $EF$
be laid down. And let the rectangular parallelogram $EK$, equal to the (square) on $AB$, have been applied to $EF$. And let $EG$, equal to
(the sum of) the (squares) on $AC$ and $CB$, have been cut  off (from $EK$).
Thus, the remainder, $HK$, is equal to twice the (rectangle contained)
by $AC$ and $CB$ [Prop. 2.4]. So, again,
let $EL$, equal to  (the sum of) the (squares) on  $AD$ and $DB$---which was shown (to be) less than (the sum of) the (squares) on $AC$ and
$CB$---have been cut off (from $EK$). And, thus, the remainder, $MK$,
(is) equal to twice the (rectangle contained) by $AD$ and $DB$. 
And since (the sum of) the (squares) on $AC$ and $CB$ is medial, $EG$ (is) thus [also]
medial. And it is applied to the rational (straight-line) $EF$. Thus, $EH$
is rational, and incommensurable in length with $EF$ [Prop. 10.22]. So, for the same (reasons), $HN$
is also rational, and incommensurable in length with $EF$. And since 
$AC$ and $CB$  are medial (straight-lines which are) commensurable
in square only, $AC$ is thus incommensurable in length with $CB$.
And as $AC$ (is) to $CB$, so the (square) on $AC$ (is) to the
(rectangle contained) by $AC$ and $CB$ [Prop. 10.21~lem.]. Thus, the (square) on $AC$ is
incommensurable with the (rectangle contained) by $AC$ and $CB$
[Prop. 10.11]. But, (the sum of) the (squares)
on $AC$ and $CB$ is commensurable with the (square) on $AC$.
For, $AC$ and $CB$ are commensurable in square [Prop. 10.15]. And
twice the (rectangle contained) by $AC$ and $CB$ is commensurable with the (rectangle contained) by $AC$ and $CB$ [Prop. 10.6]. And thus
(the sum of) the squares on $AC$ and $CB$ is incommensurable with twice
the (rectangle contained) by $AC$ and $CB$ [Prop. 10.13]. But, $EG$ is equal to (the sum
of) the (squares) on $AC$ and $CB$, and $HK$ equal to twice the (rectangle
contained) by $AC$ and $CB$. Thus, $EG$ is incommensurable
with $HK$. Hence, $EH$ is also incommensurable in length with  $HN$
[Props.~6.1, 10.11].
And (they are)  rational (straight-lines). Thus, $EH$ and $HN$ are rational
(straight-lines which are) commensurable in square only. And if two
rational (straight-lines which are) commensurable in square only are
added together then the whole (straight-line) is that irrational called
binomial [Prop. 10.36]. Thus, $EN$
is a binomial (straight-line) which has been divided (into its component terms) at $H$. So, according to
the same (reasoning), $EM$ and $MN$ can be shown (to be) rational
(straight-lines which are) commensurable in square only. And $EN$
will (thus) be a binomial (straight-line) which has been divided (into its component terms) at the different (points)
$H$ and $M$ (which is absurd [Prop. 10.42]).
And $EH$ is not the same as $MN$, since (the sum of) the
(squares) on $AC$ and $CB$ is greater than (the sum of) the (squares) on
$AD$ and $DB$. But, (the sum of) the (squares) on $AD$ and $DB$
is greater than twice the (rectangle contained) by $AD$ and $DB$ [Prop. 10.59~lem.]. Thus,  (the sum of) the (squares) on $AC$ and $CB$---that is to say, $EG$---is also much greater than twice the (rectangle contained) by 
$AD$ and $DB$---that is to say, $MK$. Hence, $EH$ is also greater
than $MN$ [Prop. 6.1]. Thus, $EH$ is not the same as $MN$. (Which is) the very thing
it was required to show.}
\end{Parallel}


\vspace{7pt}{\footnotesize\noindent$^\dag$ In other words, $k^{1/4}+{k'}^{1/2}/k^{1/4}
= {k''}^{1/4}+{k'''}^{1/2}/{k''}^{1/4}$
has only one solution: {\em i.e.}, $k''=k$ and $k'''=k'$.}

%%%%%%
% Prop 10.45
%%%%%%
\pdfbookmark[1]{Proposition 10.45}{pdf10.45}
\begin{Parallel}{}{} 
\ParallelLText{\
\begin{center}
{\large\ggn{45}.}
\end{center}\vspace*{-7pt}

\gr{<H me'izwn kat`a t`o a>ut`o m'onon shme~ion diaire~itai.}\\

\epsfysize=0.3in
\centerline{\epsffile{Book10/fig042g.eps}}

\gr{>'Estw me'izwn <h AB dih|rhm'enh kat`a t`o G, <'wste t`ac AG, GB
dun'amei >asumm'etrouc e>~inai poio'usac t`o m`en sugke'imenon
>ek t~wn >ap`o t~wn AG, GB tetrag'wnwn <rht'on, t`o d> <up`o
t~wn AG, GB m'eson; l'egw, <'oti <h AB kat> >'allo shme~ion o>u
diaire~itai.}

\gr{E>i g`ar dunat'on, dih|r'hsjw ka`i kat`a t`o D, <'wste ka`i t`ac AD, DB dun'amei
>asumm'etrouc e>~inai poio'usac t`o m`en sugke'imenon >ek t~wn
>ap`o t~wn AD, DB <rht'on, t`o d> <up> a>ut~wn m'eson. ka`i >epe'i,
<~w| diaf'erei t`a >ap`o t~wn AG, GB t~wn >ap`o t~wn AD, DB,
to'utw| diaf'erei ka`i t`o d`ic <up`o t~wn AD, DB to~u d`ic <up`o
t~wn AG, GB, >all`a t`a >ap`o t~wn AG, GB t~wn >ap`o t~wn
AD, DB <uper'eqei <rht~w|; <rht`a g`ar >amf'otera; ka`i t`o d`ic <up`o
t~wn AD, DB >'ara to~u d`ic <up`o t~wn
AG, GB <uper'eqei <rht~w| m'esa >'onta; <'oper >est`in
>ad'unaton. o>uk >'ara <h me'izwn kat> >'allo ka`i >'allo
shme~ion diaire~itai; kat`a t`o a>ut`o >'ara m'onon diaire~itai;
<'oper >'edei de~ixai.}}

\ParallelRText{
\begin{center}
{\large Proposition 45}
\end{center}

A major (straight-line) can only  be divided (into
its component terms) at the same
point.$^\dag$

\epsfysize=0.3in
\centerline{\epsffile{Book10/fig042e.eps}}

Let $AB$ be a major (straight-line) which has been divided at $C$, so
that $AC$ and $CB$ are incommensurable in square, making the sum
of the squares on $AC$ and $CB$ rational, and the (rectangle contained)
by $AC$ and $CD$ medial [Prop. 10.39].
I say that $AB$ cannot be (so) divided at another point.

For, if possible, let it also have been divided at $D$, such that $AD$ and
$DB$ are also incommensurable in square, making the
sum of the (squares) on $AD$ and $DB$ rational, and the (rectangle contained) by them medial. And since, by whatever (amount the sum of)
the (squares) on $AC$ and $CB$ differs from (the sum of) the (squares)
on $AD$ and $DB$, 
twice the (rectangle contained) by $AD$ and $DB$ also differs from
twice the (rectangle contained) by $AC$ and $CB$ by this (same amount). But, (the sum of) the
(squares) on $AC$ and $CB$ exceeds (the sum of) the (squares) on
$AD$ and $DB$ by a rational (area). For (they are) both rational (areas). Thus, twice the
(rectangle contained) by $AD$ and $DB$ also exceeds twice the
(rectangle contained) by $AC$ and $CB$ by a rational (area), (despite both) being medial (areas).
The very thing is impossible [Prop. 10.26].
Thus, a major (straight-line) cannot be divided (into its component terms) at different points. Thus,
it can only be (so) divided at the same (point). (Which is) the very thing it was required to show.}
\end{Parallel}


\vspace{7pt}{\footnotesize\noindent$^\dag$ In other words, $\sqrt{[1+k/(1+k^2)^{1/2}]/2} + \sqrt{[1-k/(1+k^2)^{1/2}]/2} =\sqrt{[1+k'/(1+{k'}^2)^{1/2}]/2} + \sqrt{[1-k'/(1+{k'}^2)^{1/2}]/2}$ has only one solution: {\em i.e.}, $k'=k$.}

%%%%%%
% Prop 10.46
%%%%%%
\pdfbookmark[1]{Proposition 10.46}{pdf10.46}
\begin{Parallel}{}{} 
\ParallelLText{
\begin{center}
{\large \ggn{46}.}
\end{center}\vspace*{-7pt}

\gr{<H <rht`on ka`i m'eson dunam'enh kaj> <`en m'onon shme~ion diaire~itai.}

\epsfysize=0.3in
\centerline{\epsffile{Book10/fig042g.eps}}

\gr{>'Estw <rht`on ka`i m'eson dunam'enh <h AB dih|rhm'enh kat`a t`o G, <'wste
t`ac AG, GB dun'amei >asumm'etrouc e>~inai poio'usac t`o m`en sugke'imenon >ek t~wn >ap`o t~wn AG, GB m'eson, t`o d`e d`ic <up`o
t~wn AG, GB <rht'on; l'egw, <'oti <h AB kat> >'allo shme~ion
o>u diaire~itai.}

\gr{E>i g`ar dunat'on, dih|r'hsjw ka`i kat`a t`o D, <'wste ka`i t`ac AD, DB
dun'amei >asumm'etrouc e>~inai poio'usac t`o m`en sugke'imenon
>ek t~wn >ap`o t~wn AD, DB m'eson, t`o d`e d`ic <up`o
t~wn AD, DB <rht'on. >epe`i o>~un, <~w| diaf'erei t`o d`ic <up`o
t~wn AG, GB to~u d`ic <up`o t~wn AD, DB, to'utw| diaf'erei ka`i
t`a >ap`o t~wn AD, DB t~wn >ap`o t~wn AG, GB, t`o d`e d`ic <up`o
t~wn AG, GB 
 to~u d`ic <up`o t~wn AD, DB <uper'eqei
<rht~w|, ka`i t`a >ap`o t~wn AD, DB >'ara t~wn >ap`o t~wn AG, GB
<uper'eqei <rht~w| m'esa >'onta; <'oper >est`in >ad'unaton. o>uk
>'ara <h <rht`on ka`i m'eson dunam'enh kat> >'allo
ka`i >'allo shme~ion diaire~itai. kat`a <`en >'ara shme~ion
diaire~itai; <'oper >'edei de~ixai.}}

\ParallelRText{
\begin{center}
{\large Proposition 46}
\end{center}
The square-root of a rational plus a  medial
(area) can be divided (into its component terms) at  one point only.$^\dag$

\epsfysize=0.3in
\centerline{\epsffile{Book10/fig042e.eps}}

Let $AB$ be the square-root of a rational plus a medial (area) which has
been divided at $C$, so that $AC$ and $CB$ are incommensurable
in square, making the sum of the (squares) on $AC$ and $CB$ medial,
and twice the (rectangle contained) by $AC$ and $CB$ rational
[Prop. 10.40]. I say that $AB$ cannot be (so)
divided at another point.

For, if possible, let it also have been divided at $D$, so that
$AD$ and $DB$ are also incommensurable in square, making the sum of 
the (squares) on $AD$ and $DB$ medial, and twice the (rectangle
contained) by $AD$ and $DB$ rational. Therefore, since by whatever
(amount) twice the (rectangle contained) by $AC$ and $CB$
differs from twice the (rectangle contained) by $AD$ and $DB$, (the sum of) the (squares) on  $AD$ and $DB$ also differs from (the sum of) the
(squares) on
$AC$ and $CB$ by this
(same amount). And twice the (rectangle contained) by $AC$ and $CB$
exceeds twice the (rectangle contained) by $AD$ and $DB$ by a
rational (area). (The sum of) the (squares)
on $AD$ and $DB$ thus also exceeds (the sum of) the (squares) on $AC$ and $CB$ by a rational (area),
(despite both) being medial (areas). The very thing is impossible [Prop. 10.26]. Thus, the square-root of a rational
plus a medial (area) cannot be divided (into its component terms) at different points. Thus, it
can  be (so) divided at one point (only). (Which is) the very thing it was required to
show.}
\end{Parallel}


\vspace{7pt}{\footnotesize\noindent$^\dag$ In other words, $\sqrt{[(1+k^2)^{1/2}+k]/[2\,(1+k^2)]} +\sqrt{[(1+k^2)^{1/2}-k]/[2\,(1+k^2)]}=\sqrt{[(1+{k'}^2)^{1/2}+k']/[2\,(1+{k'}^2)]}$\\$ +\sqrt{[(1+{k'}^2)^{1/2}-k']/[2\,(1+{k'}^2)]}$
has only one solution: {\em i.e.}, $k'=k$.}

%%%%%%
% Prop 10.47
%%%%%%
\pdfbookmark[1]{Proposition 10.47}{pdf10.47}
\begin{Parallel}{}{} 
\ParallelLText{
\begin{center}
{\large \ggn{47}.}
\end{center}\vspace*{-7pt}

\gr{<H d'uo m'esa dunam'enh kaj> <`en m'onon shme~ion diaire~itai.}\\~\\

\epsfysize=2in
\centerline{\epsffile{Book10/fig047g.eps}}

\gr{>'Estw [d'uo m'esa dunam'enh] <h AB dih|rhm'enh kat`a t`o G, <'wste t`ac
AG, GB dun'amei >asumm'etrouc e>~inai poio'usac t'o te sugke'imenon
>ek t~wn >ap`o t~wn AG, GB m'eson ka`i t`o <up`o t~wn AG, GB m'eson
ka`i >'eti >as'ummetron t~w| sugkeim'enw| >ek t~wn >ap> a>ut~wn.
l'egw, <'oti <h AB kat> >'allo shme~ion o>u diaire~itai poio~usa
t`a proke'imena.}

\gr{E>i g`ar dunat'on, dih|r'hsjw kat`a t`o D, <'wste p'alin dhlon'oti
t`hn AG t~h| DB m`h e>~inai t`hn a>ut'hn, >all`a me'izona kaj> <up'ojesin
t`hn AG, ka`i >ekke'isjw <rht`h <h EZ, ka`i parabebl'hsjw par`a t`hn EZ
to~ic m`en >ap`o t~wn AG, GB >'ison t`o EH, t~w| d`e d`ic <up`o t~wn
AG, GB >'ison t`o JK; <'olon >'ara t`o EK >'ison >est`i t~w| >ap`o
t~hc AB  tetrag'wnw|. p'alin d`h parabebl'hsjw par`a t`hn EZ to~ic >ap`o
t~wn AD, DB >'ison t`o EL; loip`on >'ara t`o d`ic <up`o t~wn AD, DB
loip~w| t~w| MK >'ison >est'in. ka`i >epe`i m'eson <up'okeitai t`o sugke'imenon >ek t~wn >ap`o t~wn AG, GB, m'eson >'ara >est`i ka`i
t`o EH. ka`i par`a <rht`hn t`hn EZ par'akeitai; <rht`h >'ara >est`in
<h JE ka`i >as'ummetroc t~h| EZ m'hkei. di`a t`a a>ut`a d`h ka`i <h JN
<rht'h >esti ka`i >as'ummetroc t~h| EZ m'hkei. ka`i >epe`i >as'ummetr'on
>esti t`o sugke'imenon >ek t~wn >ap`o t~wn AG, GB t~w| d`ic <up`o t~wn AG, GB, ka`i t`o EH >'ara t~w| HN >as'ummetr'on >estin; <'wste
ka`i <h EJ t~h| JN >as'ummetr'oc >estin. ka'i e>isi <rhta'i; a<i EJ,
JN >'ara <rhta'i e>isi dun'amei m'onon s'ummetroi; <h EN >'ara >ek
d'uo >onom'atwn >est`i dih|rhm'enh kat`a t`o J. <omo'iwc
d`h de'ixomen, <'oti ka`i kat`a t`o M di'h|rhtai. ka`i o>uk >'estin <h EJ
t~h| MN <h a>ut'h; <h >'ara >ek d'uo >onom'atwn kat> >'allo ka`i >'allo
shme~ion di'h|rhtai; <'oper >est'in >'atopon. o>uk >'ara <h d'uo
m'esa dunam'enh kat> >allo ka`i >'allo shme~ion diaire~itai;
kaj> <`en >'ara m'onon [shme~ion] diaire~itai.}}

\ParallelRText{
\begin{center}
{\large Proposition 47}
\end{center}

The square-root of (the sum of) two medial (areas) can be
divided (into its component terms) at one point only.$^\dag$

\epsfysize=2in
\centerline{\epsffile{Book10/fig047e.eps}}

Let $AB$ be [the square-root of (the sum of) two medial (areas)] which has been divided
at $C$, such that $AC$ and $CB$ are incommensurable in square, making the
sum of the (squares) on $AC$ and $CB$ medial, and the (rectangle contained) by $AC$ and $CB$ medial, and, moreover, incommensurable
with the sum of the (squares) on ($AC$ and $CB$) [Prop. 10.41]. I say that $AB$ cannot be divided
at another point fulfilling the prescribed (conditions).

For, if possible, let it have been divided at $D$, such that $AC$ is again
manifestly not the same as $DB$, but $AC$ (is), by hypothesis, greater.
And let the rational (straight-line) $EF$ be laid down. And let $EG$,
equal to (the sum of) the (squares) on $AC$ and $CB$, and $HK$, equal to twice the (rectangle contained) by $AC$ and $CB$,
have been applied to $EF$. Thus, the whole of $EK$ is equal to the
square on $AB$ [Prop. 2.4]. So, again, let $EL$, equal to (the sum of) the (squares)
on $AD$ and $DB$, have been applied to $EF$. Thus, the remainder---twice
the (rectangle contained) by $AD$ and $DB$---is equal to the remainder, $MK$. And since the sum of the (squares) on $AC$ and $CB$
was assumed (to be) medial, $EG$ is also medial. And it is applied
to the rational (straight-line) $EF$. $HE$ is thus rational, and incommensurable in length with $EF$ [Prop. 10.22].  So, for the same (reasons), 
$HN$ is also rational, and incommensurable in length with $EF$.
And since the sum of the (squares) on $AC$ and $CB$ is incommensurable
with twice the (rectangle contained) by $AC$ and $CB$, $EG$ is thus also
incommensurable with $GN$. Hence, $EH$ is also incommensurable with
$HN$ [Props.~6.1, 10.11].
And they are (both) rational (straight-lines). Thus, $EH$ and $HN$ are rational
(straight-lines which are) commensurable in  square only. Thus, $EN$
is a binomial (straight-line) which has been divided (into its component
terms) at $H$ [Prop. 10.36]. So, similarly, we can show that
it has also been (so) divided at $M$. And $EH$ is not the same  as $MN$. Thus,
a binomial (straight-line) has been divided (into its component
terms) at different points. The
very thing is absurd [Prop. 10.42]. Thus,
the square-root of (the sum of) two medial (areas) cannot be divided (into its component terms) at different
points. Thus, it can be (so) divided at one [point] only.}
\end{Parallel}


\vspace{7pt}{\footnotesize\noindent$^\dag$ In other words, ${k'}^{1/4}\sqrt{[1+k/(1+k^2)^{1/2}]/2}
+{k'}^{1/4}\sqrt{[1-k/(1+k^2)^{1/2}]/2}={k'''}^{1/4}\sqrt{[1+k''/(1+{k''}^2)^{1/2}]/2}$\\$
+{k'''}^{1/4}\sqrt{[1-k''/(1+{k''}^2)^{1/2}]/2}$ has only one solution: {\em i.e.}, $k''=k$ and $k'''=k'$.}

%%%%%%%%
% Definitions 2
%%%%%%%%
\pdfbookmark[1]{Definitions II}{def10.2}
\begin{Parallel}{}{} 
\ParallelLText{
\begin{center}
\large{\gr{<'Oroi de'uteroi}.}
\end{center}\vspace*{-7pt}

\ggn{5}.~\gr{<Upokeim'enhc <rht~hc ka`i t~hc >ek d'uo >onom'atwn
dih|rhm'enhc e>ic t`a >on'omata, <~hc t`o me~izon >'onoma
to~u >el'assonoc me~izon d'unatai t~w| >ap`o summ'etrou <eaut~h|
m'hkei, >e`an m`en t`o me~izon >'onoma s'ummetron >~h| m'hkei
t~h| >ekkeim'enh| <rht~h|, kale'isjw [<h <'olh] >ek d'uo >onom'atwn
pr'wth.}

\ggn{6}.~\gr{>E`an d`e t`o >el'asson >'onoma s'ummetron >~h| m'hkei t~h| >ekkeim'enh| <rht~h|, kale'isjw >ek d'uo >onom'atwn deut'era.}

\ggn{7}.~\gr{>E`an d`e mhd'eteron t~wn >onom'atwn s'ummetron >~h|
m'hkei t~h| >ekkeim'enh| <rht~h|, kale'isjw >ek d'uo >onom'atwn
tr'ith.}

\ggn{8}.~\gr{P'alin d`h >e`an t`o me~izon >'onoma [to~u >el'assonoc]
me~izon d'unhtai t~w| >ap`o >asumm'etrou <eaut~h| m'hkei,
>e`an m`en t`o me~izon >'onoma s'ummetron >~h| m'hkei t~h|
>ekkeim'enh| <rht~h|, kale'isjw >ek d'uo >onom'atwn tet'arth.}

\ggn{9}.~\gr{>E`an d`e t`o >'elasson, p'empth.}

\ggn{10}.~\gr{>E`an d`e mhd'eteron, <'ekth.}}

\ParallelRText{
\begin{center}
{\large Definitions II}
\end{center}

5.~Given a rational (straight-line), and  a binomial (straight-line) which has been divided into its (component) terms, of which the square on the greater term
is larger than (the square on) the lesser  by the (square) on (some straight-line) commensurable in length with (the greater) then,
if the greater term is commensurable in length with the  rational (straight-line previously) laid out,
 let [the whole] (straight-line) be called a first binomial (straight-line).
 
6.~And if the lesser term is commensurable in length with the  rational (straight-line previously) laid
out then let (the whole straight-line) be called a
second binomial (straight-line).

7.~And if neither of the terms is commensurable in length with the 
rational (straight-line previously) laid out then let (the whole straight-line) be called a third
binomial (straight-line).

8.~So, again, if the square on the greater term is larger than (the
square on) [the lesser] by the (square) on (some straight-line)
incommensurable in length with (the greater) then, if the
greater term is commensurable in length with the  rational
(straight-line previously) laid out, let (the whole straight-line) be called a fourth
binomial (straight-line).

9.~And if the lesser (term is commensurable), a fifth (binomial straight-line).

10.~And if neither (term is commensurable), a sixth (binomial straight-line).}
\end{Parallel}

%%%%
%10.48
%%%%
\pdfbookmark[1]{Proposition 10.48}{pdf10.48}
\begin{Parallel}{}{}
\ParallelLText{
\begin{center}
{\large\ggn{48}.}
\end{center}\vspace*{-7pt}

\gr{E<ure~in t`hn >ek d'uo >onom'atwn pr'wthn.}

\gr{>Ekke'isjwsan d'uo >arijmo`i o<i AG, GB, <'wste t`on sugke'imenon
>ex a>ut~wn t`on AB pr`oc m`en t`on BG l'ogon >'eqein,
<`on tetr'agwnoc >arijm`oc pr`oc tetr'agwnon >arijm'on, pr`oc d`e t`on
GA l'ogon m`h >'eqein, <`on tetr'agwnoc >arijm`oc pr`oc tetr'agwnon
>arijm'on, ka`i >ekke'isjw tic <rht`h <h D, ka`i t~h| D s'ummetroc
>'estw m'hkei <h EZ. <rht`h >'ara >est`i ka`i <h EZ. ka`i gegon'etw
<wc <o BA >arijm`oc pr`oc t`on AG, o<'utwc t`o >ap`o t~hc EZ
pr`oc t`o >ap`o t~hc ZH. <o d`e AB pr`oc t`on AG l'ogon >'eqei,
<`on >arijm`oc pr`oc >arijm'on; ka`i t`o >ap`o t~hc EZ >'ara pr`oc t`o
>ap`o t~hc ZH l'ogon >'eqei, <`on >arijm`oc pr`oc >arijm'on; <'wste
s'ummetr'on >esti t`o >ap`o t~hc EZ t~w| >ap`o t~hc ZH. ka`i >esti <rht`h
<h EZ; <rht`h >'ara ka`i <h ZH.
ka`i >epe`i
<o BA pr`oc t`on AG l'ogon o>uk >'eqei, <`on tetr'agwnoc >arijm`oc
pr`oc tetr'agwnon >arijm'on, o>ud`e t`o >ap`o t~hc EZ >'ara pr`oc t`o
>ap`o t~hc ZH l'ogon >'eqei, <`on tetr'agwnoc >arijm`oc pr`oc tetr'agwnon
>arijm'on; >as'ummetroc >'ara >est`in <h EZ t~h| ZH m'hkei. a<i EZ,
ZH >'ara <rhta'i e>isi dun'amei m'onon s'ummetroi; >ek d'uo >'ara
>onom'atwn >est`in <h EH.
l'egw, <'oti ka`i pr'wth.}\\~\\~\\~\\~\\~\\~\\

\epsfysize=1.in
\centerline{\epsffile{Book10/fig048g.eps}}

\gr{>Epe`i g'ar >estin <wc <o BA >arijm`oc pr`oc t`on AG, o<'utwc
t`o >ap`o t~hc EZ pr`oc t`o >ap`o t~hc ZH, me'izwn d`e <o BA to~u
AG, me~izon >'ara ka`i t`o >ap`o t~hc EZ to~u >ap`o t~hc ZH. >'estw
o>~un t~w| >ap`o t~hc EZ >'isa t`a >ap`o t~wn ZH, J. ka`i
>epe'i >estin <wc <o BA pr`oc t`on AG, o<'utwc t`o >ap`o t~hc
EZ pr`oc t`o >ap`o t~hc ZH, >anastr'eyanti >'ara >est`in <wc <o AB
pr`oc t`on BG, o<'utwc t`o >ap`o t~hc EZ pr`oc t`o >ap`o t~hc J. <o d`e
AB pr`oc t`on BG l'ogon >'eqei, <`on tetr'agwnoc >arijm`oc pr`oc tetr'agwnon >arijm'on. ka`i t`o >ap`o t~hc EZ >'ara pr`oc t`o >ap`o t~hc J
l'ogon >'eqei, <`on tetr'agwnoc >arijm`oc pr`oc tetr'agwnon >arijm'on.
s'ummetroc >'ara >est`in <h EZ t~h| J m'hkei; <h EZ
>'ara t~hc ZH me~izon d'unatai t~w| >ap`o summ'etrou <eaut~h|. ka'i
e>isi <rhta`i a<i EZ, ZH, ka`i s'ummetroc <h EZ t~h| D m'hkei.}

\gr{>H EH >'ara >ek d'uo >onom'atwn >est`i pr'wth; <'oper
>'edei de~ixai.}}

\ParallelRText{
\begin{center}
{\large Proposition 48}
\end{center}

To find a first binomial (straight-line).

Let  two numbers $AC$ and $CB$ be laid down such that their sum
$AB$ has to $BC$ the ratio  which (some) square number (has) to (some)
square number, and does not have to $CA$ the ratio which (some)
square number (has) to (some) square number [Prop. 10.28~lem.~I]. And let some rational (straight-line) $D$ be laid down. And let $EF$ be commensurable in length with $D$. $EF$ is thus also rational [Def. 10.3]. 
And let it have been contrived that as the number $BA$ (is) to $AC$, so the
(square) on $EF$ (is) to the (square) on $FG$ [Prop. 10.6~corr.]. And $AB$ has to $AC$ the ratio
which (some) number (has) to (some) number. Thus, the (square) on $EF$
also has to the (square) on $FG$ the ratio which (some) number
(has) to (some) number. Hence, the (square) on $EF$ is commensurable
with the (square) on $FG$ [Prop. 10.6].  And
$EF$ is rational. Thus, $FG$ (is) also rational. And since $BA$ does not
have to $AC$ the ratio which (some) square number (has) to (some)
square number, thus the (square) on $EF$ does not have to the (square)
on $FG$ the ratio which (some) square number (has) to (some) square number either. Thus, $EF$ is incommensurable in length with $FG$ [Prop~10.9]. $EF$ and $FG$ are thus rational
(straight-lines which are) commensurable in square only. Thus, $EG$
is a binomial (straight-line) [Prop. 10.36].
I say that (it is) also a first (binomial straight-line).

\epsfysize=1.in
\centerline{\epsffile{Book10/fig048e.eps}}

For since as the number $BA$ is to $AC$, so the (square) on $EF$ (is)
to the (square) on $FG$, and $BA$ (is) greater than $AC$, the
(square) on $EF$ (is) thus also greater than the (square) on $FG$ [Prop. 5.14]. Therefore, let (the sum of) the (squares)
on $FG$ and $H$ be equal to the (square) on $EF$. And since as $BA$
is to $AC$, so the (square) on $EF$ (is) to the (square) on $FG$, thus,
via conversion, as $AB$ is to $BC$, so the (square) on $EF$ (is) to
the (square) on $H$ [Prop. 5.19~corr.]. 
And $AB$ has to $BC$ the ratio which (some) square number (has) to
(some) square number. Thus, the (square) on $EF$ also has to
the (square) on $H$ the ratio which (some) square number (has) to
(some) square number. Thus,
$EF$ is commensurable in length with $H$  [Prop. 10.9]. Thus, the square on $EF$
is greater than (the square on) $FG$ by the (square) on (some straight-line)
commensurable (in length) with ($EF$). And $EF$ and $FG$ are rational (straight-lines). And $EF$ (is) commensurable in length with $D$.

Thus, $EG$ is a first binomial (straight-line) [Def. 10.5].$^\dag$
(Which is) the very thing it was required to show.}
\end{Parallel}


\vspace{7pt}{\footnotesize\noindent $^\dag$If the rational straight-line has unit length then the length of a first binomial straight-line
is  $k+k\sqrt{1-{k'}^{\,2}}$. This, and the first apotome,
whose length is $k-k\,\sqrt{1-{k'}^{\,2}}$ [Prop. 10.85],
are the roots of $x^2- 2\,k\,x+k^2\,{k'}^{\,2}=0$.}  \\~\\

%%%%
%10.49
%%%%
\pdfbookmark[1]{Proposition 10.49}{pdf10.49}
\begin{Parallel}{}{}
\ParallelLText{
\begin{center}
{\large\ggn{49}.}
\end{center}\vspace*{-7pt}

\gr{E<ure~in t`hn >ek d'uo >onom'atwn deut'eran.}

\epsfysize=1.in
\centerline{\epsffile{Book10/fig049g.eps}}

\gr{>Ekke'isjwsan d'uo >arijmo`i o<i AG, GB, <'wste t`on sugke'imenon >ex
a>ut~wn t`on AB pr`oc m`en t`on BG l'ogon >'eqein, <`on tetr'agwnoc
>arijm`oc pr`oc tetr'agwnon >arijm'on, pr`oc d`e t`on AG l'ogon
m`h >'eqein, <`on tetr'agwnoc >arijm`oc pr`oc tetr'agwnon >arijm'on,
ka`i >ekke'isjw <rht`h <h D, ka`i t~h| D s'ummetroc >'estw <h EZ m'hkei;
<rht`h
>'ara >est`in <h EZ. gegon'etw d`h ka`i <wc
<o GA >arijm`oc pr`oc t`on AB, o<'utwc t`o >ap`o t~hc EZ pr`oc
t`o >ap`o t~hc ZH; s'ummetron >'ara >est`i t`o >ap`o t~hc EZ t~w|
>ap`o t~hc ZH. <rht`h >'ara >est`i ka`i <h ZH. ka`i >epe`i <o GA >arijm`oc
pr`oc t`on AB l`ogon o>uk >'eqei, <`on tetr'agwnoc >arijm`oc pr`oc
tetr'agwnon >arijm'on, o>ud`e t`o >ap`o t~hc EZ pr`oc t`o >ap`o
t~hc ZH l'ogon >'eqei, <`on tetr'agwnoc >arijm`oc
pr`oc tetr'agwnon >arijm'on. >as'ummetroc >'ara >est`in <h EZ t~h| ZH
m'hkei; a<i EZ, ZH >'ara <rhta'i e>isi dun'amei m'onon s'ummetroi;
>ek d'uo >'ara >onom'atwn >est`in <h EH. deikt'eon d'h, <'oti ka`i
deut'era.}

\gr{>Epe`i g`ar >an'apal'in >estin <wc <o BA >arijm`oc pr`oc t`on AG,
o<'utwc t`o >ap`o t~hc HZ pr`oc t`o >ap`o t~hc ZE, me'izwn d`e <o
BA to~u AG, me~izon >'ara [ka`i] t`o >ap`o t~hc HZ to~u >ap`o t~hc
ZE. >'estw t~w| >ap`o t~hc HZ >'isa t`a >ap`o t~wn EZ, J; >anastr'eyanti
>'ara >est`in <wc <o AB pr`oc t`on BG, o<'utwc t`o >ap`o
t~hc ZH pr`oc t`o >ap`o t~hc J. >all> <o AB pr`oc t`on BG
l'ogon >'eqei, <`on tetr'agwnoc >arijm`oc pr`oc tetr'agwnon >arijm'on;
ka`i t`o >ap`o t~hc ZH >'ara pr`oc t`o >ap`o t~hc J l'ogon
>'eqei, <`on tetr'agwnoc >arijm`oc pr`oc tetr'agwnon >arijm'on.
s'ummetroc >'ara >est`in <h ZH t~h| J m'hkei; <'wste <h ZH
t~hc ZE me~izon d'unatai t~w| >ap`o summ'etrou <eaut~h|. ka'i
e>isi <rhta`i a<i ZH, ZE dun'amei m'onon s'ummetroi, ka`i t`o EZ
>'elasson >'onoma t~h| >ekkeim'enh| <rht~h| s'ummetr'on
>esti t~h| D m'hkei.}

\gr{<H EH >'ara >ek d'uo >onom'atwn >est`i deut'era; <'oper >'edei de~ixai.
}}

\ParallelRText{
\begin{center}
{\large Proposition 49}
\end{center}

To find a second binomial (straight-line).

\epsfysize=1.in
\centerline{\epsffile{Book10/fig049e.eps}}

Let the two numbers $AC$ and $CB$ be laid down such that their sum $AB$
has to $BC$ the ratio which (some) square number (has) to (some)
square number, and does not have to $AC$ the ratio which (some)
square number (has) to (some) square number [Prop. 10.28~lem.~I]. And let the rational (straight-line) $D$ be laid down. And let $EF$ be commensurable in length with $D$.
$EF$ is thus a rational (straight-line).
So, let it also have been contrived that as the number $CA$ (is) to $AB$,
so the (square) on $EF$ (is) to the (square) on $FG$ [Prop. 10.6~corr.]. Thus, the (square) on $EF$
is commensurable with the (square) on $FG$ [Prop. 10.6]. Thus, $FG$ is also a rational
(straight-line). And since the number $CA$ does not have to $AB$ the ratio
which (some) square number (has) to (some) square number, the
(square) on $EF$ does not have to the (square) on $FG$ the ratio
which (some) square number (has) to (some) square number either.
Thus, $EF$  is incommensurable in length with $FG$ [Prop. 10.9]. $EF$ and $FG$
are thus rational (straight-lines which are) commensurable in square only.
Thus, $EG$ is a binomial (straight-line) [Prop. 10.36]. So, we must show
that (it is) also a second (binomial straight-line).

For since, inversely, as the number $BA$ is to $AC$, so the (square)
on $GF$ (is) to the (square) on $FE$ [Prop. 5.7 corr.],
and $BA$ (is) greater than $AC$, the (square) on $GF$ (is)
thus [also] greater than the (square) on $FE$ [Prop. 5.14]. Let (the sum of) the (squares) on $EF$ and $H$ be equal to the (square) on $GF$. Thus, via conversion, as 
$AB$ is to $BC$, so the (square) on $FG$ (is) to the (square) on $H$
[Prop. 5.19~corr.]. But, $AB$ has to $BC$ the
ratio which (some) square number (has) to (some) square number. Thus,
the (square) on $FG$ also has to the (square) on $H$ the ratio
which (some) square number (has) to (some) square number. Thus,
$FG$ is commensurable in length with $H$ [Prop. 10.9]. Hence, the square on $FG$ is
greater than (the square on) $FE$ by the (square) on (some straight-line)
commensurable in length with ($FG$). And $FG$ and $FE$ are rational (straight-lines which
are) commensurable in square only. And  the lesser term $EF$ is commensurable in length with the rational (straight-line) $D$ (previously)
laid down.

Thus, $EG$ is a second binomial (straight-line) [Def. 10.6].$^\dag$ (Which is) the very thing it
was required to show.}
\end{Parallel}


\vspace{7pt}{\footnotesize\noindent $^\dag$ If the rational straight-line has unit length then the length of a second binomial straight-line
is  $k/\sqrt{1-{k'}^{\,2}}+k$. This, and the second apotome,
whose length is $k/\sqrt{1-{k'}^{\,2}}-k$ [Prop. 10.86],
are the roots of $x^2- (2\,k/\sqrt{1-{k'}^{\,2}})\,x+k^2\,[{k'}^{\,2}/(1-{k'}^{\,2})]=0$.}  

%%%%
%10.50
%%%%
\pdfbookmark[1]{Proposition 10.50}{pdf10.50}
\begin{Parallel}{}{}
\ParallelLText{
\begin{center}
{\large \ggn{50}.}
\end{center}\vspace*{-7pt}

\gr{E<ure~in t`hn >ek d'uo >onom'atwn tr'ithn.}

\epsfysize=1.1in
\centerline{\epsffile{Book10/fig050g.eps}}

\gr{>Ekke'isjwsan d'uo >arijmo`i o<i AG, GB, <'wste t`on sugke'imenon
>ex a>ut~wn t`on AB pr`oc m`en t`on BG l'ogon >'eqein, <`on tetr'agwnoc
>arijm`oc pr`oc tetr'agwnon >arijm'on, pr`oc d`e t`on AG l'ogon m`h
>'eqein, <`on tetr'agwnoc >arijm`oc pr`oc tetr'agwnon >arijm'on. >ekke'isjw
d'e tic ka`i >'alloc m`h tetr'agwnoc >arijm`oc <o D, ka`i pr`oc <ek'ateron
t~wn BA, AG l'ogon m`h >eq'etw, <`on tetr'agwnoc >arijm`oc pr`oc tetr'agwnon >arijm'on; ka`i >ekke'isjw tic <rht`h e>uje~ia <h E, ka`i
gegon'etw <wc <o D pr`oc t`on AB, o<'utwc t`o >ap`o t~hc E pr`oc t`o
>ap`o t~hc ZH; s'ummetron >'ara >est`i t`o >ap`o t~hc E t~w| >ap`o
t~hc ZH. ka'i >esti <rht`h <h E; <rht`h >'ara >est`i ka`i <h ZH.
ka`i >epe`i <o D pr`oc t`on AB l'ogon o>uk >'eqei, <`on tetr'agwnoc
>arijm`oc pr`oc tetr'agwnon >arijm'on, o>ud`e t`o >ap`o t~hc E pr`oc
t`o >ap`o t~hc ZH l'ogon >'eqei, <`on tetr'agwnoc >arijm`oc pr`oc
tetr'agwnon >arijm'on;
 >as'ummetroc >'ara >est`in <h E t~h|
ZH m'hkei. gegon'etw d`h p'alin <wc <h BA >arijm`oc pr`oc t`on AG,
o<'utwc t`o >ap`o t~hc ZH pr`oc t`o >ap`o t~hc HJ; s'ummetron
>'ara >est`i t`o >ap`o t~hc ZH t~w| >ap`o t~hc HJ. <rht`h d`e <h ZH;
<rht`h >'ara ka`i <h HJ. ka`i >epe`i <o BA pr`oc t`on AG l'ogon o>uk
>'eqei, <`on tetr'agwnoc >arijm`oc pr`oc tetr'agwnon >arijm'on, o>ud`e
t`o >ap`o t~hc ZH pr`oc t`o >ap`o t~hc JH l'ogon >'eqei, <`on tetr'agwnoc
>arijm`oc pr`oc tetr'agwnon >arijm'on;
>as'ummetroc >'ara >est`in <h ZH t~h| HJ m'hkei. a<i ZH, HJ
>'ara <rhta'i e>isi dun'amei m'onon s'ummetroi;
<h ZJ >'ara >ek d'uo >onom'atwn >est'in. l'egw d'h, <'oti ka`i tr'ith.}

\gr{>Epe`i g'ar >estin <wc <o D pr`oc t`on AB, o<'utwc t`o >ap`o t~hc
E pr`oc t`o >ap`o t~hc ZH, <wc d`e <o BA pr`oc t`on AG, o<'utwc
t`o >ap`o t~hc ZH pr`oc t`o >ap`o t~hc HJ, di> >'isou >'ara
>est`in <wc <o D pr`oc t`on AG, o<'utwc t`o >ap`o t~hc E pr`oc
t`o >ap`o t~hc HJ. <o d`e D pr`oc t`on AG l'ogon o>uk >'eqei, <`on
tetr'agwnoc >arijm`oc pr`oc tetr'agwnon >arijm'on; o>ud`e t`o >ap`o
t~hc E >'ara pr`oc t`o >ap`o t~hc HJ l'ogon >'eqei, <`on tetr'agwnoc
>arijm`oc pr`oc tetr'agwnon >arijm'on; >as'ummetroc >'ara >est`in <h E
t~h| HJ m'hkei. ka`i >epe'i >estin <wc <o BA pr`oc t`on AG, o<'utwc
t`o >ap`o t~hc ZH pr`oc t`o >ap`o t~hc HJ, me~izon
>'ara t`o >ap`o t~hc ZH to~u >ap`o t~hc HJ. >'estw
o>~un t~w| >ap`o t~hc ZH >'isa t`a >ap`o t~wn HJ, K; >anastr'eyanti
>'ara [>est`in] <wc <o AB pr`oc t`on BG, o<'utwc t`o >ap`o t~hc ZH
pr`oc t`o >ap`o t~hc K. <o d`e AB pr`oc t`on BG l'ogon >'eqei,
<`on tetr'agwnoc >arijm`oc pr`oc tetr'agwnon >arijm'on; ka`i
t`o >ap`o t~hc ZH >'ara pr`oc t`o >ap`o t~hc K l'ogon >'eqei, <`on
tetr'agwnoc >arijm`oc pr`oc tetr'agwnon >arijm'on; s'ummetroc
>'ara [>est`in] <h ZH t~h| K m'hkei. <h ZH >'ara t~hc HJ
me~izon d'unatai t~w| >ap`o summ'etrou <eaut~h|. ka'i e>isin a<i
ZH, HJ <rhta`i dun'amei m'onon s'ummetroi, ka`i o>udet'era a>ut~wn
s'ummetr'oc >esti t~h| E m'hkei.}

\gr{<H ZJ >'ara >ek d'uo >onom'atwn >est`i tr'ith; <'oper >'edei de~ixai.}}

\ParallelRText{
\begin{center}
{\large Proposition 50}
\end{center}

To find a third binomial (straight-line).

\epsfysize=1.1in
\centerline{\epsffile{Book10/fig050e.eps}}

Let the two numbers $AC$ and $CB$ be laid down such that their sum
$AB$ has to $BC$ the ratio which (some) square number (has) to
(some) square number, and does not have to $AC$ the ratio which (some)
square number (has) to (some) square number. And let some other non-square number
$D$ also be laid down, and let it not have to each of $BA$ and $AC$ the
ratio which (some) square number (has) to (some) square number.
And let some rational straight-line $E$ be laid down, and let it have been
contrived that as $D$ (is) to $AB$, so the (square) on $E$ (is) to
the (square) on $FG$ [Prop. 10.6~corr.]. Thus,
the (square) on $E$ is commensurable with the (square) on $FG$ [Prop. 10.6]. And $E$ is a rational (straight-line).
Thus, $FG$ is also a rational (straight-line). And since $D$ does not have to
$AB$ the ratio which (some) square number has to (some) square number,
the (square) on $E$ does not have to the (square) on $FG$ the ratio
which (some) square number (has) to (some) square number either.
$E$ is thus incommensurable in length with $FG$ [Prop. 10.9]. So, again, let it have been contrived that
as the number $BA$ (is) to $AC$, so the (square) on  $FG$ (is) to the
(square) on $GH$ [Prop. 10.6~corr.]. Thus,
the (square) on $FG$ is commensurable with the (square) on $GH$
[Prop. 10.6]. And $FG$ (is) a rational (straight-line).
Thus, $GH$ (is) also a rational (straight-line). And since $BA$ does not have to $AC$ the ratio which (some) square number (has) to (some) square number, the (square) on $FG$  does not have to the (square) on $HG$
the ratio which (some) square number (has) to (some) square number either.
Thus, $FG$ is incommensurable in length with $GH$ [Prop. 10.9].  $FG$ and $GH$ are thus rational
(straight-lines which are) commensurable in square only. Thus,
$FH$ is a binomial (straight-line) [Prop. 10.36].
So, I say that (it is) also a third (binomial straight-line).

For since as $D$ is to $AB$, so the (square) on $E$ (is) to the (square)
on $FG$, and as $BA$ (is) to $AC$, so the (square) on $FG$ (is) to
the (square) on $GH$, thus, via equality, as $D$ (is) to $AC$, so the
(square) on $E$ (is) to the (square) on $GH$ [Prop. 5.22]. And $D$ does not have to $AC$ the
ratio which (some) square number (has) to (some) square number.
Thus, the (square) on $E$ does not have to the (square) on $GH$ the ratio
which (some) square number (has) to (some) square number either.
Thus, $E$ is incommensurable in length with $GH$ [Prop. 10.9].  And since as $BA$ is to $AC$,
so the (square) on $FG$ (is) to the (square) on $GH$, the (square) on 
$FG$ (is) thus greater than the (square) on $GH$ [Prop. 5.14]. Therefore, let (the sum of) the (squares)
on $GH$ and $K$
be equal to the (square) on $FG$. Thus, via
conversion, as $AB$ [is] to $BC$, so the (square) on $FG$ (is) to the
(square) on $K$ [Prop. 5.19~corr.]. And $AB$
has to $BC$ the ratio which (some) square number (has) to (some) square
number. Thus, the (square) on $FG$ also has to the (square) on $K$
the ratio which (some) square number (has) to (some) square number.
Thus,  $FG$ [is] commensurable in length with $K$ [Prop. 10.9]. Thus, the square on
$FG$ is greater than (the square on) $GH$ by the (square) on
(some straight-line) commensurable (in length) with ($FG$). And $FG$ and $GH$
are rational (straight-lines which are) commensurable in square only,
and neither of them is commensurable in length with $E$.

Thus, $FH$ is a third binomial (straight-line) [Def. 10.7].$^\dag$ (Which is) the very thing it
was required to show.}
\end{Parallel}


\vspace{7pt}{\footnotesize\noindent$^\dag$ If the rational straight-line has unit length then the length of a third binomial straight-line
is  $k^{1/2}\,(1+\sqrt{1-{k'}^{\,2}})$. This, and the third apotome,
whose length is $k^{1/2}\,(1-\sqrt{1-{k'}^{\,2}})$ [Prop. 10.87],
are the roots of $x^2- 2\,k^{1/2}\,x+k\,{k'}^{\,2}=0$.}

%%%%
%10.51
%%%%
\pdfbookmark[1]{Proposition 10.51}{pdf10.51}
\begin{Parallel}{}{}
\ParallelLText{
\begin{center}
{\large \ggn{51}.}
\end{center}\vspace*{-7pt}

\gr{E<ure~in t`hn >ek d'uo >onom'atwn tet'arthn.}

\epsfysize=1.2in
\centerline{\epsffile{Book10/fig051g.eps}}

\gr{>Ekke'isjwsan d'uo >arijmo`i o<i AG, GB, <'wste t`on AB pr`oc t`on
BG l'ogon m`h >'eqein m'hte m`hn pr`oc t`on AG, <`on tetr'agwnoc
>arijm`oc pr`oc tetr'agwnon >arijm'on. ka`i >ekke'isjw <rht`h <h D,
ka`i t~h| D s'ummetroc >'estw m'hkei <h EZ; <rht`h >'ara >est`i
ka`i <h EZ. ka`i gegon'etw <wc <o BA >arijm`oc pr`oc t`on AG,
o<'utwc t`o >ap`o t~hc EZ pr`oc t`o >ap`o t~hc ZH; s'ummetron
>'ara >est`i t`o >ap`o t~hc EZ t~w| >ap`o t~hc ZH; <rht`h >'ara >est`i
ka`i <h ZH. ka`i >epe`i <o BA pr`oc t`on AG l'ogon o>uk >'eqei,
<`on tetr'agwnoc >arijm`oc pr`oc tetr'agwnon >arijm'on, o>ud`e
t`o >ap`o t~hc EZ  pr`oc t`o >ap`o t~hc ZH l'ogon >'eqei, <`on tetr'agwnoc
>arijm`oc pr`oc tetr'agwnon >arijm'on; >as'ummetroc >'ara >est`in
<h EZ t~h| ZH m'hkei. a<i EZ, ZH >'ara <rhta'i e>isi dun'amei m'onon
s'ummetroi; <'wste <h EH >ek d'uo >onom'atwn >est'in. l'egw d'h,
<'oti ka`i tet'arth.}

\gr{>Epe`i g'ar >estin <wc <o BA pr`oc t`on AG, o<'utwc
t`o >ap`o t~hc EZ pr`oc t`o >ap`o t~hc ZH [me'izwn
d`e <o BA to~u AG], me~izon >'ara t`o >ap`o t~hc EZ to~u >ap`o
t~hc ZH. >'estw o>~un t~w| >ap`o t~hc EZ >'isa t`a >ap`o t~wn ZH,
J; >anastr'eyanti >'ara <wc <o AB >arijm`oc pr`oc t`on BG, o<'utwc
t`o >ap`o t~hc EZ pr`oc t`o >ap`o t~hc J. <o d`e AB pr`oc t`on BG
l'ogon o>uk >'eqei, <`on tetr'agwnoc >arijm`oc pr`oc tetr'agwnon
>arijm'on; o>ud> >'ara t`o >ap`o t~hc EZ pr`oc t`o >ap`o t~hc J
l'ogon >'eqei, <`on tetr'agwnoc >arijm`oc pr`oc tetr'agwnon >arijm'on.
>as'ummetroc >'ara >est`in <h EZ t~h| J m'hkei; <h EZ >'ara t~hc
HZ me~izon d'unatai t~w| >ap`o >asumm'etrou <eaut~h|. ka'i
e>isin a<i EZ, ZH <rhta`i dun'amei m'onon s'ummetroi, ka`i
<h EZ t~h| D s'ummetr'oc >esti m'hkei.}

\gr{<H EH >'ara >ek d'uo >onom'atwn >est`i tet'arth; <'oper
>'edei de~ixai.}}

\ParallelRText{
\begin{center}
{\large Proposition 51}
\end{center}

To find a fourth binomial (straight-line).

\epsfysize=1.2in
\centerline{\epsffile{Book10/fig051e.eps}}

Let the two numbers $AC$ and $CB$ be laid down such that $AB$ does not have to $BC$,  or to $AC$ either, the ratio which (some) square number
(has) to (some) square number [Prop. 10.28~lem.~I].
And let the rational (straight-line) $D$ be laid down. And let $EF$ be
commensurable in length with $D$. Thus, $EF$ is also a rational (straight-line). And let it have been contrived that as the number $BA$ (is) to
$AC$, so the (square) on $EF$ (is) to the (square) on $FG$ [Prop. 10.6~corr.]. Thus, the (square) on $EF$
is commensurable with the (square) on $FG$ [Prop. 10.6]. Thus, $FG$ is also a rational (straight-line). And since $BA$ does not have to $AC$ the ratio which (some)
square number (has) to (some) square number,  the (square) on $EF$
does not have to the (square) on $FG$ the ratio which (some) square number
(has) to (some) square number either. Thus, $EF$ is incommensurable in length with $FG$ [Prop. 10.9]. Thus, $EF$ and
$FG$ are rational (straight-lines which are) commensurable in square only.
Hence, $EG$ is a binomial (straight-line) [Prop. 10.36]. So, I say that (it is) also
a fourth (binomial straight-line).

For since as $BA$ is to $AC$, so the (square) on $EF$ (is) to the (square) on
$FG$ [and $BA$ (is) greater than $AC$], the (square) on $EF$ (is) thus
greater than the (square) on $FG$ [Prop. 5.14]. Therefore, let (the sum of) the squares on
$FG$ and $H$ be equal to the (square) on $EF$. Thus, via conversion,
as the number $AB$ (is) to $BC$, so the (square) on $EF$ (is) to the
(square) on $H$ [Prop. 5.19~corr.]. And $AB$
does not have to $BC$ the ratio which (some) square number (has) to (some)
square number. Thus, the (square) on $EF$ does not have to the
(square) on $H$ the ratio which (some) square number (has) to (some) square
number either. Thus, $EF$ is incommensurable in length with $H$ [Prop. 10.9]. Thus, the square on $EF$ is greater than (the square on) $GF$ by the (square) on (some straight-line)
incommensurable (in length) with ($EF$). And $EF$ and $FG$ are rational (straight-lines which are) commensurable in square only.  And $EF$ is commensurable
in length with $D$.

Thus, $EG$ is a fourth binomial (straight-line) [Def. 10.8].$^\dag$ (Which is) the very thing
it was required to show.}
\end{Parallel}


\vspace{7pt}{\footnotesize\noindent $^\dag$ If the rational straight-line has unit length then the length of a fourth binomial straight-line
is  $k\,(1+1/\sqrt{1+k'})$. This, and the fourth apotome,
whose length is $k\,(1-1/\sqrt{1+k'})$ [Prop. 10.88],
are the roots of $x^2- 2\,k\,x+k^2\,k'/(1+k')=0$.}

%%%%
%10.52
%%%%
\pdfbookmark[1]{Proposition 10.52}{pdf10.52}
\begin{Parallel}{}{}
\ParallelLText{
\begin{center}
{\large \ggn{52}.}
\end{center}\vspace*{-7pt}

\gr{E<ure~in t`hn >ek d'uo >onom'atwn p'empthn.}

\epsfysize=1.3in
\centerline{\epsffile{Book10/fig052g.eps}}

\gr{>Ekke'isjwsan d'uo >arijmo`i o<i AG, GB, <'wste t`on AB pr`oc <ek'ateron
a>ut~wn l'ogon m`h >'eqein, <`on tetr'agwnoc >arijm`oc pr`oc tetr'agwnon
>arijm'on, ka`i >ekke'isjw <rht'h tic e>uje~ia <h D, ka`i t~h| D s'ummetroc
>'estw [m'hkei] <h EZ; <rht`h >'ara <h EZ. ka`i gegon'etw <wc <o GA
pr`oc t`on AB, o<'utwc t`o >ap`o t~hc EZ pr`oc t`o >ap`o t~hc ZH.
<o d`e GA pr`oc t`on AB l'ogon o>uk >'eqei, <`on tetr'agwnoc >arijm`oc
pr`oc tetr'agwnon >arijm'on; o<ud`e t`o >ap`o t~hc EZ >'ara pr`oc
t`o >ap`o t~hc ZH l'ogon >'eqei, <`on tetr'agwnoc >arijm`oc pr`oc
tetr'agwnon >arijm'on. a<i EZ, ZH >'ara <rhta'i e>isi dun'amei
m'onon s'ummetroi; >ek d'uo >'ara >onom'atwn >est`in <h EH. l'egw d'h, <'oti ka`i p'empth.}

\gr{>Epe`i g'ar >estin <wc <o GA pr`oc t`on AB, o<'utwc t`o >ap`o t~hc EZ
pr`oc t`o >ap`o t~hc ZH, >an'apalin <wc <o BA pr`oc t`on AG, o<'utwc
t`o >ap`o t~hc ZH pr`oc t`o >ap`o t~hc ZE; me~izon >'ara t`o >ap`o
t~hc HZ to~u >ap`o t~hc ZE. >'estw o>~un t~w| >ap`o t~hc
HZ >'isa t`a >ap`o t~wn EZ, J; >anastr'eyanti >'ara >est`in <wc <o AB
>arijm`oc pr`oc t`on BG, o<'utwc t`o >ap`o t~hc HZ pr`oc t`o
>ap`o t~hc J. <o d`e AB pr`oc t`on BG l'ogon o>uk >'eqei, <`on
tetr'agwnoc >arijm`oc pr`oc tetr'agwnon >arijm'on; o>ud> >'ara t`o
>ap`o t~hc ZH pr`oc t`o >ap`o t~hc J l'ogon
>'eqei, <`on tetr'agwnoc >arijm`oc pr`oc tetr'agwnon >arijm'on.
>as'ummetroc >'ara >est`in <h ZH t~h| J
m'hkei; <'wste <h ZH t~hc ZE me~izon d'unatai t~w| >ap`o >asumm'etrou
<eaut~h|. ka'i e>isin a<i HZ, ZE <rhta`i dun'amei m'onon s'ummetroi,
ka`i t`o EZ >'elatton >'onoma s'ummetr'on >esti t~h| >ekkeim'enh|
<rht~h| t~h| D m'hkei.}

\gr{<H EH >'ara >ek d'uo >onom'atwn >est`i p'empth; <'oper
>'edei de~ixai.}}

\ParallelRText{
\begin{center}
{\large Proposition 52}
\end{center}

To find a fifth binomial straight-line.

\epsfysize=1.3in
\centerline{\epsffile{Book10/fig052e.eps}}

Let the two numbers $AC$ and $CB$ be laid down such that $AB$ does
not have to either of them the ratio which (some) square number (has)
to (some) square number [Prop. 10.38~lem.]. And
let some rational straight-line $D$ be laid down. And let $EF$ be
commensurable [in length] with $D$. Thus, $EF$ (is) a rational (straight-line).
And let it have been contrived that as $CA$ (is) to $AB$, so the (square) on
$EF$ (is) to the (square) on $FG$ [Prop. 10.6~corr.].
 And $CA$ does not have to $AB$ the ratio which (some)
 square number (has) to (some) square number. Thus, the (square) on
 $EF$ does not have to the (square) on $FG$ the ratio which (some)
 square number (has) to (some) square number either. Thus, $EF$ and
 $FG$ are rational (straight-lines which are) commensurable in square only
 [Prop. 10.9].  Thus, $EG$ is a binomial (straight-line) [Prop. 10.36]. So, I say that (it is) also
 a fifth (binomial straight-line).
 
 For since as $CA$ is to $AB$, so the (square) on $EF$ (is) to the (square)
 on $FG$, inversely, as $BA$ (is) to $AC$, so the (square) on $FG$ (is)
 to the (square) on $FE$ [Prop. 5.7~corr.]. Thus,
 the (square) on $GF$ (is) greater than the (square) on $FE$ [Prop. 5.14]. Therefore, let (the sum of) the (squares)
 on $EF$ and $H$ be equal to the (square) on $GF$. Thus, via conversion, 
 as the number $AB$ is to $BC$, so the (square) on $GF$ (is) to
 the (square) on $H$ [Prop. 5.19~corr.]. 
 And $AB$ does not have to $BC$ the ratio which (some) square number
 (has) to (some) square number. Thus, the (square) on $FG$ does not have
 to the (square) on $H$ the ratio which (some) square number (has) to
 (some) square number either. Thus, $FG$ is incommensurable in length
 with $H$ [Prop. 10.9]. Hence, the square on $FG$
 is greater than (the square on) $FE$ by the (square) on (some straight-line)
 incommensurable (in length) with ($FG$). And $GF$ and $FE$ are rational (straight-lines
 which are) commensurable in square only. And the lesser term $EF$ is commensurable in length with the  rational (straight-line previously)  laid down, $D$.
 
 Thus, $EG$ is a fifth binomial (straight-line).$^\dag$  (Which is) the very thing it
 was required to show.}
\end{Parallel}


\vspace{7pt}{\footnotesize\noindent $^\dag$ If the rational straight-line has unit length then the length of a fifth binomial straight-line
is  $k\,(\sqrt{1+k'}+1)$. This, and the fifth apotome,
whose length is $k\,(\sqrt{1+k'}-1)$ [Prop. 10.89],
are the roots of $x^2- 2\,k\,\sqrt{1+k'}\,x+k^2\,k'=0$.}
 
%%%%
%10.53
%%%%
\pdfbookmark[1]{Proposition 10.53}{pdf10.53}
\begin{Parallel}{}{}
\ParallelLText{
\begin{center}
{\large \ggn{53}.}
\end{center}\vspace*{-7pt}

\gr{E<ure~in t`hn >ek d'uo >onom'atwn <'ekthn.}

\epsfysize=1.in
\centerline{\epsffile{Book10/fig053g.eps}}

\gr{>Ekke'isjwsan d'uo >arijmo`i o<i AG, GB, <'wste t`on AB pr`oc <ek'ateron
a>ut~wn l'ogon m`h >'eqein, <`on tetr'agwnoc >arijm`oc pr`oc tetr'agwnon
>arijm'on; >'estw d`e ka`i <'eteroc >arijm`oc <o D m`h tetr'agwnoc >`wn
mhd`e pr`oc <ek'ateron t~wn BA, AG l'ogon >'eqwn, <`on tetr'agwnoc
>arijm`oc pr`oc tetr'agwnon >arijm'on; ka`i >ekke'isjw tic
<rht`h e>uje~ia <h E, ka`i gegon'etw <wc <o D pr`oc t`on AB, o<'utwc
t`o >ap`o t~hc E pr`oc t`o >ap`o t~hc ZH; s'ummetron >'ara t`o >ap`o
t~hc E t~w| >ap`o t~hc ZH. ka'i >esti <rht`h <h E; <rht`h >'ara ka`i <h ZH. ka`i >epe`i o>uk >'eqei <o D pr`oc t`on AB l'ogon, <`on tetr'agwnoc
>arijm`oc pr`oc tetr'agwnon >arijm'on, o>ud`e t`o >ap`o t~hc E >'ara pr`oc t`o >ap`o t~hc ZH l'ogon >'eqei, <`on tetr'agwnoc >arijm`oc pr`oc tetr'agwnon >arijm'on; >as'ummetroc >'ara <h E t~h| ZH m'hkei. gegon'etw
d`h p'alin <wc <o BA pr`oc t`on AG, o<'utwc t`o >ap`o t~hc ZH
pr`oc t`o >ap`o t~hc HJ. s'ummetron >'ara t`o >ap`o t~hc ZH t~w|
>ap`o t~hc JH. <rht`on >'ara t`o >ap`o t~hc JH; <rht`h >'ara <h JH. ka`i
>epe`i <o BA pr`oc t`on AG l'ogon o>uk >'eqei, <`on tetr'agwnoc
>arijm`oc pr`oc tetr'agwnon >arijm'on, o>ud`e t`o >ap`o t~hc ZH pr`oc
t`o >ap`o t~hc HJ l'ogon >'eqei, <`on tetr'agwnoc >arijm`oc
pr`oc tetr'agwnon >arijm'on; >as'ummetroc >'ara >est`in <h ZH
t~h| HJ m'hkei. a<i ZH, HJ >'ara <rhta'i e>isi
dun'amei m'onon s'ummetroi; >ek d'uo >'ara >onom'atwn >est`in <h ZJ. deikt'eon d'h, <'oti ka`i <'ekth.}

\gr{>Epe`i g'ar >estin <wc <o D pr`oc t`on AB, o<'utwc t`o >ap`o t~hc E pr`oc
t`o >ap`o t~hc ZH, >'esti d`e ka`i <wc <o BA pr`oc t`on AG, o<'utwc
t`o >ap`o t~hc ZH pr`oc t`o >ap`o t~hc HJ, di> >'isou >'ara >est`in
<wc <o D pr`oc t`on AG, o<'utwc t`o >ap`o t~hc E pr`oc t`o >ap`o
t~hc HJ. <o d`e D pr`oc t`on AG l'ogon o>uk >'eqei, <`on tetr'agwnoc
>arijm`oc pr`oc tetr'agwnon >arijm'on; o>ud`e t`o >ap`o t~hc
E >'ara pr`oc t`o >ap`o t~hc HJ l'ogon >'eqei, <`on tetr'agwnoc
>arijm`oc pr`oc tetr'agwnon >arijm'on;
>as'ummetroc >'ara >est`in
<h E t~h| HJ m'hkei. >ede'iqjh d`e ka`i t~h| ZH >as'ummetroc;
<ekat'era >'ara t~wn ZH, HJ >as'ummetr'oc >esti t~h| E m'hkei.
ka`i >epe'i >estin <wc <o BA pr`oc t`on AG, o<'utwc t`o >ap`o
t~hc ZH pr`oc t`o >ap`o t~hc HJ, me~izon >'ara t`o >ap`o t~hc ZH
to~u >ap`o t~hc HJ. >'estw o>~un t~w| >ap`o [t~hc] ZH >'isa
t`a >ap`o t~wn HJ, K; >anastr'eyanti >'ara <wc <o AB pr`oc BG, o<'utwc
t`o >ap`o ZH pr`oc t`o >ap`o t~hc K. <o d`e AB pr`oc t`on BG l'ogon
o>uk >'eqei, <`on tetr'agwnoc >arijm`oc pr`oc tetr'agwnon >arijm'on; <'wste
o>ud`e t`o >ap`o ZH pr`oc t`o >ap`o t~hc K l'ogon >'eqei, <`on
tetr'agwnoc >arijm`oc pr`oc tetr'agwnon >arijm'on. >as'ummetroc >'ara
>est`in <h ZH t~h| K m'hkei; <h ZH >'ara t~hc HJ me~izon d'unatai
t~w| >ap`o >asumm'etrou <eaut~h|. ka'i e>isin a<i ZH, HJ <rhta`i dun'amei
m'onon s'ummetroi, ka`i o>udet'era a>ut~wn s'ummetr'oc >esti m'hkei
t~h| >ekkeim'enh| <rhth| t~h| E.}

\gr{<H ZJ >'ara >ek d'uo >onom'atwn >est`in <'ekth; <'oper >'edei de~ixai.}}

\ParallelRText{
\begin{center}
{\large Proposition 53}
\end{center}

To find a sixth binomial (straight-line).

\epsfysize=1.in
\centerline{\epsffile{Book10/fig053e.eps}}

Let the two numbers $AC$ and $CB$ be laid down such that $AB$ does
not have to each of them the ratio which (some) square number (has) to (some) square number.  And let $D$  also be another number, which is not
square, and does not have to each of $BA$ and $AC$ the ratio which (some)
square number (has) to (some) square number either [Prop. 10.28~lem.~I]. And let some rational straight-line $E$ be laid down. And let it have been contrived that as $D$ (is) to
$AB$, so the (square) on $E$ (is) to the (square) on $FG$ [Prop. 10.6~corr.].  Thus, the (square) on $E$
(is) commensurable with the (square) on $FG$ [Prop. 10.6]. And $E$ is rational. Thus, $FG$
(is) also rational.  And since $D$ does not have to $AB$ the ratio
which (some) square number (has) to (some) square number, the (square) on $E$ thus does not have to the (square) on $FG$ the ratio which (some) square number (has) to (some) square number either. Thus, $E$ (is) incommensurable in length with $FG$ [Prop. 10.9].
So, again, let it have be contrived that as $BA$ (is) to $AC$, so the (square)
on $FG$ (is) to the (square) on $GH$ [Prop. 10.6~corr.]. The (square) on $FG$ (is)
thus commensurable with the (square) on $HG$ [Prop. 10.6].
 The (square) on $HG$ (is) thus rational.  Thus, $HG$ (is) rational. And since $BA$ does not have to
$AC$ the ratio which (some) square number (has) to (some) square number,
the (square) on $FG$ does not have to the (square) on $GH$ the
ratio which (some) square number (has) to (some) square number either.
Thus, $FG$ is incommensurable in length with $GH$ [Prop. 10.9]. Thus, $FG$ and $GH$ are rational (straight-lines which are) commensurable in square only. Thus, $FH$
is a binomial (straight-line) [Prop. 10.36]. So,
we must show that (it is) also a sixth (binomial straight-line).

For since as $D$ is to $AB$, so the (square) on $E$ (is) to the (square) on
$FG$, and also as $BA$ is to $AC$, so the (square) on $FG$ (is) to
the (square) on $GH$, thus, via equality, as $D$ is to $AC$, so
the (square) on $E$ (is) to the (square) on $GH$ [Prop. 5.22]. And $D$ does not have to $AC$ the
ratio which (some) square number (has) to (some) square number.  Thus, 
the (square) on $E$ does not have to the (square) on $GH$ the ratio
which (some) square number (has) to (some) square number either. $E$
is thus incommensurable in length with $GH$ [Prop. 10.9]. And ($E$) was also shown (to be)
incommensurable (in length) with $FG$.  Thus,  $FG$ and $GH$ are
each incommensurable in length with $E$. And since as $BA$ is to $AC$, 
so the (square) on $FG$ (is) to the (square) on $GH$, the (square) on $FG$ (is) thus greater than the (square) on $GH$ [Prop. 5.14]. Therefore, let (the sum of) the
(squares) on $GH$ and $K$ be equal to the (square) on $FG$. Thus,
via conversion, as $AB$ (is) to $BC$,
 so the (square) on $FG$ (is) to the
(square) on $K$ [Prop. 5.19~corr.].  And $AB$ does not
have to $BC$ the ratio which (some) square number (has) to (some) square number. Hence, the (square) on $FG$ does not have to the (square) on
$K$ the ratio which (some) square number (has) to (some) square number either. Thus, $FG$ is incommensurable in length with $K$ [Prop. 10.9]. The square on $FG$ is thus
greater than (the square on) $GH$ by the (square) on (some straight-line which is) incommensurable (in length) with ($FG$). And  $FG$ and $GH$ are rational
(straight-lines which are) commensurable in square only, and neither of them
is commensurable in length with the  rational (straight-line) $E$ (previously)
laid down.

Thus, $FH$ is a  sixth binomial (straight-line) [Def. 10.10].$^\dag$ (Which is) the very thing it was required to show.}
\end{Parallel}


\vspace{7pt}{\footnotesize\noindent$^\dag$ If the rational straight-line has unit length then the length of a sixth binomial straight-line
is  $\sqrt{k}+\sqrt{k'}$. This, and the sixth apotome,
whose length is $\sqrt{k}-\sqrt{k'}$ [Prop. 10.90],
are the roots of $x^2- 2\,\sqrt{k}\,x+(k-k')=0$.}

%%%%
%10.53lem
%%%%
\begin{Parallel}{}{}
\ParallelLText{

\begin{center}
{\large \gr{L~hmma}.}
\end{center}\vspace*{-7pt}

\gr{>'Estw d'uo tetr'agwna t`a AB, BG ka`i ke'isjwsan <'wste >ep> e>uje'iac  e>~inai t`hn DB t~h| BE; >ep> e>uje'iac >'ara
>est`i ka`i <h ZB t~h| BH. ka`i sumpeplhr'wsjw t`o AG parallhl'ogrammon;
l'egw, <'oti tetr'agwn'on >esti t`o AG, ka`i <'oti t~wn AB, BG m'eson >an'alog'on >esti t`o DH, ka`i >'eti t~wn AG, GB m'eson >an'alog'on >esti
t`o DG.}

\epsfysize=1.7in
\centerline{\epsffile{Book10/fig053ag.eps}}

\gr{>Epe`i g`ar >'ish >est`in <h m`en DB t~h| BZ, <h d`e BE t~h| BH, <'olh
>'ara <h DE <'olh| t~h| ZH >estin >'ish. >all> <h m`en DE <ekat'era|
t~wn AJ, KG >estin >'ish, <h d`e ZH <ekat'era| t~wn AK, JG >estin >'ish;
ka`i <ekat'era >'ara t~wn AJ, KG <ekat'era| t~wn AK, JG >estin >'ish.
>is'opleuron >'ara >est`i t`o AG parallhl'ogrammon; >'esti d`e ka`i
>orjog'wnion; tetr'agwnon >'ara >est`i t`o AG.}

\gr{Ka`i >epe`i >estin <wc <h ZB pr`oc t`hn BH, o<'utwc <h
DB pr`oc t`hn BE, >all> <wc m`en <h ZB pr`oc t`hn BH, o<'utwc t`o AB
pr`oc t`o DH, <wc d`e <h DB pr`oc t`hn BE, o<'utwc t`o DH pr`oc
t`o BG, ka`i <wc >'ara t`o AB pr`oc t`o DH, o<'utwc t`o DH pr`oc t`o
BG. t~wn AB, BG >'ara m'eson >an'alog'on >esti t`o DH.}

\gr{L'egw d'h, <'oti ka`i t~wn AG, GB m'eson >an'alog'on [>esti] t`o DG.}

\gr{>Epe`i g'ar >estin <wc <h AD pr`oc t`hn DK, o<'utwc <h KH pr`oc
t`hn HG; >'ish g'ar [>estin] <ekat'era <ekat'era|; ka`i sunj'enti <wc <h AK
pr`oc KD, o<'utwc <h KG
pr`oc GH, >all> <wc m`en <h AK pr`oc KD, o<'utwc
t`o AG pr`oc t`o GD, <wc d`e <h KG pr`oc GH, o<'utwc t`o DG pr`oc
GB, ka`i <wc >'ara t`o AG pr`oc DG, o<'utwc t`o DG pr`oc t`o BG. t~wn AG, GB >'ara m'eson >an'alog'on >esti t`o DG; <`a pro'ekeito
de~ixai.}}

\ParallelRText{
\begin{center}
{\large Lemma}
\end{center}

Let $AB$ and $BC$ be two squares, and let them
be laid down such that $DB$ is straight-on to $BE$. $FB$ is, thus, also
straight-on to $BG$. And let the parallelogram $AC$ have been completed.
I say that $AC$ is a square, and that $DG$ is  the mean proportional to $AB$ and
$BC$, and, moreover,  $DC$ is the mean proportional to $AC$ and $CB$.

\epsfysize=1.7in
\centerline{\epsffile{Book10/fig053ae.eps}}

For since $DB$ is equal to $BF$, and $BE$ to $BG$, the whole of $DE$
is thus equal to the whole of $FG$. But $DE$ is equal to each of $AH$
and $KC$, and $FG$ is equal to each of $AK$ and $HC$ [Prop. 1.34]. Thus, $AH$ and $KC$ are also 
equal to  $AK$ and $HC$, respectively. Thus, the parallelogram $AC$
is equilateral. And (it is) also right-angled. Thus, $AC$ is a square.

And since as $FB$ is to $BG$, so $DB$ (is) to $BE$, but as $FB$
(is) to $BG$, so $AB$ (is) to $DG$, and as $DB$ (is) to $BE$, so
$DG$ (is) to $BC$ [Prop. 6.1], thus also as $AB$ (is) to $DG$, so $DG$  (is) to $BC$ [Prop. 5.11]. Thus, $DG$ is the mean proportional
to $AB$ and $BC$.

So I also say that $DC$ [is] the mean proportional to $AC$ and $CB$.

For since as $AD$ is to $DK$, so $KG$ (is) to $GC$. For [they are]  respectively equal. And, via composition, as $AK$ (is) to $KD$, so
$KC$ (is) to $CG$ [Prop. 5.18]. But as
$AK$ (is) to $KD$, so $AC$ (is) to $CD$, and as $KC$ (is) to $CG$,
so $DC$ (is) to $CB$ [Prop. 6.1]. Thus, also, as $AC$ (is) to $DC$, so
$DC$ (is) to $BC$ [Prop. 5.11]. Thus, $DC$ is the mean proportional to $AC$ and
$CB$. Which (is the very thing) it was prescribed to show.}
\end{Parallel}

%%%%
%10.54
%%%%
\pdfbookmark[1]{Proposition 10.54}{pdf10.54}
\begin{Parallel}{}{}
\ParallelLText{
\begin{center}
{\large \ggn{54}.}
\end{center}\vspace*{-7pt}

\gr{>E`an qwr'ion peri'eqhtai <up`o <rht~hc ka`i t~hc >ek d'uo >onom'atwn
pr'wthc, <h t`o qwr'ion dunam'enh >'alog'oc >estin <h kaloum'enh >ek d'uo
>onom'atwn.}\\

\epsfysize=1.3in
\centerline{\epsffile{Book10/fig054g.eps}}

\gr{Qwr'ion g`ar t`o AG perieq'esjw <up`o <rht~hc t~hc AB ka`i t~hc >ek d'uo
>onom'atwn pr'wthc t~hc AD; l'egw, <'oti <h t`o AG qwr'ion dunam'enh
>'alog'oc >estin <h kaloum'enh >ek d'uo >onom'atwn.}

\gr{>Epe`i g`ar >ek d'uo >onom'atwn >est`i pr'wth <h AD, dih|r'hsjw
e>ic t`a >on'omata kat`a t`o E, ka`i >'estw t`o me~izon >'onoma t`o
AE. faner`on d'h, <'oti a<i AE, ED <rhta'i e>isi dun'amei m'onon s'ummetroi,
ka`i <h AE t~hc ED me~izon d'unatai t~w| >ap`o summ'etrou <eauth|, ka`i
<h AE s'ummetr'oc >esti t~h| >ekkeim'enh| <rht~h| t~h| AB m'hkei.
tetm'hsjw d`h <h ED d'iqa kat`a t`o Z shme~ion. ka`i >epe`i <h AE
t~hc ED me~izon d'unatai t~w| >ap`o summ'etrou <eaut~h|, >e`an >'ara
t~w| tet'artw| m'erei to~u >ap`o t~hc >el'assonoc, tout'esti t~w| >ap`o
t~hc EZ, >'ison par`a t`hn me'izona t`hn AE parablhj~h| >elle~ipon
e>'idei tetrag'wnw|, e>ic s'ummetra a>ut`hn diaire~i. parabebl'hsjw o~>un
par`a t`hn AE t~w| >ap`o t~hc EZ >'ison t`o <up`o AH, HE; s'ummetroc
>'ara >est`in <h AH t~h| EH m'hkei. ka`i >'hqjwsan >ap`o t~wn H, E, Z
<opot'era| t~wn AB, GD par'allhloi a<i HJ, EK, ZL; ka`i t~w| m`en
AJ parallhlogr'ammw| >'ison tetr'agwnon sunest'atw t`o SN, t~w| d`e
HK >'ison t`o NP, ka`i ke'isjw <'wste >ep> e>uje'iac e>~inai t`hn
MN t~h| NX; >ep> e>uje'iac >'ara >est`i ka`i <h RN t~h| NO. ka`i
sumpeplhr'wsjw t`o SP parallhl'ogrammon; tetr'agwnon >'ara >est`i
t`o SP. ka`i >epe`i t`o <up`o t~wn AH, HE >'ison >est`i t~w| >ap`o
t~hc EZ, >'estin >'ara <wc <h AH pr`oc EZ, o<'utwc <h ZE pr`oc EH; ka`i
<wc >'ara t`o AJ pr`oc EL, t`o EL pr`oc KH; t~wn AJ, HK >'ara m'eson
>an'alog'on >esti t`o EL. >all`a t`o m`en AJ >'ison >est`i t~w| SN, t`o d`e
HK >'ison t~w| NP; t~wn SN, NP >'ara m'eson >an'alog'on >esti t`o EL.
>'esti d`e t~wn a>ut~wn t~wn SN, NP m'eson >an'alogon ka`i t`o MR; >'ison
>'ara >est`i t`o EL t~w| MR; <'wste ka`i t~w| OX >'ison >est'in. >'esti
d`e ka`i t`a AJ, HK to~ic SN, NP >'isa; <'olon >'ara t`o AG >'ison >est`in
<'olw| t~w| SP, tout'esti t~w| >ap`o t~hc MX tetrag'wnw|; t`o AG
>'ara d'unatai <h MX. l'egw, <'oti <h MX >ek d'uo >onom'atwn >est'in.}

\gr{>Epe`i g`ar s'ummetr'oc >estin <h AH t~h| HE, s'ummetr'oc >esti
ka`i <h AE <ekat'era| t~wn AH, HE. <up'okeitai d`e ka`i <h AE  t~h| AB s'ummetroc; ka`i a<i
AH, HE >'ara t~h| AB s'ummetro'i e>isin. ka'i >esti <rht`h <h AB; <rht`h
>'ara >est`i ka`i <ekat'era t~wn AH, HE; <rht`on >'ara >est`in <ek'ateron
t~wn AJ, HK, ka'i >esti s'ummetron t`o AJ t~w| HK. >all`a t`o m`en AJ
t~w| SN >'ison >est'in, t`o d`e HK t~w| NP; ka`i
t`a SN, NP >'ara,
tout'esti t`a >ap`o t~wn MN, NX, <rht'a >esti ka`i s'ummetra. ka`i
>epe`i >as'ummetr'oc >estin <h AE t~h| ED m'hkei, >all>
<h m`en AE t~h| AH >esti s'ummetroc, <h d`e DE t~h| EZ s'ummetroc,
>as'ummetroc >'ara ka`i <h AH t~h| EZ;
<'wste ka`i t`o AJ t~w| EL >as'ummetr'on >estin. >all`a t`o m`en
AJ t~w| SN >estin >'ison, t`o d`e EL t~w| MR; ka`i t`o SN >'ara
t~w| MR >as'ummetr'on >estin. >all> <wc t`o SN pr`oc MR, <h ON
pr`oc t`hn NR; >as'ummetroc >'ara >est`in <h ON t~h| NR.
>'ish d`e <h m`en ON t~h| MN, <h d`e NR
t~h| NX; >as'ummetroc >'ara >est`in <h MN t~h| NX. ka'i
>esti t`o >ap`o t~hc MN s'ummetron t~w| >ap`o t~hc NX, ka`i
<rht`on <ek'ateron; a<i MN, NX >'ara <rhta'i e>isi
dun'amei m'onon s'ummetroi.}

\gr{<H MX >'ara >ek d'uo >onom'atwn >est`i ka`i d'unatai
t`o AG; <'oper >'edei de~ixai.}}

\ParallelRText{
\begin{center}
{\large Proposition 54}
\end{center}

If an area is contained by a rational (straight-line)
and a first binomial (straight-line) then  the  square-root  of the area is
the irrational (straight-line which is) called binomial.$^\dag$

\epsfysize=1.3in
\centerline{\epsffile{Book10/fig054e.eps}}

For let the area $AC$ be contained by the rational (straight-line) $AB$
and by the first binomial (straight-line) $AD$. I say that square-root of 
area $AC$ is the irrational (straight-line which is) called binomial.

For since $AD$ is a first binomial (straight-line), let it have been divided
into its (component) terms at $E$, and let $AE$ be the greater term.
So, (it is) clear that $AE$ and $ED$ are rational (straight-lines which are)
commensurable in square only, and that the square on $AE$ is greater
than (the square on) $ED$ by the (square) on (some straight-line)
commensurable (in length) with ($AE$), and that $AE$ is commensurable 
(in length) with
the rational (straight-line) $AB$ (first) laid out [Def. 10.5]. So, let $ED$ have been cut in half
at  point $F$. And since the square on $AE$ is greater than (the square on)
$ED$ by the (square) on (some straight-line) commensurable (in length) with ($AE$),
thus if a (rectangle) equal to the fourth part of the (square) on the lesser (term)---that is to say, the (square) on $EF$---falling short by a square figure,
is applied to the greater (term) $AE$, then it divides it into (terms which are)
commensurable (in length) [Prop~10.17]. 
Therefore, let the (rectangle contained) by $AG$ and $GE$, equal to
the (square) on $EF$, have been applied to $AE$. $AG$ 
is thus commensurable in length with $EG$. And let $GH$, $EK$, and $FL$
have been drawn from (points) $G$, $E$, and $F$ (respectively), parallel to
either of $AB$ or $CD$. And let the square $SN$, equal to the
parallelogram $AH$, have been constructed, and (the square) $NQ$,
equal to (the parallelogram) $GK$ [Prop. 2.14].
And let $MN$ be laid down so as to be straight-on to $NO$. $RN$
is thus also straight-on to $NP$. And let the parallelogram $SQ$ have
been completed. $SQ$ is thus a square [Prop. 10.53~lem.].  And since the (rectangle
contained) by $AG$ and $GE$ is equal to the (square) on $EF$, thus as
$AG$ is to $EF$, so $FE$ (is) to $EG$ [Prop. 6.17]. And thus as $AH$ (is) to
$EL$, (so) $EL$ (is) to $KG$ [Prop. 6.1]. Thus,
$EL$ is the mean proportional to $AH$ and $GK$. But, $AH$ is equal to
$SN$, and $GK$ (is) equal to $NQ$. $EL$ is thus the mean proportional
to $SN$ and $NQ$. And $MR$ is also the mean proportional to the
same---(namely), $SN$ and $NQ$ [Prop. 10.53~lem.]. $EL$ is thus equal to $MR$.
Hence, it is also equal to $PO$ [Prop. 1.43]. And
$AH$ plus $GK$ is equal to $SN$ plus $NQ$. Thus, the whole of $AC$
is equal to the whole of $SQ$---that is to say, to the square on $MO$.
Thus, $MO$ (is) the square-root of (area) $AC$.  I say that $MO$ is a
binomial (straight-line).

For since $AG$ is commensurable (in length) with $GE$, $AE$ is also commensurable (in length)
with each of $AG$ and $GE$ [Prop. 10.15]. 
And $AE$ was also assumed (to be) commensurable (in length) with $AB$.  Thus,
$AG$ and $GE$ are also commensurable (in length) with $AB$
[Prop. 10.12]. And $AB$ is rational. $AG$
and $GE$ are thus each also  rational. Thus, $AH$ and $GK$ are each
rational (areas), and $AH$ is commensurable with $GK$ [Prop. 10.19]. But, $AH$ is equal to $SN$,
and $GK$ to $NQ$. $SN$ and $NQ$---that is to say, the (squares)
on $MN$ and $NO$ (respectively)---are thus also rational and commensurable. And since $AE$ is incommensurable in length with $ED$,
but $AE$ is commensurable (in length) with $AG$, and $DE$ (is) commensurable (in length)
with $EF$, $AG$ (is) thus also incommensurable (in length) with $EF$
[Prop. 10.13]. Hence, $AH$ is also incommensurable with $EL$ [Props.~6.1, 10.11]. But, $AH$ is equal to $SN$, and $EL$
to $MR$. Thus, $SN$ is also incommensurable with $MR$. But, as
$SN$ (is) to $MR$, (so) $PN$ (is) to $NR$ [Prop. 6.1].  $PN$ is thus incommensurable (in length) with $NR$ [Prop. 10.11]. And $PN$ (is) equal to $MN$, and $NR$ to
$NO$. Thus, $MN$ is incommensurable (in length) with $NO$. And the (square)
on $MN$ is commensurable with the (square) on $NO$, and each (is)
rational.  $MN$ and $NO$ are thus rational (straight-lines which are)
commensurable in square only.

Thus, $MO$ is (both) a binomial (straight-line) [Prop. 10.36], and the square-root of $AC$.
(Which is) the very thing it was required to show.}
\end{Parallel}


\vspace{7pt}{\footnotesize\noindent $^\dag$ If the rational straight-line has unit length then this proposition states that the square-root of 
a first binomial straight-line is a binomial straight-line: {\em i.e.}, 
a first binomial straight-line has a length $k+k\,\sqrt{1-{k'}^{\,2}}$ whose
square-root can be written $\rho\,(1+\sqrt{k''})$, where $\rho=\sqrt{k\,(1+k')/2}$ and $k''=(1-k')/(1+k')$. This is the length of a binomial straight-line (see Prop.~10.36), since $\rho$ is rational.}

%%%%
%10.55
%%%%
\pdfbookmark[1]{Proposition 10.55}{pdf10.55}
\begin{Parallel}{}{}
\ParallelLText{
\begin{center}
{\large \ggn{55}.}
\end{center}\vspace*{-7pt}

\gr{>E`an qwr'ion peri'eqhtai <up`o <rht~hc ka`i t~hc >ek d'uo <onom'atwn
deut'erac, <h t`o qwr'ion dunam'enh >'alog'oc >estin <h kaloum'enh
>ek d'uo m'eswn pr'wth.}\\

\epsfysize=1.3in
\centerline{\epsffile{Book10/fig054g.eps}}

\gr{Perieq'esjw g`ar qwr'ion t`o ABGD <up`o <rht~hc t~hc AB  ka`i t~hc
>ek d'uo >onom'atwn duet'erac t~hc AD; l'egw, <'oti <h t`o AG qwr'ion
dunam'enh >ek d'uo m'eswn pr'wth >est'in.}

\gr{>Epe`i g`ar >ek d'uo >onom'atwn deut'era >est`in <h AD, dih|r'hsjw e>ic
t`a >on'omata kat`a t`o E, <'wste t`o me~izon >'onoma e>~inai t`o AE;
a<i AE, ED >'ara <rhta'i e>isi dun'amei m'onon s'ummetroi, ka`i <h AE
t~hc ED me~izon d'unatai t~w| >ap`o summ'etrou <eaut~h|, ka`i t`o
>'elatton >'onoma <h ED s'ummetr'on >esti t~h| AB m'hkei. tetm'hsjw
<h ED d'iqa kat`a t`o Z, ka`i t~w| >ap`o t~hc EZ >'ison par`a t`hn AE
parabebl'hsjw >elle~ipon e>'idei tetrag'wnw| t`o <up`o t~wn
AHE; s'ummetroc >'ara <h AH t~h| HE m'hkei. ka`i  di`a t~wn H, E, Z
par'allhloi >'hqjwsan ta~ic AB, GD a<i HJ, EK, ZL, ka`i t~w| m`en
AJ parallhlogr'ammw| >'ison tetr'agwnon sunest'atw t`o SN, t~w| d`e HK
>'ison tetr'agwnon 
t`o NP, ka`i ke'isjw <'wste >ep> e>uje'iac e>~inai t`hn MN t~h| NX; >ep>
e>uje'iac >'ara [>est`i] ka`i <h RN t~h NO. ka`i sumpeplhr'wsjw
t`o SP tetr'agwnon;  faner`on d`h >ek to~u prodedeigm'enou, <'oti t`o MR
m'eson >an'alog'on >esti t~wn SN, NP, ka`i >'ison t~w| EL, ka`i
<'oti t`o AG qwr'ion d'unatai <h MX. deikt'eon d'h, <'oti <h MX >ek
d'uo m'eswn >est`i pr'wth.}

\gr{>Epe`i >as'ummetr'oc >estin <h AE t~h| ED
m'hkei, s'ummetroc d`e <h ED t~h| AB, >as'ummetroc >'ara <h AE t~h|
AB. ka`i >epe`i s'ummetr'oc >estin <h AH t~h| EH, s'ummetr'oc >esti
ka`i <h AE <ekat'era| t~wn AH, HE. >all`a <h AE >as'ummetroc t~h|
AB m'hkei; ka`i a<i AH, HE >'ara >as'ummetro'i e>isi t~h| AB. a<i BA,
AH, HE >'ara <rhta'i e>isi dun'amei m'onon s'ummetroi; <'wste m'eson
>est`in <ek'ateron t~wn AJ, HK. <'wste ka`i <ek'ateron t~wn SN, NP
m'eson >est'in. ka`i a<i MN, NX >'ara m'esai e>is'in. ka`i >epe`i
s'ummetroc <h AH t~h| HE m'hkei, s'ummetr'on >esti ka`i t`o AJ t~w|
HK, tout'esti t`o SN t~w| NP, tout'esti t`o >ap`o t~hc MN t~w| >ap`o
t~hc NX [<'wste dun'amei e>is`i s'ummetroi a<i MN, NX]. ka`i
>epe`i >as'ummetr'oc >estin <h AE t~h| ED m'hkei, >all> <h m`en AE
s'ummetr'oc >esti t~h| AH, <h d`e ED t~h| EZ s'ummetroc, >as'ummetroc >'ara <h AH t~h| EZ; <'wste ka`i t`o AJ t~w| EL >as'ummetr'on >estin,
tout'esti t`o SN t~w| MR, tout'estin <o ON t~h| NR, tout'estin
<h MN t~h| NX >as'ummetr'oc >esti m'hkei.  >ede'iqjhsan d`e a<i MN,
NX ka`i m'esai o>~usai ka`i dun'amei s'ummetroi; a<i MN, NX >'ara
m'esai e>is`i dun'amei m'onon s'ummetroi. l'egw d'h, <'oti ka`i <rht`on
peri'eqousin. >epe`i g`ar <h DE <up'okeitai <ekat'era| t~wn AB, EZ s'ummetroc, s'ummetroc >'ara ka`i <h EZ t~h| EK. ka`i <rht`h <ekat'era
a>ut~wn; <rht`on >'ara t`o EL, tout'esti t`o MR; t`o d`e MR >esti
t`o <up`o t~wn MNX. >e`an d`e d'uo m'esai dun'amei m'onon s'ummetroi
suntej~wsi <rht`on peri'eqousai, <h <'olh >'alog'oc >estin, kale~itai d`e >ek d'uo m'eswn pr'wth.}

\gr{<H >'ara MX >ek d'uo m'eswn >est`i pr'wth; <'oper >'edei de~ixai.}}

\ParallelRText{
\begin{center}
{\large Proposition 55}
\end{center}

If an area is contained by a rational (straight-line)
and a second binomial (straight-line) then the square-root of the area
is the irrational (straight-line which is) called first bimedial.$^\dag$

\epsfysize=1.3in
\centerline{\epsffile{Book10/fig054e.eps}}

For let the area $ABCD$ be contained by the rational (straight-line) $AB$
and by the second binomial (straight-line)  $AD$. I say that the square-root
of  area $AC$ is a first bimedial (straight-line).\

For since $AD$ is a second binomial (straight-line), let it have been divided
into its (component) terms at $E$, such that $AE$ is the greater term. Thus,
$AE$ and $ED$ are rational (straight-lines which are) commensurable
in square only, and the square on $AE$ is greater than (the square on)
$ED$ by the (square) on (some straight-line) commensurable (in length) with ($AE$),
and the lesser term $ED$ is commensurable in length with $AB$ [Def. 10.6].  Let $ED$ have been cut in half at $F$.
And let the (rectangle contained) by $AGE$, equal to the (square) on $EF$,
 have been applied to $AE$, falling short by a square figure. $AG$
(is) thus commensurable in length with $GE$ [Prop. 10.17]. And let $GH$, $EK$, and
$FL$ have been drawn through (points) $G$, $E$, and $F$ (respectively), parallel
to $AB$ and $CD$. And let the square $SN$, equal to the parallelogram
$AH$, have been constructed, and the square $NQ$, equal to
$GK$.  And let $MN$ be laid down so as to be straight-on to $NO$. Thus,
$RN$ [is] also straight-on to $NP$. And let the square $SQ$ have been
completed. So, (it is) clear from what has been previously
demonstrated [Prop. 10.53~lem.] that
$MR$ is the mean proportional to $SN$ and $NQ$, and (is) equal to
$EL$, and that $MO$ is the square-root of the area $AC$. So, we
must show that $MO$ is a first bimedial (straight-line).

Since $AE$
is incommensurable in length with $ED$, and $ED$ (is) commensurable 
 (in length) with $AB$, $AE$ (is) thus incommensurable (in length) with $AB$ [Prop. 10.13]. And since $AG$
is commensurable (in length) with $EG$, $AE$ is also commensurable
(in length) with each of $AG$ and $GE$ [Prop. 10.15]. 
But, $AE$ is incommensurable in length with $AB$.  Thus, $AG$ and $GE$
are also (both) 
incommensurable (in length) with $AB$ [Prop. 10.13]. 
Thus, $BA$, $AG$,  and ($BA$, and) $GE$ are (pairs of) rational (straight-lines which are)
commensurable in square only. And, hence, each of $AH$ and $GK$ is a medial
(area) [Prop. 10.21]. Hence, each of $SN$ and $NQ$
is also a medial (area). Thus, $MN$ and $NO$ are medial (straight-lines). 
And since $AG$ (is) commensurable in length with $GE$, $AH$
is also commensurable with $GK$---that is to say, $SN$ with $NQ$---that is to say, the (square)
on $MN$  with the (square) on $NO$ [hence, $MN$ and $NO$
are commensurable in square] [Props.~6.1, 10.11]. And since $AE$ is incommensurable
in length with $ED$, but $AE$ is commensurable (in length) with $AG$,
and $ED$ commensurable (in length) with $EF$, $AG$ (is) thus
incommensurable (in length) with $EF$ [Prop. 10.13]. Hence, $AH$ is also incommensurable with $EL$---that is to say, $SN$ with $MR$---that is to
say, $PN$ with $NR$---that is to say, $MN$ is incommensurable
in length with $NO$  [Props.~6.1, 10.11]. But $MN$ and $NO$ have also
been shown to be medial (straight-lines) which are commensurable in square. 
Thus, $MN$ and $NO$ are medial (straight-lines which are) commensurable
in square only.
So,
I say that they also contain a rational (area). For since $DE$ was assumed
(to be) commensurable (in length) with each of $AB$ and $EF$, $EF$
(is) thus also commensurable with $EK$ [Prop. 10.12]. And they (are) each rational.
Thus, $EL$---that is to say, $MR$---(is) rational [Prop. 10.19]. And $MR$ is the (rectangle contained) by $MNO$. And if two medial (straight-lines), commensurable
in square only, which contain a rational (area), are added together, then the
whole is (that) irrational (straight-line which is)
called first bimedial [Prop. 10.37].

Thus, $MO$ is a first bimedial (straight-line). (Which is) the very thing it
was required to show.}
\end{Parallel}


\vspace{7pt}{\footnotesize\noindent$^\dag$ If the rational straight-line has unit length then this proposition states that the square-root of 
a second binomial straight-line is a first bimedial straight-line: {\em i.e.}, 
a second binomial straight-line has a length $k/\sqrt{1-{k'}^{\,2}}+k$ whose
square-root can be written $\rho\,({k''}^{1/4}+{k''}^{3/4})$, where $\rho=\sqrt{(k/2)\,(1+k')/(1-k')}$ and $k''=(1-k')/(1+k')$. This is the length of a first bimedial straight-line (see Prop.~10.37), since $\rho$ is rational.}

%%%%
%10.56
%%%%
\pdfbookmark[1]{Proposition 10.56}{pdf10.56}
\begin{Parallel}{}{}
\ParallelLText{
\begin{center}
{\large \ggn{56}.}
\end{center}\vspace*{-7pt}

\gr{>E`an qwr'ion peri'eqhtai <up`o <rht~hc ka`i t~hc >ek
d'uo >onom'atwn tr'ithc, <h t`o qwr'ion dunam'enh >'alog'oc >estin
<h kaloum'enh >ek d'uo m'eswn deut'era.}\\

\epsfysize=1.3in
\centerline{\epsffile{Book10/fig054g.eps}}

\gr{Qwr'ion g`ar t`o ABGD perieq'esjw <up`o <rht~hc t~hc AB ka`i t~hc
>ek d'uo >onom'atwn tr'ithc t~hc AD dih|rhm'enhc e>ic t`a >on'omata
kat`a t`o E, <~wn me~iz'on >esti t`o AE; l'egw,  <'oti <h t`o AG
qwr'ion dunam'enh >'alog'oc >estin <h kaloum'enh >ek d'uo m'eswn
deut'era.}

\gr{Kateskeu'asjw g`ar t`a a>ut`a to~ic pr'oteron. ka`i >epe`i >ek d'uo
>onom'atwn >est`i tr'ith <h AD, a<i AE, ED >'ara <rhta'i e>isi
dun'amei m'onon s'ummetroi, ka`i <h AE t~hc ED me~izon d'unatai
t~w| >ap`o summ'etrou <eaut~h|, ka`i o>udet'era t~wn AE, ED
s'ummetr'oc [>esti] t~h| AB m'hkei. <omo'iwc d`h to~ic prodedeigm'enoic
de'ixomen, <'oti <h MX >estin <h t`o AG qwr'ion dunam'enh, ka`i a<i MN,
NX m'esai e>is`i dun'amei m'onon s'ummetroi; <'wste <h MX >ek d'uo
m'eswn >est'in. deikt'eon d'h, <'oti ka`i deut'era.}

\gr{\mbox{[}Ka`i] >epe`i >as'ummetr'oc >estin <h DE t~h| AB m'hkei, tout'esti
t~h| EK, s'ummetroc d`e <h DE t~h| EZ, >as'ummetroc >'ara >est`in
<h EZ t~h| EK m'hkei. ka'i e>isi <rhta'i; a<i ZE, EK >'ara <rhta'i
e>isi dun'amei m'onon s'ummetroi. m'eson >'ara [>est`i] t`o EL,
tout'esti t`o MR; ka`i peri'eqetai <up`o t~wn MNX; m'eson >'ara >est`i
t`o <up`o t~wn MNX.}

\gr{<H MX >'ara >ek d'uo m'eswn >est`i deut'era; <'oper >'edei de~ixai.}}

\ParallelRText{
\begin{center}
{\large Proposition 56}
\end{center}

If an area is contained by a rational (straight-line)
and a third binomial (straight-line) then the square-root of the area is
the irrational (straight-line which is) called second bimedial.$^\dag$

\epsfysize=1.3in
\centerline{\epsffile{Book10/fig054e.eps}}

For let the area $ABCD$ be contained by the rational (straight-line)
$AB$ and by the third binomial (straight-line) $AD$, which has been divided into
its (component) terms at $E$, of which $AE$ is the greater.  I say that
the square-root of area $AC$ is the irrational (straight-line which is)
called second bimedial.

For let the same construction be made as  previously. And since
$AD$ is a third binomial (straight-line), $AE$ and $ED$
are thus rational (straight-lines which are) commensurable in square only,
and the square on $AE$ is greater than (the square on) $ED$
by the (square) on (some straight-line) commensurable (in length) with
($AE$), and neither of $AE$ and $ED$ [is] commensurable in length
with $AB$ [Def. 10.7]. So, similarly
to that which has been previously demonstrated, we can show that
$MO$ is the square-root of  area $AC$, and $MN$ and $NO$
are medial (straight-lines which are) commensurable in square only.
Hence, $MO$ is bimedial. So, we must show that (it is)
also second (bimedial).

\mbox{[}And] since $DE$ is incommensurable in length with $AB$---that is to say,
with $EK$---and $DE$ (is) commensurable (in length) with $EF$,
$EF$ is thus incommensurable in length with $EK$ [Prop. 10.13]. And they are (both)
rational (straight-lines). Thus, $FE$ and $EK$ are rational (straight-lines which are) commensurable in square only. $EL$---that is to say, $MR$---[is] thus medial [Prop. 10.21]. And it
is contained by $MNO$.  Thus, the (rectangle contained) by $MNO$
is medial.

Thus, $MO$ is a second bimedial (straight-line) [Prop. 10.38]. (Which is) the very thing it was required to show.}
\end{Parallel}


\vspace{7pt}{\footnotesize\noindent $^\dag$ If the rational straight-line has unit length then this proposition states that the square-root of 
a third binomial straight-line is a second bimedial straight-line: {\em i.e.}, 
a third binomial straight-line has a length $k^{1/2}\,(1+\sqrt{1-{k'}^{\,2}})$ whose
square-root can be written $\rho\,(k^{1/4}+{k''}^{1/2}/k^{1/4})$, where $\rho=\sqrt{(1+k')/2}$ and $k''=k\,(1-k')/(1+k')$. This is the length of a second bimedial straight-line (see Prop.~10.38), since $\rho$ is rational.}

%%%%
%10.57
%%%%
\pdfbookmark[1]{Proposition 10.57}{pdf10.57}
\begin{Parallel}{}{}
\ParallelLText{
\begin{center}
{\large \ggn{57}.}
\end{center}\vspace*{-7pt}

\gr{>E`an qwr'ion peri'eqhtai <up`o <rht~hc ka`i t~hc >ek d'uo >onom'atwn
tet'arthc, <h t`o qwr'ion dunam'enh >'alog'oc >estin <h kaloum'enh 
me'izwn.}\\

\epsfysize=1.3in
\centerline{\epsffile{Book10/fig054g.eps}}

\gr{Qwr'ion g`ar t`o AG perieq'esjw <up`o <rht~hc t~hc AB ka`i t~hc
>ek d'uo >onom'atwn tet'arthc t~hc AD dih|rhm'enhc e>ic t`a
>on'omata kat`a t`o E, <~wn me~izon >'estw t`o AE; l'egw, <'oti
<h t`o AG qwr'ion dunam'enh >'alog'oc >estin <h kaloum'enh
me'izwn.}

\gr{>Epe`i g`ar <h AD >ek d'uo >onom'atwn >est`i tet'arth, a<i AE, ED
>'ara <rhta'i e>isi dun'amei m'onon s'ummetroi, ka`i <h AE t~hc
ED me~izon d'unatai t~w| >ap`o >asumm'etrou <eaut~h|, ka`i <h AE
t~h| AB s'ummetr'oc [>esti] m'hkei. tetm'hsjw <h DE d'iqa kat`a
t`o Z, ka`i t~w| >ap`o t~hc EZ >'ison par`a t`hn AE parabebl'hsjw
parallhl'ogrammon t`o <up`o AH, HE; >as'ummetroc >'ara >est`in
<h AH t~h| HE m'hkei. >'hqjwsan par'allhloi t~h| AB a<i HJ, EK, 
ZL, ka`i t`a loip`a t`a a>ut`a to~ic pr`o to'utou gegon'etw;
faner`on d'h, <'oti <h t`o AG qwr'ion dunam'enh >est`in <h
MX. deikt'eon d'h, <'oti <h MX >'alog'oc >estin <h kaloum'enh
me'izwn.}

\gr{>Epe`i >as'ummetr'oc >estin <h AH t~h| EH m'hkei, >as'ummetr'on
>esti ka`i t`o AJ t~w| HK, tout'esti t`o SN t~w| NP; a<i MN, NX >'ara dun'amei e>is`in >as'ummetroi. ka`i >epe`i s'ummetr'oc >estin <h AE
t~h| AB m'hkei, <rht'on >esti t`o AK; ka'i >estin >'ison to~ic
>ap`o t~wn MN, NX; <rht`on >'ara [>est`i] ka`i t`o sugke'imenon
>ek t~wn >ap`o t~wn MN, NX. ka`i >epe`i >as'ummetr'oc
[>estin] <h DE t~h| AB m'hkei, tout'esti  t~h| EK, >all`a
<h DE s'ummetr'oc >esti t~h| EZ, >as'ummetroc >'ara <h EZ
t~h| EK m'hkei. a<i EK, EZ >'ara <rhta'i e>isi dun'amei m'onon s'ummetroi;
m'eson >'ara t`o LE, tout'esti t`o MR. ka`i peri'eqetai <up`o t~wn
MN, NX; m'eson >'ara >est`i t`o <up`o t~wn MN, NX. ka`i <rht`on
t`o [sugke'imenon] >ek t~wn >ap`o t~wn MN, NX, ka'i e>isin >as'ummetroi
a<i MN, NX dun'amei. >e`an d`e d'uo e>uje~iai dun'amei >as'ummetroi
suntej~wsi poio~usai t`o m`en sugke'imenon >ek t~wn >ap> a>ut~wn
tetrag'wnwn <rht'on, t`o d> <up> a>ut~wn m'eson, <h <'olh >'alog'oc
>estin, kale~itai d`e me'izwn.}

\gr{<H MX >'ara >'alog'oc >estin <h kaloum'enh me'izwn, ka`i d'unatai
t`o AG qwr'ion; <'oper >'edei de~ixai.}}

\ParallelRText{
\begin{center}
{\large Proposition 57}
\end{center}

If an area is contained by a rational (straight-line)
and a fourth binomial (straight-line) then the square-root of the area
is the irrational (straight-line which is) called major.$^\dag$

\epsfysize=1.3in
\centerline{\epsffile{Book10/fig054e.eps}}

For let the area $AC$ be contained by the rational (straight-line) $AB$
and the fourth binomial (straight-line) $AD$, which has been divided into
its (component) terms at $E$, of which let $AE$ be the greater. I say that
the square-root of $AC$ is the irrational (straight-line which is) called major.

For since $AD$ is a fourth binomial (straight-line), $AE$ and $ED$
are thus rational (straight-lines which are) commensurable in square only,
and the square on $AE$ is greater than (the square on) $ED$ by the
(square) on (some straight-line) incommensurable (in length) with ($AE$),
and $AE$ [is] commensurable in length with $AB$ [Def. 10.8]. Let $DE$ have been cut in half at $F$,
and let the parallelogram (contained by) $AG$ and $GE$, equal to the
(square) on $EF$, (and falling short by a square figure) have been applied to
$AE$. $AG$ is thus incommensurable in length with $GE$ [Prop. 10.18]. Let $GH$, $EK$, and $FL$
have been drawn parallel to $AB$, and let the rest (of the construction) have been made the same as the (proposition) before this. So, it is clear that
$MO$ is the square-root of  area $AC$. So, we must show that $MO$
is the irrational (straight-line which is) called major.

Since $AG$ is incommensurable in length with $EG$, $AH$
is also incommensurable with $GK$---that is to say, $SN$
 with $NQ$ [Props.~6.1, 10.11].  Thus, $MN$ and $NO$
 are incommensurable in square. And since $AE$ is
 commensurable in length with $AB$, $AK$ is rational [Prop. 10.19].  And it is equal to the (sum of the squares) on $MN$ and $NO$.  Thus, the sum of the (squares) on 
 $MN$ and $NO$ [is] also rational. And since $DE$ [is]  incommensurable
 in length with $AB$ [Prop. 10.13]---that is to say, with $EK$---but $DE$ is commensurable (in length) with $EF$, $EF$ (is) thus incommensurable
 in length with $EK$ [Prop. 10.13].
 Thus, $EK$ and $EF$ are rational (straight-lines which are) commensurable
 in square only. $LE$---that is to say, $MR$---(is) thus medial
 [Prop. 10.21]. And it is contained by $MN$ and $NO$. The (rectangle contained) by $MN$ and $NO$ is thus medial. And
 the [sum] of the (squares) on $MN$ and $NO$ (is) rational, and
 $MN$ and $NO$ are incommensurable in square. And if two
 straight-lines (which are) incommensurable in square, making the
 sum of the squares on them rational, and the (rectangle contained) by them
 medial, are added together, then the whole is the irrational (straight-line which is) called major [Prop. 10.39].
 
 Thus, $MO$ is the irrational (straight-line which is) called major.
 And (it is) the square-root of area $AC$. (Which is) the very thing it
 was required to show.}
\end{Parallel}
 

\vspace{7pt}{\footnotesize\noindent$^\dag$ If the rational straight-line has unit length then this proposition states that the square-root of 
a fourth binomial straight-line is a major straight-line: {\em i.e.}, 
a fourth binomial straight-line has a length $k\,(1+1/\sqrt{1+k'})$ whose
square-root can be written $\rho\,\sqrt{[1+k''/(1+{k''}^{\,2})^{1/2}]/2}+\rho\,\sqrt{[1-k''/(1+{k''}^{\,2})^{1/2}]/2}$, where $\rho=\sqrt{k}$ and ${k''}^{\,2}=k'$. This is the length of a major straight-line (see Prop.~10.39), since $\rho$ is rational.}

%%%%
%10.58
%%%%
\pdfbookmark[1]{Proposition 10.58}{pdf10.58}
\begin{Parallel}{}{}
\ParallelLText{
\begin{center}
{\large \ggn{58}.}
\end{center}\vspace*{-7pt}

\gr{>E`an qwr'ion peri'eqhtai <up`o <rht~hc ka`i t~hc >ek d'uo >onom'atwn
p'empthc, <h t`o qwr'ion dunam'enh >'alog'oc >estin <h kaloum'enh
<rht`on ka`i m'eson dunam'enh.}

\gr{Qwr'ion g`ar t`o AG perieq'esjw <up`o <rht~hc t~hc AB ka`i t~hc >ek
d'uo >onom'atwn p'empthc t~hc AD dih|rhm'enhc e>ic t`a >on'omata
kat`a t`o E, <'wste t`o me~izon >'onoma e>~inai t`o AE; l'egw [d'h],
<'oti <h t`o AG qwr'ion dunam'enh >'alog'oc >estin <h kaloum'enh
<rht`on ka`i m'eson dunam'enh.}\\~\\~\\

\epsfysize=1.35in
\centerline{\epsffile{Book10/fig054g.eps}}

\gr{Kateskeu'asjw g`ar t`a a>ut`a to~ic pr'oteron dedeigm'enoic; faner`on
d'h, <'oti <h t`o AG qwr'ion dunam'enh >est`in <h MX. deikt'eon
d'h, <'oti <h MX >estin <h <rht`on ka`i m'eson dunam'enh.}

\gr{>Epe`i
g`ar >as'ummetr'oc >estin <h AH t~h| HE, >as'ummet\-ron >'ara >est`i
ka`i t`o AJ t~w| JE, tout'esti t`o >ap`o t~hc MN t~w| >ap`o
t~hc NX; a<i MN, NX >'ara dun'amei e>is`in >as'ummetroi. ka`i
>epe`i <h AD >ek d'uo >onom'atwn >est`i p'empth, ka'i [>estin]
>'elasson a>ut~hc tm~hma t`o ED, s'ummetroc >'ara <h ED t~h| AB
m'hkei. >all`a <h AE t~h| ED >estin >as'ummetroc; ka`i <h AB >'ara
t~h| AE >estin >as'ummetroc m'hkei [a<i BA, AE <rhta'i e>isi dun'amei
m'onon s'ummetroi]; m'eson >'ara >est`i t`o AK, tout'esti t`o sugke'imenon
>ek t~wn >ap`o t~wn MN, NX. ka`i >epe`i s'ummetr'oc >estin <h DE
t~h| AB m'hkei, tout'esti t~h| EK, >all`a <h DE t~h| EZ s'ummetr'oc
>estin, ka`i <h EZ >'ara t~h| EK s'ummetr'oc >estin. ka`i <rht`h <h
EK; <rht`on >'ara ka`i t`o EL, tout'esti t`o MR, tout'esti t`o <up`o
MNX; a<i MN, NX >'ara dun'amei >as'ummetro'i e>isi poio~usai
t`o m`en sugke'imenon >ek t~wn >ap> a>ut~wn tetrag'wnwn
m'eson, t`o d> <up> a>ut~wn <rht'on.}

\gr{<H MX >'ara <rht`on ka`i m'eson dunam'enh >est`i ka`i d'unatai
t`o AG qwr'ion; <'oper >'edei de~ixai.}}

\ParallelRText{
\begin{center}
{\large Proposition 58}
\end{center}

If an area is contained by a rational (straight-line)
and a fifth binomial (straight-line) then the square-root of the area is
the irrational (straight-line which is) called the square-root of a rational
plus a medial (area).$^\dag$

For let the area $AC$ be contained by the rational (straight-line) $AB$
and the fifth binomial (straight-line) $AD$, which has been divided into its
(component) terms at $E$, such that $AE$ is the greater term. [So] I say that
the square-root of area $AC$ is the irrational (straight-line which is)
called the square-root of a rational plus a medial (area).

\epsfysize=1.35in
\centerline{\epsffile{Book10/fig054e.eps}}

For let the same construction  be made as that shown previously.
So, (it is) clear that $MO$ is the square-root of area $AC$.
So, we must show that $MO$ is the square-root of a rational plus a medial
(area).

 For since $AG$ is incommensurable (in length) with $GE$ [Prop. 10.18], $AH$ is thus also
 incommensurable with $HE$---that is to say, the (square) on $MN$
 with the (square) on $NO$ [Props.~6.1, 10.11]. Thus, $MN$ and $NO$
 are incommensurable in square. And since $AD$ is a fifth
 binomial (straight-line), and $ED$ [is] its lesser segment, $ED$
 (is) thus commensurable in length with $AB$ [Def. 10.9]. But, $AE$ is incommensurable (in length) with $ED$. Thus, $AB$ is also incommensurable in length with $AE$
 [$BA$ and $AE$ are rational (straight-lines which are) commensurable
 in square only] [Prop. 10.13]. Thus, $AK$---that
 is to say, the sum of the (squares) on $MN$ and $NO$---is medial [Prop. 10.21]. And since $DE$ is
 commensurable in length with $AB$---that is to say, with $EK$---but,
 $DE$ is commensurable (in length) with $EF$, $EF$ is thus
also  commensurable (in length) with $EK$ [Prop. 10.12]. And $EK$ (is) rational. Thus, $EL$---that is to say, $MR$---that is to say, the (rectangle contained) by
$MNO$---(is) also rational [Prop. 10.19]. 
$MN$ and $NO$ are thus (straight-lines which are) incommensurable in square, making
the sum of the squares on them medial, and the (rectangle contained)
by them rational.

Thus, $MO$ is the square-root of a rational plus a medial (area) [Prop. 10.40]. And (it is) the square-root
of area $AC$. (Which is) the very thing it was required to show.}
\end{Parallel}


\vspace{7pt}{\footnotesize\noindent $^\dag$
If the rational straight-line has unit length then this proposition states that the square-root of 
a fifth binomial straight-line is the square root of a rational plus a medial area: {\em i.e.}, 
a fifth binomial straight-line has a length $k\,(\sqrt{1+k'}+1)$ whose
square-root can be written $\rho\,\sqrt{[(1+{k''}^{\,2})^{1/2}+k'']/[2\,(1+{k''}^{\,2})]}+\rho\,\sqrt{[(1+{k''}^{\,2})^{1/2}-k'']/[2\,(1+{k''}^{\,2})]}$, where $\rho=\sqrt{k\,(1+{k''}^{\,2})}$ and ${k''}^{\,2}=k'$. This is the length of the square root of a rational plus a medial area (see Prop.~10.40), since $\rho$ is rational.}

%%%%
%10.59
%%%%
\pdfbookmark[1]{Proposition 10.59}{pdf10.59}
\begin{Parallel}{}{}
\ParallelLText{
\begin{center}
{\large \ggn{59}.}
\end{center}\vspace*{-7pt}

\gr{>E`an qwr'ion peri'eqhtai <up`o <rht~hc ka`i t~hc >ek d'uo >onom'atwn
<'ekthc, <h t`o qwr'ion dunam'enh >'alog'oc >estin <h kaloum'enh
d'uo m'esa dunam'enh.}\\

\epsfysize=1.35in
\centerline{\epsffile{Book10/fig054g.eps}}

\gr{Qwr'ion g`ar t`o ABGD perieq'esjw <up`o <rht~hc t~hc AB ka`i t~hc
>ek d'uo >onom'atwn <'ekthc t~hc AD dih|rhm'enhc e>ic t`a >on'omata
kat`a t`o E, <'wste t`o me~izon >'onoma e>~inai t`o AE; l'egw, <'oti
<h t`o AG dunam'enh <h d'uo m'esa dunam'enh >est'in.}

\gr{Kateskeu'asjw [g`ar] t`a a>ut`a to~ic prodedeigm'enoic. faner`on d'h,
<'oti [<h] t`o AG dunam'enh >est`in <h MX, ka`i <'oti >as'ummetr'oc >estin
<h MN t~h| NX dun'amei. ka`i >epe`i >as'ummetr'oc >estin <h EA t~h|
AB m'hkei, a<i EA, AB >'ara <rhta'i e>isi dun'amei m'onon s'ummetroi; m'eson >'ara >est`i t`o AK, tout'esti t`o sugke'imenon >ek t~wn >ap`o
t~wn MN, NX. p'alin, >epe`i >as'ummetr'oc >estin <h ED t~h| AB m'hkei,
>as'ummetroc >'ara >est`i ka`i <h ZE t~h| EK; a<i ZE, EK >'ara
<rhta'i e>isi dun'amei m'onon s'ummetroi; m'eson >'ara >est`i t`o EL,
tout'esti t`o MR, tout'esti t`o <up`o t~wn MNX. ka`i >epe`i >as'ummetroc
<h AE t~h| EZ, ka`i t`o AK t~w| EL >as'ummetr'on >estin. >all`a t`o m`en
AK >esti t`o sugke'imenon >ek t~wn >ap`o t~wn MN, NX,
t`o d`e EL >esti t`o <up`o t~wn MNX; >as'ummetron >'ara >est`i t`o
sugke'imenon >ek t~wn >ap`o t~wn MNX t~w| <up`o
t~wn MNX. ka'i >esti m'eson <ek'ateron a>ut~wn, ka`i a<i MN, NX
dun'amei e>is`in >as'ummetroi.}

\gr{<H MX >'ara d'uo m'esa dunam'enh >est`i ka`i d'unatai t`o 
AG; <'oper >'edei de~ixai.}}

\ParallelRText{
\begin{center}
{\large Proposition 59}
\end{center}

If an area is contained by a rational (straight-line)
and a sixth binomial (straight-line) then the square-root of the
area is the irrational (straight-line which is)  called the square-root
of (the sum of) two medial (areas).$^\dag$

\epsfysize=1.35in
\centerline{\epsffile{Book10/fig054e.eps}}

For let the area $ABCD$ be contained by the rational (straight-line)
$AB$ and the sixth binomial (straight-line) $AD$, which has been
divided into its (component) terms at $E$, such that $AE$ is the greater
term. So, I say that the square-root of $AC$ is the square-root
of (the sum of) two medial (areas).

\mbox{[}For] let the same construction  be made as that shown previously. So, (it is)
clear that $MO$ is the square-root of $AC$, and that $MN$ is incommensurable
in square with $NO$. And since $EA$ is incommensurable
in length with $AB$ [Def. 10.10], $EA$ and $AB$ are thus rational (straight-lines
which are) commensurable in square only. Thus, $AK$---that is to
say, the sum of the (squares) on $MN$ and $NO$---is medial [Prop. 10.21]. Again, since  $ED$ is incommensurable in length with $AB$  [Def. 10.10], $FE$ is thus also incommensurable 
(in length) with $EK$ [Prop. 10.13]. 
Thus, $FE$ and $EK$ are rational (straight-lines which are) commensurable
in square only. Thus, $EL$---that is to say,  $MR$---that is to say,
the (rectangle contained) by $MNO$---is medial [Prop. 10.21]. And since $AE$ is incommensurable
(in length) with $EF$, $AK$ is also incommensurable with $EL$
[Props.~6.1, 10.11]. 
But, $AK$ is the sum of the (squares) on $MN$ and $NO$, and $EL$
is the (rectangle contained) by $MNO$. Thus,
the sum of the (squares) on $MNO$ is incommensurable with the (rectangle
contained) by $MNO$. And each of them is medial.  And $MN$ and
$NO$ are incommensurable in square.

Thus, $MO$ is the square-root of (the sum of) two medial (areas) [Prop. 10.41].
And (it is) the square-root of $AC$. (Which is) the very thing it was required
to show.}
\end{Parallel}


\vspace{7pt}{\footnotesize\noindent$^\dag$ If the rational straight-line has unit length then this proposition states that the square-root of 
a sixth binomial straight-line is the square root of the sum of  two  medial areas: {\em i.e.}, 
a sixth binomial straight-line has a length $\sqrt{k}+\sqrt{k'}$ whose
square-root can be written\\ $k^{1/4}\left(\sqrt{[1+k''/(1+{k''}^{\,2})^{1/2}]/2}+\sqrt{[1-k''/(1+{k''}^{\,2})^{1/2}]/2}\right)$, where ${k''}^{\,2}=(k-k')/k'$. This is the length of the square-root of the sum of two medial areas (see Prop.~10.41).}~\\

%%%%
%10.59a
%%%%
\begin{Parallel}{}{}
\ParallelLText{
\begin{center}
{\large \gr{L~hmma}.}
\end{center}\vspace*{-7pt}

\gr{>E`an e>uje~ia gramm`h tmhj~h| e>ic >'anisa, t`a >ap`o t~wn >an'iswn
tetr'agwna me'izon'a >esti to~u d`ic <up`o t~wn >an'iswn perieqom'enou
>orjogwn'iou.}

\epsfysize=0.3in
\centerline{\epsffile{Book10/fig059ag.eps}}

\gr{>'Estw e>uje~ia <h AB ka`i tetm'hsjw e>ic >'anisa kat`a t`o G, ka`i >'estw
me'izwn <h AG; l'egw, <'oti t`a >ap`o t~wn AG, GB me'izon'a >esti
to~u d`ic <up`o t~wn AG, GB.}

\gr{Tetm'hsjw g`ar <h AB d'iqa kat`a t`o D. >epe`i o>~un e>uje~ia
gramm`h t'etmhtai e>ic m`en >'isa kat`a t`o D, e>ic
d`e >'anisa kat`a t`o G, t`o >'ara <up`o t~wn AG, GB met`a
to~u >ap`o GD >'ison >est`i t~w| >ap`o AD; <'wste t`o <up`o
t~wn AG, GB >'elatt'on >esti to~u >ap`o AD; t`o >'ara d`ic
<up`o t~wn AG, GB >'elatton
 >`h dipl'asi'on >esti to~u >ap`o
AD. >all`a t`a >ap`o t~wn AG, GB dipl'asi'a [>esti] t~wn
>ap`o t~wn AD, DG; t`a >'ara >ap`o t~wn AG, GB me'izon'a
>esti to~u d`ic <up`o t~wn AG, GB; <'oper >'edei de~ixai.}}

\ParallelRText{
\begin{center}
{\large Lemma}
\end{center}

If a straight-line is cut unequally then 
(the sum of) the squares on the unequal (parts) is greater than twice
the rectangle contained by the unequal (parts).

\epsfysize=0.3in
\centerline{\epsffile{Book10/fig059ae.eps}}

Let $AB$ be a straight-line, and let it have been cut unequally at $C$, and
let $AC$ be greater (than $CB$). I say that (the sum of) the (squares) on
$AC$ and $CB$ is greater than twice the (rectangle contained) by $AC$ and
$CB$.

For let $AB$ have been cut in half at $D$. Therefore, since a straight-line
has been cut into equal (parts) at $D$, and into unequal (parts) at $C$,
 the (rectangle contained) by $AC$ and $CB$, plus the (square)
on $CD$, is thus equal to the (square) on $AD$ [Prop. 2.5]. Hence, the (rectangle contained) by $AC$ and $CB$ is less than the (square) on $AD$. Thus, twice the
(rectangle contained) by $AC$ and $CB$ is less than double the (square) on
$AD$. But, (the sum of) the (squares) on $AC$ and $CB$ [is] double (the sum of) the (squares) on $AD$ and $DC$ [Prop. 2.9].
Thus, (the sum of) the (squares) on $AC$ and $CB$ is greater than twice
the (rectangle contained) by $AC$ and $CB$. (Which is) the very thing it
was required to show.}
\end{Parallel}

%%%%
%10.60
%%%%
\pdfbookmark[1]{Proposition 10.60}{pdf10.60}
\begin{Parallel}{}{}
\ParallelLText{
\begin{center}
{\large \ggn{60}.}
\end{center}\vspace*{-7pt}

\gr{T`o >ap`o t~hc >ek d'uo >onom'atwn par`a <rht`hn  paraball'omenon pl'atoc
poie~i t`hn >ek d'uo >onom'atwn pr'wthn.}\\

\epsfysize=1.5in
\centerline{\epsffile{Book10/fig060g.eps}}

\gr{>'Estw >ek d'uo >onom'atwn <h AB dih|rhm'enh e>ic t`a >on'omata kat`a
t`o G, <'wste t`o me~izon >'onoma e>~inai t`o AG, ka`i >ekke'isjw
<rht`h <h DE, ka`i t~w| >ap`o t~hc AB >'ison par`a t`hn DE parabebl'hsjw
t`o DEZH pl'atoc poio~un t`hn DH; l'egw, <'oti <h DH >ek d'uo >onom'atwn
>est`i pr'wth.}

\gr{Parabebl'hsjw g`ar par`a t`hn DE t~w| m`en >ap`o t~hc AG >'ison t`o DJ,
t~w| d`e >ap`o t~hc BG >'ison t`o KL; loip`on >'ara t`o d`ic <up`o t~wn
AG, GB >'ison >est`i t~w| MZ. tetm'hsjw <h MH d'iqa kat`a t`o N, ka`i
par'allhloc >'hqjw <h NX [<ekat'era| t~wn ML, HZ]. <ek'ateron >'ara
t~wn MX, NZ >'ison >est`i t~w| <'apax <up`o t~wn AGB. ka`i >epe`i
>ek d'uo >onom'atwn >est`in <h AB dih|rhm'enh e>ic t`a >on'omata
kat`a t`o G, a<i AG, GB >'ara <rhta'i e>isi dun'amei m'onon s'ummetroi;
t`a >'ara >ap`o t~wn AG, GB <rht'a >esti ka`i s'ummetra >all'hloic;
<'wste ka`i t`o sugke'imenon >ek t~wn >ap`o t~wn AG, GB.
ka'i >estin >'ison t~w| DL; <rht`on >'ara
>est`i t`o DL. ka`i par`a <rht`hn t`hn DE par'akeitai; <rht`h >'ara >est`in
<h DM ka`i s'ummetroc t~h| DE m'hkei. p'alin, >epe`i a<i AG, GB <rhta'i
e>isi dun'amei m'onon s'ummetroi, m'eson >'ara >est`i t`o d`ic <up`o
t~wn AG, GB, tout'esti t`o MZ. ka`i par`a <rht`hn t`hn ML par'akeitai;
<rht`h >'ara ka`i <h MH ka`i >as'ummetroc t~h| ML, tout'esti
t~h| DE, m'hkei. >'esti d`e ka`i <h MD <rht`h ka`i t~h| DE m'hkei
s'ummetroc; >as'ummetroc >'ara >est`in <h DM t~h| MH m'hkei. ka'i e>isi
<rhta'i; a<i DM, MH >'ara <rhta'i e>isi dun'amei m'onon s'ummetroi;
>ek d'uo >'ara >onom'atwn >est`in <h DH. deikt'eon d'h, <'oti
ka`i pr'wth.}

\gr{>Epe`i t~wn >ap`o t~wn AG, GB m'eson >an'alog'on >esti t`o <up`o t~wn
AGB, ka`i t~wn DJ, KL >'ara m'eson >an'alog'on >esti t`o MX. >'estin
>'ara <wc t`o DJ pr`oc t`o MX, o<'utwc t`o MX pr`oc t`o
KL, tout'estin <wc <h DK pr`oc t`hn MN, <h MN pr`oc t`hn MK;
t`o >'ara <up`o t~wn DK, KM >'ison >est`i t~w| >ap`o t~hc MN.
ka`i >epe`i s'ummetr'on >esti t`o >ap`o t~hc AG t~w| >ap`o t~hc GB,
s'ummetr'on >esti ka`i t`o DJ t~w| KL; <'wste ka`i <h DK t~h|
KM s'ummetr'oc >estin. ka`i >epe`i me'izon'a >esti t`a >ap`o t~wn AG,
GB to~u d`ic <up`o t~wn AG, GB, me~izon >'ara ka`i t`o DL to~u
MZ; <'wste ka`i <h DM t~hc MH me'izwn >est'in. ka'i >estin >'ison
t`o <up`o t~wn DK, KM t~w| >ap`o t~hc MN, tout'esti t~w| tet'artw|
 to~u >ap`o t~hc MH, ka`i s'ummetroc <h DK t~h| KM. >e`an
d`e >~wsi d'uo e>uje~iai >'anisoi,  t~w| d`e tet'artw| m'erei to~u >ap`o t~hc
>el'assonoc >'ison par`a t`hn me'izona parablhj~h| >elle~ipon
e>'idei tetrag'wnw|  ka`i e>ic s'ummetra a>ut`hn diair~h|, <h me'izwn
t~hc >el'assonoc me~izon d'unatai t~w| >ap`o summ'etrou <eaut~h|; <h
DM >'ara t~hc MH me~izon d'unatai t~w| >ap`o summ'etrou
<eaut~h|. ka'i e>isi <rhta`i a<i DM, MH, ka`i <h DM me~izon
>'onoma o>~usa s'ummetr'oc >esti t~h| >ekkeim'enh| <rht~h| t~h|
DE m'hkei.}

\gr{<H DH >'ara >ek d'uo >onom'atwn >est`i pr'wth; <'oper >'edei
de~ixai.}}

\ParallelRText{
\begin{center}
{\large Proposition 60}
\end{center}

The square on a binomial (straight-line) applied to a rational (straight-line) produces as breadth a first binomial (straight-line).$^\dag$

\epsfysize=1.5in
\centerline{\epsffile{Book10/fig060e.eps}}

Let $AB$ be a binomial (straight-line), having been divided into its
(component) terms at $C$, such that $AC$ is the greater term. And let the
rational (straight-line) $DE$ be laid down. And let the (rectangle)
$DEFG$, equal to the (square) on $AB$, have been applied to $DE$,
producing $DG$ as breadth. I say that $DG$ is a first binomial (straight-line).

For let $DH$, equal to the (square) on $AC$,  and $KL$, equal to the
(square) on $BC$, have been applied to $DE$. Thus, the remaining twice
the (rectangle contained) by $AC$ and $CB$ is equal to $MF$ [Prop. 2.4].  Let $MG$ have been cut in half
at $N$, and let $NO$ have been drawn parallel [to each of $ML$ and
$GF$]. $MO$ and $NF$ are thus each equal to once the (rectangle contained)
by $ACB$. And since $AB$ is a binomial (straight-line), having been divided into its (component) terms at $C$,  $AC$ and $CB$ are thus rational (straight-lines which are) commensurable in square only [Prop. 10.36]. Thus, the (squares) on $AC$ and $CB$ are rational, and commensurable with one another.  And hence the
sum of the (squares) on $AC$ and $CB$ (is rational) [Prop. 10.15], and is equal to $DL$. Thus, $DL$
is rational. And it is applied to the rational (straight-line) $DE$. $DM$
is thus rational, and commensurable in length with $DE$ [Prop. 10.20]. Again, since $AC$ and $CB$
are rational (straight-lines which are) commensurable in square only, twice the (rectangle contained) by $AC$ and $CB$---that is to say, $MF$---is thus medial [Prop. 10.21]. And it is applied to the
rational (straight-line) $ML$. $MG$ is thus also rational, and incommensurable in length with $ML$---that is to say, with $DE$
[Prop. 10.22].  And $MD$ is also rational, and
commensurable in length with $DE$. Thus, $DM$ is  incommensurable in length with $MG$ [Prop. 10.13]. And
they are rational. $DM$ and $MG$ are thus rational (straight-lines which
are) commensurable in square only. Thus, $DG$ is a binomial (straight-line)
[Prop. 10.36]. So, we must show that (it is)
also a first (binomial straight-line).

Since the (rectangle contained) by $ACB$ is the mean proportional to
 the squares on $AC$ and $CB$ [Prop. 10.53~lem.], $MO$ is thus also the
mean proportional to   $DH$ and $KL$. Thus,
as $DH$ is to $MO$, so $MO$ (is) to $KL$---that is to say, as $DK$
(is) to $MN$, (so) $MN$ (is) to $MK$ [Prop. 6.1].
Thus, the (rectangle contained) by $DK$ and $KM$ is equal
to the (square) on $MN$ [Prop. 6.17]. And since
the (square) on $AC$ is commensurable with the (square) on $CB$,
$DH$ is also commensurable with $KL$. Hence, $DK$
is also commensurable with $KM$ [Props.~6.1, 10.11].
And since (the sum of) the squares on $AC$ and $CB$ is greater
than twice the (rectangle contained) by $AC$ and $CB$ [Prop. 10.59~lem.], $DL$ (is) thus also
greater than $MF$. Hence, $DM$ is also greater than $MG$
[Props.~6.1, 5.14]. And
the (rectangle contained) by $DK$ and $KM$ is equal to
the (square) on $MN$---that is to say, to one quarter the (square) on $MG$.
And $DK$ (is) commensurable (in length) with $KM$. And if there are
two unequal straight-lines, and a (rectangle) equal to the fourth part
of the (square) on the lesser, falling short by a square figure, is applied to
the greater, and divides it into (parts which are) commensurable (in length),
then the square on the greater is larger than (the square on) the lesser by
the (square) on (some straight-line) commensurable (in length) with the greater [Prop. 10.17]. Thus,  the square on $DM$
is greater than (the square on) $MG$ by the (square) on (some straight-line)
commensurable (in length) with ($DM$). And $DM$ and $MG$ are rational. And
$DM$, which is the greater term, is commensurable in length with the (previously) laid down rational (straight-line) $DE$.

Thus, $DG$ is a first binomial (straight-line) [Def. 10.5]. (Which is) the very thing it was required to show.}
\end{Parallel}


\vspace{7pt}{\footnotesize\noindent$^\dag$ In other words, the square of a binomial  is a
first binomial. See Prop.~10.54.}

%%%%
%10.61
%%%%
\pdfbookmark[1]{Proposition 10.61}{pdf10.61}
\begin{Parallel}{}{}
\ParallelLText{
\begin{center}
{\large \ggn{61}.}
\end{center}\vspace*{-7pt}

\gr{T`o >ap`o t~hc >ek d'uo m'eswn pr'wthc par`a <rht`hn paraball'omenon pl'atoc poie~i t`hn >ek d'uo >onom'atwn deut'eran.}\\

\epsfysize=1.55in
\centerline{\epsffile{Book10/fig060g.eps}}

\gr{>'Estw >ek d'uo m'eswn pr'wth <h AB dih|rhm'enh e>ic t`ac m'esac kat`a
t`o G, <~wn me'izwn <h AG, ka`i >ekke'isjw <rht`h <h DE, ka`i parabebl'hsjw par`a t`hn DE t~w| >ap`o t~hc AB >'ison parallhl'ogrammon
t`o DZ pl'atoc poio~un t`hn DH; l'egw, <'oti <h DH >ek d'uo >onom'atwn >est`i deut'era.}

\gr{Kateskeu'asjw g`ar t`a a>ut`a to~ic pr`o to'utou. ka`i >epe`i <h AB
>ek d'uo m'eswn >est`i pr'wth dih|rhm'enh kat`a t`o G, a<i AG, GB >'ara
m'esai e>is`i dun'amei m'onon s'ummetroi <rht`on peri'eqousai; <'wste
ka`i t`a >ap`o t~wn AG, GB m'esa >est'in. m'eson >'ara >est`i t`o DL.
ka`i par`a <rht`hn t`hn DE parab'eblhtai; <rht`h >'ara >est'in <h MD
ka`i >as'ummetroc t~h| DE m'hkei. p'alin, >epe`i <rht'on >esti t`o d`ic
<up`o t~wn AG, GB, <rht'on >esti ka`i t`o MZ. ka`i par`a <rht`hn t`hn
ML par'akeitai; <rht`h >'ara [>est`i] ka`i <h MH ka`i m'hkei s'ummetroc
t~h| ML, tout'esti t~h| DE; >as'ummetroc >'ara >est`in <h DM t~h|
MH m'hkei. ka'i e>isi <rhta'i; a<i DM, MH >'ara <rhta'i e>isi dun'amei
m'onon s'ummetroi; >ek d'uo >'ara >onom'atwn >est`in <h DH. deikt'eon
d'h, <'oti ka`i deut'era.}

\gr{>Epe`i g`ar t`a >ap`o t~wn AG, GB me'izon'a >esti to~u d`ic <up`o t~wn
AG, GB, me~izon >'ara ka`i t`o DL to~u MZ; <'wste ka`i <h DM t~hc
MH. ka`i >epe`i s'ummetr'on >esti t`o >ap`o t~hc AG t~w| >ap`o t~hc
GB, s'ummetr'on >esti ka`i t`o DJ t~w| KL; <'wste ka`i <h DK t~h| KM
s'ummetr'oc >estin. ka'i >esti t`o <up`o t~wn DKM >'ison t~w| >ap`o
t~hc MN; <h DM >'ara t~hc MH me~izon d'unatai t~w| >ap`o summ'etrou
<eaut~h|. ka'i >estin <h MH s'ummetroc t~h| DE m'hkei.}

\gr{<H DH >'ara >ek d'uo >onom'atwn >est`i deut'era.}}

\ParallelRText{
\begin{center}
{\large Proposition 61}
\end{center}

The square on a first bimedial (straight-line)
applied to a rational (straight-line) produces as breadth a second binomial
(straight-line).$^\dag$

\epsfysize=1.55in
\centerline{\epsffile{Book10/fig060e.eps}}

Let $AB$ be a first bimedial (straight-line) having been divided into its (component)
medial (straight-lines) at $C$, of which $AC$ (is) the greater. And let
the rational (straight-line) $DE$ be laid down. And let the parallelogram
$DF$, equal to the (square) on $AB$, have been applied to $DE$, producing
$DG$ as breadth. I say that $DG$ is a second binomial (straight-line).

For let the same construction have been made as in the (proposition) before
this. And since $AB$ is a first bimedial (straight-line), having been divided at $C$, 
$AC$ and $CB$ are thus medial (straight-lines) commensurable
in square only, and containing a rational (area) [Prop. 10.37]. Hence, the (squares) on $AC$ and
$CB$ are also medial [Prop. 10.21]. Thus,
$DL$ is medial [Props.~10.15, 10.23~corr.].  And it has been applied to the
rational (straight-line) $DE$. $MD$ is thus rational, and incommensurable
in length with $DE$ [Prop. 10.22]. Again,
since twice the (rectangle contained) by $AC$ and $CB$ is rational,
$MF$ is also rational. And it is applied to the rational (straight-line)
$ML$. Thus, $MG$ [is] also rational, and commensurable in length
with $ML$---that is to say, with $DE$ [Prop. 10.20]. $DM$ is thus incommensurable in length with $MG$ [Prop. 10.13]. And they are rational. $DM$ and $MG$ are thus rational, and commensurable in square only. $DG$ is thus a binomial (straight-line) [Prop. 10.36]. So, we must show that (it
is) also a second (binomial straight-line).

For since (the sum of) the squares on $AC$ and $CB$ is greater than twice
the (rectangle contained) by $AC$ and $CB$ [Prop. 10.59], $DL$ (is) thus also greater
than $MF$. Hence, $DM$ (is) also (greater) than $MG$ [Prop. 6.1]. And since
the (square) on $AC$ is commensurable with the (square) on $CB$, $DH$
is also commensurable with $KL$. Hence, $DK$ is also commensurable
(in length) with $KM$ [Props.~6.1, 10.11].  And the (rectangle contained) by $DKM$
is equal to the (square) on $MN$. Thus, the square on $DM$ is greater
than (the square on) $MG$ by the (square) on (some straight-line)
commensurable (in length) with ($DM$) [Prop. 10.17]. And $MG$ is commensurable in length
with $DE$.

Thus, $DG$ is a second binomial (straight-line) [Def. 10.6].}
\end{Parallel}


\vspace{7pt}{\footnotesize\noindent $^\dag$In other words, the square of a first bimedial is a
second binomial. See Prop.~10.55.}

%%%%
%10.62
%%%%
\pdfbookmark[1]{Proposition 10.62}{pdf10.62}
\begin{Parallel}{}{}
\ParallelLText{
\begin{center}
{\large \ggn{62}.}
\end{center}\vspace*{-7pt}


\gr{T`o >ap`o t~hc >ek d'uo m'eswn deut'erac par`a <rht`hn paraball'omenon
pl'atoc poie~i t`hn >ek d'uo >onom'atwn tr'ithn.}\\

\epsfysize=1.5in
\centerline{\epsffile{Book10/fig060g.eps}}

\gr{>'Estw >ek d'uo m'eswn deut'era <h AB dih|rhm'enh e>ic t`ac m'esac
kat`a t`o G, <'wste t`o me~izon tm~hma e>~inai t`o AG, <rht`h d'e tic
>'estw <h DE, ka`i par`a t`hn DE t~w| >ap`o t~hc AB >'ison parallhl'ogrammon parabebl'hsjw t`o DZ pl'atoc poio~un t`hn DH;
l'egw, <'oti <h DH >ek d'uo >onom'atwn >est`i tr'ith.}

\gr{Kateskeu'asjw t`a a>ut`a to~ic prodedeigm'enoic. ka`i >epe`i >ek d'uo
m'eswn deut'era >est`in <h AB dih|rhm'enh kat`a t`o G, a<i AG, GB
>'ara m'esai e>is`i dun'amei m'onon s'ummetroi m'eson
peri'eqousai; <'wste ka`i t`o sugke'imenon >ek t~wn >ap`o t~wn AG, GB
m'eson >est'in. ka'i >estin >'ison t~w| DL; m'eson >'ara ka`i t`o DL.
ka`i par'akeitai par`a <rht`hn t`hn DE; <rht`h >'ara >est`i ka`i <h
MD ka`i >as'ummetroc t~h| DE m'hkei. di`a t`a a>ut`a d`h ka`i <h
MH <rht'h >esti ka`i >as'ummetroc t~h| ML, tout'esti t~h| DE, m'hkei; <rht`h
>'ara >est`in <ekat'era t~wn DM, MH ka`i >as'ummetroc t~h| DE m'hkei.
ka`i >epe`i >as'ummetr'oc >estin <h AG t~h| GB m'hkei, <wc
d`e <h AG pr`oc t`hn GB, o<'utwc t`o >ap`o t~hc AG pr`oc t`o <up`o t~wn
AGB, >as'ummetron >'ara ka`i t`o >ap`o t~hc AG t~w| <up`o
t~wn AGB. <'wste ka`i t`o sugke'imenon >ek t~wn >ap`o t~wn AG, GB
t~w| d`ic <up`o t~wn AGB >as'ummetr'on >estin, tout'esti t`o DL t~w|
MZ;  <'wste ka`i <h DM t~w| MH >as'ummetr'oc >estin. ka'i e>isi
<rhta'i; >ek d'uo >'ara >onom'atwn >est`in <h DH. deikt'eon
[d'h], <'oti ka`i tr'ith.}

\gr{<Omo'iwc d`h to~ic prot'eroic >epilogio'umeja, <'oti me'izwn >est`in <h DM
t~hc MH, ka`i s'ummetroc <h DK t~h| KM. ka'i >esti t`o <up`o t~wn
DKM >'ison t~w| >ap`o t~hc MN; <h DM >'ara t~hc MH me~izon
d'unatai t~w| >ap`o summ'etrou <eaut~h|. ka`i o>udet'era t~wn DM, MH
s'ummetr'oc >esti t~h| DE m'hkei.}

\gr{<H DH >'ara >ek d'uo >onom'atwn >est`i tr'ith; <'oper >'edei
de~ixai.}}

\ParallelRText{
\begin{center}
{\large Proposition 62}
\end{center}

The square on a second bimedial (straight-line)
applied to  a rational (straight-line) produces as breadth a third binomial
(straight-line).$^\dag$

\epsfysize=1.5in 
\centerline{\epsffile{Book10/fig060e.eps}}

Let $AB$ be a second bimedial (straight-line) having been divided into its (component)
medial (straight-lines) at $C$, such that $AC$ is the greater segment. And
let $DE$ be some rational (straight-line). And let the parallelogram
$DF$, equal to the (square) on $AB$, have been applied to $DE$,
producing $DG$ as breadth. I say that $DG$ is a third binomial (straight-line).

Let the same construction  be made as that shown previously. And since $AB$ is a second bimedial (straight-line), having been divided at $C$, $AC$ and
$CB$ are thus medial (straight-lines) commensurable in square only,
and containing a medial (area) [Prop. 10.38]. Hence, the sum  of the (squares) on
$AC$ and $CB$ is also medial [Props.~10.15, 10.23~corr.]. And it is equal to $DL$. Thus, $DL$ (is)
also medial. And it is applied to the rational (straight-line) $DE$. $MD$
is thus also rational, and incommensurable in length with $DE$
[Prop. 10.22]. So, for the same (reasons),
$MG$ is also rational, and incommensurable in length with $ML$---that is to say, with $DE$. Thus, $DM$ and $MG$ are each rational, and incommensurable in length with $DE$. And since	$AC$ is incommensurable
in length with $CB$, and as $AC$ (is) to $CB$, so the (square)
on $AC$ (is) to the (rectangle contained) by $ACB$ [Prop. 10.21~lem.], the (square) on $AC$
(is) also incommensurable with the (rectangle contained) by $ACB$
[Prop. 10.11]. And hence the sum of the
(squares) on $AC$ and $CB$ is incommensurable with twice the (rectangle
contained) by $ACB$---that is to say, $DL$ with $MF$ [Props.~10.12, 10.13].  Hence,
$DM$ is also incommensurable (in length) with $MG$ [Props.~6.1, 10.11].
And they are rational.  $DG$ is thus a binomial (straight-line)
[Prop. 10.36]. [So]
we must show that (it is) also a third (binomial straight-line).

So, similarly to the previous (propositions), we can conclude that
$DM$ is greater than $MG$, and $DK$ (is) commensurable (in length) 
with $KM$.  And the (rectangle contained) by $DKM$ is equal to the
(square) on $MN$. Thus, the square on $DM$ is greater than (the square on)
$MG$ by the (square) on (some straight-line) commensurable (in length) with ($DM$)
[Prop. 10.17]. And neither of $DM$ and $MG$
is commensurable in length with $DE$.

Thus, $DG$ is a third binomial (straight-line) [Def. 10.7]. (Which is) the very thing it
was required to show.}
\end{Parallel}


\vspace{7pt}{\footnotesize\noindent$^\dag$ In other words, the square of a second bimedial  is a
third binomial. See Prop.~10.56.}

%%%%
%10.63
%%%%
\pdfbookmark[1]{Proposition 10.63}{pdf10.63}
\begin{Parallel}{}{}
\ParallelLText{
\begin{center}
{\large\ggn{63}.}
\end{center}\vspace*{-7pt}

\gr{T`o >ap`o t~hc me'izonoc par`a <rht`hn paraball'omen\-on pl'atoc poie~i
t`hn >ek d'uo >onom'atwn tet'arthn.}\\

\epsfysize=1.6in
\centerline{\epsffile{Book10/fig060g.eps}}

\gr{>'Estw me'izwn <h AB dih|rhm'enh kat`a t`o G, <'wste me'izona e>~inai
t`hn AG t~hc GB, <rht`h d`e <h DE, ka`i t~w| >ap`o t~hc AB >'ison
par`a t`hn DE parabebl'hsjw t`o DZ parallhl'ogrammon pl'atoc poio~un
t`hn DH; l'egw, <'oti <h DH >ek d'uo >onom'atwn >est`i tet'arth.}

\gr{Kateskeu'asjw t`a a>ut`a to~ic prodedeigm'enoic. ka`i >epe`i me'izwn
>est`in <h AB dih|rhm'enh kat`a t`o G, a<i AG, GB dun'amei e>is`in >as'ummetroi poio~usai t`o m`en sugke'imenon >ek t~wn >ap> a>ut~wn
tetrag'wnwn <rht'on, t`o d`e <up> a>ut~wn m'eson. >epe`i o>~un <rht'on
>esti t`o sugke'imenon >ek t~wn >ap`o t~wn AG, GB, <rht`on >'ara >est`i
t`o DL; <rht`h >'ara ka`i <h DM ka`i s'ummetroc t~h| DE m'hkei.
p'alin, >epe`i m'eson >est`i t`o d`ic <up`o t~wn AG, GB, tout'esti t`o MZ,
ka`i par`a <rht'hn >esti t`hn ML, <rht`h >'ara >est`i ka`i <h MH ka`i
>as'ummetroc  t~h| DE m'hkei; >as'ummetroc >'ara >est`i ka`i <h DM
t~h| MH m'hkei. a<i DM, MH >'ara <rhta'i e>isi dun'amei m'onon s'ummetroi; >ek d'uo >'ara >onom'atwn >est`in <h DH. deikt'eon
[d'h], <'oti ka`i tet'arth.}

\gr{<Omo'iwc d`h de'ixomen to~ic pr'oteron, <'oti me'izwn >est`in <h DM
t~hc MH, ka`i <'oti t`o <up`o DKM >'ison >est`i t~w| >ap`o t~hc
MN. >epe`i o>~un >as'ummetr'on >esti t`o >ap`o t~hc AG t~w| >ap`o
t~hc GB, >as'ummetron >'ara >est`i ka`i t`o DJ t~w| KL; <'wste
>as'ummetroc ka`i <h DK t~h| KM >estin. >e`an d`e >~wsi d'uo e>uje~iai
>'anisoi, t~w| d`e tet'artw| m'erei to~u >ap`o t~hc >el'assonoc >'ison
parallhl'ogrammon par`a t`hn me'izona parablhj~h| >elle~ipon e>'idei
tetrag'wnw| ka`i e>ic >as'ummetra a>ut`hn diair~h|, <h me'izwn t~hc
>el'assonoc me~izon dun'hsetai t~w| >ap`o >as'umm'etrou <eaut~h|
m'hkei; <h DM >'ara t~hc MH me~izon d'unatai t~w| >ap`o >asumm'etrou
<eaut~h|. ka'i e>isin a<i DM, MH <rhta`i dun'amei m'onon s'ummetroi,
ka`i <h DM s'ummetr'oc >esti t~h| >ekkeim'enh| <rht~h| t~h| DE.}

\gr{<H DH >'ara >ek d'uo >onom'atwn >est`i tet'arth; <'oper >'edei de~ixai.}}

\ParallelRText{
\begin{center}
{\large Proposition 63}
\end{center}

The square on a major (straight-line) applied
to a rational (straight-line)  produces as breadth a fourth binomial (straight-line).$^\dag$

\epsfysize=1.6in 
\centerline{\epsffile{Book10/fig060e.eps}}

Let $AB$ be a major (straight-line) having been divided at $C$, such that
$AC$ is greater than $CB$, and (let) $DE$ (be)  a rational (straight-line). And
let the parallelogram $DF$, equal to the (square) on $AB$, have been
applied to $DE$, producing $DG$ as breadth. I say that
$DG$ is a fourth binomial (straight-line).

Let the same construction be made as that shown previously.
And since $AB$ is a major (straight-line), having been divided at $C$, $AC$ and
$CB$ are incommensurable in square, making the sum of the squares
on them rational, and the (rectangle contained) by them medial [Prop. 10.39]. Therefore, since the sum of the
(squares) on $AC$ and $CB$ is rational, $DL$ is thus rational.
Thus, $DM$ (is) also rational, and commensurable in length with $DE$ [Prop. 10.20]. Again, since twice the (rectangle
contained) by $AC$ and $CB$---that is to say, $MF$---is medial,
and is (applied to) the rational (straight-line) $ML$,  $MG$
is thus also rational, and incommensurable in length with $DE$ [Prop. 10.22]. $DM$ is thus also incommensurable
in length with $MG$ [Prop. 10.13]. $DM$
and $MG$ are thus rational (straight-lines which are) commensurable
in square only. Thus, $DG$ is a binomial (straight-line) [Prop. 10.36]. [So] we must show that (it is)
also a fourth (binomial straight-line).

So, similarly to the previous (propositions), we can show that $DM$ is
greater than $MG$, and that the (rectangle contained) by $DKM$
is equal to the (square) on $MN$. Therefore, since the (square) on $AC$
is incommensurable with the (square) on $CB$, $DH$ is also incommensurable
with $KL$. Hence, $DK$ is also incommensurable with $KM$
[Props.~6.1, 10.11].
And if there are two unequal straight-lines, and a parallelogram
equal to the fourth part of the (square) on the lesser, falling short
by a square figure, is applied to the greater, and divides it
into (parts which are) incommensurable (in length), then the square on
the greater will be larger than (the square on) the lesser by the
(square) on (some straight-line) incommensurable in length
with the greater [Prop. 10.18]. Thus,
the square on $DM$ is greater than (the square on) $MG$ by the
(square) on (some straight-line) incommensurable (in length) with ($DM$). And
$DM$ and $MG$ are rational (straight-lines which are) commensurable
in square only. And $DM$ is commensurable (in length)
with the (previously) laid down rational (straight-line) $DE$.

Thus, $DG$ is a fourth binomial (straight-line) [Def. 10.8]. (Which is) the very thing it was required
to show.}
\end{Parallel}


\vspace{7pt}{\footnotesize\noindent$^\dag$ In other words, the square of a major is a
fourth binomial. See Prop.~10.57.}

%%%%
%10.64
%%%%
\pdfbookmark[1]{Proposition 10.64}{pdf10.64}
\begin{Parallel}{}{}
\ParallelLText{
\begin{center}
{\large \ggn{64}.}
\end{center}\vspace*{-7pt}

\gr{T`o >ap`o t~hc <rht`on ka`i m'eson dunam'enhc par`a <rht`hn
paraball'omenon pl'atoc poie~i t`hn >ek d'uo >onom'atwn
p'empthn.}

\gr{>'Estw <rht`on ka`i m'eson dunam'enh <h AB dih|rhm'enh e>ic
t`ac e>uje'iac kat`a t`o G, <'wste me'izona e>~inai t`hn AG,
ka`i >ekke'isjw <rht`h <h DE, ka`i t~w| >ap`o t~hc AB >'ison
par`a t`hn DE parabebl'hsjw t`o DZ pl'atoc poio~un t`hn DH;
l'egw, <'oti <h DH >ek d'uo >onom'atwn >est`i p'empth.}\\~\\~\\

\epsfysize=1.5in
\centerline{\epsffile{Book10/fig060g.eps}}

\gr{Kateskeu'asjw t`a a>uta to~ic pr`o to'utou. >epe`i o>~un
<rht`on ka`i m'eson dunam'enh >est`in <h AB dih|rhm'enh
kat`a t`o G, a<i AG, GB >'ara dun'amei e>is`in >as'ummetroi
poio~usai t`o m`en sugke'imenon >ek t~wn >ap> a>ut~wn
tetrag'wnwn m'eson, t`o d> <up> a>ut~wn <rht'on. >epe`i o>~un
m'eson >est`i t`o sugke'imenon >ek t~wn >ap`o t~wn AG, GB,
m'eson >'ara >est`i t`o DL; <'wste <rht'h >estin <h DM ka`i
m'hkei >as'ummetroc t~h| DE. p'alin, >epe`i <rht'on >esti t`o d`ic
<up`o t~wn AGB, tout'esti t`o MZ, <rht`h >'ara <h MH
ka`i s'ummetroc t~h| DE. >as'ummetroc >'ara <h DM t~h|
MH; a<i DM, MH >'ara <rhta'i e>isi dun'amei m'onon s'ummetroi;
>ek d'uo >'ara >onom'atwn >est`in <h DH. l'egw d'h, <'oti ka`i
p'empth.}

\gr{<Omo'iwc g`ar dieqj'hsetai, <'oti t`o <up`o t~wn DKM >'ison
>est`i t~w| >ap`o t~hc MN, ka`i >as'ummetroc <h DK t~h| KM
m'hkei; <h DM >'ara t~hc MH me~izon d'unatai t~w| >ap`o
>asumm'etrou <eaut~h|. ka'i e>isin a<i DM, MH [<rhta`i] dun'amei
m'onon s'ummetroi, ka`i <h >el'asswn <h MH s'ummetroc t~h|
DE m'hkei.}

\gr{<H DH >'ara >ek d'uo >onom'atwn >est`i p'empth; <'oper
>'edei de~ixai.}}

\ParallelRText{
\begin{center}
{\large Proposition 64}
\end{center}

The square on the square-root of a rational
plus a medial (area) applied to a rational (straight-line) produces as breadth
a fifth binomial (straight-line).$^\dag$

Let $AB$ be the square-root of a rational plus a medial (area) having been divided
into its (component) straight-lines at $C$, such that $AC$ is greater. 
And let the rational (straight-line) $DE$ be laid down. And let the
(parallelogram) $DF$, equal to the (square) on $AB$, have been applied
to $DE$, producing $DG$ as breadth. I say that $DG$ is a fifth
binomial straight-line.

\epsfysize=1.5in 
\centerline{\epsffile{Book10/fig060e.eps}}

Let the same construction  be made as in the (propositions)
before this. Therefore, since $AB$ is the square-root of a
rational plus a medial (area), having been  divided at $C$, $AC$ and $CB$ are thus
incommensurable in square, making the sum of the squares on them
medial, and the (rectangle contained) by them rational [Prop. 10.40]. Therefore, since the
sum of the (squares) on $AC$ and $CB$ is medial, $DL$ is thus
medial. Hence, $DM$ is rational and incommensurable
in length with $DE$ [Prop. 10.22]. Again,
since twice the (rectangle contained) by $ACB$---that is to say,
$MF$---is rational, $MG$ (is) thus rational and commensurable
(in length) with $DE$ [Prop. 10.20]. 
$DM$ (is) thus incommensurable (in length) with $MG$ [Prop. 10.13]. Thus, $DM$ and $MG$
are rational (straight-lines which are) commensurable in square only. Thus,
$DG$ is a binomial (straight-line) [Prop. 10.36].
So, I say that (it is) also a fifth (binomial straight-line).

For, similarly (to the previous propositions), it can be shown that the
(rectangle contained) by $DKM$ is equal to the (square) on $MN$,
and $DK$ (is) incommensurable in length with $KM$. Thus, the
square on $DM$ is greater than (the square on) $MG$ by the (square)
on (some straight-line) incommensurable (in length) with ($DM$)
[Prop. 10.18]. And  $DM$ and $MG$
are [rational] (straight-lines which are) commensurable in square only,
and the lesser $MG$ is commensurable in length with $DE$.

Thus, $DG$ is a fifth binomial (straight-line) [Def. 10.9]. (Which is) the very thing it was required to show.}
\end{Parallel}


\vspace{7pt}{\footnotesize\noindent$^\dag$ In other words, the square of the square-root of a rational plus medial  is a
fifth binomial. See Prop.~10.58.}

%%%%
%10.65
%%%%
\pdfbookmark[1]{Proposition 10.65}{pdf10.65}
\begin{Parallel}{}{}
\ParallelLText{
\begin{center}
{\large \ggn{65}.}
\end{center}\vspace*{-7pt}

\gr{T`o >ap`o t~hc d'uo m'esa dunam'enhc par`a <rht`hn paraball'omenon
pl'atoc poie~i t`hn >ek d'uo >onom'atwn <'ekthn.}

\gr{>'Estw d'uo m'esa dunam'enh <h AB dih|rhm'enh kat`a t`o G, <rht`h
d`e >'estw <h DE, ka`i par`a t`hn DE t~w| >ap`o t~hc AB >'ison parabebl'hsjw t`o DZ pl'atoc poio~un t`hn DH; l'egw, <'oti <h DH
>ek d'uo >onom'atwn >est`in <'ekth.}\\~\\~\\

\epsfysize=1.6in
\centerline{\epsffile{Book10/fig060g.eps}}

\gr{Kateskeu'asjw g`ar t`a a>ut`a to~ic pr'oteron. ka`i >epe`i <h AB
d'uo m'esa dunam'enh >est`i dih|rhm'enh kat`a t`o G, a<i AG, GB
>'ara dun'amei e>is`in >as'ummetroi poio~usai t'o te sugke'imenon
>ek t~wn >ap> a>ut~wn tetrag'wnwn m'eson ka`i t`o <up> a>ut~wn
m'eson ka`i >'eti >as'ummetron t`o >ek t~wn >ap> a>ut~wn
tetrag'wnwn sugke'imenon t~w| <up> a>ut~wn; <'wste kat`a t`a
prodedeigm'ena m'eson >est`in <ek'ateron t~wn DL, MZ. ka`i par`a
<rht`hn t`hn DE par'akeitai; <rht`h >'ara >est`in <ekat'era t~wn
DM, MH ka`i >as'ummetroc t~h| DE m'hkei. ka`i >epe`i >as'ummetr'on
>esti t`o sugke'imenon >ek t~wn >ap`o t~wn AG, GB t~w| d`ic
<up`o t~wn AG, GB, >as'ummetron >'ara >est`i t`o DL t~w| MZ. >as'ummetroc >'ara ka`i <h DM t~h| MH; a<i DM, MH >'ara <rhta'i
e>isi dun'amei m'onon s'ummetroi; >ek d'uo >'ara >onom'atwn >est`in
<h DH. l'egw d'h, <'oti ka`i <'ekth.}

\gr{<Omo'iwc d`h p'alin de~ixomen, <'oti t`o <up`o t~wn DKM >'ison
>est`i t~w| >ap`o t~hc MN, ka`i <'oti <h DK t~h| KM m'hkei >est`in
>as'ummetroc; ka`i di`a t`a a>ut`a d`h <h DM t~hc MH me~izon
d'unatai t~w| >ap`o >asumm'etrou <eaut~h| m'hkei. ka`i
o>udet'era t~wn DM, MH s'ummetr'oc >esti t~h| >ekkeim'enh| <rht~h|
t~h| DE m'hkei.}

\gr{<H DH >'ara >ek d'uo >onom'atwn >est`in <'ekth; <'oper
>'edei de~ixai.}}

\ParallelRText{
\begin{center}
{\large Proposition 65}
\end{center}

The square on the square-root of (the sum of) two medial
(areas) applied to a rational (straight-line) produces as breadth  a sixth binomial (straight-line).$^\dag$

Let $AB$ be the square-root of (the sum of) two medial (areas), having been divided
at $C$. And let $DE$ be a  rational (straight-line). And let the (parallelogram) $DF$, equal to the (square) on $AB$, have been applied to $DE$, producing $DG$
as breadth. I say that $DG$ is a sixth binomial (straight-line).

\epsfysize=1.6in 
\centerline{\epsffile{Book10/fig060e.eps}}

For  let the same construction  be made  as in the previous (propositions). And since $AB$ is the square-root of (the sum of) two medial
(areas), having been divided at $C$, $AC$ and $CB$ are thus incommensurable
in square, making the sum of the squares on them medial, and the (rectangle
contained) by them medial, and, moreover, the sum of the squares on them
incommensurable with the (rectangle contained) by them [Prop. 10.41]. Hence, according to
what has been previously  demonstrated, $DL$ and $MF$ are each medial.
And they are applied to the rational (straight-line) $DE$. Thus, $DM$ and
$MG$ are each rational, and incommensurable in length with $DE$ [Prop. 10.22]. And since the sum of the (squares)
on $AC$ and $CB$ is incommensurable with twice the (rectangle contained)
by $AC$ and $CB$, $DL$ is thus incommensurable with $MF$. Thus,
$DM$ (is) also incommensurable (in length) with $MG$
[Props.~6.1, 10.11]. 
$DM$ and $MG$ are thus rational (straight-lines which are) commensurable
in square only. Thus, $DG$ is a binomial (straight-line) [Prop. 10.36]. So, I say that (it is) also a sixth
(binomial straight-line).

So, similarly (to the previous propositions), we can again show that the (rectangle contained) by $DKM$ is equal to the (square) on $MN$, and that
$DK$ is incommensurable in length with $KM$. And so, for the same (reasons), the square on $DM$ is greater than (the square on) $MG$ by the
(square) on (some straight-line) incommensurable in length with ($DM$) [Prop. 10.18]. And neither of $DM$ and $MG$
is commensurable in length with the (previously) laid down rational (straight-line) $DE$.

Thus, $DG$ is a sixth binomial (straight-line) [Def. 10.10]. (Which is) the very thing it
was required to show.}
\end{Parallel}


\vspace{7pt}{\footnotesize\noindent $^\dag$ In other words, the square of the square-root of  two  medials  is a
sixth binomial. See Prop.~10.59.}

%%%%
%10.66
%%%%
\pdfbookmark[1]{Proposition 10.66}{pdf10.66}
\begin{Parallel}{}{}
\ParallelLText{
\begin{center}
{\large \ggn{66}.}
\end{center}\vspace*{-7pt}

\gr{<H t~h| >ek d'uo >onom'atwn m'hkei s'ummetroc ka`i a>ut`h >ek d'uo
>onom'atwn >est`i ka`i t~h| t'axei <h a>ut'h.}

\gr{>'Estw >ek d'uo >onom'atwn <h AB, ka`i t~h| AB m'hkei s'ummetroc
>'estw <h GD; l'egw, <'oti <h GD >ek d'uo >onom'atwn >est`i ka`i t~h|
t'axei <h a>ut`h t~h| AB.}\\

\epsfysize=0.7in
\centerline{\epsffile{Book10/fig066g.eps}}

\gr{>Epe`i g`ar >ek d'uo >onom'atwn >est`in <h AB, dih|r'hsjw e>ic t`a
>on'omata kat`a t`o E, ka`i >'estw me~izon >'onoma t`o AE; a<i AE, EB >'ara
<rhta'i e>isi dun'amei m'onon s'ummetroi. gegon'etw <wc <h AB pr`oc
t`hn GD, o<'utwc <h AE pr`oc t`hn GZ; ka`i loip`h >'ara <h EB pr`oc
loip`hn t`hn ZD >estin, <wc <h AB pr`oc t`hn GD. s'ummetroc d`e <h AB t~h| GD m'hkei; s'ummetroc >'ara >est`i ka`i <h m`en AE t~h| GZ, <h
d`e EB t~h| ZD. ka'i e>isi <rhta`i a<i AE, EB; <rhta`i >'ara e>is`i ka`i a<i
GZ, ZD. ka`i  >estin <wc <h AE pr`oc GZ, <h EB pr`oc ZD. >enall`ax
>'ara >est`in <wc <h AE pr`oc EB, <h GZ pr`oc ZD.
a<i d`e
AE, EB dun'amei m'onon [e>is`i] s'ummetroi; ka`i a<i GZ, ZD >'ara
dun'amei m'onon  e>is`i s'ummetroi.
ka'i e>isi <rhta'i;
>ek d'uo >'ara >onom'atwn >est`in <h GD. l'egw d'h, <'oti t~h| t'axei
>est`in <h a>ut`h t~h| AB.}

\gr{<H g`ar AE t~hc EB me~izon d'unatai >'htoi t~w| >ap`o summ'etrou
<eaut~h| >`h t~w| >ap`o >asumm'etrou. e>i m`en o>~un <h AE t~hc
EB me~izon d'unatai t~w| >ap`o summ'etrou <eaut~h|, ka`i <h GZ t~hc
ZD me~izon dun'hsetai t~w| >ap`o summ'etrou <eaut~h|. ka`i e>i m`en
s'ummetr'oc >estin <h AE t~h| >ekkeim'enh| <rht~h|, ka`i <h GZ s'ummetroc
a>ut~h| >'estai, ka`i di`a to~uto <ekat'era t~wn AB, GD >ek d'uo >onom'atwn
>est`i pr'wth, tout'esti t~h| t'axei <h a>ut'h. e>i d`e <h EB s'ummetr'oc
>esti t~h| >ekkeim'enh| <rht~h|, ka`i <h ZD s'ummetr'oc >estin a>ut~h|, ka`i
di`a to~uto p'alin t~h| t'axei <h a>ut`h >'estai t~h| AB; <ekat'era g`ar
a>ut~wn >'estai >ek d'uo >onom'atwn deut'era. e>i d`e o>udet'era
t~wn AE, EB s'ummetr'oc >esti t~h| >ekkeim'enh| <rht~h|, o>udet'era
t~wn GZ, ZD s'ummetroc a>ut~h| >'estai, ka'i >estin <ekat'era tr'ith. e>i d`e
<h AE t~hc EB me~izon d'unatai t~w| >ap`o >asumm'etrou <eaut~h|,
ka`i <h GZ t`hc ZD me~izon d'unatai t~w| >ap`o >asumm'etrou
<eaut~h|. ka`i e>i m`en <h AE s'ummetr'oc >esti t~h| >ekkeim'enh| <rhth|,
ka`i <h GZ s'ummetr'oc >estin a>ut~h|, ka`i >estin <ekat'era tet'arth.
e>i d`e <h EB, ka`i <h ZD, ka`i >'estai <ekat'era p'empth. e>i d`e o>udet'era
t~wn AE, EB, ka`i t~wn GZ, ZD o>udet'era s'ummetr'oc >esti t~h| >ekkeim'enh| <rht~h|, ka`i >'estai <ekat'era <'ekth.}

\gr{<'Wste <h t~h| >ek d'uo >onom'atwn m'hkei s'ummetroc >ek d'uo
>onom'atwn >est`i ka`i t~h| t'axei <h a>ut'h; <'oper >'edei de~ixai.
}}

\ParallelRText{
\begin{center}
{\large Proposition 66}
\end{center}

A (straight-line) commensurable in length with
a binomial (straight-line) is itself also binomial, and the same in order.

Let $AB$ be a binomial (straight-line), and let $CD$ be commensurable
in length with $AB$. I say that $CD$ is a binomial (straight-line),
and (is)  the same in order as $AB$.

\epsfysize=0.7in 
\centerline{\epsffile{Book10/fig066e.eps}}

For since $AB$ is a binomial (straight-line), let it have been divided into
its (component) terms at $E$, and let $AE$ be the greater term. $AE$ and
$EB$ are thus rational (straight-lines which are) commensurable in square only [Prop. 10.36]. Let it have been contrived
that as $AB$ (is) to $CD$, so $AE$ (is) to $CF$ [Prop. 6.12]. Thus, the remainder $EB$ is also
to the remainder $FD$, as $AB$ (is) to $CD$ [Props.~6.16, 5.19~corr.]. 
And $AB$ (is) commensurable in length with $CD$. Thus, $AE$ is also
commensurable (in length) with $CF$, and $EB$ with $FD$ [Prop. 10.11]. And $AE$ and $EB$ are rational.
Thus, $CF$ and $FD$ are also rational. And  as $AE$
is to $CF$, (so) $EB$ (is) to $FD$ [Prop. 5.11]. 
Thus, alternately, as $AE$ is to $EB$, (so) $CF$ (is) to $FD$ [Prop. 5.16]. And $AE$ and $EB$ [are]
commensurable in square only. Thus, $CF$ and $FD$ are also
commensurable in square only [Prop. 10.11]. 
And they are rational. $CD$ is thus a binomial (straight-line) [Prop. 10.36]. So, I say that it is the same in order as $AB$.

For the square on $AE$ is  greater than (the square on) $EB$ by the
(square) on (some straight-line) either commensurable or incommensurable
(in length) with ($AE$).  Therefore, if the square on $AE$ is greater than (the square on)
$EB$ by the (square) on (some straight-line) commensurable (in length)
with ($AE$) then the square on $CF$ will also be greater than (the square on)
$FD$ by the (square) on (some straight-line) commensurable (in length)
with ($CF$) [Prop. 10.14]. And if $AE$
is commensurable (in length) with (some previously) laid down rational (straight-line) then $CF$ will also be commensurable (in length) with it [Prop. 10.12]. And, on account of this,
$AB$ and $CD$ are each first binomial (straight-lines) [Def. 10.5]---that is to say, the same  in order.
And if $EB$ is commensurable (in length) with the (previously) laid down
rational (straight-line) then $FD$ is also  commensurable (in length) with it [Prop. 10.12], and, again, on account of this,
($CD$) will be  the same in order as $AB$. For each of them will be second
binomial (straight-lines) [Def. 10.6]. And if neither
of $AE$ and $EB$ is commensurable (in length) with the (previously) laid down
rational (straight-line) then neither of $CF$ and $FD$ will be commensurable (in length)
with it [Prop. 10.13], and each (of $AB$ and $CD$) is a third (binomial straight-line) [Def. 10.7]. And if the square on $AE$
is greater than (the square on) $EB$ by the (square) on (some straight-line)
incommensurable (in length) with ($AE$) then the square on $CF$
is also greater than (the square on) $FD$ by the (square) on (some straight-line) incommensurable (in length) with ($CF$) [Prop. 10.14]. And if $AE$ is commensurable
(in length) with the (previously) laid down rational (straight-line) then $CF$ is also
commensurable (in  length) with it [Prop. 10.12], 
and each (of $AB$ and $CD$) is a fourth (binomial straight-line)
[Def. 10.8]. And
if $EB$ (is commensurable in length with the previously laid down rational
straight-line) then $FD$ (is) also (commensurable in length with it), and
each (of $AB$ and $CD$) will be a fifth (binomial straight-line) [Def. 10.9]. And if neither of $AE$ and  $EB$
(is commensurable in length with the previously laid down rational
straight-line) then also neither of $CF$ and $FD$ is   commensurable (in length) with the laid down rational (straight-line), and
each (of $AB$ and $CD$) will be a sixth (binomial straight-line) [Def. 10.10].

Hence, a (straight-line) commensurable in length with a binomial (straight-line) is a binomial (straight-line), and the same in order. (Which is) the
very thing it was required to show.}
\end{Parallel}

%%%%
%10.67
%%%%
\pdfbookmark[1]{Proposition 10.67}{pdf10.67}
\begin{Parallel}{}{}
\ParallelLText{
\begin{center}
{\large\ggn{67}.}
\end{center}\vspace*{-7pt}

\gr{<H t~h| >ek d'uo m'eswn m'hkei s'ummetroc ka`i a>ut`h >ek d'uo m'eswn
>est`i ka`i t~h| t'axei <h a>ut'h.}\\

\epsfysize=0.7in
\centerline{\epsffile{Book10/fig066g.eps}}

\gr{>'Estw >ek d'uo m'eswn <h AB, ka`i t~h| AB s'ummetroc >'estw m'hkei
<h GD; l'egw, <'oti <h GD >ek d'uo m'eswn >est`i ka`i t~h| t'axei
<h a>ut`h t~h| AB.}

\gr{>Epe`i g`ar >ek d'uo m'eswn >est`in <h AB, dih|r'hsjw e>ic t`ac
m'esac kat`a t`o E; a<i AE, EB >'ara m'esai e>is`i dun'amei m'onon
s'ummetroi. ka`i gegon'etw <wc <h AB pr`oc GD, <h AE pr`oc GZ;
ka`i loip`h >'ara <h EB pr`oc loip`hn t`hn ZD >estin, <wc <h AB pr`oc GD. s'ummetroc d`e <h AB t~h| GD m'hkei; s'ummetroc >'ara ka`i <ekat'era t~wn
AE, EB <ekat'era| t~wn GZ, ZD. m'esai d`e a<i AE, EB; m'esai >'ara ka`i a<i
GZ, ZD. ka`i >epe'i >estin <wc <h AE pr`oc EB, <h GZ pr`oc ZD, a<i 
d`e AE, EB dun'amei m'onon s'ummetro'i e>isin, ka`i a<i GZ, ZD [>'ara]
dun'amei m'onon s'ummetro'i e>isin, >ede'iqjhsan d`e ka`i m'esai; <h
GD >'ara >ek d'uo m'eswn >est'in. l'egw d'h, <'oti ka`i t~h| t'axei <h
a>ut'h >esti t~h| AB.}

\gr{>Epe`i g'ar >estin <wc <h AE pr`oc EB, <h GZ pr`oc ZD, ka`i <wc
>'ara t`o >ap`o t~hc AE pr`oc t`o <up`o t~wn AEB, o<'utwc
t`o >ap`o t~hc GZ pr`oc t`o <up`o t~wn GZD; >enall`ax <wc t`o
>ap`o t~hc AE pr`oc t`o >ap`o t~hc GZ, o<'utwc t`o <up`o t~wn AEB
pr`oc t`o <up`o t~wn GZD. s'ummetron d`e t`o >ap`o t~hc
AE t~w| >ap`o t~hc GZ; s'ummetron
>'ara ka`i t`o <up`o t~wn AEB t~w| <up`o t~wn GZD. e>'ite o>~un <rht'on
>esti t`o <up`o t~wn AEB, ka`i t`o <up`o t~wn GZD <rht'on >estin
[ka`i di`a to~ut'o >estin >ek d'uo m'eswn pr'wth]. e>'ite m'eson, m'eson,
ka'i >estin <ekat'era deut'era.}

\gr{Ka`i di`a to~uto >'estai <h GD t~h| AB t~h| t'axei <h
a>ut'h; <'oper >'edei de~ixai.}}

\ParallelRText{
\begin{center}
{\large Proposition 67}
\end{center}

A (straight-line) commensurable
in length with a bimedial (straight-line) is  itself also bimedial, and
 the same in order.

\epsfysize=0.7in 
\centerline{\epsffile{Book10/fig066e.eps}}
 
Let $AB$ be a bimedial (straight-line), and let $CD$ be commensurable
in length with $AB$. I say that $CD$ is bimedial, and the same in order as
$AB$.

For since $AB$ is a bimedial (straight-line), let it have been divided into its
(component) medial (straight-lines) at $E$. Thus, $AE$ and $EB$
are medial (straight-lines which are) commensurable in square only [Props.~10.37, 10.38]. And
let it have been contrived that as $AB$ (is) to $CD$, (so) $AE$ (is)
to $CF$ [Prop. 6.12]. And thus as the remainder $EB$ is to the remainder $FD$, so $AB$ (is) to $CD$ [Props.~5.19~corr., 6.16].
And $AB$ (is) commensurable in length with $CD$. Thus, $AE$
and $EB$ are also commensurable (in length) with $CF$ and $FD$, respectively [Prop. 10.11]. And 
$AE$ and $EB$ (are) medial. Thus, $CF$ and $FD$ (are) also
medial [Prop. 10.23]. And since as $AE$
is to $EB$, (so) $CF$ (is) to $FD$, and $AE$ and $EB$ are commensurable
in square only, $CF$ and $FD$ are [thus] also commensurable in square only
[Prop. 10.11]. And they were also shown (to be)
medial. Thus, $CD$ is a bimedial (straight-line). So, I say that it is also the same in order as $AB$.

For since as $AE$ is to $EB$, (so) $CF$ (is) to $FD$, thus also as the
(square) on $AE$ (is) to the (rectangle contained) by $AEB$, so the (square)
on $CF$ (is) to the (rectangle contained) by $CFD$ [Prop. 10.21~lem.]. Alternately, as the (square)
on $AE$ (is) to the (square) on $CF$, so the (rectangle
contained) by $AEB$ (is) to the (rectangle contained) by $CFD$ [Prop. 5.16]. And the (square) on $AE$
(is) commensurable with the (square) on $CF$. Thus, the (rectangle
contained) by $AEB$ (is) also commensurable with the (rectangle contained)
by $CFD$ [Prop. 10.11]. Therefore,  either the
(rectangle contained) by $AEB$ is rational, and  the (rectangle contained) by
$CFD$ is rational [and, on account of this, ($AE$ and $CD$) are first
bimedial (straight-lines)], or (the rectangle contained by $AEB$ is) medial,
and (the rectangle contained by $CFD$ is) medial, and ($AB$ and
$CD$) are each second (bimedial straight-lines) [Props.~10.23, 10.37, 10.38].

And, on account of this, $CD$ will be  the same in order as $AB$.
(Which is) the very thing it was required to show.}
\end{Parallel}

%%%%
%10.68
%%%%
\pdfbookmark[1]{Proposition 10.68}{pdf10.68}
\begin{Parallel}{}{}
\ParallelLText{
\begin{center}
{\large\ggn{68}.}
\end{center}\vspace*{-7pt}

\gr{<H t~h| me'izoni s'ummetroc ka`i a>ut`h me'izwn >est'in.}\\

\epsfysize=0.7in
\centerline{\epsffile{Book10/fig066g.eps}}

\gr{>'Estw me'izwn <h AB, ka`i t~h| AB s'ummetroc >'estw <h GD; l'egw,
<'oti <h GD me'izwn >est'in.}

\gr{Dih|r'hsjw <h AB kat`a t`o E; a<i AE, EB >'ara dun'amei e>is`in >as'ummetroi poio~usai t`o m`en sugke'imenon >ek t~wn >ap> a>ut~wn
tetrag'wnwn <rht'on, t`o d> <up> a>ut~wn m'eson; ka`i gegon'etw
t`a a>ut`a to~ic pr'oteron. ka`i >epe'i >estin <wc <h AB pr`oc t`hn
GD, o<'utwc <'h te AE pr`oc t`hn GZ ka`i <h EB pr`oc t`hn ZD, ka`i <wc >'ara <h AE pr`oc t`hn GZ, o<'utwc <h EB pr`oc t`hn ZD. s'ummetroc
d`e <h AB t~h| GD; s'ummetroc >'ara ka`i <ekat'era t~wn AE, EB <ekat'era|
t~wn GZ, ZD. ka`i >epe'i >estin <wc <h AE pr`oc t`hn GZ, o<'utwc
<h EB pr`oc t`hn ZD,  ka`i
>enall`ax <wc <h AE pr`oc EB, o<'utwc <h GZ pr`oc ZD, ka`i
sunj'enti >'ara <est`in <wc <h AB pr`oc t`hn BE, o<'utwc  <h GD pr`oc
t`hn DZ; ka`i <wc >'ara t`o >ap`o t~hc AB pr`oc t`o >ap`o t~hc BE, o<'utwc
t`o >ap`o t~hc GD pr`oc t`o >ap`o t~hc DZ. <omo'iwc d`h de'ixomen,
<'oti ka`i <wc t`o >ap`o t~hc AB pr`oc t`o >ap`o t~hc AE, o<'utwc
t`o >ap`o t~hc GD pr`oc t`o >ap`o t~hc GZ. ka`i <wc >'ara t`o >ap`o
t~hc AB pr`oc t`a >ap`o t~wn AE, EB, o<'utwc t`o >ap`o t~hc GD
pr`oc t`a >ap`o t~wn GZ, ZD; ka`i >enall`ax >'ara >est`in <wc t`o
>ap`o t~hc AB pr`oc t`o >ap`o t~hc GD, o<'utwc t`a >ap`o t~wn
AE, EB pr`oc t`a >ap`o t~wn GZ, ZD. s'ummetron d`e t`o >ap`o
t~hc AB t~w| >ap`o t~hc GD; s'ummetra >'ara ka`i t`a >ap`o t~wn AE, EB
to~ic >ap`o t~wn GZ, ZD. ka'i >esti t`a >ap`o t~wn AE, EB <'ama
<rht'on, ka`i t`a >ap`o t~wn GZ, ZD <'ama <rht'on >estin. <omo'iwc
d`e ka`i t`o d`ic <up`o t~wn AE, EB s'ummetr'on >esti t~w| d`ic
<up`o t~wn
GZ, ZD. ka'i >esti m'eson t`o d`ic <up`o t~wn AE, EB;
m'eson >'ara ka`i t`o d`ic <up`o t~wn GZ, ZD. a<i GZ, ZD
>'ara dun'amei >as'ummetro'i
e>isi poio~usai t`o m`en sugke'imenon >ek t~wn >ap> a>ut~wn
tetrag'wnwn <'ama <rht'on, t`o d`e d`ic <up> a>ut~wn m'eson;
<'olh >'ara <h GD >'alog'oc >estin <h kaloum'enh me'izwn.}

\gr{<H >'ara t~h| me'izoni s'ummetroc me'izwn >est'in; <'oper
>'edei de~ixai.}}

\ParallelRText{
\begin{center}
{\large Proposition 68}
\end{center}

A (straight-line) commensurable (in length)
with a major (straight-line) is  itself also major.

\epsfysize=0.7in 
\centerline{\epsffile{Book10/fig066e.eps}}

Let $AB$ be a major (straight-line), and let $CD$ be commensurable (in
length) with $AB$. I say that $CD$ is a major (straight-line).

Let $AB$ have been divided (into its component terms) at $E$.
$AE$ and $EB$ are thus incommensurable in square, making the sum
of the squares on them rational, and the (rectangle contained) by them
medial [Prop. 10.39]. And let (the) same (things) have been contrived as in the previous (propositions). And since
as $AB$ is to $CD$, so $AE$ (is) to $CF$ and $EB$ to $FD$, 
thus also as $AE$ (is) to $CF$, so $EB$ (is) to $FD$ [Prop. 5.11]. And $AB$ (is) commensurable
(in length) with $CD$. Thus, $AE$ and $EB$ (are) also commensurable
(in length) with $CF$ and $FD$, respectively [Prop. 10.11].  And since as $AE$ is to $CF$, so
$EB$ (is) to $FD$, also, alternately, as $AE$ (is) to $EB$, so
$CF$ (is) to $FD$ [Prop. 5.16], and thus, via composition, as $AB$ is to $BE$, so $CD$ (is) to $DF$ [Prop. 5.18]. And thus as the (square) on $AB$
(is) to the (square) on $BE$, so the (square) on $CD$ (is) to the (square)
on $DF$ [Prop. 6.20]. So, similarly, we can also show that as the (square) on $AB$ (is) to the (square) on $AE$, so the
(square) on $CD$ (is) to the (square) on $CF$. And thus as the (square)
on $AB$ (is) to (the sum of) the (squares) on $AE$ and $EB$, so
the (square) on $CD$ (is) to (the sum of) the (squares) on $CF$ and $FD$.
And thus, alternately, as the (square) on $AB$ is to the (square) on $CD$,
so (the sum of) the (squares) on $AE$ and $EB$ (is) to (the sum of)
the (squares) on $CF$ and $FD$ [Prop. 5.16]. And the (square) on $AB$
(is) commensurable with the (square) on $CD$. Thus, (the sum of) the
(squares) on $AE$ and $EB$ (is) also commensurable with (the sum of)
the (squares) on $CF$ and $FD$ [Prop. 10.11]. 
And the (squares) on $AE$ and $EB$ (added) together are rational.
The (squares) on $CF$ and $FD$ (added) together (are) thus also
rational. So, similarly, twice the (rectangle contained) by $AE$ and
$EB$ is also commensurable with twice the (rectangle contained)
by $CF$ and $FD$. And twice the (rectangle contained) by $AE$ and $EB$
is medial. Therefore, twice the (rectangle contained) by $CF$ and
$FD$ (is) also medial [Prop. 10.23~corr.]. 
$CF$ and $FD$ are thus (straight-lines which are) incommensurable in square [Prop~10.13], simultaneously making the sum of the
squares on them rational, and twice the (rectangle contained) by them medial.
The whole, $CD$, is thus that irrational (straight-line) called major
[Prop. 10.39].

Thus, a (straight-line) commensurable (in length) with a major
(straight-line) is major. (Which is) the very thing it was required to show.}
\end{Parallel}

%%%%
%10.69
%%%%
\pdfbookmark[1]{Proposition 10.69}{pdf10.69}
\begin{Parallel}{}{}
\ParallelLText{
\begin{center}
{\large \ggn{69}.}
\end{center}\vspace*{-7pt}

\gr{<H t~h| <rht`on ka`i m'eson dunam'enh| s'ummetroc [ka`i a>ut`h]
<rht`on ka`i m'eson dunam'enh >est'in.}\\

\epsfysize=0.7in
\centerline{\epsffile{Book10/fig066g.eps}}

\gr{>'Estw <rht`on ka`i m'eson dunam'enh <h AB, ka`i t~h|
AB s'ummetroc >'estw <h GD; deikt'eon, <'oti ka`i <h GD
<rht`on ka`i m'eson dunam'enh >est'in.}

\gr{Dih|r'hsjw <h AB e>ic t`ac e>uje'iac kat`a t`o E; a<i AE, EB >'ara
dun'amei e>is`in >as'ummetroi poio~usai
t`o m`en sugke'imenon >ek t~wn >ap> a>ut~wn tetrag'wnwn
m'eson, t`o d> <up> a>ut~wn <rht'on; ka`i t`a a>ut`a kateskeu'asjw
to~ic pr'oteron. <omo'iwc d`h de'ixomen, <'oti ka`i a<i GZ,
ZD dun'amei e>is`in >as'ummetroi, ka`i s'ummetron t`o m`en
sugke'imenon >ek t~wn >ap`o t~wn AE, EB t~w| sugkeim'enw|
>ek t~wn >ap`o t~wn GZ, ZD, t`o d`e <up`o AE, EB t~w|
<up`o GZ, ZD; <'wste ka`i t`o [m`en] sugke'imenon >ek t~wn
>ap`o t~wn GZ, ZD tetrag'wnwn >est`i m'eson, t`o d> <up`o
t~wn GZ, ZD <rht'on.}

\gr{<Rht`on >'ara ka`i m'eson dunam'enh >est`in <h GD; <'oper
>'edei de~ixai.}}

\ParallelRText{
\begin{center}
{\large Proposition 69}
\end{center}

A (straight-line) commensurable (in length)
with the square-root of a rational plus a medial (area) is [itself also]
the square-root of a rational plus a medial (area).

\epsfysize=0.7in 
\centerline{\epsffile{Book10/fig066e.eps}}

Let $AB$ be the square-root of a rational plus a medial (area), 
and let $CD$ be commensurable (in length) with $AB$. We must  show
that $CD$ is also the square-root of a rational plus a medial (area).

Let $AB$ have been divided into its (component) straight-lines at $E$.
$AE$ and $EB$ are thus incommensurable in square, making 
the sum of the squares on them medial, and the (rectangle contained)
by them rational [Prop. 10.40]. And let the
same construction have been made as in the previous (propositions).
So, similarly, we can show that $CF$ and $FD$ are also incommensurable
in square, and that the sum of the (squares) on $AE$ and $EB$
(is) commensurable with the sum of the (squares) on $CF$ and $FD$,
and the (rectangle contained) by $AE$ and $EB$ with the
(rectangle contained) by $CF$ and $FD$. And hence the sum of the
squares on $CF$ and $FD$ is medial, and the (rectangle contained) by
$CF$ and $FD$ (is) rational.

Thus, $CD$ is the square-root of a rational plus a medial (area) [Prop. 10.40]. (Which is) the very thing it was required to show.}
\end{Parallel}

%%%%
%10.70
%%%%
\pdfbookmark[1]{Proposition 10.70}{pdf10.70}
\begin{Parallel}{}{}
\ParallelLText{
\begin{center}
{\large \ggn{70}.}
\end{center}\vspace*{-7pt}

\gr{<H t~h| d'uo m'esa dunam'enh| s'ummetroc d'uo m'esa dunam'enh >est'in.}\\

\epsfysize=0.7in
\centerline{\epsffile{Book10/fig066g.eps}}

\gr{>'Estw d'uo m'esa dunam'enh <h AB, ka`i t~h| AB s'ummetroc
<h GD; deikt'eon, <'oti ka`i <h GD d'uo m'esa dunam'enh >est'in.}

\gr{>Epe`i g`ar d'uo m'esa dunam'enh >est`in <h AB, dih|r'hsjw
e>ic t`ac e>uje'iac kat`a t`o E; a<i AE, EB >'ara dun'amei e>is`in
>as'ummetroi poio~usai t'o te sugke'imenon >ek t~wn >ap> a>ut~wn
[tetrag'wnwn] m'eson ka`i t`o <up> a>ut~wn m'eson ka`i >'eti
>as'ummetron t`o sugke'imenon >ek t~wn >ap`o t~wn AE, EB
tetrag'wnwn t~w| <up`o t~wn AE, EB; ka`i kateskeu'asjw t`a a>ut`a
to~ic pr'oteron. <omo'iwc d`h de'ixomen, <'oti ka`i a<i GZ, ZD dun'amei
e>is`in >as'ummetroi ka`i s'ummetron t`o m`en sugke'imenon >ek
t~wn >ap`o t~wn AE, EB t~w| sugkeim'enw| >ek t~wn >ap`o t~wn
GZ, ZD, t`o d`e <up`o t~wn AE, EB t~w| <up`o t~wn GZ, ZD;
<'wste ka`i t`o sugke'imenon >ek t~wn >ap`o t~wn GZ, ZD
 tetrag'wnwn
m'eson >est`i ka`i t`o <up`o t~wn GZ, ZD m'eson ka`i >'eti >as'ummetron
t`o sugke'imenon >ek t~wn >ap`o t~wn GZ, ZD tetrag'wnwn t~w| <up`o
t~wn GZ, ZD.}

\gr{<H >'ara GD d'uo m'esa dunam'enh >est'in; <'oper >'edei de~ixai.}}

\ParallelRText{
\begin{center}
{\large Proposition 70}
\end{center}

A (straight-line) commensurable (in length)
with the square-root of (the sum of) two medial (areas) is (itself also)
the square-root of (the sum of) two medial (areas).

\epsfysize=0.7in 
\centerline{\epsffile{Book10/fig066e.eps}}

Let $AB$ be the square-root of (the sum of) two medial (areas),
and (let) $CD$ (be) commensurable (in length) with $AB$. We must show that
$CD$ is also the square-root of (the sum of) two medial (areas).

For since $AB$ is the square-root of (the sum of) two medial (areas),
let it have been divided into its (component) straight-lines at $E$.
Thus, $AE$ and $EB$ are incommensurable in square, making the sum of
the [squares] on them medial, and the (rectangle contained) by them
medial, and, moreover, the sum of the (squares) on $AE$ and $EB$
incommensurable with the (rectangle) contained by $AE$ and $EB$ [Prop. 10.41]. And let the same construction
have been made as in the previous (propositions). So, similarly, we
can show that $CF$ and $FD$ are also incommensurable in square,
and (that) the sum of the (squares) on $AE$ and $EB$ (is)
commensurable with the sum of the (squares) on $CF$ and $FD$, and the
(rectangle contained) by $AE$ and $EB$ with the (rectangle contained)
by $CF$ and $FD$. Hence, the sum of the squares on $CF$ and
$FD$ is also medial, and the (rectangle contained) by $CF$ and $FD$
(is) medial, and, moreover, the sum of the squares on $CF$ and $FD$
(is) incommensurable with the (rectangle contained) by $CF$ and $FD$.

Thus, $CD$ is the square-root of (the sum of) two medial (areas) [Prop. 10.41].
(Which is) the very thing it was required to show.}
\end{Parallel}

%%%%
%10.71
%%%%
\pdfbookmark[1]{Proposition 10.71}{pdf10.71}
\begin{Parallel}{}{}
\ParallelLText{
\begin{center}
{\large \ggn{71}.}
\end{center}\vspace*{-7pt}

\gr{<Rhto~u ka`i m'esou suntijem'enou t'essarec >'alogoi g'ignon\-tai
>'htoi >ek d'uo >onom'atwn >`h >ek d'uo m'eswn pr'wth >`h
me'izwn >`h <rht`on ka`i m'eson dunam'enh.}

\gr{>'Estw <rht`on m`en t`o AB, m'eson d`e t`o GD; l'egw, <'oti <h t`o
AD qwr'ion dunam'enh >'htoi >ek d'uo >onom'atwn >est`in
>`h >ek d'uo m'eswn pr'wth >`h me'izwn >`h <rht`on ka`i
m'eson dunam'enh.}\\~\\~\\

\epsfysize=2.in
\centerline{\epsffile{Book10/fig071g.eps}}

\gr{T`o g`ar AB to~u GD >'htoi me~iz'on >estin >`h >'elasson.
>'estw pr'oteron me~izon; ka`i >ekke'isjw <rht`h <h EZ, ka`i
parabebl'hsjw par`a t`hn EZ t~w| AB >'ison t`o EH
pl'atoc poio~un t`hn EJ; t~w| d`e DG >'ison par`a t`hn EZ parabebl'hsjw
t`o JI pl'atoc poio~un t`hn JK. ka`i >epe`i <rht'on >esti t`o AB ka'i
>estin >'ison t~w| EH, <rht`on >'ara ka`i t`o EH. ka`i par`a [<rht`hn]
t`hn EZ parab'eblhtai pl'atoc poio~un t`hn EJ; <h EJ >'ara <rht'h >esti
ka`i s'ummetroc t~h| EZ m'hkei. p'alin, >epe`i m'eson >est`i t`o GD ka'i
>estin >'ison t~w| JI, m'eson >'ara >est`i ka`i t`o JI. ka`i par`a <rht`hn
t`hn EZ par'akeitai pl'atoc poio~un t`hn JK; <rht`h >'ara >est`in <h JK
ka`i >as'ummetroc t~h| EZ m'hkei. ka`i >epe`i m'eson >est`i t`o GD,
<rht`on d`e t`o AB, >as'ummetron >'ara >est`i t`o AB t~w| GD; <'wste
ka`i t`o EH >as'ummetr'on >esti t~w| JI. <wc d`e t`o EH pr`oc t`o JI,
o<'utwc >est`in <h EJ pr`oc t`hn JK; >as'ummetroc >'ara >est`i ka`i
<h EJ t~h| JK m'hkei. ka'i e>isin >amf'oterai <rhta'i; a<i EJ, JK
>'ara <rhta'i e>isi dun'amei m'onon s'ummetroi; >ek d'uo >'ara
>onom'atwn >est`in <h EK dih|rhm'enh kat`a t`o J. ka`i >epe`i
me~iz'on >esti t`o AB to~u GD, >'ison d`e t`o m`en AB t~w| EH, t`o d`e
GD t~w| JI, me~izon >'ara ka`i t`o EH to~u JI; ka`i <h EJ
>'ara me'izwn >est`i t~hc JK. >'htoi o>~un <h EJ t~hc JK me~izon
d'unatai t~w| >ap`o summ'etrou <eaut~h| m'hkei >`h t~w|
>ap`o >asumm'etrou. dun'asjw pr'oteron t~w| >ap`o summ'etrou
<eaut~h|; ka'i >estin <h me'izwn <h JE s'ummetroc t~h| >ekkeim'enh|
<rhth| t~h| EZ; <h >'ara EK >ek d'uo >onom'atwn >est`i pr'wth. <rht`h
d`e <h EZ; >e`an d`e qwr'ion peri'eqhtai <up`o <rht~hc ka`i t~hc >ek
d'uo >onom'atwn pr'wthc, <h t`o qwr'ion dunam'enh >ek d'uo >onom'atwn
>est'in. <h >'ara t`o EI dunam'enh >ek d'uo >onom'atwn >est'in; <'wste
ka`i <h t`o AD dunam'enh >ek d'uo >onom'atwn >est'in. >all`a d`h dun'asjw
<h EJ t~hc JK me~izon t~w| >ap`o >asumm'etrou <eaut~h|; ka'i
>estin <h me'izwn <h EJ s'ummetroc t~h| >ekkeim'enh| <rht~h|
t~h| EZ m'hkei; <h >'ara EK >ek d'uo >onom'atwn >est`i tet'arth.
<rht`h d`e <h EZ; >e`an d`e qwr'ion peri'eqhtai <up`o <rht~hc
ka`i t~hc >ek d'uo >onom'atwn tet'arthc, <h t`o qwr'ion dunam'enh >'alog'oc
>estin <h kaloum'enh me'izwn. <h >'ara t`o EI qwr'ion dunam'enh me'izwn
>est'in; <'wste ka`i <h t`o AD dunam'enh me'izwn >est'in.}

\gr{>All`a d`h >'estw >'elasson t`o AB to~u GD; ka`i t`o EH >'ara >'elass'on
>esti to~u JI; <'wste ka`i <h EJ >el'asswn >est`i t~hc JK. >'htoi d`e <h JK
t~hc EJ
 me~izon d'unatai t~w| >ap`o summ'etrou
<eaut~h| >`h t~w| >ap`o >asumm'etrou. dun'asjw pr'oteron t~w| >ap`o
summ'etrou <eaut~h| m'hkei; ka'i >estin <h >el'asswn <h EJ s'ummetroc
t~h| >ekkeim'enh| <rht~h| t~h| EZ m'hkei; <h >'ara EK >ek d'uo
>onom'atwn >est`i deut'era. <rht`h d`e <h EZ; >e`an d`e qwr'ion
peri'eqhtai <up`o <rht~hc ka`i t~hc >ek d'uo >onom'atwn deut'erac,
<h t`o qwr'ion dunam'enh >ek d'uo m'eswn >est`i pr'wth. <h >'ara
t`o EI qwr'ion dunam'enh >ek d'uo m'eswn >est`i pr'wth; <'wste
ka`i <h t`o AD dunam'enh >ek d'uo m'eswn >est`i pr'wth. >all`a d`h
<h JK t~hc JE me~izon dun'asjw t~w| >ap`o >asumm'etrou <eaut~h|.
ka'i >estin <h >el'asswn <h EJ s'ummetroc t~h| >ekkeim'enh|
<rht~h| t~h| EZ; <h >'ara EK >ek d'uo >onom'atwn >est`i p'empth.
<rht`h d`e <h EZ; >e`an d`e qwr'ion peri'eqhtai <up`o <rht~hc ka`i t~hc
>ek d'uo >onom'atwn p'empthc, <h t`o qwr'ion dunam'enh <rht`on ka`i m'eson dunam'enh >est'in. <h >'ara t`o EI qwr'ion dunam'enh <rht`on
ka`i m'eson dunam'enh >est'in; <'wste ka`i <h t`o AD qwr'ion dunam'enh
<rht`on ka`i m'eson dunam'enh >est'in.}

\gr{<Rhto~u >'ara ka`i m'esou suntijem'enou t'essarec >'alogoi g'ignontai
>'htoi >ek d'uo >onom'atwn >`h >ek d'uo m'eswn pr'wth >`h me'izwn
>`h <rht`on ka`i m'eson dunam'enh; <'oper >'edei de~ixai.}}

\ParallelRText{
\begin{center}
{\large Proposition 71}
\end{center}

When a rational and a medial (area) are added
together, four irrational (straight-lines) arise (as the square-roots of the
total area)---either a binomial, or a first bimedial, or a major, or the
square-root of a rational plus a medial (area).

Let $AB$ be a rational (area), and $CD$ a medial (area). I say that the
square-root of area $AD$ is either binomial, or first bimedial, or
major, or the square-root of a rational plus a medial (area).

\epsfysize=2.in 
\centerline{\epsffile{Book10/fig071e.eps}}

For $AB$ is either greater or less than $CD$. Let it, first of all,
be greater. And let the rational (straight-line) $EF$ be laid down. And
let (the rectangle) $EG$, equal to $AB$, have been applied to $EF$,
producing $EH$ as breadth. And let (the recatangle) $HI$, equal to $DC$,
have been applied to $EF$, producing $HK$ as breadth. And since
$AB$ is rational, and is equal to $EG$, $EG$ is thus also rational. And it
has been applied to the [rational] (straight-line) $EF$, producing $EH$
as breadth. $EH$ is thus rational, and commensurable in length
with $EF$ [Prop. 10.20]. Again, since
$CD$ is medial, and is equal to $HI$, $HI$ is thus also medial. 
And it is applied to the rational (straight-line) $EF$, producing
$HK$ as breadth. $HK$ is thus rational, and incommensurable
in length with $EF$ [Prop. 10.22]. And since
$CD$ is medial, and $AB$ rational, $AB$ is thus incommensurable
with $CD$. Hence, $EG$ is also incommensurable with $HI$. And
as $EG$ (is) to $HI$, so $EH$ is to $HK$ [Prop. 6.1].
Thus, $EH$ is also incommensurable in length with $HK$
[Prop. 10.11]. And they are both rational. Thus,
$EH$ and $HK$ are rational (straight-lines which are) commensurable in
 square only. $EK$ is thus a binomial (straight-line), having been divided
(into its component terms) at $H$ [Prop. 10.36].
And since $AB$ is greater than $CD$, and $AB$ (is) equal to $EG$, and
$CD$ to $HI$, $EG$ (is) thus also greater than $HI$. Thus, $EH$
is also greater than $HK$ [Prop. 5.14]. Therefore,
 the square on $EH$ is greater than (the square on) $HK$
either by the (square) on (some straight-line) commensurable in length
with ($EH$), or by the (square) on (some straight-line)  incommensurable
(in length with $EH$). 
Let it, first of all, be greater by the (square) on (some straight-line)
commensurable (in length with $EH$). And the greater (of the two components of $EK$) $HE$ is commensurable
(in length) with the (previously) laid down (straight-line) $EF$. $EK$ is thus a first binomial
(straight-line) [Def. 10.5]. And $EF$ (is) rational. And if an area is contained by a rational (straight-line) and a first
binomial (straight-line) then the square-root  of the  area is a binomial
(straight-line)  [Prop. 10.54]. Thus,
the square-root of $EI$ is a binomial (straight-line). Hence the square-root of $AD$ is also a binomial (straight-line).  And, so, let the
 square on $EH$ be greater than (the square on) $HK$ by the (square) on 
 (some straight-line) incommensurable (in length) with ($EH$). And the
 greater (of the two components of $EK$) $EH$ is commensurable in length with the (previously)
 laid down rational (straight-line) $EF$. Thus, $EK$ is a fourth
 binomial (straight-line) [Def. 10.8]. And
 $EF$ (is) rational. And if an area
 is contained by a rational (straight-line) and a fourth binomial (straight-line) 
 then the square-root of the area is the irrational (straight-line) called
 major [Prop. 10.57]. Thus, the square-root
 of area $EI$ is a major (straight-line). Hence, the square-root of
 $AD$ is also major.
 
 And so, let $AB$ be less than $CD$. Thus, $EG$ is also less
 than $HI$. Hence, $EH$ is also less than $HK$ [Props.~6.1, 5.14].
 And the square on $HK$ is greater than (the square on) $EH$ either
 by the (square) on (some straight-line) commensurable (in length)
 with ($HK$), or by the (square) on (some straight-line) incommensurable
 (in length) with ($HK$). Let it, first of all, be greater by the square
 on (some straight-line) commensurable in length with ($HK$).
 And the lesser (of the two components of $EK$) $EH$ is  commensurable in length with the
 (previously) laid down rational (straight-line) $EF$. Thus, $EK$
 is a second binomial (straight-line) [Def. 10.6]. 
 And $EF$ (is) rational. And if an area is contained by a rational (straight-line)
 and a second binomial (straight-line)  then the square-root of the area is a
 first bimedial (straight-line) [Prop. 10.55]. 
 Thus, the square-root of area $EI$ is a first bimedial (straight-line).
 Hence, the square-root of $AD$ is also a first bimedial (straight-line).
 And so, let the square on $HK$ be greater than (the square on) $HE$
 by the (square) on (some straight-line) incommensurable (in length) with
 ($HK$). 
 And the lesser (of the two components of $EK$) $EH$ is commensurable (in length) with the (previously) laid down rational (straight-line) $EF$.  Thus, $EK$ is a fifth binomial
 (straight-line) [Def. 10.9]. And $EF$ (is) rational. And if an area is
 contained by a rational (straight-line) and a fifth binomial (straight-line) 
 then the square-root of the area is the square-root of a rational plus
 a medial (area) [Prop. 10.58]. Thus, the square-root of area $EI$ is the square-root of a rational plus a medial (area). Hence,
 the square-root of area $AD$ is also the square-root of a rational plus
 a medial (area).
 
 Thus, when a rational and a medial area are added together, four irrational
(straight-lines) arise (as the square-roots of the
total area)---either a binomial, or a first bimedial, or a major, or the
square-root of a rational plus a medial (area). (Which is) the very thing
it was required to show.}
\end{Parallel}

%%%%
%10.72
%%%%
\pdfbookmark[1]{Proposition 10.72}{pdf10.72}
\begin{Parallel}{}{}
\ParallelLText{
\begin{center}
{\large \ggn{62}.}
\end{center}\vspace*{-7pt}

\gr{D'uo m'eswn >asumm'etrwn >all'hloic suntijem'enwn a<i loipa`i 
d'uo >'alogoi g'ignontai >'htoi >ek d'uo m'eswn deut'era >`h [<h]
d'uo m'esa dunam'enh.}\\~\\

\epsfysize=2in
\centerline{\epsffile{Book10/fig071g.eps}}

\gr{Sugke'isjw g`ar d'uo m'esa >as'ummetra >all'hloic t`a AB, GD; l'egw,
<'oti <h t`o AD qwr'ion dunam'enh >'htoi >ek d'uo m'eswn
>est`i deut'era >`h d'uo m'esa dunam'enh.}

\gr{T`o g`ar AB to~u GD >'htoi me~iz'on >estin  >`h >'elasson. >'estw,
e>i t'uqon, pr'oteron me~izon t`o AB to~u GD; ka`i >ekke'isjw <rht`h
<h EZ, ka`i t~w| m`en AB >'ison par`a t`hn EZ parabebl'hsjw t`o EH
pl'atoc poio~un t`hn EJ, t~w| d`e GD >'ison t`o JI pl'atoc poio~un t`hn JK.
ka`i >epe`i m'eson >est`in <ek'ateron t~wn AB, GD, m'eson
>'ara ka`i >ek'ateron t~wn EH, JI. ka`i par`a <rht`hn t`hn ZE par'akeitai
pl'atoc poio~un t`ac EJ, JK; <ekat'era >'ara t~wn EJ, JK <rht'h
>esti ka`i >as'ummetroc t~h| EZ m'hkei. ka`i >epe`i >as'ummetr'on
>esti t`o AB t~w| GD, ka'i >estin >'ison t`o m`en AB t~w| EH, 
t`o d`e GD t~w| JI, >as'ummetron >'ara >est`i ka`i t`o EH t~w| JI.
<wc d`e t`o EH pr`oc t`o JI, o<'utwc >est`in <h EJ
pr`oc JK; >as'ummetroc >'ara >est`in <h EJ t~h| JK m'hkei. a<i 
EJ, JK >'ara <rhta'i e>isi dun'amei m'onon s'ummetroi; >ek
d'uo >'ara >onom'atwn >est`in <h EK. >'htoi d`e <h EJ t~hc JK
me~izon d'unatai t~w| >ap`o summ'etrou <eaut~h| >`h t~w| >ap`o
>asumm'etrou. dun'asjw pr'oteron t~w| >ap`o summ'etrou <eaut~h|
m'hkei; ka`i o>udet'era t~wn EJ, JK s'ummetr'oc >esti t~h| >ekkeim'enh|
<rht~h| t~h| EZ m'hkei; <h EK >'ara >ek d'uo >onom'atwn >est`i tr'ith.
<rht`h d`e <h EZ; >e`an d`e qwr'ion peri'eqhtai <up`o <rht~hc 
ka`i t~hc >ek d'uo >onom'atwn tr'ithc, <h t`o qwr'ion dunam'enh
>ek d'uo m'eswn >est`i deut'era; <h >'ara t`o EI, tout'esti t`o AD,
dunam'enh >ek d'uo m'eswn >est`i deut'era. >alla d`h <h EJ t~hc JK
me~izon dun'asjw t~w| >ap`o >asumm'etrou <eaut~h| m'hkei; ka`i
>as'ummetr'oc >estin <ekat'era t~wn EJ, JK t~h| EZ m'hkei; <h >'ara
EK >ek d'uo >onom'atwn >est`in <'ekth. >e`an d`e qwr'ion peri'eqhtai
<up`o <rht~hc ka`i t~hc >ek d'uo >onom'atwn <'ekthc, <h t`o
qwr'ion dunam'enh <h d'uo m'esa dunam'enh >est'in; <'wste ka`i <h
t`o AD qwr'ion dunam'enh <h d'uo m'esa dunam'enh >est'in.}

\gr{\mbox{[}<Omo'iwc d`h de'ixomen, <'oti k>`an >'elatton >~h| t`o AB
to~u GD, <h t`o AD qwr'ion dunam'enh >`h >ek d'uo m'eswn
deut'era >est`in >'htoi d'uo m'esa dunam'enh].}

\gr{D'uo >'ara m'eswn >asumm'etrwn >all'hloic suntijem'enwn a<i loipa`i
d'uo >'alogoi g'ignontai >'htoi >ek d'uo m'eswn deut'era >`h d'uo m'esa
dunam'enh.}\\

\gr{>H >ek d'uo >onom'atwn ka`i a<i met> a>ut`hn >'alogoi o>'ute
t~h| m'esh| o>'ute >all'hlaic e>is`in a<i a>uta'i. t`o m`en g`ar >ap`o
m'eshc par`a <rht`hn paraball'omenon pl'atoc poie~i <rht`hn ka`i
>as'ummetron t~h| par> <`hn par'akeitai m'hkei. t`o d`e >ap`o t~hc
>ek d'uo >onom'atwn par`a <rht`hn paraball'omenon pl'atoc poie~i
t`hn >ek d'uo >onom'atwn pr'wthn. t`o d`e >ap`o t~hc >ek d'uo m'eswn
pr'wthc par`a <rht`hn paraball'omenon pl'atoc poie~i t`hn >ek d'uo
>onom'atwn deut'eran. t`o d`e >ap`o t~hc >ek d'uo m'eswn deut'erac par`a
<rht`hn paraball'omenon pl'atoc poie~i t`hn >ek d'uo >onom'atwn
tr'ithn. t`o d`e >ap`o t~hc me'izonoc par`a <rht`hn paraball'omenon
pl'atoc poie~i t`hn >ek d'uo >onom'atwn
tet'arthn. t`o d`e >ap`o t~hc <rht`on ka`i m'eson dunam'enhc
par`a <rht`hn paraball'omenon pl'atoc poie~i t`hn >ek d'uo
>onom'atwn p'empthn. t`o d`e >ap`o t~hc d'uo m'esa dunam'enhc
par`a <rht`hn paraball'omenon pl'atoc poie~i t`hn >ek d'uo >onom'atwn
<'ekthn. t`a d> e>irhm'ena pl'ath diaf'erei to~u te pr'wtou ka`i >all'hlwn,
to~u m`en pr'wtou, <'oti <rht'h >estin, >all'hlwn d'e, <'oti t~h| t'axei
o>uk e>is`in a<i a>uta'i; <'wste ka`i a>uta`i a<i >'alogoi diaf'erousin
>all'hlwn.}}

\ParallelRText{
\begin{center}
{\large Proposition 72}
\end{center}

When two medial (areas which are) incommensurable with one another are added together, the remaining
two irrational (straight-lines) arise (as the square-roots of the total area)---either a second bimedial, or the square-root of (the sum of)
two medial (areas).

\epsfysize=2in 
\centerline{\epsffile{Book10/fig071e.eps}}

For let the two medial (areas) $AB$ and $CD$, (which are) incommensurable
with one another, have been added together. I say that the square-root
of area $AD$ is either a second bimedial, or the square-root of (the sum of)
two medial (areas).

For $AB$ is either greater than or less than $CD$. By chance, let $AB$,
first of all,  be greater than $CD$. And let the rational (straight-line)
$EF$ be laid down. And let $EG$, equal to $AB$, have been applied
to $EF$, producing $EH$ as breadth, and $HI$, equal to $CD$, producing
$HK$ as breadth. And since $AB$ and $CD$ are each medial,
$EG$ and $HI$ (are) thus also each medial. And they are applied to
the rational straight-line $FE$, producing $EH$ and $HK$ (respectively) as breadth. Thus, $EH$ and $HK$ are  each rational (straight-lines which are) incommensurable in length with $EF$ [Prop. 10.22]. And since $AB$ is incommensurable
with $CD$, and $AB$ is equal to $EG$, and $CD$ to $HI$, $EG$
is thus also incommensurable with $HI$. And as $EG$ (is) to $HI$,
so $EH$ is to $HK$ [Prop. 6.1].  $EH$ is thus
incommensurable in length with $HK$ [Prop. 10.11]. Thus, $EH$ and $HK$ are rational
(straight-lines which are) commensurable in square only. $EK$
is thus a binomial (straight-line) [Prop. 10.36]. 
And the square on $EH$ is  greater than (the square on) $HK$
either by the (square) on (some straight-line) commensurable (in length) with ($EH$), or
by the (square) on (some straight-line) incommensurable (in length
with $EH$). Let it, first of all, be greater by the square on (some
straight-line) commensurable in length with ($EH$). And neither of
$EH$ or $HK$ is commensurable in length with the (previously)
laid down rational (straight-line) $EF$.  Thus, $EK$ is a third
binomial (straight-line) [Def. 10.7]. And $EF$
(is) rational. And if an area is contained by a rational (straight-line) and
a third binomial (straight-line)  then the square-root of the area
is a second bimedial  (straight-line) [Prop. 10.56]. Thus,
the square-root of $EI$---that is to say, of $AD$---is a second bimedial.
And so, let the square on $EH$ be greater than (the square) on $HK$
by the (square)  on (some straight-line) incommensurable in length
with ($EH$).  And $EH$ and $HK$
are each incommensurable
in length with $EF$. Thus, $EK$ is a sixth binomial (straight-line)
[Def. 10.10]. And if an area
is contained by
a rational (straight-line) and a sixth binomial (straight-line)  then the square-root of the area is the square-root of (the sum of) two medial (areas) 
[Prop. 10.59]. Hence, the
square-root of area $AD$ is also the square-root of (the sum of)
two medial (areas).

\mbox{[}So, similarly, we can show that, even if $AB$ is less than $CD$, the
square-root of area $AD$ is either a second bimedial or the square-root
of (the sum of) two medial (areas).]

Thus, when two medial (areas which are) incommensurable with one another are added together, the remaining
two irrational (straight-lines) arise (as the square-roots of the total area)---either a second bimedial, or the square-root of (the sum of)
two medial (areas).\\

A binomial (straight-line), and the (other) irrational (straight-lines) 
after it, are neither the same as a medial (straight-line) nor (the same) as one another.
For the (square) on a medial (straight-line), applied to a rational (straight-line),
produces as breadth a rational (straight-line which is) also incommensurable
in length with (the straight-line) to which it is applied [Prop. 10.22]. And the (square) on a binomial
(straight-line), applied to a rational (straight-line), produces
as breadth a first binomial [Prop. 10.60]. 
And the (square) on a first bimedial (straight-line), applied
to a rational (straight-line), produces as breadth a  second binomial
[Prop. 10.61]. And the (square) on a second bimedial (straight-line), applied
to a rational (straight-line), produces as breadth a  third binomial
[Prop. 10.62]. And the (square) on a major (straight-line), applied
to a rational (straight-line), produces as breadth a  fourth binomial
[Prop. 10.63].  And the (square) on the square-root of a rational plus a medial (area), applied
to a rational (straight-line), produces as breadth a  fifth binomial
[Prop. 10.64]. And the (square) on the square-root of (the sum of) two medial (areas), applied
to a rational (straight-line), produces as breadth a  sixth binomial
[Prop. 10.65]. And the aforementioned breadths
differ from the first (breadth), and from one another---from the first, because it is rational---and from one another, because they are not the same in order. Hence, the (previously mentioned) irrational (straight-lines) themselves
also differ from one another.}
\end{Parallel}

%%%%
%10.73
%%%%
\pdfbookmark[1]{Proposition 10.73}{pdf10.73}
\begin{Parallel}{}{}
\ParallelLText{
\begin{center}
{\large \ggn{73}.}
\end{center}\vspace*{-7pt}

\gr{>E`an >ap`o <rht~hc <rht`h >afairej~h| dun'amei m'onon s'ummetroc o>~usa
t~h| <'olh|, <h loip`h >'alog'oc >estin; kale'isjw d`e >apotom'h.}\\~\\

\epsfysize=0.3in
\centerline{\epsffile{Book10/fig073g.eps}}

\gr{>Ap`o g`ar <rht~hc t~hc AB <rht`h >afh|r'hsjw <h BG dun'amei m'onon
s'ummetroc o>~usa t~h| <'olh|; l'egw, <'oti <h loip`h <h AG >'alog'oc
>estin <h kaloum'enh >apotom'h.}

\gr{>Epe`i g`ar >as'ummetr'oc >estin <h AB t~h| BG m'hkei, ka'i >estin
<wc <h AB pr`oc t`hn BG, o<'utwc t`o >ap`o t~hc AB pr`oc t`o <up`o
t~wn AB, BG, >as'ummetron >'ara >est`i t`o >ap`o t~hc AB t~w|
<up`o t~wn AB, BG. >all`a t~w| m`en >ap`o t~hc AB s'ummetr'a >esti
t`a >ap`o t~wn AB, BG tetr'agwna, t~w| d`e <up`o t~wn AB, BG s'ummetr'on
>esti t`o d`ic <up`o t~wn AB, BG. ka`i >epeid'hper t`a >ap`o t~wn AB, BG
>'isa >est`i t~w| d`ic <up`o t~wn AB, BG met`a to~u >ap`o GA, ka`i
loip~w| >'ara t~w| >ap`o t~hc AG >as'ummetr'a >esti t`a >ap`o t~wn AB, BG.
<rht`a d`e t`a >ap`o t~wn AB, BG; >'alogoc >'ara >est`in <h AG; kale'isjw
d`e >apotom'h. <'oper >'edei de~ixai.}}

\ParallelRText{
\begin{center}
{\large Proposition 73}
\end{center}

If  a  rational (straight-line), which is commensurable in square only with the
whole, is subtracted from a(nother) rational (straight-line)  then the remainder is an irrational (straight-line). Let it be called an apotome.

\epsfysize=0.3in 
\centerline{\epsffile{Book10/fig073e.eps}}

For let the rational (straight-line) $BC$, which commensurable in square only with the whole,  have been subtracted from the
rational (straight-line) $AB$. I say that the remainder $AC$ is that irrational
(straight-line) called an apotome.

For since $AB$ is incommensurable in length with $BC$, and as $AB$ is to
$BC$, so the (square) on $AB$ (is) to the (rectangle contained) by $AB$ and
$BC$ [Prop. 10.21~lem.], the
(square) on $AB$ is thus incommensurable with the (rectangle contained)
by $AB$ and $BC$ [Prop. 10.11]. 
But, the (sum of the) squares on $AB$ and $BC$ is commensurable with the
(square) on $AB$ [Prop. 10.15], and twice
the (rectangle contained) by $AB$ and $BC$ is commensurable with the
(rectangle contained) by $AB$ and $BC$ [Prop. 10.6]. And, inasmuch as the (sum of the squares) on $AB$ and $BC$ is equal to twice the (rectangle contained) by
$AB$ and $BC$ plus the (square) on $CA$ [Prop. 2.7], the (sum of the squares) on $AB$ and $BC$
is thus also incommensurable with the remaining (square) on $AC$ [Props.~10.13, 10.16]. 
And the (sum of the squares) on $AB$ and $BC$ is rational. $AC$ is
thus an irrational (straight-line) [Def. 10.4]. And let it be called an
apotome.$^\dag$ (Which is) the very thing it was required to show.}
\end{Parallel}


\vspace{7pt}{\footnotesize\noindent $^\dag$ See footnote to Prop.~10.36.}

%%%%
%10.74
%%%%
\pdfbookmark[1]{Proposition 10.74}{pdf10.74}
\begin{Parallel}{}{}
\ParallelLText{
\begin{center}
{\large \ggn{74}.}
\end{center}\vspace*{-7pt}

\gr{>E`an >ap`o m'eshc m'esh >afairej~h| dun'amei m'onon s'ummetroc
o>~usa t~h| <'olh|, met`a d`e t~hc <'olhc <rht`on peri'eqousa, <h loip`h
>'alog'oc >estin; kale'isjw d`e m'eshc >apotom`h pr'wth.}\\~\\~\\

\epsfysize=0.3in
\centerline{\epsffile{Book10/fig073g.eps}}

\gr{>Ap`o g`ar m'eshc t~hc AB m'esh >afh|r'hsjw <h BG dun'amei
m'onon s'ummetroc o>~usa t~h| AB, met`a d`e t~hc AB <rht`on
poio~usa t`o <up`o t~wn AB, BG; l'egw, <'oti
<h loip`h <h AG >'alog'oc >estin; kale'isjw d`e m'eshc >apotom`h
pr'wth.}

\gr{>Epe`i g`ar a<i  AB, BG m'esai e>is'in, m'esa >est`i ka`i t`a >ap`o t~wn
AB, BG. <rht`on d`e t`o d`ic <up`o t~wn AB, BG; >as'ummetra >'ara t`a
>ap`o t~wn AB, BG t~w| d`ic <up`o
t~wn AB, BG; ka`i loip~w| >'ara t~w| >ap`o t~hc AG >as'ummetr'on >esti
t`o d`ic <up`o t~wn AB, BG, >epe`i k>`an t`o <'olon <en`i a>ut~wn
>as'ummetron >~h|, ka`i t`a >ex >arq~hc meg'ejh >as'ummetra >'estai.
<rht`on d`e t`o d`ic <up`o t~wn AB, BG;
>'alogon >'ara t`o >ap`o t~hc AG; >'alogoc >'ara >est`in <h AG;
kale'isjw d`e m'eshc >apotom`h pr'wth.}}

\ParallelRText{
\begin{center}
{\large Proposition 74}
\end{center}

If  a medial (straight-line), which is commensurable in square only
with the whole, and which contains a rational (area) with the whole,  is subtracted from a(nother) medial
(straight-line)   then
the remainder is an irrational (straight-line). Let it be called a
first  apotome of a medial (straight-line).

\epsfysize=0.3in 
\centerline{\epsffile{Book10/fig073e.eps}}

For let the medial (straight-line) $BC$, which is commensurable in square only with $AB$,
and which makes with $AB$ the rational (rectangle contained) by $AB$ and $BC$,
have been subtracted from the medial (straight-line) $AB$ [Prop. 10.27]. I say that
the remainder $AC$ is an irrational (straight-line). Let it be
called the first apotome of a medial (straight-line).

For since $AB$ and $BC$ are medial (straight-lines), the (sum of the squares)
on $AB$ and $BC$ is also medial. And twice the (rectangle contained)
by $AB$ and $BC$ (is) rational. The (sum of the squares) on $AB$
and $BC$ (is) thus incommensurable with twice the (rectangle contained)
by $AB$ and $BC$. Thus, twice the (rectangle contained) by
$AB$ and $BC$ is also incommensurable with the remaining (square) on $AC$ [Prop. 2.7], since  if the whole is
incommensurable with one of the (constituent magnitudes)   then the original magnitudes will also be incommensurable (with one another) [Prop. 10.16]. And twice the (rectangle
contained) by $AB$ and $BC$ (is) rational. Thus, the (square) on $AC$
is irrational. Thus, $AC$ is an irrational (straight-line) [Def. 10.4]. Let it be called a first apotome
of a medial (straight-line).$^\dag$}
\end{Parallel}


\vspace{7pt}{\footnotesize\noindent $^\dag$ See footnote to Prop.~10.37.} 

%%%%
%10.75
%%%%
\pdfbookmark[1]{Proposition 10.75}{pdf10.75}
\begin{Parallel}{}{}
\ParallelLText{
\begin{center}
{\large \ggn{75}.}
\end{center}\vspace*{-7pt}

\gr{>E`an >ap`o m'eshc m'esh >afairej~h| dun'amei m'onon s'ummetroc
o>~usa t~h| <'olh, met`a d`e t~hc <'olhc m'eson peri'eqousa, <h
loip`h >'alog'oc >estin; kale'isjw d`e m'eshc >apotom`h deut'era.}

\gr{>Ap`o g`ar m'eshc t~hc AB m'esh >afh|r'hsjw <h GB dun'amei m'onon
s'ummetroc o>~usa t~h| <'olh| t~h| AB, met`a d`e t~hc <'olhc t~hc AB
m'eson peri'eqousa t`o <up`o t~wn AB, BG; l'egw, <'oti
<h loip`h <h AG >'alog'oc >estin; kale'isjw d`e m'eshc >apotom`h
deut'era.}\\~\\~\\~\\~\\

\epsfysize=1.7in
\centerline{\epsffile{Book10/fig075g.eps}}

\gr{>Ekke'isjw g`ar <rht`h <h DI, ka`i to~ic m`en >ap`o t~wn
AB, BG >'ison par`a t`hn DI parabebl'hsjw t`o DE
pl'atoc poio~un t`hn DH, t~w| d`e d`ic <up`o t~wn AB, BG
>'ison par`a t`hn DI parabebl'hsjw t`o DJ pl'atoc poio~un t`hn
DZ; loip`on >'ara t`o ZE >'ison >est`i t~w| >ap`o t~hc AG.
ka`i >epe`i m'esa ka`i s'ummetr'a >esti t`a >ap`o t~wn AB, BG,
m'eson >'ara ka`i t`o DE. ka`i par`a <rht`hn t`hn DI par'akeitai
pl'atoc poio~un t`hn DH; <rht`h >'ara >est`in <h DH ka`i >as'ummetroc
t~h| DI m'hkei. p'alin, >epe`i m'eson >est`i t`o <up`o t~wn AB, BG, ka`i
t`o d`ic >'ara <up`o t~wn AB, BG m'eson >est'in. ka'i >estin >'ison
t~w| DJ; ka`i t`o DJ >'ara m'eson >est'in. ka`i par`a <rht`hn t`hn DI
parab'eblhtai pl'atoc poio~un t`hn DZ; <rht`h >'ara >est`in <h DZ
ka`i >as'ummetroc t~h| DI m'hkei. ka`i >epe`i a<i AB, BG dun'amei
m'onon s'ummetro'i e>isin, >as'ummetroc >'ara >est`in <h AB
t~h| BG m'hkei; >as'ummetron >'ara ka`i t`o >ap`o t~hc AB
tetr'agwnon t~w| <up`o t~wn AB, BG. >all`a t~w| m`en >ap`o
t~hc AB s'ummetr'a >esti t`a >ap`o t~wn AB, BG, t~w| d`e <up`o
t~wn AB, BG s'ummetr'on >esti t`o d`ic <up`o t~wn AB, BG;
>as'ummetron >'ara >est`i t`o d`ic <up`o t~wn AB, BG to~ic
>ap`o t~wn AB, BG. >'ison d`e to~ic m`en >ap`o t~wn AB, BG
t`o DE, t~w| d`e d`ic <up`o t~wn AB, BG t`o DJ; >as'ummetron
>'ara [>est`i] t`o DE t~w| DJ. <wc d`e t`o DE pr`oc t`o DJ,
o<'utwc <h HD pr`oc t`hn DZ; >as'ummetroc >'ara >est`in <h HD
t~h| DZ. ka'i e>isin >amf'oterai <rhta'i; a<i >'ara HD, DZ
<rhta'i e>isi dun'amei m'onon s'ummetroi; <h ZH >'ara >apotom'h
>estin. <rht`h d`e <h DI; t`o d`e <up`o <rht~hc ka`i >al'ogou
perieq'omenon >'alog'on >estin, ka`i <h dunam'enh a>ut`o
>'alog'oc >estin. ka`i d'unatai t`o ZE <h AG; <h AG >'ara >'alog'oc
>estin; kale'isjw d`e m'eshc >apotom`h deut'era. <'oper
>'edei de~ixai.}}

\ParallelRText{
\begin{center}
{\large Proposition 75}
\end{center}

If  a medial (straight-line), which is commensurable in square only
with the whole, and which contains a medial (area) with the whole, is subtracted from a(nother) medial
(straight-line)   then
the remainder is an irrational (straight-line). Let it be called a second apotome of a medial (straight-line).

For let the medial (straight-line) $CB$, which is commensurable in square
only with the whole, $AB$, and which contains with the whole, $AB$, the
medial (rectangle contained) by $AB$ and $BC$, have been subtracted from the medial (straight-line) $AB$ [Prop. 10.28].  I say that the remainder $AC$ is an
irrational (straight-line). Let it be called a second apotome of a medial (straight-line).

\epsfysize=1.7in 
\centerline{\epsffile{Book10/fig075e.eps}}

For let the rational (straight-line) $DI$ be laid down. And let $DE$, equal
to the (sum of the squares) on $AB$ and $BC$, have been applied to
$DI$, producing $DG$ as breadth. And let $DH$, equal to twice the
(rectangle contained) by $AB$ and $BC$, have been applied to $DI$,
producing $DF$ as breadth. The remainder $FE$ is thus equal to the
(square) on $AC$ [Prop. 2.7]. And since the
(squares) on $AB$ and $BC$ are medial and commensurable (with one another), $DE$ (is) thus also medial [Props.~10.15, 10.23~corr.]. And it is applied to the
rational (straight-line) $DI$, producing $DG$ as breadth. Thus, $DG$
is rational, and incommensurable in length with $DI$ [Prop. 10.22].  Again, since the (rectangle contained)
by $AB$ and $BC$ is medial, twice the (rectangle contained) by $AB$
and $BC$ is thus also medial [Prop. 10.23~corr.]. 
And it is equal to $DH$. Thus, $DH$ is also medial. And it has been
applied to the rational (straight-line) $DI$, producing $DF$ as breadth.
$DF$ is thus rational, and incommensurable in length with $DI$
[Prop. 10.22].  And since $AB$ and $BC$ are commensurable in square only, $AB$ is thus incommensurable in length
with $BC$. Thus, the square on $AB$ (is) also incommensurable
with the (rectangle contained) by $AB$ and $BC$ [Props.~10.21~lem.,
10.11]. But, the (sum of the squares) on $AB$ and $BC$  is commensurable with the (square) on $AB$  [Prop. 10.15], and twice the (rectangle contained)
by $AB$ and $BC$ is commensurable with the (rectangle contained)
by $AB$ and $BC$ [Prop. 10.6]. Thus, twice
the (rectangle contained) by $AB$ and $BC$ is incommensurable
with the (sum of the squares) on $AB$ and $BC$ [Prop. 10.13]. And $DE$ is equal
to the (sum of the squares) on $AB$ and $BC$, and $DH$ to
twice the (rectangle contained) by $AB$ and $BC$. Thus, $DE$
[is] incommensurable with $DH$. And as $DE$ (is) to $DH$, so
$GD$ (is) to $DF$ [Prop. 6.1]. Thus,
$GD$ is incommensurable with $DF$ [Prop. 10.11]. And they are both rational (straight-lines). Thus, $GD$ and $DF$ are rational (straight-lines which are)
commensurable in square only.
Thus, $FG$ is an apotome [Prop. 10.73]. 
And $DI$ (is) rational. And the (area) contained by a rational and
an irrational (straight-line) is irrational [Prop. 10.20], and its square-root 
is irrational. And $AC$ is the square-root of $FE$. Thus, $AC$ is an
irrational (straight-line) [Def. 10.4]. And
let it be called the second apotome of a medial (straight-line).$^\dag$
(Which is) the very thing it
was required to show.}
\end{Parallel}


\vspace{7pt}{\footnotesize\noindent$^\dag$ See footnote to Prop.~10.38.} 

%%%%
%10.76
%%%%
\pdfbookmark[1]{Proposition 10.76}{pdf10.76}
\begin{Parallel}{}{}
\ParallelLText{
\begin{center}
{\large \ggn{76}.}
\end{center}\vspace*{-7pt}

\gr{>E`an >ap`o e>uje'iac e>uje~ia >afairej`h| dun'amei >as'ummet\-roc
o>~usa t~h| <'olh|, met`a d`e t~hc <'olhc poio~usa t`a m`en >ap>
a>ut~wn <'ama <rht'on, t`o d> <up> a>ut~wn m'eson, <h loip`h
>'alog'oc >estin; kale'isjw d`e >el'asswn.}\\~\\

\epsfysize=0.27in
\centerline{\epsffile{Book10/fig073g.eps}}

\gr{>Ap`o g`ar e>uje'iac t~hc AB e>uje~ia >afh|r'hsjw <h BG dun'amei
>as'ummetroc o>~usa t~h| <'olh| poio~usa t`a proke'imena. l'egw,
<'oti <h loip`h <h AG >'alog'oc >estin <h kaloum'enh >el'asswn.}

\gr{>Epe`i g`ar t`o m`en sugke'imenon >ek t~wn >ap`o t~wn AB, BG tetrag'wnwn <rht'on >estin, t`o d`e d`ic <up`o t~wn AB, BG m'eson,
>as'ummetra >'ara >est`i t`a >ap`o t~wn AB, BG t~w| d`ic <up`o
t~wn AB, BG; ka`i >anastr'eyanti loip~w| t~w| >ap`o t~hc AG
>as'ummetr'a >esti t`a >ap`o t~wn AB, BG.  <rht`a d`e t`a
>ap`o t~wn AB, BG;
>'alogon >'ara
t`o >ap`o t~hc AG;
 >'alogoc
>'ara <h AG; kale'isjw d`e >el'asswn. <'oper >'edei de~ixai.}}

\ParallelRText{
\begin{center}
{\large Proposition 76}
\end{center}

If  a straight-line, which is incommensurable in square with the whole,
and  with the whole makes the (squares) on them (added) together rational, and the
(rectangle contained) by them medial, is subtracted from a(nother) straight-line
 then the remainder is an
irrational (straight-line). Let it be called a minor (straight-line).

\epsfysize=0.27in 
\centerline{\epsffile{Book10/fig073e.eps}}

For let the straight-line $BC$, which is incommensurable in square with the whole, and fulfils the (other) prescribed (conditions), have been subtracted from
the straight-line $AB$ [Prop. 10.33]. I say that the remainder $AC$ is that irrational (straight-line)
called minor.

For since the sum of the squares on $AB$ and $BC$ is rational, and twice
the (rectangle contained) by $AB$ and $BC$ (is) medial, the (sum of the squares) on $AB$ and $BC$ is thus incommensurable with twice the
(rectangle contained) by $AB$ and $BC$. And, via conversion, 
the (sum of the squares) on $AB$ and $BC$ is incommensurable with the
remaining (square) on $AC$ [Props.~2.7, 10.16]. And the (sum of the squares) on $AB$
and $BC$ (is) rational. The (square) on $AC$ (is) thus irrational. Thus, 
$AC$ (is) an irrational (straight-line) [Def. 10.4]. Let it be
called a minor (straight-line).$^\dag$ (Which is) the very thing it
was required to show.}
\end{Parallel}


\vspace{7pt}{\footnotesize\noindent$^\dag$ See footnote to Prop.~10.39.}

%%%%
%10.77
%%%%
\pdfbookmark[1]{Proposition 10.77}{pdf10.77}
\begin{Parallel}{}{}
\ParallelLText{
\begin{center}
{\large \ggn{77}.}
\end{center}\vspace*{-7pt}

\gr{>E`an >ap`o e>uje'iac e>uje~ia >afairej~h| dun'amei >as'ummet\-roc
o>~usa t~h| <'olh|, met`a d`e t~hc <'olhc poio~usa t`o m`en sugke'imenon
>ek t~wn >ap> a>ut~wn tetrag'wnwn m'eson, t`o d`e d`ic <up>
a>ut~wn <rht'on, <h loip`h >'alog'oc >estin; kale'isjw d`e <h met`a <rhto~u
m'eson t`o <'olon poio~usa.}\\~\\

\epsfysize=0.27in
\centerline{\epsffile{Book10/fig073g.eps}}

\gr{>Ap`o g`ar e>uje'iac t~hc AB e>uje~ia >afh|r'hsjw <h BG dun'amei
>as'ummetoc o>~usa t~h| AB poio~usa t`a proke'imena; l'egw,
<'oti <h loip`h <h AG >'alog'oc >estin <h proeirhm'enh.}

\gr{>Epe`i g`ar t`o m`en sugke'imenon >ek t~wn >ap`o t~wn AB, BG
tetrag'wnwn m'eson >est'in, t`o d`e d`ic <up`o t~wn AB, BG <rht'on,
>as'ummetra >'ara >est`i t`a >ap`o t~wn AB, BG t~w|
d`ic <up`o t~wn AB, BG; ka`i loip`on >'ara t`o >ap`o t~hc AG >as'ummetr'on
>esti t~w| d`ic <up`o t~wn AB, BG. ka'i >esti t`o d`ic <up`o t~wn
AB, BG <rht'on; t`o >'ara >ap`o t~hc AG >'alog'on >estin; >'alogoc
>'ara >est`in <h AG; kale'isjw d`e <h met`a <rhto~u m'eson t`o <'olon
poio~usa. <'oper >'edei de~ixai.}}

\ParallelRText{
\begin{center}
{\large Proposition 77}
\end{center}

If  a straight-line, which is incommensurable in  square with the whole, and with the
whole makes the sum of the squares on them medial, and twice
the (rectangle contained) by them rational, is subtracted from a(nother) straight-line 
 then the remainder
is an irrational (straight-line). Let it be called that which makes
with a rational (area) a medial whole.

\epsfysize=0.27in 
\centerline{\epsffile{Book10/fig073e.eps}}

For let the straight-line $BC$, which is incommensurable in
square with $AB$, and fulfils the (other) prescribed (conditions), have been
subtracted from the straight-line $AB$ [Prop. 10.34]. I say that the remainder
$AC$ is the aforementioned irrational (straight-line).

For since the sum of the squares on $AB$ and $BC$ is medial, and
twice the (rectangle contained) by $AB$ and $BC$ rational,
the (sum of the squares) on $AB$ and $BC$ is thus incommensurable
with twice the (rectangle contained) by $AB$ and $BC$. Thus, the
remaining (square) on $AC$ is also incommensurable with
twice the (rectangle contained) by $AB$ and $BC$ [Props.~2.7, 10.16]. And twice
the (rectangle contained) by $AB$ and $BC$ is rational. 
Thus, the (square) on $AC$ is irrational. Thus, $AC$ is an
irrational (straight-line) [Def. 10.4]. And let it be
called that which makes
with a rational (area) a medial whole.$^\dag$ (Which is) the very thing it
was required to show.}
\end{Parallel}


\vspace{7pt}{\footnotesize\noindent$^\dag$ See footnote to Prop.~10.40.}

%%%%
%10.78
%%%%
\pdfbookmark[1]{Proposition 10.78}{pdf10.78}
\begin{Parallel}{}{}
\ParallelLText{
\begin{center}
{\large \ggn{78}.}
\end{center}\vspace*{-7pt}

\gr{>E`an >ap`o e>uje'iac e>uje~ia >afairej~h| dun'amei >as'umm\-etroc
o>~usa t~h| <'olh|, met`a d`e t~hc <'olhc poio~usa t'o te sugke'imenon
>ek t~wn >ap> a>ut~wn tetrag'wnwn m'eson t'o te d`ic <up> a>ut~wn
m'eson ka`i >'eti t`a >ap> a>ut~wn tetr'agwna >as'ummetra t~w|
d`ic <up> a>ut~wn, <h loip`h >'alog'oc >estin; kale'isjw d`e <h met`a
m'esou m'eson t`o <'olon poio~usa.}\\~\\~\\

\epsfysize=1.7in
\centerline{\epsffile{Book10/fig078g.eps}}

\gr{>Ap`o g`ar e>uje'iac t~hc AB e>uje~ia >afh|r'hsjw <h BG dun'amei
>as'ummetroc o>~usa t~h| AB poio~usa t`a proke'imena; l'egw,
<'oti <h loip`h <h AG >'alog'oc >estin <h kaloum'enh <h met`a
m'esou m'eson t`o <'olon poio~usa.}

\gr{>Ekke'isjw g`ar <rht`h <h DI, ka`i to~ic m`en >ap`o t~wn AB, BG >'ison
par`a t`hn DI parabebl'hsjw t`o DE pl'atoc poio~un t`hn DH, t~w| d`e d`ic
<up`o t~wn AB, BG >'ison >afh|r'hsjw t`o DJ [pl'atoc poio~un t`hn
DZ]. loip`on >'ara t`o ZE >'ison >est`i t~w| >ap`o t~hc AG;
<'wste <h AG d'unatai t`o ZE. ka`i >epe`i t`o sugke'imenon >ek
t~wn >ap`o t~wn AB, BG tetrag'wnwn m'eson >est`i ka'i >estin
>'ison t~w| DE, m'eson >'ara [>est`i] t`o DE. ka`i par`a <rht`hn
t`hn DI par'akeitai pl'atoc poio~un t`hn DH; <rht`h
>'ara >est`in <h DH ka`i >as'ummetroc t~h| DI m'hkei. p'alin,
>epe`i t`o d`ic <up`o t~wn AB, BG m'eson >est`i ka'i >estin >'ison
t~w| DJ, t`o >'ara DJ m'eson >est'in. ka`i par`a <rht`hn t`hn
DI par'akeitai pl'atoc poio~un  t`hn DZ;
<rht`h >'ara >est`i ka`i <h DZ ka`i >as'ummetroc t~h| DI m'hkei.
 ka`i >epe`i >as'ummetr'a >esti t`a >ap`o t~wn
AB, BG t~w| d`ic <up`o t~wn AB, BG, >as'ummetron >'ara ka`i
t`o DE t~w| DJ. <wc d`e t`o DE pr`oc t`o DJ, o<'utwc >est`i
ka`i <h DH pr`oc t`hn DZ; >as'ummetroc >'ara <h DH t~h| DZ.
ka'i e>isin >amf'oterai <rhta'i; a<i HD, DZ >'ara <rhta'i
e>isi dun'amei m'onon s'ummetroi. >apotom`h >'ara >est'in <h
ZH; <rht`h d`e <h ZJ. t`o d`e <up`o <rht~hc ka`i >apotom~hc
perieq'omenon [>orjog'wnion] >'alog'on >estin, ka`i <h dunam'enh
a>ut`o >'alog'oc >estin; ka`i d'unatai t`o ZE <h AG; <h AG >'ara >'alog'oc >estin;
 kale'isjw d`e <h met`a m'esou m'eson t`o <'olon
poio~usa. <'oper >'edei de~ixai.}}

\ParallelRText{
\begin{center}
{\large Proposition 78}
\end{center}

If a straight-line, which is incommensurable in square with the whole, and with the whole
makes the sum of the squares on them medial, and twice the (rectangle
contained) by them medial, and, moreover, the (sum of the) squares on them
incommensurable with twice the (rectangle contained) by them, is
subtracted from a(nother) straight-line  then the
remainder is an irrational (straight-line). Let it be called that which makes
with a medial (area) a medial whole.

\epsfysize=1.7in 
\centerline{\epsffile{Book10/fig078e.eps}}

For let the straight-line $BC$, which is incommensurable in square
$AB$, and fulfils the (other) prescribed (conditions), have been subtracted from the
(straight-line) $AB$ [Prop. 10.35].  I say that the remainder $AC$ is the irrational (straight-line) called  that which makes with a medial (area) a medial whole.

For let the rational (straight-line) $DI$ be laid down. And let $DE$, equal to
the (sum of the squares) on $AB$ and $BC$, have been applied to $DI$,
producing $DG$ as breadth. And let $DH$, equal to twice the (rectangle
contained) by $AB$ and $BC$, have been subtracted (from $DE$) [producing
$DF$ as breadth]. Thus, the remainder $FE$ is equal to the (square) on 
$AC$ [Prop. 2.7]. Hence, $AC$ is the square-root of $FE$. And since the sum of the squares on $AB$ and $BC$ is medial, and
is equal to $DE$, $DE$ [is] thus medial. And it is applied to the
rational (straight-line) $DI$, producing $DG$ as breadth. Thus, $DG$
is rational, and incommensurable in length with $DI$ [Prop~10.22]. Again, since twice the (rectangle
contained) by $AB$ and $BC$ is medial, and is equal to $DH$, $DH$
is thus medial. And it is applied to the rational (straight-line)
$DI$, producing $DF$ as breadth.  Thus, $DF$ is also rational, and incommensurable in length with $DI$ [Prop. 10.22]. And since  the (sum of the squares)
on $AB$ and $BC$ is incommensurable with twice the (rectangle contained)
by $AB$ and $BC$, $DE$ (is) also incommensurable with $DH$. 
And as $DE$ (is) to $DH$, so $DG$  also is to $DF$ [Prop. 6.1]. Thus, $DG$ (is) incommensurable (in length) with
$DF$ [Prop. 10.11]. And they are both rational. 
Thus, $GD$ and $DF$ are rational (straight-lines which are) commensurable
in square only. Thus, $FG$ is an apotome [Prop. 10.73]. And $FH$ (is) rational. And the
[rectangle] contained by a rational (straight-line) and an apotome is
irrational [Prop. 10.20], and its square-root
is irrational. And $AC$ is the square-root of $FE$.  Thus, $AC$
is irrational. Let it be called that which makes with a medial (area) a medial
whole.$^\dag$ 
(Which is) the very thing it was required to show.}
\end{Parallel}


\vspace{7pt}{\footnotesize\noindent$^\dag$ See footnote to Prop.~10.41.}

%%%%
%10.79
%%%%
\pdfbookmark[1]{Proposition 10.79}{pdf10.79}
\begin{Parallel}{}{}
\ParallelLText{
\begin{center}
{\large \ggn{79}.}
\end{center}\vspace*{-7pt}

\gr{T~h| >apotom~h| m'ia [m'onon] prosarm'ozei e>uje~ia <rht`h dun'amei
m'onon s'ummetroc o>~usa t~h| <'olh|.}\\

\epsfysize=0.27in
\centerline{\epsffile{Book10/fig079g.eps}}

\gr{>'Estw >apotom`h <h AB, prosarm'ozousa d`e a>ut~h| <h BG; a<i AG, GB
>'ara <rhta'i e>isi dun'amei m'onon s'ummetroi; l'egw, <'oti t~h| AB <et'era o>u
prosarm'ozei <rht`h dun'amei m'onon s'ummetroc o>~usa t~h| <'ol~h|.}

\gr{E>i g`ar dunat'on, prosarmoz'etw <h BD; ka`i a<i AD, DB >'ara <rhta'i
e>isi dun'amei m'onon s'ummetroi. ka`i >epe'i, <~w| <uper'eqei t`a
>ap`o t~wn AD, DB to~u d`ic <up`o t~wn AD, DB, to'utw| <uper'eqei
ka`i t`a >ap`o t~wn AG, GB to~u d`ic <up`o t~wn AG, GB; t~w| g`ar
a>ut~w| t~w| >ap`o t~hc AB >amf'otera <uper'eqei; >enall`ax >'ara, <~w|
<uper'eqei t`a >ap`o t~wn AD, DB t~wn >ap`o t~wn AG, GB, to'utw| <uper'eqei [ka`i] t`o d`ic <up`o t~wn AD, DB
to~u d`ic <up`o t~wn AG, GB. t`a d`e >ap`o t~wn AD, DB t~wn >ap`o t~wn
AG, GB <uper'eqei <rht~w|; <rht`a g`ar >amf'otera. ka`i t`o d`ic >'ara <up`o
t~wn AD, DB to~u d`ic <up`o t~wn AG, GB <uper'eqei <rht~w|; <'oper >est`in >ad'unaton; m'esa g`ar >amf'otera, m'eson d`e m'esou o>uq <uper'eqei
<rht~w|. t~h| >'ara AB <et'era o>u prosarm'ozei <rht`h dun'amei
m'onon s'ummetroc o>~usa t~h| <'olh|.}

\gr{M'ia >'ara m'onh t~h| >apotom~h| prosarm'ozei <rht`h dun'amei
m'onon s'ummetroc o>~usa t~h| <'olh|; <'oper >'edei
de~ixai.}}

\ParallelRText{
\begin{center}
{\large Proposition 79}
\end{center}

\mbox{[}Only] one rational straight-line, which is
commensurable in square only with the whole, can be attached to an apotome.$^\dag$

\epsfysize=0.26in 
\centerline{\epsffile{Book10/fig079e.eps}}

Let $AB$ be an apotome, with $BC$  (so) attached to it. $AC$ and $CB$ are thus
rational (straight-lines which are) commensurable in square only [Prop. 10.73]. I say that another rational (straight-line), which is commensurable in  square only with the whole, cannot
be attached to $AB$. 

For, if possible, let $BD$ be (so) attached (to $AB$). Thus, $AD$ and $DB$ are also
rational (straight-lines which are) commensurable in square only [Prop. 10.73]. And since by whatever (area)
the (sum of the squares) on $AD$ and $DB$ exceeds twice the
(rectangle contained) by $AD$ and $DB$, the (sum of the squares)
on $AC$ and $CB$ also exceeds twice the (rectangle contained) by $AC$ and
$CB$ by this (same area). For both exceed by the same (area)---(namely), the (square)
on $AB$ [Prop. 2.7]. Thus, alternately, 
by whatever (area) the (sum of the squares) on $AD$ and 
$DB$ exceeds the (sum of the squares) on $AC$ and $CB$, 
 twice the (rectangle contained) by $AD$ and $DB$
[also] exceeds twice the (rectangle contained) by $AC$ and $CB$ by this (same area).
And the (sum of the squares) on $AD$ and $DB$ exceeds the (sum
of the squares) on $AC$ and $CB$ by a rational (area). For both
(are) rational (areas). Thus, twice the (rectangle contained) by $AD$ and $DB$ also
exceeds twice the (rectangle contained) by $AC$ and $CB$ by a
rational (area). The very thing is impossible. For both are
medial (areas) [Prop. 10.21],
and a medial (area) cannot exceed a(nother) medial (area) by a rational
(area) [Prop. 10.26].  
Thus, another rational (straight-line), which is commensurable in  square only with the whole, cannot
be attached to $AB$.

Thus, only one rational (straight-line), which is
commensurable in square only with the whole, can be attached to an apotome.
(Which is) the very thing it was required to show.}
\end{Parallel}


\vspace{7pt}{\footnotesize\noindent$^\dag$ This proposition is equivalent to 
Prop.~10.42, with minus signs instead of
plus signs.}

%%%%
%10.80
%%%%
\pdfbookmark[1]{Proposition 10.80}{pdf10.80}
\begin{Parallel}{}{}
\ParallelLText{
\begin{center}
{\large \ggn{80}.}
\end{center}\vspace*{-7pt}

\gr{T~h| m'eshc >apotom~h| pr'wth| m'ia m'onon prosarm'ozei e>uje~ia
m'esh dun'amei m'onon s'ummetroc o>~usa t~h| <'olh|, met`a d`e t~hc <'olhc
<rht`on peri'eqousa.}\\

\epsfysize=0.26in
\centerline{\epsffile{Book10/fig079g.eps}}

\gr{>'Estw g`ar m'eshc >apotom`h pr'wth <h AB, ka`i t~h| AB prosarmoz'etw <h BG; a<i AG,
GB >'ara m'esai e>is`i dun'amei m'onon s'ummetroi <rht`on peri'eqousai
t`o <up`o t~wn AG, GB; l'egw, <'oti t~h| AB <et'era o>u prosarm'ozei
m'esh dun'amei m'onon s'ummetroc o>~usa t~h| <'olh|, met`a d`e
t~hc <'olhc <rht`on peri'eqousa.}

\gr{E>i g`ar dunat'on, prosarmoz'etw ka`i <h DB; a<i >'ara 
AD, DB m'esai e>is`i dun'amei m'onon s'ummetroi <rht`on peri'eqousai
t`o <up`o t~wn AD, DB. ka`i >epe'i, <~w| <uper'eqei t`a >ap`o t~wn
AD, DB to~u d`ic <up`o t~wn AD, DB, to'utw| <uper'eqei ka`i t`a >ap`o
t~wn AG, GB to~u d`ic <up`o t~wn AG, GB; t~w| g`ar a>ut~w| [p'alin]
<uper'eqousi t~w| >ap`o t~hc AB; >enall`ax >'ara, <~w| <uper'eqei t`a
>ap`o t~wn AD, DB t~wn >ap`o t~wn AG, GB, to'utw| <uper'eqei
ka`i t`o d`ic <up`o t~wn
AD, DB to~u d`ic <up`o t~wn
AG, GB. t`o d`e d`ic <up`o t~wn AD, DB to~u d`ic <up`o t~wn AG, GB
<uper'eqei <rht~w|; <rht`a g`ar >amf'otera. ka`i t`a >ap`o t~wn AD, DB
>'ara t~wn >ap`o t~wn AG, GB [tetrag'wnwn]
<uper'eqei <rht~w|; <'oper >est`in >ad'unaton; m'esa g'ar >estin >amf'otera,
m'eson d`e m'esou o>uq <uper'eqei <rht~w|.}

\gr{T~h| >'ara m'eshc >apotom~h| pr'wth| m'ia m'onon prosarm'ozei
e>uje~ia m'esh dun'amei m'onon s'ummetroc o>~usa t~h| <'olh|,
met`a d`e t~hc <'olhc <rht`on peri'eqousa; <'oper >'edei de~ixai.}}

\ParallelRText{
\begin{center}
{\large Proposition 80}
\end{center}

Only one medial straight-line,
which is commensurable in square only with the whole, and
contains a rational (area) with the whole, can be attached
to a first apotome of a medial (straight-line).$^\dag$

\epsfysize=0.26in 
\centerline{\epsffile{Book10/fig079e.eps}}

For let $AB$ be a first apotome of a medial (straight-line), and let $BC$ be (so) attached
to $AB$. Thus, $AC$ and $CB$ are medial (straight-lines which are)
commensurable in square only, containing a rational (area)---(namely, that
contained) by $AC$ and $CB$ [Prop. 10.74]. 
I say that a(nother) medial (straight-line), which is commensurable in square only with the whole, and
contains a rational (area) with the whole, cannot be attached to $AB$.

For, if possible, let $DB$ also be (so) attached to $AB$. Thus, $AD$ and
$DB$ are medial (straight-lines which are) commensurable in square only, containing a rational (area)---(namely, that)
contained by $AD$ and $DB$ [Prop. 10.74]. 
And since by whatever (area)
the (sum of the squares) on $AD$ and $DB$ exceeds twice the
(rectangle contained) by $AD$ and $DB$,  the (sum of the squares)
on $AC$ and $CB$ also exceeds twice the (rectangle contained) by $AC$ and
$CB$ by this (same area). For [again] both exceed by the same (area)---(namely), the (square)
on $AB$ [Prop. 2.7]. Thus, alternately, 
by whatever (area) the (sum of the squares) on $AD$ and 
$DB$ exceeds the (sum of the squares) on $AC$ and $CB$, 
 twice the (rectangle contained) by $AD$ and $DB$
also exceeds twice the (rectangle contained) by $AC$ and $CB$ by this (same area).
And twice the (rectangle contained) by $AD$ and $DB$ exceeds twice the (rectangle contained) by $AC$ and $CB$ by a rational (area). For both
(are) rational (areas). Thus,  the (sum of the squares) on $AD$ and $DB$ also
exceeds the (sum of the) [squares] on $AC$ and $CB$ by a
rational (area). The very thing is impossible. For both are
medial (areas) [Props.~10.15, 10.23~corr.],
and a medial (area) cannot exceed a(nother) medial (area) by a rational
(area) [Prop. 10.26].

Thus, only one medial (straight-line),
which is commensurable in square only with the whole, and
contains a rational (area) with the whole, can be attached
to a first apotome of a medial (straight-line). (Which is) the very thing it was required to
show.}
\end{Parallel}


\vspace{7pt}{\footnotesize\noindent$^\dag$ This proposition is equivalent to 
Prop.~10.43, with minus signs instead of
plus signs.}

%%%%
%10.81
%%%%
\pdfbookmark[1]{Proposition 10.81}{pdf10.81}
\begin{Parallel}{}{}
\ParallelLText{
\begin{center}
{\large \ggn{81}.}
\end{center}\vspace*{-7pt}

\gr{T~h| m'eshc >apotom~h| deut'era| m'ia m'onon prosarm'ozei e>uje~ia
m'esh dun'amei m'onon s'ummetroc t~h| <'olh|, met`a d`e t~hc <'olhc
m'eson peri'eqousa.}\\

\epsfysize=1.6in
\centerline{\epsffile{Book10/fig081g.eps}}

\gr{>'Estw m'eshc >apotom`h deut'era <h AB ka`i t~h| AB prosarm'ozousa
<h BG; a<i >'ara AG, GB m'esai e>is`i dun'amei m'onon s'ummetroi
m'eson peri'eqousai t`o <up`o t~wn AG, GB; l'egw, <'oti t~h| AB
<et'era o>u prosarm'osei e>uje~ia m'esh dun'amei m'onon s'ummetroc
o>~usa t~h| <'olh|, met`a d`e t~hc <'olhc m'eson peri'eqousa.}

\gr{E>i g`ar dunat'on, prosarmoz'etw <h BD; ka`i a<i AD,
DB >'ara m'esai e>is`i dun'amei m'onon s'ummetroi m'eson peri'eqousai
t`o <up`o t~wn AD, DB. ka`i >ekke'isjw <rht`h <h EZ, ka`i to~ic m`en
>ap`o t~wn AG, GB >'ison par`a t`hn EZ parabebl'hsjw t`o EH pl'atoc poio~un
t`hn EM; t~w| d`e d`ic <up`o t~wn AG, GB >'ison >afh|r'hsjw t`o JH
pl'atoc poio~un t`hn JM; loip`on >'ara t`o EL >'ison >est`i t~w|
>ap`o t~hc AB; <'wste <h AB d'unatai t`o EL. p'alin d`h to~ic
>ap`o t~wn AD, DB >'ison par`a t`hn EZ parabebl'hsjw t`o EI pl'atoc
poio~un t`hn EN; >'esti d`e ka`i t`o EL >'ison t~w| >ap`o t~hc AB
tetrag'wnw|; loip`on >'ara t`o JI >'ison >est`i t~w| d`ic <up`o t~wn AD, DB.
ka`i >epe`i m'esai e>is`in a<i AG, GB, m'esa >'ara >est`i ka`i t`a
>ap`o t~wn AG, GB. ka'i >estin >'isa t~w| EH; m'eson >'ara ka`i t`o
EH. ka`i par`a <rht`hn t`hn EZ par'akeitai pl'atoc poio~un t`hn EM;
<rht`h >'ara >est`in <h EM ka`i >as'ummetroc t~h| EZ m'hkei. 
p'alin,
>epe`i m'eson >est`i t`o <up`o t~wn AG, GB, ka`i t`o d`ic <up`o t~wn
AG, GB m'eson >est'in. ka'i >estin >'ison t~w| JH; ka`i t`o JH >'ara
m'eson >est'in. ka`i par`a <rht`hn t`hn EZ par'akeitai pl'atoc poio~un
t`hn JM; <rht`h >'ara >est`i ka`i <h JM ka`i >as'ummetroc t~h| EZ
m'hkei. 
ka`i >epe`i a<i AG, GB dun'amei m'onon s'ummetro'i e>isin,
>as'ummetroc >'ara >est`in <h AG t~h| GB m'hkei. <wc d`e <h AG pr`oc
t`hn GB, o<'utwc >est`i t`o >ap`o t~hc AG pr`oc t`o <up`o t~wn AG, GB;
>as'ummetron >'ara >est`i t`o >ap`o t~hc AG  t~w| <up`o
t~wn AG, GB. >all`a t~w| m`en >ap`o t~hc AG
s'ummetr'a >esti t`a >ap`o
t~wn AG, GB, t~w| d`e <up`o t~wn AG, GB s'ummetr'on >esti t`o d`ic
<up`o t~wn AG, GB; >as'ummetra >'ara >est`i t`a >ap`o t~wn AG, GB
t~w| d`ic <up`o t~wn AG, GB. ka'i >esti to~ic m`en >ap`o t~wn
AG, GB >'ison t`o EH, t~w| d`e d`ic <up`o t~wn AG, GB >'ison t`o HJ;
 >as'ummetron >'ara >est`i t`o EH t~w| JH.
<wc d`e t`o EH
pr`oc t`o JH, o<'utwc >est`in <h EM pr`oc t`hn JM; 
>as'ummetroc
>'ara >est`in <h EM t~h| MJ m'hkei. ka'i e>isin >amf'oterai <rhta'i;
a<i EM, MJ >'ara <rhta'i e>isi dun'amei m'onon s'ummetroi;
>apotom`h >'ara >est`in <h EJ, prosarm'ozousa d`e a>ut~h| 
<h JM. <omo'iwc d`h de'ixomen, <'oti ka`i <h JN a>ut~h| 
prosarm'ozei;
t~h| >'ara >apotom~h| >'allh ka`i >'allh prosarm'ozei e>uje~ia dun'amei
m'onon s'ummetroc o>~usa t~h| <'olh|; <'oper >est`in >ad'unaton.}

\gr{T~h| >'ara m'eshc >apotom~h| deut'era| m'ia m'onon prosarm'ozei e>uje~ia
m'esh dun'amei m'onon s'ummetroc o>~usa t~h| <'olh|, met`a d`e t~hc
<'olhc m'eson peri'eqousa; <'oper >'edei de~ixai.}}

\ParallelRText{
\begin{center}
{\large Proposition 81}
\end{center}

Only one medial straight-line,
which is commensurable in square only with the whole, and
contains a medial (area) with the whole, can be attached
to a second apotome of a medial (straight-line).$^\dag$

\epsfysize=1.6in 
\centerline{\epsffile{Book10/fig081e.eps}}

Let $AB$ be a second apotome of a medial (straight-line), with $BC$ (so) attached
to $AB$. Thus, $AC$ and $CB$ are medial (straight-lines which
are) commensurable in square only, containing a medial (area)---(namely,
that contained) by $AC$ and $CB$ [Prop. 10.75].
I say that a(nother) medial straight-line, which is commensurable in square
only with the whole, and contains a medial (area) with the whole, cannot
be attached to $AB$.

For, if possible, let $BD$ be (so) attached. Thus, $AD$ and $DB$ are also medial (straight-lines which
are) commensurable in square only, containing a medial (area)---(namely,
that contained) by $AD$ and $DB$ [Prop. 10.75].
And let the rational (straight-line) $EF$ be laid down. And let $EG$,
equal to the (sum of the squares) on $AC$ and $CB$, have been applied
to $EF$, producing $EM$ as breadth. And let $HG$, equal to twice
the (rectangle contained) by $AC$ and $CB$, have been subtracted
(from $EG$), producing $HM$ as breadth. The remainder $EL$ is
thus equal to the (square) on $AB$ [Prop. 2.7]. 
Hence, $AB$ is the square-root of $EL$. So, again, let $EI$, equal to
the (sum of the squares) on $AD$ and $DB$ have been applied to $EF$,
producing $EN$ as breadth. And $EL$ is also equal to the square
on $AB$. Thus, the remainder 
$HI$ is equal to twice the (rectangle contained) by $AD$ and $DB$ [Prop. 2.7]. And since $AC$ and $CB$ are
(both) medial (straight-lines), the (sum of the squares) on $AC$ and $CB$ is also medial.
And it is equal to $EG$. Thus, $EG$ is also medial [Props.~10.15, 10.23~corr.]. And it is
applied to the rational (straight-line) $EF$, producing $EM$ as breadth.
Thus, $EM$ is rational, and incommensurable in length with $EF$
[Prop. 10.22]. Again, since the (rectangle
contained) by $AC$ and $CB$ is medial, twice the (rectangle
contained) by $AC$ and $CB$ is also medial [Prop. 10.23~corr.]. And it is equal to $HG$.
Thus, $HG$ is also medial. And it is applied to the rational (straight-line)
$EF$, producing $HM$ as breadth. Thus, $HM$ is also rational, and
incommensurable in length with $EF$ [Prop. 10.22]. And since $AC$ and $CB$ are commensurable in square only, $AC$ is thus incommensurable in length with $CB$.  And as $AC$ (is)
to $CB$, so the (square) on $AC$
is to the (rectangle contained) by $AC$ and $CB$ [Prop. 10.21~corr.]. Thus, the (square) on $AC$
is incommensurable with the (rectangle contained) by $AC$ and $CB$
[Prop. 10.11].
But, the (sum of the squares) on $AC$ and $CB$ is commensurable
with the (square) on $AC$, and twice the (rectangle contained) by $AC$ and
$CB$ is commensurable with the (rectangle contained) by $AC$ and $CB$ [Prop. 10.6].
Thus, the (sum of the squares) on $AC$ and $CB$ is incommensurable
with twice the (rectangle contained) by $AC$ and $CB$ [Prop. 10.13]. And $EG$ is equal to
the (sum of the squares) on $AC$ and $CB$. And $GH$ is equal
to twice the (rectangle contained) by $AC$ and $CB$. Thus, 
$EG$ is incommensurable with $HG$. And as $EG$ (is) to $HG$,
so $EM$ is to $HM$ [Prop. 6.1]. 
Thus, $EM$ is incommensurable in length with $MH$ [Prop. 10.11]. And they are both rational (straight-lines).
Thus, $EM$ and $MH$ are rational (straight-lines which are)
commensurable in square only. Thus, $EH$ is an apotome
[Prop. 10.73], and $HM$ (is) attached to
it. So, similarly, we can show that $HN$ (is) also (commensurable
in square only with $EN$ and is) attached to ($EH$). 
Thus, different straight-lines, which are commensurable
in square only with the whole, are attached to an apotome.
The very thing is impossible [Prop. 10.79].

Thus, only one medial straight-line,
which is commensurable in square only with the whole, and
contains a medial (area) with the whole, can be attached
to a second apotome of a medial (straight-line). (Which is) the very thing it was required to show.}
\end{Parallel}


\vspace{7pt}{\footnotesize\noindent$^\dag$ This proposition is equivalent to 
Prop.~10.44, with minus signs instead of
plus signs.}

%%%%
%10.82
%%%%
\pdfbookmark[1]{Proposition 10.82}{pdf10.82}
\begin{Parallel}{}{}
\ParallelLText{
\begin{center}
{\large \ggn{82}.}
\end{center}\vspace*{-7pt}

\gr{T~h| >el'assoni m'ia m'onon prosarm'ozei e>uje~ia dun'amei
>as'ummetroc o>~usa t~h| <'olh| poio~usa met`a t~hc <'olhc
t`o m`en >ek t~wn >ap> a>ut~wn tetrag'wnwn <rht'on,
t`o d`e d`ic <up> a>ut~wn m'eson.}\\

\epsfysize=0.25in
\centerline{\epsffile{Book10/fig079g.eps}}

\gr{>'Estw <h >el'asswn <h AB, ka`i t~h| AB prosarm'ozousa >'estw
<h BG; a<i >'ara AG, GB dun'amei e>is`in >as'ummetroi
poio~usai t`o m`en sugke'imenon >ek t~wn >ap> a>ut~wn tetrag'wnwn
<rht'on, t`o d`e d`ic <up> a>ut~wn m'eson; l'egw, <'oti
t~h| AB <et'era e>uje~ia o>u prosarm'osei t`a a>ut`a poio~usa.}

\gr{E>i g`ar dunat'on, prosarmoz'etw <h BD; ka`i a<i AD,
DB >'ara dun'amei e>is`in >as'ummetroi poio~usai t`a proeirhm'ena.
ka`i >epe'i, <~w| <uper'eqei t`a >ap`o t~wn AD, DB
t~wn >ap`o t~wn AG, GB, to'utw| <uper'eqei ka`i t`o d`ic
<up`o t~wn AD, DB to~u d`ic <up`o t~wn AG, GB, t`a d`e
>ap`o t~wn AD, DB tetr'agwna t~wn >ap`o t~wn AG, GB
 tetrag'wnwn <uper'eqei
<rht~w|; <rht`a g'ar >estin >amf'otera; ka`i t`o d`ic <up`o t~wn AD, DB
>'ara to~u d`ic <up`o t~wn AG, GB <uper'eqei <rht~w|; <'oper
>est`in >ad'unaton; m'esa g'ar >estin >amf'otera.}

\gr{T~h| >'ara >el'assoni m'ia m'onon prosarm'ozei e>uje~ia dun'amei
>as'ummetroc o>~usa t~h| <'olh| ka`i poio~usa t`a m`en >ap> 
a>ut~wn tetr'agwna <'ama <rht'on, t`o d`e d`ic <up> a>ut~wn
m'eson; <'oper >'edei de~ixai.}}

\ParallelRText{
\begin{center}
{\large Proposition 82}
\end{center}

Only one straight-line, which is incommensurable
in square with the whole, and (together) with the whole makes the (sum of the) squares on them
rational, and twice the (rectangle contained) by them medial, can be
attached to a minor (straight-line).

\epsfysize=0.25in 
\centerline{\epsffile{Book10/fig079e.eps}}

Let $AB$ be a minor (straight-line), and let $BC$ be (so) attached to $AB$.
Thus, $AC$ and $CB$ are (straight-lines which are) incommensurable  in square, making the sum of
the squares on them rational, and twice the (rectangle contained) by them
medial [Prop. 10.76]. I say that another
another straight-line fulfilling the same (conditions) cannot be
attached to $AB$.

For, if possible, let $BD$ be (so) attached (to $AB$). Thus, $AD$
and $DB$ are also (straight-lines which are) incommensurable in square, fulfilling the (other) aforementioned
(conditions) [Prop. 10.76]. And since
by whatever (area) the (sum of the squares) on $AD$ and $DB$
exceeds the (sum of the squares) on $AC$ and $CB$, 
twice the (rectangle contained) by $AD$ and $DB$ also exceeds
twice the (rectangle contained) by $AC$ and $CB$  by this (same area) [Prop. 2.7]. And the (sum of the) squares on
$AD$ and $DB$ exceeds the (sum of the) squares on $AC$ and $CB$
by a rational (area). For both are rational (areas). Thus, twice the
(rectangle contained) by $AD$ and $DB$ also exceeds
twice the (rectangle contained) by $AC$ and $CB$ by a rational (area).
The very thing is impossible. For both are medial (areas) [Prop. 10.26].

Thus,  only one straight-line, which is  incommensurable in square
with the whole, and (with the whole) makes the  squares on them
(added) together rational, and twice the (rectangle contained) by them medial, can be
attached to a minor (straight-line). (Which is) the very thing it was required to
show.}
\end{Parallel}


\vspace{7pt}{\footnotesize\noindent$^\dag$ This proposition is equivalent to 
Prop.~10.45, with minus signs instead of plus signs.}

%%%%
%10.83
%%%%
\pdfbookmark[1]{Proposition 10.83}{pdf10.83}
\begin{Parallel}{}{}
\ParallelLText{
\begin{center}
{\large \ggn{83}.}
\end{center}\vspace*{-7pt}

\gr{T~h| met`a <rhto~u m'eson t`o <'olon poio'ush| m'ia m'onon prosarm'ozei
e>uje~ia dun'amei >as'ummetroc o>~usa t~h| <'olh|, met`a d`e t~hc
<'olhc poio~usa t`o m`en sugke'imenon >ek t~wn >ap>
a>ut~wn tetrag'wnwn m'eson, t`o d`e d`ic <up> a>ut~wn <rht'on.}\\~\\

\epsfysize=0.25in
\centerline{\epsffile{Book10/fig079g.eps}}

\gr{>'Estw <h met`a <rhto~u m'eson t`o <'olon poio~usa <h AB,
ka`i t~h| AB prosarmoz'etw <h BG; a<i >'ara AG, GB dun'amei
e>is`in >as'ummetroi poio~usai t`a proke'imena; l'egw, <'oti
t~h| AB <et'era o>u prosarm'osei t`a a>ut`a poio~usa.}

\gr{E>i g`ar dunat'on, prosarmoz'etw <h BD; ka`i a<i AD, DB >'ara
e>uje~iai dun'amei e>is`in >as'ummetroi poio~usai t`a
proke'imena. >epe`i o>un, <~w| <uper'eqei t`a >ap`o t~wn
AD, DB t~wn >ap`o t~wn AG, GB, to'utw| <uper'eqei ka`i t`o
d`ic <up`o t~wn AD, DB to~u d`ic <up`o t~wn AG, GB
>akolo'ujwc to~ic pr`o a>uto~u, t`o d`e d`ic <up`o t~wn AD, DB
to~u d`ic <up`o t~wn AG, GB <uper'eqei <rht~w|; <rht`a g'ar
>estin >amf'otera; ka`i t`a >ap`o t~wn AD, DB >'ara t~wn
>ap`o t~wn AG, GB <uper'eqei <rht~w|; <'oper >est`in
>ad'unaton; m'esa g'ar >estin >amf'otera.}

 \gr{O>uk >'ara t~h| AB
<et'era prosarm'osei e>uje~ia dun'amei >as'ummetroc o>~usa t~h|
<'olh|, met`a d`e t~hc <'olhc poio~usa t`a proeirhm'ena;
m'ia >'ara m'onon prosarm'osei; <'oper >'edei de~ixai.}}

\ParallelRText{
\begin{center}
{\large Proposition 83}
\end{center}

Only  one straight-line, which is incommensurable
in square with the whole, and (together) with the whole makes the sum of the squares on them
medial, and twice the (rectangle contained) by them rational, can be
attached to that (straight-line) which with a rational (area) makes a
medial whole.$^\dag$

\epsfysize=0.25in 
\centerline{\epsffile{Book10/fig079e.eps}}

Let $AB$ be a (straight-line) which with a rational (area) makes a medial
whole,  and let $BC$ be (so) attached to $AB$. Thus, $AC$ and $CB$
are (straight-lines which are) incommensurable in square, fulfilling the (other) proscribed (conditions) [Prop. 10.77]. I say that another (straight-line)
fulfilling the same (conditions) cannot be attached to $AB$.

For, if possible, let $BD$ be (so) attached (to $AB$). Thus, $AD$ and
$DB$ are also straight-lines (which are) incommensurable in square,
fulfilling the (other) prescribed  (conditions) [Prop. 10.77]. Therefore, analogously to the
(propositions) before this, since by whatever (area)
the (sum of the squares) on $AD$ and $DB$ exceeds the (sum of the
squares) on $AC$ and $CB$,  twice the (rectangle contained)
by $AD$ and $DB$ also exceeds twice the (rectangle contained) by
$AC$ and $CB$ by this (same area). And twice the (rectangle contained) by $AD$ and
$DB$ exceeds twice the (rectangle contained) by $AC$ and
$CB$ by a rational (area). For they are (both) rational (areas).
Thus, the (sum of the squares) on 
$AD$ and $DB$ also exceeds the (sum of the squares) on $AC$ and $CB$
by a rational (area). The very thing is impossible. For both are
medial (areas) [Prop. 10.26].

 Thus, another
straight-line cannot be attached to $AB$, which is incommensurable
in square with the whole, and fulfills the (other) aforementioned (conditions) with the
whole. Thus, only one (such straight-line) can be (so) attached. (Which is)
the very thing it was required to show.}
\end{Parallel}


\vspace{7pt}{\footnotesize\noindent$^\dag$
This proposition is equivalent to Prop.~10.46, with minus signs instead of
plus signs.}

%%%%
%10.84
%%%%
\pdfbookmark[1]{Proposition 10.84}{pdf10.84}
\begin{Parallel}{}{}
\ParallelLText{
\begin{center}
{\large \ggn{84}.}
\end{center}\vspace*{-7pt}

\gr{T~h| met`a m'esou m'eson t`o <'olon poio'ush| m'ia m'onh prosarm'ozei e>uje~ia dun'amei >as'ummetroc o>~usa t~h| <'olh|, met`a d`e t~hc <'olhc poio~usa t'o te sugke'imenon >ek t~wn >ap> a>ut~wn tetrag'wnwn
m'eson t'o te d`ic <up> a>ut~wn m'eson ka`i >'eti >as'ummetron
t~w| sugkeim'enw| >ek t~wn >ap> a>ut~wn.}

\gr{>'Estw <h met`a m'esou m'eson t`o <'olon poio~usa <h AB,
prosarm'ozousa d`e a>ut~h| <h BG; a<i >'ara AG, GB dun'amei e>is`in
>as'ummetroi poio~usai t`a proeirhm'ena. l'egw, <'oti t~h| AB <et'era
o>u prosarm'osei poio~usa proeirhm'ena.}\\~\\~\\~\\~\\

\epsfysize=1.6in
\centerline{\epsffile{Book10/fig081g.eps}}

\gr{E>i g`ar dunat'on, prosarmoz'etw <h BD, <'wste ka`i t`ac AD, DB dun'amei
>asumm'etrouc e>~inai poio'usac t'a te >ap`o t~wn AD, DB tetr'agwna
<'ama m'eson ka`i t`o d`ic <up`o t~wn AD, DB m'eson ka`i >'eti t`a
>ap`o t~wn AD, DB >as'ummetra t~w| d`ic <up`o t~wn AD, DB; ka`i
>ekke'isjw <rht`h <h EZ, ka`i to~ic m`en >ap`o t~wn AG, GB >'ison
par`a t`hn EZ parabebl'hsjw t`o EH pl'atoc poio~un t`hn EM, t~w|
d`e d`ic <up`o t~wn AG, GB >'ison par`a t`hn EZ parabebl'hsjw t`o JH
pl'atoc poio~un t`hn JM; loip`on
>'ara t`o >ap`o t~hc AB >'ison >est`i t~w| EL; <h >'ara AB d'unatai
t`o EL. p'alin to~ic >ap`o t~wn AD, DB >'ison
par`a t`hn EZ parabebl'hsjw t`o EI pl'atoc poio~un t`hn EN.
 >'esti
d`e ka`i t`o >ap`o t~hc AB >'ison t~w| EL;
loip`on >'ara t`o d`ic <up`o t~wn AD, DB >'ison [>est`i]
t~w| JI. ka`i >epe`i m'eson >est`i t`o sugke'imenon >ek t~wn
>ap`o t~wn AG, GB ka'i >estin >'ison t~w| EH, m'eson >'ara >est`i
ka`i t`o EH. ka`i par`a <rht`hn t`hn EZ par'akeitai pl'atoc
poio~un t`hn EM;
 <rht`h >'ara >est`in <h EM ka`i >as'ummetroc t~h|
EZ m'hkei. p'alin, >epe`i m'eson >est`i t`o d`ic <up`o t~wn AG, GB ka'i
>estin >'ison t~w| JH, m'eson >'ara ka`i t`o JH. ka`i par`a <rht`hn t`hn
EZ par'akeitai pl'atoc poio~un t`hn JM; 
<rht`h >'ara >est`in <h JM
ka`i >as'ummetroc t~h| EZ m'hkei. ka`i >epe`i >as'ummetr'a >esti
t`a >ap`o t~wn AG, GB t~w| d`ic <up`o t~wn AG, GB, >as'ummetr'on
>esti ka`i t`o EH t~w| JH; >as'ummetroc >'ara >est`i ka`i <h EM t~h|
MJ m'hkei. ka'i e>isin >amf'oterai <rhta'i; a<i >'ara EM, MJ <rhta'i
e>isi dun'amei m'onon s'ummetroi; >apotom`h >'ara >est`in <h EJ, prosarm'ozousa d`e a>ut~h| <h JM. <omo'iwc d`h de'ixomen, <'oti <h EJ p'alin >apotom'h
>estin, prosarm'ozousa d`e a>ut~h| <h JN. t~h| >'ara >apotom~h|
>'allh ka`i >'allh prosarm'ozei <rht`h dun'amei m'onon s'ummetroc
o>~usa t~h| <'olh|; <'oper >ede'iqjh >ad'unaton. o>uk >'ara t~h| AB <et'era
prosarm'osei e>uje~ia.}

\gr{T~h| >'ara AB m'ia m'onon prosarm'ozei e>uje~ia dun'amei
>as'ummetroc o>~usa t~h| <'olh|, met`a d`e t~hc <'olhc poio~usa
t'a te >ap> a>ut~wn tetr'agwna <'ama m'eson ka`i t`o d`ic <up>
a>ut~wn m'eson ka`i >'eti t`a >ap> a>ut~wn tetr'agwna >as'ummetra
t~w| d`ic <up> a>ut~wn; <'oper >'edei de~ixai.}}

\ParallelRText{
\begin{center}
{\large Proposition 84}
\end{center}

Only one straight-line, which is incommensurable
in square with the whole, and (together) with the whole makes the sum of the
squares on them medial, and twice the (rectangle contained) by
them medial, and, moreover, incommensurable with the sum of
the (squares) on them, can be attached to that (straight-line) which with
a medial (area) makes a medial whole.$^\dag$

Let $AB$ be a (straight-line) which with a medial (area) makes a medial
whole, $BC$ being (so) attached to it. Thus, $AC$ and $CB$ are incommensurable in square, fulfilling the (other) aforementioned (conditions)
[Prop. 10.78]. I say that a(nother) (straight-line)
fulfilling the aforementioned (conditions) cannot be attached to $AB$.

\epsfysize=1.6in 
\centerline{\epsffile{Book10/fig081e.eps}}

For, if possible, let $BD$ be (so) attached. Hence, $AD$ and
$DB$ are also (straight-lines which are) incommensurable in square, making the
squares on $AD$ and $DB$ (added) together medial, and twice the (rectangle contained) by $AD$ and $DB$ medial, and, moreover, the (sum of the squares)
on $AD$ and $DB$ incommensurable with twice the (rectangle contained)
by $AD$ and $DB$ [Prop. 10.78].
And let the rational (straight-line) $EF$ be laid down. And let $EG$, equal
to the (sum of the squares) on $AC$ and $CB$, have been applied to
$EF$, producing $EM$ as breadth. And let $HG$, equal to twice the (rectangle contained) by $AC$ and $CB$, have been applied to $EF$, producing $HM$ as breadth. Thus, the remaining (square) on  $AB$ is equal
to $EL$ [Prop. 2.7].  Thus, $AB$ is the square-root
of $EL$. Again, let $EI$, equal to the (sum of the squares) on $AD$ and $DB$, have been applied to $EF$, producing $EN$ as breadth. And the (square) on $AB$
 is also equal to $EL$. Thus, the remaining twice the
(rectangle contained) by $AD$ and $DB$ [is] equal to $HI$ [Prop. 2.7].  And since the sum of the (squares) on $AC$ and $CB$ is medial, and is equal to $EG$, $EG$ is thus also medial.
And it is applied to the rational (straight-line) $EF$, producing $EM$ as breadth. $EM$ is thus rational, and incommensurable in length with $EF$
[Prop. 10.22]. Again, since twice the
(rectangle contained) by $AC$ and $CB$ is medial, and is equal to $HG$,
$HG$ is thus also medial. And it is applied to the rational (straight-line) $EF$,
producing $HM$ as breadth. $HM$ is thus rational, and incommensurable in
length with $EF$ [Prop. 10.22]. 
And since the (sum of the squares) on
 $AC$ and $CB$ is incommensurable
with twice the (rectangle contained) by $AC$ and $CB$, $EG$ is also incommensurable with $HG$. Thus, $EM$ is also incommensurable in length
with $MH$ [Props.~6.1, 10.11]. And they are both  rational (straight-lines). Thus, $EM$ and $MH$ are rational (straight-lines which are) commensurable in square
only. Thus, $EH$ is an apotome [Prop. 10.73], with $HM$ attached to it. So, similarly, we can show that $EH$ is again an apotome, with $HN$ attached to it. Thus, different rational (straight-lines),
which are commensurable in square only with the whole, are attached to
an apotome. The very thing was shown (to be) impossible [Prop. 10.79]. Thus, another straight-line cannot
be (so) attached to $AB$.

Thus, only one straight-line, which is incommensurable
in square with the whole, and (together) with the whole makes the 
squares on them (added) together medial, and twice the (rectangle contained) by
them medial, and, moreover, the (sum of the)
squares on them incommensurable with the (rectangle contained)
by them, can be attached to $AB$. (Which is) the very thing it
was required to show.}
\end{Parallel}


\vspace{7pt}{\footnotesize\noindent$^\dag$ This proposition is equivalent to 
Prop.~10.47, with minus signs instead of
plus signs.}

%%%%%%%%%
% Definitions III
%%%%%%%%%
\pdfbookmark[1]{Definitions III}{def10.3}
\begin{Parallel}{}{}
\ParallelLText{
\begin{center}
\large{\gr{<'Oroi tr'itoi}.}
\end{center}\vspace*{-7pt}

\ggn{11}.~\gr{<Upokeim'enhc <rht~hc ka`i >apotom~hc, >e`an m`en <h <'olh
t~hc prosarmozo'ushc me~izon d'unhtai t~w| >ap`o summ'etrou <eaut~h|
m'hkei, ka`i <h <'olh s'ummetroc >~h| t~h| >ekkeim'enh| <rht~h| m'hkei,
kale'isjw >apotom`h pr'wth.}

\ggn{12}.~\gr{>E`an d`e <h prosarm'ozousa s'ummetroc >~h| t~h| >ekkeim'enh|
<rht~h| m'hkei, ka`i <h <'olh t~hc prosarmozo'ushc me~izon d'unhtai
t~w| >ap`o summ'etrou <eaut~h|, kale'isjw >apotom`h deut'era.}

\ggn{13}.~\gr{>E`an d`e mhdet'era s'ummetroc >~h| t~h| >ekkeim'enh|
<rht~h| m'hkei, <h d`e <'olh t~hc prosarmozo'ushc me~izon d'unhtai
t~w| >ap`o summ'etrou <eaut~h|, kale'isjw >apotom`h tr'ith.}

\ggn{14}.~\gr{P'alin, >e`an <h <'olh t~hc prosarmozo'ushc me~izon
d'unhtai t~w| >ap`o >asumm'etrou <eaut~h| [m'hkei], >e`an
m`en <h <'olh s'ummetroc >~h| t~h| >ekkeim'enh| <rht~h|
m'hkei, kale'isjw >apotom`h tet'arth.}

\ggn{15}.~\gr{>E`an d`e <h prosarm'ozousa, p'empth.}

\ggn{16}.~\gr{>E`an d`e mhdet'era, <'ekth.}}

\ParallelRText{
\begin{center}
{\large Definitions III}
\end{center}

11.~Given a rational (straight-line) and an apotome, if the square
on the whole is greater than the (square on a straight-line) attached (to the apotome) by the (square)
on (some straight-line) commensurable in length  with (the whole), and the
whole is commensurable in length with the 
(previously) laid down rational (straight-line), then let the (apotome) be
called a first apotome.

12.~And if the attached (straight-line)
is commensurable in length with the (previously) laid down rational
(straight-line), and the square on the whole is greater than (the
square on) the attached (straight-line) by the (square) on (some
straight-line) commensurable (in length) with (the whole), then let
the (apotome) be called a second apotome.

13.~And if neither of (the whole
or the attached straight-line) is commensurable
in length with the (previously) laid down rational (straight-line),
and the square on the whole is greater than (the
square on) the attached (straight-line) by the (square) on (some
straight-line) commensurable (in length) with (the whole), then let
the (apotome) be called a third apotome.

14.~Again, if the square
on the whole is greater than (the square on) the attached (straight-line) by the (square)
on (some straight-line) incommensurable [in length]  with (the whole), and the
whole is commensurable in length with the
(previously) laid down rational (straight-line), then let the (apotome) be
called a fourth apotome.

15.~And if the
attached (straight-line is commensurable),   a fifth (apotome).

16.~And if neither (the whole
nor the attached straight-line is commensurable), a sixth (apotome).}
\end{Parallel}

%%%%
%10.85
%%%%
\pdfbookmark[1]{Proposition 10.85}{pdf10.85}
\begin{Parallel}{}{}
\ParallelLText{
\begin{center}
{\large \ggn{85}.}
\end{center}\vspace*{-7pt}

\gr{E<ure~in t`hn pr'wthn >apotom'hn.}

\epsfysize=0.65in
\centerline{\epsffile{Book10/fig085g.eps}}

\gr{>Ekke'isjw <rht`h <h A, ka`i t~h| A m'hkei s'ummetroc >'estw <h BH;
<rht`h >'ara >est`i ka`i <h BH. ka`i >ekke'isjwsan d'uo tetr'agwnoi
>arijmo`i o<i DE, EZ, <~wn <h <uperoq`h <o ZD m`h >'estw tetr'agwnoc;
o>ud> >'ara <o ED pr`oc t`on DZ l'ogon >'eqei, <`on tetr'agwnoc
>arijm`oc pr`oc tetr'agwnon >arijm'on. ka`i pepoi'hsjw <wc <o
ED pr`oc t`on DZ, o<'utwc t`o >ap`o t~hc BH tetr'agwnon pr`oc t`o >ap`o
t`hc HG tetr'agwnon; s'ummetron >'ara >est`i t`o >ap`o t~hc BH t~w|
>ap`o t~hc HG. <rht`on d`e t`o >ap`o t~hc BH; <rht`on >'ara ka`i t`o
>ap`o t~hc HG; <rht`h >'ara >est`i ka`i <h HG. ka`i >epe`i <o ED pr`oc 
t`on DZ l'ogon o>uk >'eqei, <`on tetr'agwnoc >arijm`oc pr`oc tetr'agwnon
>arijm'on, o>ud> >'ara t`o >ap`o t~hc BH pr`oc t`o >ap`o t~hc HG
l'ogon >'eqei, <`on tetr'agwnoc >arijm`oc pr`oc tetr'agwnon
>arijm'on; >as'ummetroc >'ara >est`in <h BH t~h| HG m'hkei.
ka'i e>isin >amf'oterai <rhta'i; a<i BH, HG >'ara <rhta'i e>isi
dun'amei m'onon s'ummetroi; <h >'ara BG >apotom'h >estin. l'egw
d'h, <'oti ka`i pr'wth.}

\gr{<~Wi g`ar me~iz'on >esti t`o >ap`o t~hc BH to~u >ap`o t~hc HG, >'estw
t`o >ap`o t~hc J. ka`i >epe'i >estin <wc <o ED pr`oc t`on ZD,
o<'utwc t`o >ap`o t~hc BH pr`oc t`o >ap`o t~hc HG, ka`i >anastr'eyanti
>'ara >est`in <wc <o DE pr`oc t`on EZ, o<'utwc t`o >ap`o t~hc HB
pr`oc t`o >ap`o t~hc J. <o d`e DE pr`oc t`on EZ l'ogon >'eqei, <`on
tetr'agwnoc >arijm`oc pr`oc tetr'agwnon >arijm'on; <ek'ateroc
g`ar tetr'agwn'oc >estin; ka`i t`o >ap`o t~hc HB >'ara pr`oc t`o
>ap`o t~hc J l'ogon >'eqei, <`on tetr'agwnoc >arijm`oc pr`oc
tetr'agwnon >arijm'on; s'ummetroc >'ara >est`in <h BH t~h| J m'hkei.
ka`i d'unatai <h BH t~hc HG me~izon t~w| >ap`o t~hc J; <h BH >'ara
t~hc HG me~izon d'unatai t~w| >ap`o summ'etrou <eaut~h| m'hkei.
ka'i >estin <h <'olh <h BH s'ummetroc t~h| >ekkeim'enh| <rht~h| m'hkei
t~h| A. <h BG >'ara >apotom'h >esti pr'wth.}

\gr{E<'urhtai >'ara <h pr'wth >apotom`h <h BG; <'oper >'edei e<ure~in.}}

\ParallelRText{
\begin{center}
{\large Proposition 85}
\end{center}

To find a first apotome.

\epsfysize=0.65in
\centerline{\epsffile{Book10/fig085e.eps}}

Let the rational (straight-line) $A$ be laid down. And let $BG$
be commensurable in length with $A$. $BG$ is thus also a rational (straight-line).
And let two square numbers
$DE$ and $EF$ be laid down, and let their difference $FD$ be  not 
square [Prop. 10.28~lem.~I]. Thus,
$ED$ does not have to $DF$ the ratio which (some) square number
(has) to (some) square number. And let it have been contrived that
as $ED$ (is) to $DF$, so the square on $BG$ (is) to the square on $GC$
[Prop. 10.6.~corr.]. Thus, the (square) on $BG$
is commensurable with the (square) on $GC$ [Prop. 10.6]. And the (square) on $BG$ (is)
rational. Thus, the (square) on $GC$ (is) also rational. Thus, $GC$ is also
rational. And since $ED$ does not have to $DF$ the ratio which
(some) square number (has) to (some) square number, 
the (square) on $BG$ thus does not have to the (square) on $GC$ the ratio
which (some) square number (has) to (some) square number either. Thus,
$BG$ is incommensurable in length with $GC$ [Prop. 10.9]. And they are both
rational (straight-lines). Thus, $BG$ and $GC$ are rational (straight-lines
which are) commensurable in square only. Thus, $BC$ is an apotome
[Prop. 10.73]. So, I say that (it is)
also a first (apotome).

Let the (square) on $H$ be that (area) by which the (square) on $BG$
is greater than the (square) on $GC$ [Prop. 10.13~lem.]. And since as $ED$ is to
$FD$, so the (square) on $BG$ (is) to the (square) on $GC$, thus, via
conversion, as $DE$ is to $EF$, so the (square) on $GB$ (is) to the
(square) on $H$ [Prop. 5.19~corr.]. 
And $DE$ has to $EF$ the ratio which (some) square-number
(has) to (some) square-number. For each is a square (number). 
Thus, the (square) on $GB$ also has to the (square) on $H$ the ratio which
(some) square number (has) to (some) square number. Thus, $BG$
is commensurable in length with $H$ [Prop. 10.9].
And the square on $BG$ is greater than (the square on) $GC$ by the
(square) on $H$. Thus, the square on $BG$ is greater than (the square on) $GC$ by the
(square) on (some straight-line) commensurable in length with ($BG$).
And the whole, $BG$, is commensurable in length with the (previously)
laid down rational (straight-line) $A$. Thus, $BC$ is a first
apotome [Def. 10.11].

Thus, the first apotome $BC$ has been found. (Which is) the
very thing it was required to find.}
\end{Parallel}


\vspace{7pt}{\footnotesize\noindent$^\dag$ See footnote to Prop.~10.48.}

%%%%
%10.86
%%%%
\pdfbookmark[1]{Proposition 10.86}{pdf10.86}
\begin{Parallel}{}{}
\ParallelLText{
\begin{center}
{\large \ggn{86}.}
\end{center}

\gr{E<ure~in t`hn deut'eran >apotom'hn.}

\gr{>Ekke'isjw <rht`h <h A ka`i t~h| A s'ummetroc m'hkei <h HG. <rht`h
>'ara >est`in <h HG. ka`i >ekke'isjwsan d'uo tetr'agwnoi >arijmo`i
o<i DE, EZ, <~wn <h <uperoq`h <o DZ m`h >'estw tetr'agwnoc. ka`i
pepoi'hsjw <wc <o ZD pr`oc t`on DE, o<'utwc t`o >ap`o t~hc GH tetr'agwnon
pr`oc t`o >ap`o t~hc HB tetr'agwnon. s'ummetron >'ara >est`i t`o >ap`o
t~hc GH tetr'agwnon t~w| >ap`o t~hc HB tetrag'wnw|. <rht`on d`e t`o
>ap`o t~hc GH. <rht`on >'ara [>est`i] ka`i t`o >ap`o t~hc HB;
<rht`h >'ara >est`in <h BH. ka`i >epe`i t`o >ap`o t~hc HG tetr'agwnon
pr`oc t`o >ap`o t~hc HB l'ogon o>uk >'eqei, <`on tetr'agwnoc >arijm`oc pr`oc
tetr'agwnon >arijm'on, >as'ummetr'oc >estin <h GH t~h| HB
m'hkei. ka'i e>isin >amf'oterai <rhta'i; a<i GH, HB >'ara  rhta'i e>isi
dun'amei m'onon s'ummetroi; <h BG >'ara >apotom'h >estin.
l'egw d'h, <'oti ka`i deut'era.}\\~\\~\\~\\~\\

\epsfysize=0.65in
\centerline{\epsffile{Book10/fig086g.eps}}

\gr{<~Wi g`ar me~iz'on >esti t`o >ap`o t~hc BH to~u >ap`o t~hc
HG, >'estw t`o >ap`o t~hc J. >epe`i o>~un >estin <wc t`o >ap`o
t~hc BH pr`oc t`o >ap`o t~hc HG, o<'utwc <o ED >arijm`oc
pr`oc t`on DZ >arijm'on, >anastr'eyanti >'ara >est`in <wc t`o >ap`o
t~hc BH pr`oc t`o >ap`o t~hc J, o<'utwc <o DE pr`oc t`on EZ.  ka'i
>estin <ek'ateroc t~wn DE, EZ tetr'agwnoc; t`o >'ara >ap`o t~hc BH
pr`oc t`o >ap`o t~hc J l'ogon >'eqei, <`on tetr'agwnoc >arijm`oc
pr`oc tetr'agwnon >arijm'on; s'ummetroc >'ara >est`in <h BH t~h| J
m'hkei. ka`i d'unatai <h BH t~hc HG me~izon t~w| >ap`o t~hc J;
<h BH >'ara t~hc HG me~izon d'unatai t~w| >ap`o summ'etrou
<eaut~h| m'hkei. ka'i >estin <h prosarm'ozousa <h GH t~h|
>ekkeim'enh| <rht~h| s'ummetroc t~h| A. <h BG >'ara >apotom'h
>esti deut'eta.}

\gr{E<'urhtai >'ara deut'era >apotom`h <h BG;
<'oper >'edei de~ixai.}}

\ParallelRText{
\begin{center}
{\large Proposition 86}
\end{center}\vspace*{-7pt}

To find a second apotome.

Let the rational (straight-line) $A$, and $GC$ (which is)
commensurable in length with $A$, be laid down. Thus, $GC$ is a rational (straight-line).
And let the two square numbers $DE$ and $EF$ be laid down, and let
their
difference $DF$ be not square [Prop. 10.28~lem.~I]. 
And let it have been contrived that as $FD$ (is) to $DE$, so the square on $CG$ (is) to the square on $GB$ [Prop. 10.6~corr.].
Thus, the square on $CG$ is commensurable with the square on $GB$ [Prop. 10.6]. And the (square) on $CG$ (is) rational.
Thus, the (square) on $GB$ [is] also rational. Thus, $BG$ is a rational
(straight-line).  And since the square on $GC$ does not have to the
(square) on $GB$ the ratio which (some) square number (has) to
(some) square number, $CG$ is incommensurable in length with $GB$
[Prop. 10.9]. And they are both rational (straight-lines). Thus, $CG$ and $GB$ are rational (straight-lines which are)
commensurable in square only. Thus, $BC$ is an apotome [Prop. 10.73]. So, I say that it is also a second
(apotome).

\epsfysize=0.65in
\centerline{\epsffile{Book10/fig086e.eps}}

For let the (square) on $H$ be that (area) by which the (square) on $BG$
is greater than the (square) on $GC$ [Prop. 10.13~lem.]. Therefore, since
as the (square) on $BG$ is to the (square) on $GC$, so the number $ED$
(is) to the number $DF$, thus, also, via conversion, as the (square) on $BG$
is to the (square) on $H$, so $DE$ (is) to $EF$ [Prop. 5.19~corr.]. And $DE$ and $EF$ are each
square (numbers). Thus, the (square) on $BG$ has to the (square)
on $H$ the ratio which (some) square number (has) to (some) square number.
Thus, $BG$ is commensurable in length with $H$ [Prop. 10.9]. 
And the square on $BG$ is greater than (the square on) $GC$ by the (square)
on $H$.
Thus, the square on $BG$ is greater
than (the square on) $GC$ by the (square) on (some straight-line)
commensurable in length with ($BG$). And the attachment $CG$
is commensurable (in length) with the (prevously) laid down rational (straight-line) $A$.
Thus, $BC$ is a second apotome [Def. 10.12].$^\dag$

Thus, the second apotome $BC$ has been found. (Which is) the very thing
it was required to show.}
\end{Parallel}


\vspace{7pt}{\footnotesize\noindent$^\dag$ See footnote to Prop.~10.49.}

%%%%
%10.87
%%%%
\pdfbookmark[1]{Proposition 10.87}{pdf10.87}
\begin{Parallel}{}{}
\ParallelLText{
\begin{center}
{\large \ggn{87}.}
\end{center}\vspace*{-7pt}

\gr{E<ure~in t`hn tr'ithn >apotom'hn.}

\epsfysize=1.in
\centerline{\epsffile{Book10/fig087g.eps}}

\gr{>Ekke'isjw <rht`h <h A, ka`i >ekke'isjwsan tre~ic >arijmo`i
o<i E, BG, GD l'ogon m`h >'eqontec pr`oc >all'hlouc, <`on
tetr'agwnoc >arijm`oc pr`oc tetr'agwnon >arijm'on, <o d`e
GB pr`oc t`on BD l'ogon >eq'etw, <`on tetr'agwnoc >arijm`oc pr`oc
tetr'agwnon >arijm'on, ka`i pepoi'hsjw <wc m`en <o E pr`oc t`on
BG, o<'utwc t`o >ap`o t~hc A tetr'agwnon pr`oc t`o >ap`o t~hc
ZH tetr'agwnon, <wc d`e <o BG pr`oc t`on GD, o<'utwc t`o >ap`o
t~hc ZH tetr'agwnon pr`oc t`o >ap`o t`hc HJ. >epe`i o>~un >estin
<wc <o E pr`oc t`on BG, o<'utwc t`o >ap`o t~hc A tetr'agwnon pr`oc
t`o >ap`o t~hc ZH tetr'agwnon, s'ummetron >'ara >est`i t`o >ap`o
t~hc A tetr'agwnon t~w| >ap`o t~hc ZH tetrag'wnw|. <rht`on d`e
t`o >ap`o t~hc A tetr'agwnon. <rht`on >'ara ka`i t`o >ap`o t~hc ZH;
<rht`h >'ara >est`in <h ZH. ka`i >epe`i <o E pr`oc t`on BG
l'ogon o>uk >'eqei, <`on tetr'agwnoc >arijm`oc pr`oc tetr'agwnon
>arijm'on, o>ud> >'ara t`o >ap`o t~hc A
tetr'agwnon pr`oc t`o >ap`o t~hc ZH [tetr'agwnon] l'ogon >'eqei,
<'on tetr'agwnoc >arijm`oc pr`oc tetr'agwnon >arijm'on; >as'ummetroc
>'ara >est`in <h A t~h| ZH m'hkei. p'alin, >epe'i >estin <wc <o BG
pr`oc t`on GD, o<'utwc t`o >ap`o t~hc ZH tetr'agwnon pr`oc t`o >ap`o
t~hc HJ, s'ummetron >'ara >est`i t`o >ap`o t~hc ZH t~w| >ap`o t~hc HJ.
<rht`on d`e t`o >ap`o t~hc ZH; <rht`on >'ara ka`i t`o >ap`o t~hc HJ; <rht`h
>'ara >est`in <h HJ. ka`i >epe`i <o BG pr`oc t`on GD l'ogon o>uk
>'eqei, <`on tetr'agwnoc >arijm`oc pr`oc tetr'agwnon >arijm'on, o>ud>
>'ara t`o >ap`o t~hc ZH pr`oc t`o >ap`o t~hc HJ l'ogon >'eqei, <`on
tetr'agwnoc >arijm`oc pr`oc tetr'agwnon >arijm'on;
>as'ummetroc >'ara >est`in <h ZH t~h| HJ m'hkei. ka'i e>isin >amf'oterai
<rhta'i; a<i ZH, HJ >'ara <rhta'i e>isi dun'amei m'onon s'ummetroi;
>apotom`h >'ara >est`in <h ZJ. l'egw d'h, <'oti ka`i tr'ith.}

\gr{>Epe`i g'ar >estin <wc m`en <o E pr`oc t`on BG, o<'utwc t`o >ap`o t~hc
A tetr'agwnon pr`oc t`o >ap`o t~hc ZH, <wc d`e <o BG pr`oc t`on GD,
o<'utwc t`o >ap`o t~hc ZH pr`oc  t`o >ap`o t~hc JH, di> >'isou >'ara
>est`in <wc <o E pr`oc t`on GD, o<'utwc t`o >ap`o t~hc A pr`oc t`o
>ap`o t~hc JH. <o d`e E pr`oc t`on GD l'ogon o>uk >'eqei, <`on
tetr'agwnoc >arijm`oc pr`oc tetr'agwnon >arijm'on; o>ud> >'ara t`o
>ap`o t~hc A pr`oc t`o >ap`o t~hc HJ l'ogon >'eqei, <`on tetr'agwnoc
>arijm`oc
pr`oc tetr'agwnon >arijm'on; >as'ummetroc >'ara <h A t~h| HJ
m'hkei. o>udet'era >'ara t~wn ZH, HJ s'ummetr'oc >esti t~h| >ekkeim'enh|
<rht~h| t~h| A m'hkei. <~w| o>~un me~iz'on >esti t`o >ap`o t~hc ZH
to~u >ap`o t~hc HJ, >'estw t`o >ap`o t~hc K. >epe`i o>~un >estin <wc
<o BG pr`oc t`on GD, o<'utwc t`o >ap`o t~hc ZH pr`oc t`o >ap`o t~hc
HJ, >anastr'eyanti >'ara >est`in <wc <o BG pr`oc t`on BD, o<'utwc
t`o >ap`o t~hc ZH tetr'agwnon pr`oc t`o >ap`o t~hc K. <o d`e
BG pr`oc t`on BD l'ogon >'eqei, <`on tetr'agwnoc >arijm`oc pr`oc
tetr'agwnon >arijm'on. ka`i t`o <ap`o t~hc ZH >'ara pr`oc t`o >ap`o
t~hc K l'ogon >'eqei, <`on tetr'agwnoc >arijm`oc pr`oc tetr'agwnon >arijm'on.
s'ummetr'oc >'ara >est`in <h ZH t~h| K m'hkei,
ka`i d'unatai <h ZH t~hc HJ me~izon t~w| >ap`o summ'etrou <eaut~h|.
ka`i o>udet'era t~wn ZH, HJ s'ummetr'oc >esti t~h| >ekkeim'enh| <rht~h|
t~h| A m'hkei; <h ZJ >'ara >apotom'h >esti tr'ith.}

\gr{E<'urhtai >'ara <h tr'ith >apotom`h <h ZJ; <'oper
>'edei de~ixai.}}

\ParallelRText{
\begin{center}
{\large Proposition 87}
\end{center}

To find a third apotome.

\epsfysize=1.in
\centerline{\epsffile{Book10/fig087e.eps}}

Let the rational (straight-line) $A$ be laid down. And let the
three numbers, $E$, $BC$, and $CD$, not having to one another
the ratio which (some) square number (has) to (some) square number,
be laid down. And let $CB$ have to $BD$ the ratio which (some)
square number (has) to (some) square number. And let it have been
contrived that as $E$ (is) to $BC$, so the square on $A$ (is) to the square
on $FG$, and as $BC$ (is) to $CD$, so the square on $FG$ (is) to the (square)
on $GH$ [Prop. 10.6~corr.]. Therefore,
since  as $E$ is to $BC$, so the square on $A$ (is) to the square on
$FG$, the square on $A$ is thus commensurable with the square on
$FG$ [Prop. 10.6]. And the square on $A$
(is) rational. Thus, the (square) on $FG$ (is) also rational. Thus,
$FG$ is a rational (straight-line). And since  $E$ does not have to $BC$
the ratio which (some) square number (has) to (some) square number,
the square on $A$ thus does not have to the [square] on $FG$ the ratio
which (some) square number (has) to (some) square number either. Thus,
$A$ is incommensurable in length with $FG$ [Prop. 10.9]. Again, since as $BC$ is to $CD$, so
the square on $FG$ is to the (square) on $GH$, the square on $FG$
is thus commensurable with the (square) on $GH$ [Prop. 10.6]. And the (square) on $FG$ (is)
rational. Thus, the (square) on $GH$ (is) also rational. Thus, $GH$ is
a rational (straight-line). And since $BC$ does not have to $CD$ the
ratio which (some) square number (has) to (some) square number, the
(square) on $FG$ thus does not have to the (square) on $GH$ the
ratio which (some) square number (has) to (some) square number either.
Thus, $FG$ is incommensurable in length with $GH$ [Prop. 10.9]. And both are rational (straight-lines).
$FG$ and $GH$ are thus rational (straight-lines which are) commensurable
in square only. Thus, $FH$ is an apotome [Prop. 10.73]. So, I say that (it is) also a third
(apotome).

For since as $E$ is to $BC$, so the square on $A$ (is) to the (square)
on $FG$, and as $BC$ (is) to $CD$, so the (square) on $FG$
(is) to the (square) on $HG$, thus, via equality, as $E$ is to
$CD$, so the (square) on $A$ (is) to the (square) on $HG$ [Prop. 5.22]. And $E$ does not have to $CD$
the ratio which (some) square number (has) to (some) square number. Thus,
the (square) on $A$ does not have to the (square) on $GH$ the ratio
which (some) square number (has) to (some) square number either. 
$A$ (is) thus incommensurable in length with $GH$ [Prop. 10.9]. Thus, neither of $FG$ and $GH$
is commensurable in length with the (previously) laid down rational
(straight-line) $A$. Therefore, let the (square) on $K$ be that (area) by which the (square) on
$FG$ is greater than the (square) on $GH$ [Prop. 10.13~lem.]. Therefore, since as $BC$ is to
$CD$, so the (square) on $FG$ (is) to the (square) on $GH$, thus,
via conversion, as $BC$ is to $BD$, so the square on $FG$ (is) to the square on
$K$ [Prop. 5.19~corr.]. And $BC$ has to $BD$ the ratio which
(some) square number (has) to (some) square number. Thus, the (square)
on $FG$ also has to the (square) on $K$ the ratio which (some)
square number (has) to (some) square number.
$FG$ is thus commensurable in length with $K$ [Prop. 10.9]. And the square on $FG$ is (thus) greater
than (the square on) $GH$ by the (square) on (some straight-line)
commensurable (in length) with ($FG$). And neither of $FG$ and
$GH$ is commensurable in length with the (previously) laid down
rational (straight-line) $A$. Thus, $FH$ is a third
apotome [Def. 10.13].

Thus, the third apotome $FH$ has been found. (Which is) very thing
it was required to show.}
\end{Parallel}


\vspace{7pt}{\footnotesize\noindent$^\dag$ See footnote
to Prop.~10.50.}

%%%%
%10.88
%%%%
\pdfbookmark[1]{Proposition 10.88}{pdf10.88}
\begin{Parallel}{}{}
\ParallelLText{
\begin{center}
{\large\ggn{88}.}
\end{center}\vspace*{-7pt}

\gr{E<ure~in t`hn tet'arthn >apotom'hn.}

\epsfysize=0.65in
\centerline{\epsffile{Book10/fig088g.eps}}

\gr{>Ekke'isjw <rht`h <h A ka`i t~h| A m'hkei s'ummetroc <h BH;
<rht`h >'ara >est`i ka`i <h BH. ka`i >ekke'isjwsan d'uo >arijmo`i o<i DZ, ZE,
<'wste t`on DE <'olon pr`oc <ek'ateron t~wn DZ, EZ l'ogon m`h
>'eqein, <`on tetr'agwnoc >arijm`oc pr`oc tetr'agwnon >arijm'on. ka`i
pepoi'hsjw <wc <o DE pr`oc t`on EZ, o<'utwc t`o >ap`o t~hc BH
tetr'agwnon pr`oc t`o >ap`o t~hc HG; s'ummetron >'ara >est`i t`o
>ap`o t~hc BH t~w| >ap`o t~hc HG.
 <rht`on d`e t`o >ap`o t~hc BH; 
 <rht`on >'ara ka`i t`o >ap`o t~hc HG; 
 <rht`h >'ara >est`in <h HG.
ka`i >epe`i <o DE pr`oc t`on EZ l'ogon o>uk >'eqei, <`on tetr'agwnoc >arijm`oc pr`oc tetr'agwnon >arijm'on, o>ud> >'ara t`o >ap`o t~hc
BH pr`oc t`o >ap`o t~hc HG l'ogon >'eqei, <`on tetr'agwnoc >arijm`oc
pr`oc tetr'agwnon >arijm'on;
>as'ummetroc >'ara >est`in <h BH t~h| HG m'hkei. ka'i e>isin >amf'oterai
<rhta'i; a<i BH, HG >'ara <rhta'i e>isi dun'amei m'onon s'ummetroi;
>apotom`h >'ara >est`in <h BG.  [l'egw d'h, <'oti ka`i tet'arth.]}

\gr{<~Wi o>~un me~iz'on >esti t`o >ap`o t~hc BH to~u >ap`o t~hc
HG, >'estw t`o >ap`o t~hc J. >epe`i o>~un >estin <wc <o DE pr`oc
t`on EZ, o<'utwc t`o >ap`o t~hc BH pr`oc t`o >ap`o t~hc
HG, ka`i >anastr'eyanti >'ara >est`in <wc <o ED pr`oc t`on DZ,
o<'utwc t`o >ap`o t~hc HB pr`oc t`o  >ap`o t~hc J.
<o d`e ED pr`oc t`on DZ l'ogon o>uk >'eqei, <`on tetr'agwnoc
>arijm`oc pr`oc tetr'agwnon >arijm'on; o>ud> >'ara t`o >ap`o t~hc HB
pr`oc t`o >ap`o t~hc J l'ogon >'eqei, <`on tetr'agwnoc >arijm`oc
pr`oc tetr'agwnon >arijm'on;
 >as'ummetroc >'ara >est`in
<h BH t~h| J m'hkei. ka`i d'unatai <h BH t~hc HG me~izon t~w|
>ap`o t~hc J; <h >'ara BH t~hc HG me~izon d'unatai
t~w| >ap`o >asumm'etrou <eaut~h|.
ka`i >estin <'olh <h BH
 s'ummetroc t~h| >ekkeim'enh| <rht~h| m'hkei t~h| A.
<h >'ara BG >apotom'h >esti tet'arth.}

\gr{E<'urhtai >'ara <h tet'arth >apotom'h; <'oper >'edei de~ixai.}}

\ParallelRText{
\begin{center}
{\large Proposition 88}
\end{center}

To find a fourth apotome.

\epsfysize=0.65in
\centerline{\epsffile{Book10/fig088e.eps}}

Let the rational (straight-line) $A$, and $BG$ (which is)
commensurable in length with $A$, be laid down.
Thus, $BG$ is also a rational (straight-line). And let the two numbers $DF$
and $FE$ be laid down such that the whole, $DE$, does not have to
each of $DF$ and $EF$ the ratio which (some) square number (has)
to (some) square number. And let it have been contrived that as $DE$
(is) to $EF$, so the square on $BG$ (is) to the (square) on $GC$
[Prop. 10.6~corr.]. The (square) on $BG$ is
thus commensurable with the (square) on $GC$ [Prop. 10.6]. And the (square) on $BG$ (is)
rational. Thus, the (square) on $GC$ (is) also rational. Thus, $GC$
(is) a rational (straight-line). And since $DE$ does not have to $EF$
the ratio which (some) square number (has) to (some) square number,
the (square) on $BG$ thus does not have to the (square) on $GC$ the
ratio which (some) square number (has) to (some) square number either.
Thus, $BG$ is incommensurable in length with $GC$ [Prop. 10.9]. And they are both rational (straight-lines). Thus, $BG$ and $GC$ are rational (straight-lines which are)
commensurable in square only. Thus, $BC$ is an apotome [Prop. 10.73]. [So, I say that
(it is) also a fourth (apotome).]

Now, let the (square) on $H$ be that (area) by which the (square)
on $BG$ is greater than the (square) on $GC$ [Prop. 10.13~lem.]. Therefore, since as
$DE$ is to $EF$, so the (square) on $BG$ (is) to the (square)
on $GC$, thus, also, via conversion, as $ED$ is to $DF$, so the
(square) on $GB$ (is) to the (square) on $H$ [Prop. 5.19~corr.]. And $ED$ does not have to
$DF$ the ratio which (some) square number (has) to (some) square number.
Thus, the (square) on $GB$ does not have to the (square) on $H$
the ratio which (some) square number (has) to (some) square number either.
Thus, $BG$ is incommensurable in length with $H$ [Prop. 10.9]. And the square on $BG$ is greater
than (the square on) $GC$ by the (square) on $H$. Thus, the
square on $BG$ is greater than (the square) on $GC$ by the (square)
on (some straight-line) incommensurable (in length) with ($BG$). And the whole,
$BG$, is commensurable in length with the the (previously) laid
down rational (straight-line) $A$. Thus, $BC$ is a fourth apotome
[Def. 10.14].$^\dag$

Thus, a fourth apotome has been found. (Which is) the very thing it was required to show.}
\end{Parallel}


\vspace{7pt}{\footnotesize\noindent$^\dag$ See footnote to Prop.~10.51.}

%%%%
%10.89
%%%%
\pdfbookmark[1]{Proposition 10.89}{pdf10.89}
\begin{Parallel}{}{}
\ParallelLText{
\begin{center}
{\large \ggn{89}.}
\end{center}\vspace*{-7pt}

\gr{E>ure~in t`hn p'empthn >apotom'hn.}

\epsfysize=0.6in
\centerline{\epsffile{Book10/fig089g.eps}}

\gr{>Ekke'isjw <rht`h <h A, ka`i t~h| A m'hkei s'ummetroc >'estw <h GH;
<rht`h >'ara [>est`in] <h GH. ka`i >ekke'isjwsan d'uo >arijmo`i
o<i DZ, ZE, <'wste t`on DE pr`oc <ek'ateron t~wn DZ, ZE l'ogon
p'alin m`h >'eqein, <`on tetr'agwnoc >arijm`oc pr`oc tetr'agwnon
>arijm'on; 
ka`i pepoi'hsjw <wc <o ZE pr`oc t`on ED, o<'utwc t`o >ap`o t~hc
GH pr`oc t`o >ap`o t~hc HB. 
<rht`on >'ara ka`i t`o >ap`o t~hc HB; 
<rht`h >'ara >est`i ka`i <h BH. 
ka`i
>epe'i >estin <wc <o DE pr`oc t`on EZ, o<'utwc t`o >ap`o t~hc BH
pr`oc t`o >ap`o t~hc HG, <o d`e DE pr`oc t`on EZ l'ogon o>uk
>'eqei, <`on tetr'agwnoc >arijm`oc pr`oc tetr'agwnon >arijm'on,
o>ud> >'ara t`o >ap`o t~hc BH pr`oc t`o >ap`o t~hc HG l'ogon
>'eqei, <`on tetr'agwnoc >arijm`oc pr`oc tetr'agwnon >arijm'on;
>as'ummetroc >'ara >est`in <h BH t~h| HG m'hkei. ka'i e>isin
>amf'oterai <rhta'i; a<i BH, HG >'ara <rhta'i e>isi dun'amei
m'onon s'ummetroi; <h BG >'ara >apotom'h >estin. l'egw d'h,
<'oti ka`i p'empth.}

\gr{<~Wi g`ar me~iz'on >esti t`o >ap`o t~hc BH to~u >ap`o t~hc
HG, >'estw t`o >ap`o t~hc J. >epe`i o>~un >estin <wc t`o >ap`o
t`hc BH pr`oc t`o >ap`o t~hc HG, o<'utwc <o DE pr`oc t`on EZ,
>anastr'eyanti >'ara >est`in  <wc <o ED pr`oc t`on DZ, o<'utwc
t`o >ap`o t~hc BH pr`oc t`o >ap`o t~hc J, <o d`e ED pr`oc t`on DZ
l'ogon o>uk >'eqei, <`on tetr'agwnoc >arijm`oc pr`oc tetr'agwnon
>arijm'on; o>ud> >'ara t`o >ap`o t~hc BH pr`oc t`o >ap`o t~hc J
l'ogon >'eqei, <`on tetr'agwnoc >arijm`oc pr`oc tetr'agwnon >arijm'on;
>as'ummetroc >'ara >est`in <h BH t~h| J m'hkei. ka`i d'unatai
<h BH t~hc HG me~izon t~w| >ap`o t~hc J; <h HB >'ara t~hc HG
me~izon d'unatai t~w| >ap`o >asumm'etrou <eaut~h| m'hkei. ka'i
>estin <h prosarm'ozousa <h GH s'ummetroc t~h| >ekkeim'enh| <rht~h|
t~h| A m'hkei; <h >'ara BG >apotom'h >esti p'empth.}

\gr{E<'urhtai >'ara <h p'empth >apotom`h <h BG;
<'oper >'edei de~ixai.}}

\ParallelRText{
\begin{center}
{\large Proposition 89}
\end{center}

To find a fifth apotome.

\epsfysize=0.6in
\centerline{\epsffile{Book10/fig089e.eps}}

Let the rational (straight-line) $A$ be laid down, and let
$CG$ be commensurable in length with $A$. Thus, $CG$ [is] a rational
(straight-line). And let the two numbers $DF$ and $FE$ be laid
down such that $DE$ again does not have to each of $DF$ and $FE$
the ratio which (some) square number (has) to (some) square number. 
And let it have been contrived that as $FE$ (is) to $ED$, so the
(square) on $CG$ (is) to the (square) on $GB$. Thus, the (square) on
$GB$ (is) also rational [Prop. 10.6]. 
Thus, $BG$ is also rational.
And since as $DE$ is to $EF$, so the (square) on $BG$ (is) to
the (square) on $GC$. And $DE$ does not have to $EF$ the
ratio which (some) square number (has) to (some) square number.
The (square) on $BG$ thus does not have to the (square) on $GC$
the ratio which (some) square number (has) to (some) square number either.
Thus, $BG$ is incommensurable in length with $GC$ [Prop. 10.9]. And they are both rational (straight-lines). $BG$ and $GC$ are thus rational (straight-lines which are)
commensurable in square only. Thus, $BC$ is an apotome [Prop. 10.73]. So, I say that (it is) also a fifth
(apotome).

For, let the (square) on $H$ be that (area) by which the (square) on $BG$
is greater than the (square) on $GC$ [Prop. 10.13~lem.]. Therefore, since
as the (square) on $BG$ (is) to the (square) on $GC$, so $DE$  (is) to
$EF$, thus, via conversion, as $ED$ is to $DF$, so the (square)
on $BG$ (is) to the (square) on $H$ [Prop. 5.19~corr.]. And $ED$ does not have to
$DF$ the ratio which (some) square number (has) to (some) square number.
Thus, the (square) on $BG$ does not have to the (square) on $H$
the ratio which (some) square number (has) to (some) square number
either. Thus, $BG$ is incommensurable in length with $H$ [Prop. 10.9]. And the square on $BG$ is greater
than (the square on) $GC$ by the (square) on $H$.
Thus, the square on $GB$ is greater
than (the square on) $GC$ by the (square) on (some straight-line)
incommensurable in length with ($GB$). And the attachment $CG$ is
commensurable in length with the (previously) laid down rational (straight-line) $A$. Thus, $BC$ is a fifth apotome [Def. 10.15].$^\dag$

Thus, the fifth apotome $BC$ has been found. (Which is) the very thing it
was required to show.}
\end{Parallel}


\vspace{7pt}{\footnotesize\noindent$^\dag$ See footnote to Prop.~10.52.}

%%%%
%10.90
%%%%
\pdfbookmark[1]{Proposition 10.90}{pdf10.90}
\begin{Parallel}{}{}
\ParallelLText{
\begin{center}
{\large \ggn{90}.}
\end{center}

\gr{E<ure~in t`hn <'ekthn >apotom'hn.}

\gr{>Ekke'isjw <rht`h <h A ka`i tre~ic >arijmo`i o<i E, BG, GD
l'ogon m`h >'eqontec pr`oc >all'hlouc, <`on tetr'agwnoc >arijm`oc
pr`oc tetr'agwnon >arijm'on; >'eti d`e ka`i <o GB pr`oc t`on BD l'ogon
m`h >eqet'w, <`on tetr'agwnoc >arijm`oc pr`oc tetr'agwnon >arijm'on;
ka`i pepoi'hsjw <wc m`en <o E pr`oc t`on BG, o<'utwc t`o >ap`o t~hc
A pr`oc t`o >ap`o t~hc ZH, <wc d`e <o BG pr`oc t`on GD, o<'utwc t`o >ap`o
t~hc ZH pr`oc t`o >ap`o t~hc HJ.}\\~\\

\epsfysize=1.1in
\centerline{\epsffile{Book10/fig090g.eps}}

\gr{>Epe`i o>~un >estin <wc <o E pr`oc t`on BG, o<'utwc t`o >ap`o t~hc
A pr`oc t`o >ap`o t~hc ZH, s'ummetron >'ara t`o >ap`o t~hc A t~w|
>ap`o t~hc ZH. <rht`on d`e t`o >ap`o t~hc A; <rht`on >'ara ka`i
t`o >ap`o t~hc ZH; <rht`h >'ara >est`i ka`i <h ZH. ka`i >epe`i <o E
pr`oc t`on BG l'ogon o>uk >'eqei, <`on tetr'agwnoc >arijm`oc
pr`oc tetr'agwnon >arijm'on, o>ud> >'ara t`o >ap`o t~hc A pr`oc t`o
>ap`o t~hc ZH l'ogon >'eqei, <`on tetr'agwnoc >arijm`oc pr`oc tetr'agwnon
>arijm'on; >as'ummetroc >'ara >est`in <h A t~h ZH m'hkei. p'alin,
>epe'i >estin <wc <o BG pr`oc t`on GD, o<'utwc t`o >ap`o t~hc ZH
pr`oc t`o >ap`o t~hc HJ, s'ummetron >'ara t`o >ap`o t~hc ZH t~w|
>ap`o t~hc HJ. <rht`on d`e t`o >ap`o t~hc ZH; <rht`on >'ara ka`i t`o
>ap`o t~hc HJ; <rht`h >'ara ka`i <h HJ. ka`i >epe`i <o BG pr`oc
t`on GD l'ogon o>uk >'eqei, <`on tetr'agwnoc >arijm`oc pr`oc tetr'agwnon
>arijm'on, o>ud> >'ara t`o >ap`o t~hc ZH pr`oc t`o >ap`o t~hc HJ
l'ogon >'eqei, <`on tetr'agwnoc >arijm`oc pr`oc tetr'agwnon >arijm'on;
>as'ummetroc >'ara >est`in <h ZH t~h| HJ m'hkei. ka'i e>isin
>amf'oterai <rhta'i; a<i ZH, HJ >'ara <rhta'i e>isi dun'amei m'onon
s'ummetroi; <h >'ara ZJ >apotom'h >estin. l'egw d'h, <'oti ka`i
<'ekth.}

\gr{>Epe`i g'ar >estin <wc m`en <o E pr`oc t`on BG, o<'utwc t`o >ap`o t~hc
A pr`oc t`o >ap`o t~hc ZH, <wc d`e <o BG pr`oc t`on GD, o<'utwc
t`o >ap`o t~hc ZH pr`oc t`o >ap`o t~hc HJ, di> >'isou >'ara >est`in
<wc <o E pr`oc t`on GD, o<'utwc t`o >ap`o t~hc A pr`oc t`o >ap`o t~hc
HJ. <o d`e E pr`oc t`on GD l'ogon o>uk >'eqei, <`on tetr'agwnoc
>arijm`oc pr`oc tetr'agwnon >arijm'on; o>ud> >'ara t`o >ap`o t~hc
A pr`oc t`o >ap`o t~hc HJ l'ogon >'eqei, <`on tetr'agwnoc >arijm`oc
pr`oc tetr'agwnon >arijm'on; >as'ummetroc >'ara >est`in <h 
A t~h| HJ m'hkei; o>udet'era >'ara t~wn ZH, HJ s'ummetr'oc >esti
t~h| A <rht~h| m'hkei. <~w| o>~un me~iz'on
>esti t`o >ap`o t~hc ZH to~u >ap`o t~hc HJ, >'estw t`o >ap`o
t~hc K. >epe`i o>~un >estin <wc <o BG pr`oc t`on GD, o<'utwc
t`o >ap`o t~hc ZH pr`oc t`o >ap`o t~hc HJ, >anastr'eyanti >'ara
>est`in
<wc <o GB pr`oc t`on BD, o<'utwc t`o >ap`o t~hc ZH pr`oc t`o
>ap`o t~hc K. <o d`e GB pr`oc t`on BD l'ogon o>uk >'eqei,
<`on tetr'agwnoc >arijm`oc pr`oc tetr'agwnon >arijm'on;
o>ud> >'ara t`o >ap`o t~hc ZH pr`oc t`o >ap`o t~hc K
l'ogon >'eqei, <`on tetr'agwnoc >arijm`oc pr`oc tetr'agwnon
>arijm'on; >as'ummetroc >'ara >est`in <h ZH t~h| K m'hkei.
ka`i d'unatai <h ZH t~hc HJ me~izon t~w| >ap`o t~hc K; <h
ZH >'ara t~hc HJ me~izon d'unatai t~w| >ap`o  >asumm'etrou
<eaut~h| m'hkei. ka`i o>udet'era t~wn ZH, HJ s'ummetr'oc >esti t~h|
>ekkeim'enh| <rht~h| m'hkei t~h| A. <h >'ara ZJ >apotom'h >estin
<'ekth.}

\gr{E<'urhtai >'ara <h <'ekth >apotom`h <h ZJ; <'oper >'edei
de~ixai.}}

\ParallelRText{
\begin{center}
{\large Proposition 90}
\end{center}

To find a sixth apotome.

Let the rational (straight-line) $A$, and the three numbers
$E$, $BC$, and $CD$, not having to one another the ratio which (some)
square  number (has) to (some) square number, be laid down. Furthermore, let $CB$
also not have to $BD$ the ratio which (some) square number (has) to
(some) square number. And let it have been contrived that as $E$
(is) to $BC$, so the (square) on $A$ (is) to the (square) on $FG$,
and as $BC$ (is) to $CD$, so the (square) on $FG$ (is) to the (square)
on $GH$ [Prop. 10.6~corr.].

\epsfysize=1.1in
\centerline{\epsffile{Book10/fig090e.eps}}

Therefore, since as $E$ is to $BC$, so the (square) on $A$ (is)
to the (square) on $FG$, the (square) on $A$ (is) thus commensurable
with the (square) on $FG$ [Prop. 10.6]. 
And the (square) on $A$ (is) rational. Thus, the (square) on $FG$
(is) also rational. Thus, $FG$  is also a rational (straight-line). And
since $E$ does not have to $BC$ the ratio which (some) square
number (has) to (some) square number, the (square) on $A$
thus does not have to the (square) on $FG$ the ratio which (some)
square number (has) to (some) square number either. Thus,
$A$ is incommensurable in length with $FG$ [Prop. 10.9]. Again, since as $BC$ is to $CD$, so
the (square) on $FG$ (is) to the (square) on $GH$, the (square) on
$FG$ (is) thus commensurable with the (square) on $GH$ [Prop. 10.6]. And the (square) on $FG$ (is) rational.
Thus, the (square) on $GH$ (is) also rational. Thus, $GH$
(is) also rational. And since $BC$ does not have to $CD$ the ratio
which (some) square number (has) to (some) square number, the
(square) on $FG$ thus does not have to the (square) on $GH$
the ratio which (some) square (number) has to (some) square (number)
either. Thus, $FG$ is incommensurable in length with $GH$ [Prop. 10.9]. And both are rational (straight-lines).
Thus, $FG$ and $GH$ are rational (straight-lines which are) commensurable
in square only. Thus, $FH$ is an apotome [Prop. 10.73]. So, I say that (it is) also a sixth (apotome).

For since as $E$ is to $BC$, so the (square) on $A$ (is) to the (square)
on $FG$, and as $BC$ (is) to $CD$, so the (square) on $FG$ (is) to
the (square) on $GH$, thus, via equality, as $E$ is to $CD$, so the
(square) on $A$ (is) to the (square) on $GH$ [Prop. 5.22]. And $E$ does not have to $CD$
the ratio which (some) square number (has) to (some) square number.
Thus, the (square) on $A$ does not have to the (square) $GH$ the
ratio which (some) square number (has) to (some) square number either. $A$ is thus incommensurable in length with $GH$ [Prop. 10.9]. Thus, neither of $FG$ and $GH$
is commensurable in length with the rational (straight-line) $A$. Therefore, let the (square)
on $K$ be that (area) by which the (square) on $FG$ is greater than the (square) on $GH$ [Prop. 10.13~lem.].
 Therefore, since as $BC$ is to $CD$, so the (square) on $FG$ (is) to the
(square) on $GH$, thus, via conversion, as $CB$ is to $BD$, so the (square)
on $FG$ (is) to the (square) on $K$ [Prop. 5.19~corr.]. And $CB$ does not have
to $BD$ the ratio which (some) square number (has) to (some) square
number. Thus, the (square) on $FG$ does not have to the (square) on
$K$ the ratio which (some) square number (has) to (some) square number
either. $FG$ is thus incommensurable in length with $K$ [Prop. 10.9]. And the square on $FG$ is greater
than (the square on) $GH$ by the (square) on $K$.
Thus, the square on
$FG$ is greater than (the square on) $GH$ by the (square)
on (some straight-line) incommensurable in length with ($FG$).
And neither of $FG$ and $GH$ is commensurable in length
with the (previously) laid down rational (straight-line) $A$. Thus,
$FH$ is a sixth apotome [Def. 10.16].

Thus, the sixth apotome $FH$ has been found. (Which is) the
very thing it was required to show.}
\end{Parallel}


\vspace{7pt}{\footnotesize\noindent$^\dag$ See footnote to Prop.~10.53.}

%%%%
%10.91
%%%%
\pdfbookmark[1]{Proposition 10.91}{pdf10.91}
\begin{Parallel}{}{}
\ParallelLText{
\begin{center}
{\large \ggn{91}.}
\end{center}\vspace*{-7pt}

\gr{>E`an qwr'ion peri'eqhtai <up`o <rht~hc ka`i >apotom~hc pr'wthc,
<h t`o qwr'ion dunam'enh >aporom'h >estin.}

\gr{Perieq'esjw g`ar qwr'ion t`o AB <up`o <rht~hc t~hc AG ka`i >apotom~hc
pr'wthc t~hc AD; l'egw, <'oti <h t`o AB qwr'ion dunam'enh >apotom'h
>estin.}\\

\epsfysize=1.3in
\centerline{\epsffile{Book10/fig091g.eps}}

\gr{>Epe`i g`ar >apotom'h >esti pr'wth <h AD, >'estw a>ut~h| prosarm'ozousa
<h DH; a<i AH, HD >'ara <rhta'i e>isi dun'amei m'onon s'ummetroi.
ka`i <'olh <h AH s'ummetr'oc >esti t~h| >ekkeim'enh| <rht~h| t~h| AG,
ka`i <h AH t~hc HD me~izon d'unatai t~w| >ap`o summ'etrou <eaut~h|
m'hkei; >e`an >'ara t~w| tet'artw| m'erei to~u >ap`o t~hc DH >'ison
par`a t`hn AH parablhj~h| >elle~ipon e>'idei tetrag'wnw|, e>ic s'ummetra
a>ut`hn diaire~i. tetm'hsjw <h DH d'iqa kat`a t`o E, ka`i t~w| >ap`o
t~hc EH >'ison par`a t`hn AH parabebl'hsjw >elle~ipon e>'idei
tetrag'wnw|, ka`i >'estw t`o <up`o t~wn AZ, ZH; s'ummetroc >'ara
>est`in <h AZ t~h| ZH. ka`i di`a t~wn E, Z, H shme'iwn t~h| AG
par'allhloi >'hqjwsan a<i EJ, ZI, HK.}

\gr{Ka`i >epe`i s'ummetr'oc >estin <h AZ t~h| ZH m'hkei, ka`i <h AH
>'ara <ekat'era| t~wn AZ, ZH s'ummetr'oc >esti m'hkei. >all`a
<h AH s'ummetr'oc >esti t~h| AG; ka`i <ekat'era
>'ara t~wn AZ, ZH s'ummetr'oc >esti t~h| AG m'hkei. ka'i >esti
<rht`h <h AG; <rht`h >'ara ka`i <ekat'era t~wn AZ, ZH; <'wste ka`i
<ek'ateron t~wn AI, ZK <rht'on >estin. ka`i >epe`i s'ummetr'oc
>estin <h DE t~h| EH m'hkei, ka`i <h DH >'ara <ekat'era| t~wn
DE, EH s'ummetr'oc >esti m'hkei. <rht`h d`e <h DH ka`i
>as'ummetroc t~h| AG m'hkei; <rht`h >'ara ka`i <ekat'era t~wn
DE, EH ka`i >as'ummetroc t~h| AG m'hkei; <ek'ateron >'ara t~wn
DJ, EK m'eson >est'in.}

\gr{Ke'isjw d`h t~w| m`en AI >'ison tetr'agwnon t`o LM,
t~w| d`e ZK >'ison tetr'agwnon >afh|r'hsjw koin`hn gwn'ian
>'eqon a>ut~w| t`hn <up`o LOM t`o NX; per`i t`hn
a>ut`hn >'ara di'ametr'on >esti t`a LM, NX tetr'agwna. >'estw
a>ut~wn di'ametroc <h OR, ka`i katagegr'afjw t`o sq~hma.
>epe`i o>~un >'ison >est`i t`o <up`o t~wn AZ, ZH perieq'omenon >orjog'wnion t~w| >ap`o t~hc EH tetrag'wnw|, >'estin >'ara <wc
<h AZ pr`oc t`hn EH, o<'utwc <h EH pr`oc t`hn ZH.
>all> <wc m`en <h AZ pr`oc t`hn EH, o<'utwc t`o AI pr`oc
t`o EK, <wc d`e <h EH pr`oc t`hn ZH, o<'utwc >est`i t`o EK pr`oc
t`o KZ; t~wn >'ara AI, KZ m'eson >an'alog'on >esti t`o EK. >'esti
d`e ka`i t~wn LM, NX m'eson >an'alogon t`o MN,  <wc
>en to~ic >'emprosjen >ede'iqjh, ka'i >esti t`o [m`en] AI t~w| LM
tetrag'wnw| >'ison, t`o d`e KZ t~w| NX; ka`i t`o MN >'ara t~w| EK >'ison
 >est'in. >all`a t`o m`en EK t~w| DJ >estin >'ison,
t`o d`e MN t~w| LX; t`o >'ara DK >'ison >est`i t~w| UFQ gn'wmoni
ka`i t~w| NX. >'esti d`e ka`i t`o AK >'ison to~ic LM, NX tetrag'wnoic;
loip`on >'ara t`o AB >'ison >est`i t~w| ST. t`o d`e ST t`o >ap`o
t~hc LN >esti tetr'agwnon; t`o >'ara >ap`o t~hc LN tetr'agwnon >'ison
>est`i t~w| AB; <h LN >'ara d'unatai t`o AB. l'egw d'h, <'oti
<h LN >apotom'h >estin.}

\gr{>Epe`i g`ar <rht'on >estin <ek'ateron t~wn AI, ZK, ka'i >estin >'ison
to~ic LM, NX, ka`i <ek'ateron >'ara t~wn LM, NX <rht'on >estin,
tout'esti t`o >ap`o <ekat'erac t~wn LO, ON; ka`i <ekat'era >'ara t~wn
LO, ON <rht'h >estin. p'alin, >epe`i m'eson >est`i t`o DJ ka'i >estin
>'ison t~w| LX, m'eson >'ara >est`i ka`i t`o LX. >epe`i o>~un t`o m`en
LX m'eson >est'in, t`o d`e NX <rht'on, >as'ummetron >'ara >est`i
t`o LX t~w| NX; <wc d`e t`o LX pr`oc t`o NX, o<'utwc
>est`in <h LO pr`oc t`hn ON; >as'ummetroc >'ara >est`in <h  LO
t~h| ON m'hkei. ka'i e>isin >amf'oterai <rhta'i; a<i LO, ON >'ara
<rhta'i e>isi dun'amei m'onon s'ummetroi; >apotom`h
>'ara >est`in <h LN. ka`i d'unatai t`o AB qwr'ion;
<h >'ara t`o AB qwr'ion dunam'enh >apotom'h >estin.}

\gr{>E`an >'ara qwr'ion peri'eqhtai <up`o <rht~hc ka`i t`a
<ex~hc.}}

\ParallelRText{
\begin{center}
{\large Proposition 91}
\end{center}

If an area is contained by a rational (straight-line)
and a first apotome then the square-root of the area is an apotome.

For let the area $AB$ have been contained by the rational (straight-line) $AC$ and the first apotome $AD$. I say that the square-root of area $AB$ is an apotome.

\epsfysize=1.3in
\centerline{\epsffile{Book10/fig091e.eps}}

For since $AD$ is a first apotome, let $DG$ be its attachment. Thus,
$AG$ and $DG$ are rational (straight-lines which are) commensurable in
square only [Prop. 10.73]. And the whole, $AG$,
is commensurable (in length) with the (previously) laid down rational (straight-line)
$AC$, and the square on $AG$ is greater than (the square on) $GD$
by the (square) on  (some straight-line) commensurable in length with ($AG$) [Def. 10.11]. Thus, if (an area) equal to the
fourth part of the (square) on $DG$ is applied to $AG$, falling short
by a square figure, then it divides ($AG$) into (parts which are) commensurable
(in length) [Prop. 10.17]. Let $DG$
have been cut in half at $E$. And let (an area) equal to the (square) on
$EG$ have been applied to $AG$, falling short by a square figure. And let
it be the (rectangle contained) by $AF$ and $FG$. $AF$ is
 thus commensurable (in length) with $FG$. And let $EH$, $FI$, and $GK$
have been drawn through points $E$, $F$, and $G$ (respectively),
parallel to $AC$.

And since $AF$ is commensurable in length with $FG$, $AG$ is thus also
commensurable in length with each of $AF$ and $FG$ [Prop. 10.15]. But $AG$ is commensurable (in length)
with $AC$. Thus, each of $AF$ and $FG$ is also commensurable in length with $AC$ 
[Prop. 10.12]. 
And $AC$ is a rational (straight-line). Thus,  $AF$ and $FG$ (are) each
also rational (straight-lines). Hence, $AI$ and
$FK$ are also each rational (areas) [Prop. 10.19]. And since $DE$ is commensurable
in length with $EG$, $DG$ is thus also commensurable
in length with each of $DE$ and $EG$ [Prop. 10.15]. And $DG$ (is) rational, and
incommensurable in length with $AC$. $DE$ and $EG$ (are) thus each
rational, and incommensurable in length with $AC$ [Prop. 10.13]. Thus, $DH$ and $EK$ are
each medial (areas) [Prop. 10.21].

So let the square $LM$, equal to $AI$, be laid down. And let the square $NO$,
equal to $FK$, have been
subtracted (from $LM$), having with it the common angle $LPM$. Thus,
the squares $LM$ and $NO$ are about the same diagonal [Prop. 6.26]. Let $PR$ be their (common) diagonal, and
let the (rest of the) figure have been drawn. Therefore, since
the rectangle contained by $AF$ and $FG$ is equal to the square $EG$,
thus as $AF$ is to $EG$, so $EG$ (is) to $FG$ [Prop. 6.17]. But, as $AF$ (is) to $EG$, so $AI$ (is)
to $EK$, and as $EG$ (is) to $FG$, so $EK$ is to $KF$ [Prop. 6.1]. Thus, $EK$ is the mean proportional
to $AI$ and $KF$ [Prop. 5.11]. And $MN$ is also the mean proportional to
$LM$ and $NO$, as shown before [Prop. 10.53~lem.].  And $AI$ is equal to the
square $LM$, and $KF$ to $NO$. Thus, $MN$ is also equal to $EK$.
But, $EK$ is equal to $DH$, and $MN$ to $LO$ [Prop. 1.43]. Thus, $DK$ is equal to the
gnomon $UVW$ and  $NO$. And $AK$ is also equal to 
(the sum of) the squares $LM$ and $NO$.  Thus, the remainder $AB$ is equal
to $ST$. And $ST$ is the square on $LN$. Thus, the square on $LN$
is equal to $AB$. Thus, $LN$ is the square-root of $AB$. So, I say that
$LN$ is an apotome.

For since $AI$ and $FK$ are each rational (areas), and are equal to
$LM$ and $NO$ (respectively), thus  $LM$ and $NO$---that is to say, the
(squares) on each of $LP$ and $PN$ (respectively)---are  also
each rational (areas). Thus, $LP$ and $PN$ are also each rational (straight-lines). 
Again, since $DH$ is a medial (area), and is equal to $LO$, $LO$ is thus
also a medial (area). Therefore, since $LO$ is medial, and $NO$ rational, 
$LO$ is thus incommensurable with $NO$. And as $LO$ (is) to $NO$,
so $LP$ is to $PN$ [Prop. 6.1]. $LP$ is
thus incommensurable in length with $PN$ [Prop. 10.11]. And they are both rational
(straight-lines). Thus, $LP$ and $PN$ are rational (straight-lines which are)
commensurable in square only. Thus, $LN$ is an apotome [Prop. 10.73]. And it is the square-root of area $AB$. Thus, the square-root of area $AB$ is an apotome.

Thus, if an area is contained by a rational (straight-line), and
so on \ldots.}
\end{Parallel}

%%%%
%10.92
%%%%
\pdfbookmark[1]{Proposition 10.92}{pdf10.92}
\begin{Parallel}{}{}
\ParallelLText{
\begin{center}
{\large \ggn{92}.}
\end{center}\vspace*{-7pt}

\gr{>E`an qwr'ion peri'eqhtai <up`o <rht~hc ka`i >apotom~hc deut'erac,
<h t`o qwr'ion dunam'enh m'eshc >apotom'h >esti pr'wth.}\\

\epsfysize=1.3in
\centerline{\epsffile{Book10/fig092g.eps}}

\gr{Qwr'ion g`ar t`o AB perieq'esjw <up`o <rht~hc t~hc AG ka`i >apotom~hc
deut'erac t~hc AD; l'egw, <'oti <h t`o AB qwr'ion dunam'enh m'eshc
>apotom'h >esti pr'wth.}

\gr{>'Estw g`ar t~h| AD prosarm'ozousa <h DH; a<i >'ara AH, HD
<rhta'i e>isi dun'amei m'onon s'ummetroi, ka`i <h prosarm'ozousa
<h DH s'ummetr'oc >esti t~h| >ekkeim'enh|
<rht~h| t~h| AG, <h d`e <'olh <h AH t~hc prosarmozo'ushc t~hc
HD me~izon d'unatai t~w| >ap`o summ'etrou <eaut~h| m'hkei.
>epe`i o>~un <h AH t~hc HD me~izon d'unatai t~w| >ap`o
summ'etrou <eaut~h|, >e`an >'ara t~w| tet'artw| m'erei to~u >ap`o t~hc
HD >'ison par`a t`hn AH parablhj~h| >elle~ipon e>'idei
tetrag'wnw|, e>ic s'ummetra a>ut`hn diaire~i. tetm'hsjw o>~un <h
DH d'iqa kat`a t`o E; ka`i t~w| >ap`o t~hc EH >'ison par`a t`hn AH
parabebl'hsjw >elle~ipon e>'idei tetrag'wnw|, ka`i >'estw t`o <up`o
t~wn AZ, ZH; s'ummetroc >'ara >est`in <h AZ t~h| ZH m'hkei. ka`i
<h AH >'ara <ekat'era| t~wn AZ, ZH s'ummetr'oc >esti m'hkei.
<rht`h d`e <h AH ka`i >as'ummetroc t~h| AG m'hkei; ka`i <ekat'era
>'ara t~wn AZ, ZH <rht'h >esti ka`i >as'ummetroc t~h| AG m'hkei;
<ek'ateron >'ara t~wn AI, ZK m'eson >est'in. p'alin, >epe`i
s'ummetr'oc >estin <h DE t~h| EH, ka`i <h DH >'ara <ekat'era| t~wn
DE, EH s'ummetr'oc >estin. >all> <h DH s'ummetr'oc >esti t~h|
AG m'hkei [<rht`h >'ara ka`i <ekat'era t~wn DE, EH ka`i s'ummetroc
t~h| AG m'hkei]. <ek'ateron >'ara t~wn DJ, EK <rht'on >estin.}

\gr{Sunest'atw o>~un t~w| m`en AI >'ison tetr'agwnon t`o LM, t~w|
d`e ZK >'ison >afh|r'hsjw t`o NX per`i t`hn a>ut`hn gwn'ian >`on
t~w| LM t`hn <up`o t~wn LOM; per`i t`hn a>ut`hn >'ara >est`i
di'ametron t`a LM, NX tetr'agwna. >'estw a>ut~wn di'ametroc
<h OR, ka`i katagegr'afjw t`o sq~hma. >epe`i o>~un t`a AI, ZK m'esa
>est`i ka'i >estin >'isa to~ic >ap`o t~wn LO, ON, ka`i t`a >ap`o t~wn
LO, ON [>'ara] m'esa >est'in; ka`i a<i LO, ON >'ara m'esai e>is`i
dun'amei m'onon s'ummetroi. ka`i >epe`i t`o <up`o t~wn AZ,
ZH >'ison >est`i t~w| >ap`o t~hc EH, >'estin >'ara <wc <h AZ pr`oc
t`hn EH, o<'utwc <h EH pr`oc t`hn ZH; >all> <wc m`en <h AZ
pr`oc t`hn EH, o<'utwc t`o AI pr`oc t`o EK; <wc d`e <h EH pr`oc t`hn
ZH, o<'utwc [>est`i] t`o EK pr`oc t`o ZK; t~wn >'ara AI, ZK m'eson
>an'alog'on >esti t`o EK. >'esti d`e ka`i t~wn LM, NX tetrag'wnwn
m'eson >an'alogon t`o MN; ka'i >estin >'ison t`o m`en AI t~w| LM,
t`o d`e ZK t~w| NX;  ka`i  t`o MN >'ara >'ison >est`i t~w| EK. >all`a
t~w| m`en EK >'ison [>est`i] t`o DJ, t~w| d`e MN >'ison t`o LX; <'olon
>'ara t`o DK >'ison >est`i t~w| UFQ gn'wmoni ka`i t~w| NX. >epe`i
o>~un <'olon t`o AK >'ison >est`i to~ic LM, NX, <~wn t`o DK
>'ison >est`i t~w| UFQ gn'wmoni ka`i t~w| NX, loip`on >'ara t`o AB
>'ison >est`i t~w| TS. t`o d`e TS >esti t`o >ap`o t~hc LN;
t`o >ap`o t~hc LN >'ara >'ison >est`i t~w| AB qwr'iw|; <h LN
>'ara d'unatai t`o AB qwr'ion. l'egw [d'h], <'oti <h LN m'eshc
>apotom'h >esti pr'wth.}

\gr{>Epe`i g`ar <rht'on >esti t`o EK ka'i >estin >'ison t~w| LX,
<rht`on >'ara >est`i t`o LX, tout'esti t`o <up`o t~wn
LO, ON. m'eson d`e >ede'iqjh t`o NX; >as'ummetron >'ara
>est`i t`o LX t~w| NX; <wc d`e t`o LX pr`oc t`o NX,
o<'utwc >est`in <h LO pr`oc ON; a<i LO, ON >'ara >as'ummetro'i
e>isi m'hkei. a<i >'ara LO, ON m'esai e>is`i dun'amei m'onon s'ummetroi
<rht`on peri'eqousai; <h LN >'ara m'eshc >apotom'h >esti pr'wth;
ka`i d'unatai t`o AB qwr'ion.}

\gr{<H >'ara t`o AB qwr'ion dunam'enh m'eshc >apotom'h >esti
pr'wth; <'oper >'edei de~ixai.}}

\ParallelRText{
\begin{center}
{\large Proposition 92}
\end{center}

If an area is contained by a rational (straight-line)
and a second apotome then the square-root of the area is a first apotome of a medial (straight-line).

\epsfysize=1.3in
\centerline{\epsffile{Book10/fig092e.eps}}

For let the area $AB$ have been contained by the rational (straight-line) $AC$ and the second apotome $AD$. I say that the square-root of area $AB$ is the first apotome of a medial (straight-line).

For let $DG$ be an attachment to $AD$. Thus, $AG$ and $GD$
are rational (straight-lines which are) commensurable in square only [Prop. 10.73],
and the attachment $DG$ is commensurable (in length) with the
(previously) laid down rational (straight-line) $AC$, and the square on
the whole, $AG$, is greater than (the square on) the attachment, $GD$,
by the (square) on (some straight-line) commensurable in length with ($AG$) 
[Def. 10.12]. Therefore, since the square on $AG$ is
greater than (the square on) $GD$ by the (square) on (some
straight-line) commensurable (in length) with ($AG$), thus if  (an area)
equal to the fourth part of the (square) on $GD$ is applied
to $AG$, falling short by a square figure, then it divides ($AG$) into (parts which
are) commensurable (in length) [Prop. 10.17]. 
Therefore, let $DG$ have been cut in half at $E$. And let (an area) equal to
the (square) on $EG$ have been applied to $AG$, falling short
by a square figure. And let it be the (rectangle contained) by $AF$ and $FG$.
Thus, $AF$ is commensurable in length with $FG$. $AG$ is thus
also commensurable in length with each of $AF$ and $FG$ [Prop. 10.15]. And $AG$ (is) a rational (straight-line), and incommensurable in length with $AC$. $AF$ and $FG$
are thus also each rational (straight-lines), and incommensurable in length with
$AC$ [Prop. 10.13].
Thus, $AI$
and $FK$ are each medial (areas) [Prop. 10.21]. 
Again, since $DE$ is commensurable (in length) with $EG$, thus
$DG$ is also commensurable (in length) with each of $DE$ and $EG$ [Prop. 10.15]. But, $DG$ is commensurable
in length with $AC$ [thus, $DE$ and $EG$ are also each rational, and
commensurable in length with $AC$]. Thus, $DH$ and $EK$ are each
rational (areas) [Prop. 10.19].

Therefore, let the square $LM$, equal to $AI$, have been constructed.
And let $NO$, equal to $FK$, which is about the
same angle $LPM$ as $LM$, have been subtracted (from $LM$).
Thus, the squares $LM$ and $NO$ are about the same diagonal
[Prop. 6.26]. Let $PR$ be their (common) diagonal, and
let the (rest of the) figure have been drawn. Therefore, since $AI$ and $FK$
are medial (areas), and are equal to the (squares) on $LP$ and $PN$
(respectively), [thus] the (squares) on $LP$ and $PN$ are also medial.
Thus, $LP$ and $PN$ are  also medial (straight-lines which are)
commensurable in square only.$^\dag$ And since the (rectangle contained) by
$AF$ and $FG$ is equal to the (square) on $EG$, thus as $AF$ is to $EG$, so $EG$ (is) to $FG$ [Prop. 10.17]. But, as
$AF$ (is) to $EG$, so $AI$ (is) to $EK$. And as $EG$ (is) to $FG$, so
$EK$ [is] to $FK$ [Prop. 6.1]. Thus, $EK$ is the mean proportional to $AI$ and $FK$ [Prop. 5.11].
And $MN$ is also the mean proportional to the squares $LM$ and $NO$ [Prop. 10.53~lem.]. 
And $AI$ is equal to $LM$, and $FK$ to $NO$. 
Thus, $MN$ is also equal to
$EK$. But, $DH$ [is] equal to $EK$, and $LO$ equal to $MN$ [Prop. 1.43]. Thus, the whole  (of) $DK$ is equal to
the gnomon $UVW$ and $NO$. Therefore, since the whole (of) $AK$
is equal to $LM$ and $NO$, of which $DK$ is equal to the
gnomon $UVW$ and $NO$,  the remainder $AB$ is thus equal to $TS$.
And $TS$ is the (square) on $LN$. Thus, the (square) on $LN$
is equal to the area $AB$. $LN$ is thus the square-root of area $AB$.
[So], I say that $LN$ is the first apotome of a medial (straight-line).

For since $EK$ is a rational (area), and is equal to $LO$, $LO$---that is to
say, the (rectangle contained) by $LP$ and $PN$---is thus
a rational (area). And $NO$ was shown (to be) a medial (area). Thus,
$LO$ is incommensurable  with $NO$. And as $LO$ (is) to $NO$,
so $LP$ is to $PN$ [Prop. 6.1]. Thus,  $LP$ and
$PN$ are incommensurable in length [Prop. 10.11]. 
$LP$ and $PN$ are thus medial (straight-lines which are) commensurable
in square only, and which contain a rational (area). Thus, $LN$ is the
first apotome of a medial (straight-line) [Prop. 10.74]. And it is the square-root of area $AB$.

Thus, the square root of area $AB$ is the first apotome of a medial (straight-line). (Which is) the very thing it was required to show.}
\end{Parallel}


\vspace{7pt}{\footnotesize\noindent$^\dag$ There is an error in the argument here. It should just say that $LP$ and $PN$ are commensurable in square,
rather than in square only, since $LP$ and $PN$ are only shown
to be incommensurable in length later on.}

%%%%
%10.93
%%%%
\pdfbookmark[1]{Proposition 10.93}{pdf10.93}
\begin{Parallel}{}{}
\ParallelLText{
\begin{center}
{\large \ggn{93}.}
\end{center}\vspace*{-7pt}

\gr{>E`an qwr'ion peri'eqhtai <up`o <rht~hc ka`i >apotom~hc tr'ithc,
<h t`o qwr'ion dunam'enh m'eshc >apotom'h >esti deut'era.}\\

\epsfysize=1.3in
\centerline{\epsffile{Book10/fig093g.eps}}

\gr{Qwr'ion g`ar t`o AB perieq'esjw <up`o <rht~hc t~hc AG ka`i >apotom~hc
tr'ithc t~hc AD; l'egw, <'oti <h t`o AB qwr'ion dunam'enh m'eshc
>apotom'h >esti deut'era.}

\gr{>'Estw g`ar t~h| AD prosarm'ozousa <h DH; a<i AH, HD >'ara <rhta'i e>isi
dun'amei m'onon s'ummetroi, ka`i o>udet'era t~wn AH, HD s'ummetr'oc
>esti m'hkei t~h| >ekkeim'enh| <rht~h| t~h| AG, <h d`e <'olh <h AH
t`hc prosarmozo'ushc t~hc DH me~izon d'unatai t~w| >ap`o
summ'etrou <eaut~h|. >epe`i o>~un <h AH t~hc HD me~izon
d'unatai t~w| >ap`o summ'etrou <eaut~h|, >e`an >'ara t~w| tet'artw|
m'erei to~u >ap`o t~hc DH >'ison par`a t`hn AH parablhj~h| >elle~ipon
e>'idei tetrag'wnw|, e>ic s'ummetra a>ut`hn diele~i. tetm'hsjw
o>~un <h DH d'iqa kat`a t`o E, ka`i t~w| >ap`o t~hc EH >'ison
par`a t`hn AH parabebl'hsjw >elle~ipon e>'idei tetrag'wnw|, ka`i
>'estw t`o <up`o t~wn AZ, ZH. ka`i >'hqjwsan di`a t~wn E, Z, H shme'iwn
t~h| AG par'allhloi a<i EJ, ZI, HK; s'ummetroi >'ara e>is`in a<i AZ, ZH;
s'ummetron >'ara ka`i t`o AI t~w| ZK. ka`i >epe`i a<i AZ, ZH s'ummetro'i
e>isi m'hkei, ka`i <h AH >'ara <ekat'era| t~wn AZ, ZH s'ummetr'oc
>esti m'hkei. <rht`h d`e <h AH ka`i >as'ummetroc t~h| AG m'hkei; <'wste
ka`i a<i AZ, ZH. <ek'ateron >'ara t~wn AI, ZK m'eson >est'in. p'alin, >epe`i
s'ummetr'oc >estin <h DE t~h| EH m'hkei, ka`i <h DH >'ara <ekat'era|
t~wn DE, EH s'ummetr'oc >esti m'hkei. <rht`h d`e <h HD ka`i
>as'ummetroc t~h| AG m'hkei; <rht`h >'ara ka`i <ekat'era t~wn DE,
EH ka`i >as'ummetroc t~h| AG m'hkei; <ek'ateron >'ara t~wn DJ, EK
m'eson >est'in. ka`i >epe`i a<i AH, HD dun'amei m'onon s'ummetro'i
e>isin, >as'ummetroc >'ara >est`i m'hkei <h AH t~h| HD. >all>
<h m`en AH t~h| AZ s'ummetr'oc >esti m'hkei <h d`e DH t~h| EH;
>as'ummetroc >'ara >est`in <h AZ t~h| EH m'hkei. <wc d`e <h AZ
pr`oc t`hn EH, o<'utwc >est`i t`o AI pr`oc t`o EK;
>as'ummetron >'ara >est`i t`o AI t~w| EK.}

\gr{Sunest'atw o>~un t~w| m`en AI >'ison tetr'agwnon t`o LM, t~w| d`e
ZK >'ison >af~h|r'hsjw t`o NX per`i t`hn a>ut`hn gwn'ian >`on t~w|
LM; per`i t`hn a>ut`hn >'ara di'ametr'on >esti t`a LM, NX. >'estw
a>ut~wn di'ametroc <h OR, ka`i katagegr'afjw t`o sq~hma. >epe`i
o>~un t`o <up`o t~wn AZ, ZH >'ison >est`i t~w| >ap`o t~hc EH,
>'estin >'ara <wc <h AZ pr`oc t`hn EH, o<'utwc <h EH pr`oc t`hn
ZH. >all> <wc m`en <h AZ pr`oc t`hn EH, o<'utwc >est`i t`o AI
pr`oc t`o EK; <wc d`e <h EH pr`oc t`hn ZH, o<'utwc >est`i t`o EK pr`oc t`o
ZK; ka`i <wc >'ara t`o AI pr`oc t`o EK, o<'utwc t`o EK pr`oc t`o ZK;
t~wn >'ara AI, ZK m'eson >an'alog'on >esti t`o EK. >'esti d`e ka`i t~wn
LM, NX tetrag'wnwn m'eson >an'alogon t`o MN; ka'i >estin >'ison
t`o m`en AI t~w| LM, t`o d`e ZK t~w| NX; ka`i t`o EK >'ara  >'ison
>est`i t~w| MN. >all`a t`o m`en MN >'ison >est`i t~w| LX,
t`o d`e EK >'ison [>est`i] t~w| DJ; ka`i <'olon >'ara t`o
DK >'ison >est`i t~w| UFQ gn'wmoni ka`i t~w|
NX. >'esti d`e ka`i t`o AK >'ison to~ic LM, NX; loip`on >'ara t`o
AB >'ison >est`i t~w| ST, tout'esti t~w| >ap`o t~hc LN tetrag'wnw|;
<h LN >'ara d'unatai t`o AB qwr'ion. l'egw, <'oti <h LN m'eshc
>apotom'h >esti deut'era.}

\gr{>Epe`i g`ar m'esa >ede'iqjh t`a AI, ZK ka'i >estin >'isa to~ic >ap`o
t~wn LO, ON, m'eson >'ara ka`i <ek'ateron t~wn >ap`o t~wn LO, ON;
m'esh >'ara <ekat'era t~wn LO, ON. ka`i >epe`i s'ummetr'on >esti
t`o AI t~w| ZK, s'ummetron >'ara ka`i t`o >ap`o t~hc LO t~w| >ap`o
t~hc ON. p'alin, >epe`i >as'ummetron >ede'iqjh t`o AI t~w| EK,
>as'ummetron >'ara >est`i ka`i t`o LM t~w| MN, tout'esti t`o
>ap`o t~hc LO t~w| <up`o t~wn LO, ON; <'wste ka`i <h LO
>as'ummetr'oc >esti m'hkei t~h| ON; a<i LO, ON >'ara m'esai
e>is`i dun'amei m'onon s'ummetroi. l'egw d'h, <'oti ka`i m'eson peri'eqousin.}

\gr{>Epe`i g`ar m'eson >ede'iqjh t`o EK ka'i >estin >'ison t~w| <up`o t~wn
LO, ON, m'eson >'ara >est`i ka`i t`o <up`o t~wn LO, ON; <'wste a<i
LO, ON m'esai e>is`i dun'amei m'onon s'ummetroi m'eson peri'eqousai.
<h LN >'ara m'eshc >apotom'h >esti deut'era; ka`i d'unatai t`o AB
qwr'ion.}

\gr{<H >'ara t`o AB qwr'ion dunam'enh m'eshc >apotom'h >esti deut'era;
<'oper >'edei de~ixai.}}

\ParallelRText{
\begin{center}
{\large Proposition 93}
\end{center}

If an area is contained by a rational (straight-line)
and a third apotome then the square-root of the area is a second apotome
of a medial (straight-line).

\epsfysize=1.3in
\centerline{\epsffile{Book10/fig093e.eps}}

For let the area $AB$ have been contained by the rational (straight-line)
$AC$ and the third apotome $AD$. I say that the square-root of area $AB$
is the second apotome of a medial (straight-line).

For let $DG$ be an attachment to $AD$. Thus, $AG$ and $GD$ are rational (straight-lines which are) commensurable in square only [Prop. 10.73], and neither of $AG$ and $GD$
is commensurable in length with the (previously) laid down rational (straight-line) $AC$, and the square on the whole, $AG$, is greater than
(the square on) the attachment, $DG$, by the (square) on (some straight-line)
commensurable (in length) with ($AG$) [Def. 10.13]. 
Therefore, since the square on $AG$ is greater than (the square on)
$GD$ by the (square) on (some straight-line) commensurable (in length)
with ($AG$), thus if (an area) equal to the fourth part of the square on $DG$
is applied to $AG$, falling short by a square figure, then it divides ($AG$)
into (parts which are) commensurable (in length) [Prop. 10.17]. Therefore, let $DG$ have been
cut in half at $E$. And let (an area) equal to the (square) on $EG$
have been applied to $AG$, falling short by a square figure.
And let it be the (rectangle contained) by $AF$ and $FG$. And let
$EH$, $FI$, and $GK$ have been drawn through points $E$, $F$, and $G$
(respectively), parallel to $AC$. Thus, $AF$ and $FG$ are
commensurable (in length). $AI$ (is) thus also commensurable with
$FK$ [Props.~6.1, 10.11]. 
 And since $AF$ and $FG$ are commensurable
in length, $AG$ is thus also commensurable in length with each of
$AF$ and $FG$ [Prop. 10.15]. And
$AG$ (is) rational, and incommensurable in length with $AC$. 
Hence, $AF$ and $FG$ (are) also (rational, and incommensurable in length
with $AC$) [Prop. 10.13]. 
Thus, $AI$ and $FK$ are each medial (areas) [Prop. 10.21]. Again, since $DE$ is commensurable
in length with $EG$, $DG$ is also commensurable
in length with each of $DE$ and $EG$ [Prop. 10.15]. And $GD$ (is) rational, and incommensurable in length with $AC$. Thus, $DE$ and $EG$ (are)
each also rational, and incommensurable in length with $AC$ [Prop. 10.13]. $DH$ and $EK$ are thus
each medial (areas) [Prop. 10.21]. And since 
$AG$ and $GD$ are commensurable in square only, $AG$ is thus
incommensurable in length with $GD$. But, $AG$ is commensurable in
length with $AF$, and $DG$ with $EG$. Thus, $AF$ is incommensurable
in length with $EG$ [Prop. 10.13]. 
And as $AF$ (is) to $EG$, so $AI$ is to $EK$ [Prop. 6.1]. Thus, $AI$ is incommensurable with $EK$ [Prop. 10.11].

Therefore, let the square $LM$, equal to $AI$, have been constructed. And
let $NO$, equal to $FK$, which is about the same angle as $LM$, have
been subtracted (from $LM$). Thus, $LM$ and $NO$ are about the same
diagonal [Prop. 6.26]. Let $PR$ be their (common) diagonal,
and let the (rest of the) figure have been drawn. Therefore, since
the (rectangle contained) by $AF$ and $FG$ is equal to the (square) on
$EG$, thus as $AF$ is to $EG$, so $EG$ (is) to $FG$
[Prop. 6.17]. But, as $AF$ (is) to $EG$, so
$AI$ is to $EK$ [Prop. 6.1]. And as $EG$ (is)
to $FG$, so $EK$ is to $FK$ [Prop. 6.1]. And thus
as $AI$ (is) to $EK$, so $EK$ (is) to $FK$ [Prop. 5.11]. Thus, $EK$
is the mean proportional to $AI$ and $FK$. And $MN$ is also the
mean proportional to the squares $LM$ and $NO$ [Prop. 10.53~lem.]. And $AI$
is equal to $LM$, and $FK$ to $NO$. Thus, $EK$ is also equal to $MN$.
But, $MN$ is equal to $LO$, and $EK$ [is] equal to $DH$ [Prop. 1.43]. And thus the whole
of $DK$ is equal to the gnomon $UVW$ and $NO$. And $AK$ (is) also
equal to $LM$ and $NO$. Thus, the remainder $AB$ is equal to $ST$---that 
is to say, to the square on $LN$. Thus, $LN$ is the square-root of area $AB$.
I say that $LN$ is the second apotome of a medial (straight-line).

For since $AI$ and $FK$ were shown (to be)
medial (areas), and are equal to the (squares) on $LP$ and $PN$ (respectively), the (squares) on each of $LP$ and $PN$ (are) thus also
medial. Thus, $LP$ and $PN$ (are) each medial (straight-lines).
And since $AI$ is commensurable with $FK$ [Props.~6.1, 10.11], the (square) on $LP$
(is) thus also commensurable with the (square) on $PN$.  Again, since
$AI$ was shown (to be) incommensurable with $EK$, $LM$ is thus
also incommensurable with $MN$---that is to say, the (square) on 
$LP$ with the (rectangle contained) by $LP$ and $PN$.
Hence, $LP$ is also incommensurable in length with $PN$ [Props.~6.1, 10.11]. Thus, $LP$ and $PN$
are medial (straight-lines which are) commensurable in square only.
So, I say that they also contain a medial (area).

For since $EK$ was shown (to be) a medial (area), and is equal to the
(rectangle contained) by $LP$ and $PN$, the (rectangle contained) by
$LP$ and $PN$ is thus also medial. Hence, $LP$ and $PN$
are medial (straight-lines which are) commensurable in square only, and
which contain a medial (area). Thus, $LN$ is the
second apotome of a medial (straight-line) [Prop. 10.75]. And it is the square-root of area
$AB$.

Thus, the square-root of area $AB$ is the second apotome of a medial (straight-line). (Which is) the very thing it was required to show.}
\end{Parallel}

%%%%
%10.94
%%%%
\pdfbookmark[1]{Proposition 10.94}{pdf10.94}
\begin{Parallel}{}{}
\ParallelLText{
\begin{center}
{\large \ggn{94}.}
\end{center}\vspace*{-7pt}

\gr{>E`an qwr'ion peri'eqhtai <up`o <rht~hc ka`i >apotom~hc tet'arthc,
<h t`o qwr'ion dunam'enh >el'asswn >est'in.}\\

\epsfysize=1.3in
\centerline{\epsffile{Book10/fig093g.eps}}

\gr{Qwr'ion g`ar t`o AB perieq'esjw <up`o <rht~hc t~hc AG ka`i >apotom~hc 
tet'arthc t~hc AD; l'egw, <'oti <h t`o AB qwr'ion dunam'enh >el'asswn
>est'in.}

\gr{>'Estw g`ar t~h| AD prosarm'ozousa <h DH; a<i >'ara AH, HD <rhta'i
e>isi dun'amei m'onon s'ummetroi, ka`i <h AH s'ummetr'oc >esti
t~h| >ekkeim'enh| <rht~h| t~h| AG m'hkei, <h d`e <'olh <h AH t~hc
prosarmozo'ushc t~hc DH me~izon d'unatai t~w| >ap`o >asumm'etrou
<eaut~h| m'hkei. >epe`i o>~un <h AH t~hc HD me~izon d'unatai
t~w| >ap`o >asumm'etrou <eaut~h| m'hkei, >e`an >'ara t~w| tet'artw|
m'erei to~u >ap`o t~hc DH >'ison par`a t`hn AH parablhj~h|
>elle~ipon e>'idei tetrag'wnw|, e>ic >as'ummetra a>ut`hn diele~i.
tetm'hsjw o>~un <h DH d'iqa kat`a t`o E, ka`i t~w| >ap`o t~hc
EH >'ison par`a t`hn AH parabebl'hsjw >elle~ipon e>'idei
tetrag'wnw|, ka`i >'estw t`o <up`o t~wn AZ, ZH; >as'ummetroc
>'ara >est`i m'hkei <h AZ t~h| ZH. >'hqjwsan o>~un di`a t~wn E, Z, H
par'allhloi ta~ic AG, BD a<i EJ, ZI, HK. >epe`i o>~un <rht'h >estin
<h AH ka`i s'ummetroc t~h| AG m'hkei, <rht`on >'ara >est`in <'olon
t`o AK.
 p'alin, >epe`i >as'ummetr'oc >estin <h DH t~h| AG m'hkei, ka'i
 e>isin >amf'oterai <rhta'i, m'eson
 >'ara >est`i t`o DK. p'alin, >epe`i >as'ummetr'oc >est`in <h
 AZ t~h| ZH m'hkei,
>as'ummetron >'ara ka`i t`o AI t~w| ZK.}

\gr{Sunest'atw o>~un t~w| m`en
AI >'ison tetr'agwnon t`o LM, t~w| d`e ZK >'ison >afh|r'hsjw
per`i t`hn a>ut`hn gwn'ian t`hn <up`o t~wn LOM t`o NX. per`i
t`hn a>ut`hn >'ara di'amet'on >esti t`a LM, NX tetr'agwna. >'estw
a>ut~wn di'ametoc <h OR, ka`i katagegr'afjw t`o sq~hma.
>epe`i o>~un t`o <up`o t~wn AZ, ZH >'ison >est`i t~w| >ap`o t~hc
EH, >an'alogon >'ara >est`in <wc <h AZ pr`oc t`hn EH, o<'utwc
<h EH pr`oc t`hn ZH. >all> <wc m`en <h AZ pr`oc t`hn EH, o<'utwc
>est`i t`o AI pr`oc t`o EK, <wc d`e <h EH pr`oc t`hn ZH, o<'utwc
>est`i t`o EK pr`oc t`o ZK; t~wn >'ara AI, ZK m'eson >an'alog'on
>esti t`o EK. >'esti d`e ka`i t~wn LM, NX tetrag'wnwn m'eson >an'alogon
t`o MN, ka'i >estin >'ison t`o m`en AI t~w| LM, t`o d`e ZK t~w| NX;
ka`i t`o EK >'ara >'ison >est`i t~w| MN. >all`a t~w| m`en EK >'ison
>est`i t`o DJ, t~w| d`e MN >'ison >est`i  t`o LX; <'olon >'ara t`o DK >'ison >est`i
t~w| UFQ gn'wmoni ka`i t~w| NX. >epe`i o>un <'olon t`o AK >'ison >est`i
to~ic LM, NX tetrag'wnoic, <~wn t`o DK >'ison >est`i  t~w| UFQ gn'wmoni ka`i
t~w| NX tetrag'wnw|, loip`on >'ara
t`o AB >'ison >est`i t~w| ST, tout'esti t~w| >ap`o t~hc LN tetrag'wnw|;
<h LN >'ara d'unatai t`o AB qwr'ion. l'egw, <'oti <h LN >'alog'oc
>estin <h kaloum'enh >el'asswn.}

\gr{>Epe`i g`ar <rht'on >esti t`o AK ka'i >estin >'ison to~ic >ap`o t~wn
LO,  ON tetr'agwnoic, t`o >'ara sugke'imenon >ek t~wn >ap`o
t~wn LO, ON <rht'on >estin. p'alin, >epe`i t`o DK m'eson >est'in,
ka'i >estin >'ison t`o DK t~w| d`ic <up`o t~wn LO, ON, t`o
>'ara d`ic <up`o t~wn LO, ON m'eson >est'in. ka`i >epe`i >as'ummetron
>ede'iqjh t`o AI t~w| ZK, >as'ummetron >'ara ka`i t`o >ap`o
t~hc LO tetr'agwnon t~w| >ap`o t~hc ON tetrag'wnw|. a<i LO, ON
>'ara dun'amei e>is`in >as'ummetroi poio~usai t`o m`en
sugke'imenon >ek t~wn >ap> a>ut~wn tetrag'wnwn <rht'on, t`o d`e d`ic
<up> a>ut~wn m'eson. <h LN >'ara >'alog'oc >estin <h kaloum'enh
>el'asswn; ka`i d'unatai t`o AB qwr'ion.}

\gr{<H >'ara t`o AB qwr'ion dunam'enh >el'asswn >est'in; <'oper
>'edei de~ixai.}}

\ParallelRText{
\begin{center}
{\large Proposition 94}
\end{center}

If an area is contained by a rational (straight-line)
and a fourth apotome then the square-root of the area is a minor (straight-line).

\epsfysize=1.3in
\centerline{\epsffile{Book10/fig093e.eps}}

For let the area $AB$ have been contained by the rational (straight-line)
$AC$ and the fourth apotome $AD$. I say that the square-root of area $AB$
is a minor (straight-line).
For let $DG$ be an attachment to $AD$. Thus, $AG$ and $DG$ are rational (straight-lines which are) commensurable in square only [Prop. 10.73], and $AG$ is commensurable
in length with the (previously) laid down rational (straight-line) $AC$, and
the square on the whole, $AG$, is greater than (the square on) the attachment,
$DG$, by the square on (some straight-line) incommensurable
in length with ($AG$) [Def. 10.14]. Therefore,
since the square on $AG$ is greater than (the square on)  $GD$ by the
(square) on (some straight-line) incommensurable in length with ($AG$),
thus if (some area), equal to the fourth part of the (square) on $DG$,
is applied to $AG$, falling short by a square figure, then it divides
($AG$) into (parts which are) incommensurable (in length) 
[Prop. 10.18]. Therefore, let $DG$ have been
cut in half at $E$, and let (some area), equal to the (square) on
$EG$, have been applied to $AG$, falling short by a square figure, and
let it be the (rectangle contained) by $AF$ and $FG$. Thus, $AF$
is incommensurable in length with $FG$. Therefore, let $EH$, $FI$, and $GK$
have been drawn through $E$, $F$, and $G$ (respectively), parallel
to $AC$ and $BD$. Therefore, since $AG$ is rational, and
commensurable in length with $AC$, the whole (area) $AK$
is thus rational [Prop. 10.19]. Again, since $DG$ is incommensurable in length with
$AC$, and both are rational (straight-lines), $DK$ is thus
a medial (area) [Prop. 10.21]. Again, since
$AF$ is incommensurable in length with $FG$, $AI$ (is) thus also
incommensurable with $FK$ [Props.~6.1, 10.11].

Therefore, let the square $LM$, equal  to $AI$, have been constructed. And let $NO$, equal to $FK$, (and) about the same angle, $LPM$,  have been subtracted (from $LM$). Thus, the squares $LM$ and $NO$ are about the
same diagonal [Prop. 6.26]. Let $PR$ be their (common) diagonal, and let the (rest of the) figure have been drawn. Therefore, since
the (rectangle contained) by $AF$ and $FG$ is equal to the (square) on $EG$, thus, proportionally, as $AF$ is to $EG$, so $EG$ (is) to $FG$
[Prop. 6.17]. 
But, as $AF$ (is) to $EG$, so
$AI$ is to $EK$, and as $EG$ (is) to $FG$, so $EK$ is to $FK$ [Prop. 6.1]. Thus, $EK$ is the mean proportional to
$AI$ and $FK$ [Prop. 5.11]. And $MN$ is also the mean proportional to the squares $LM$
and $NO$ [Prop. 10.13~lem.], and $AI$ is equal to $LM$, and $FK$ to $NO$. $EK$ is thus also equal to $MN$. But, $DH$ is equal to $EK$, and $LO$ is equal to $MN$ [Prop. 1.43]. 
Thus, the whole of $DK$ is equal to the gnomon $UVW$ and $NO$.
Therefore, since the whole of $AK$ is equal to the (sum of the) squares $LM$ and
$NO$, of which $DK$ is equal to the gnomon $UVW$ and the square
$NO$, the remainder $AB$ is thus equal to $ST$---that is to say, to the
square on $LN$. Thus, $LN$ is the square-root of area $AB$. I say
that $LN$ is the irrational (straight-line which is) called minor.

For since $AK$ is rational, and is equal to the (sum of the) squares $LP$ and $PN$,
the sum of the (squares) on $LP$ and $PN$ is thus rational. Again, since
$DK$ is medial, and $DK$ is equal to twice the (rectangle contained)
by $LP$ and $PN$, thus twice the (rectangle contained) by $LP$
and $PN$ is medial. And since $AI$ was shown (to be) incommensurable
with $FK$, the square on $LP$ (is) thus also incommensurable with
the square on $PN$. Thus, $LP$ and $PN$ are (straight-lines which are) incommensurable in square,
making the sum of the squares on them rational, and twice the (rectangle
contained) by them medial. $LN$ is thus the irrational (straight-line)
called minor [Prop. 10.76]. And it is
the square-root of area $AB$.

Thus, the square-root of area $AB$ is a minor (straight-line). (Which is)
the very thing it was required to show.}
\end{Parallel}

%%%%
%10.95
%%%%
\pdfbookmark[1]{Proposition 10.95}{pdf10.95}
\begin{Parallel}{}{}
\ParallelLText{
\begin{center}
{\large \ggn{95}.}
\end{center}\vspace*{-7pt}

\gr{>E`an qwr'ion peri'eqhtai <up`o <rht~hc ka`i >apotom~hc p'empthc,
<h t`o qwr'ion dunam'enh [<h] met`a <rhto~u m'eson t`o <'olon
poio~us'a >estin.}\\

\epsfysize=1.3in
\centerline{\epsffile{Book10/fig093g.eps}}

\gr{Qwr'ion g`ar t`o AB perieq'esjw <up`o <rht~hc t~hc AG ka`i >apotom~hc
p'empthc t~hc AD; l'egw, <'oti <h t`o AB qwr'ion dunam'enh [<h]
met`a <rhto~u m'eson t`o <'olon poio~us'a >estin.}

\gr{>'Estw g`ar t~h| AD prosarm'ozousa <h DH; a<i >'ara AH, HD <rhta'i
e>isi dun'amei m'onon s'ummetroi, ka`i <h prosarm'ozousa <h HD
s'ummetr'oc >esti m'hkei t~h| >ekkeim'enh| <rht~h| t~h| AG,
<h d`e <'olh <h AH t~hc prosarmozo'ushc t~hc DH me~izon d'unatai
t~w| >ap`o >asumm'etrou <eaut~h|. >e`an >'ara t~w| tet'artw| m'erei
to~u >ap`o t~hc DH >'ison par`a t`hn AH parablhj~h| >elle~ipon
e>'idei tetrag'wnw|, e>ic >as'ummetra a>ut`hn diele~i. tetm'hsjw
o>~un <h DH d'iqa kat`a t`o E shme~ion, ka`i t~w| >ap`o t~hc
EH >'ison par`a t`hn AH parabebl'hsjw >elle~ipon e>'idei
tetrag'wnw| ka`i >'estw t`o <up`o t~wn AZ, ZH; >as'ummetroc
>'ara >est`in <h AZ t~h| ZH m'hkei. ka`i >epe`i >as'ummetr'oc
 >est`in <h AH t~h| GA m'hkei, ka'i e>isin >amf'oterai
<rhta'i, m'eson >'ara >est`i t`o AK. p'alin, >epe`i <rht'h
>estin <h DH ka`i s'ummetroc t~h| AG m'hkei, <rht'on >esti t`o DK.}

\gr{Sunest'atw o>~un t~w| m`en AI >'ison tetr'agwnon t`o LM,
t~w| d`e ZK >'ison tetr'agwnon >afh|r'hsjw t`o NX per`i
t`hn a>ut`hn gwn'ian t`hn <up`o LOM; per`i t`hn a>ut`hn >'ara
di'ametr'on >esti t`a LM, NX tetr'agwna. >'estw a>ut~wn di'ametroc
<h OR, ka`i katagegr'afjw t`o sq~hma. <omo'iwc
d`h de'ixomen, <'oti <h LN d'unatai t`o AB qwr'ion. l'egw, <'oti
<h LN <h met`a <rhto~u m'eson t`o <'olon poio~us'a >estin.}

\gr{>Epe`i g`ar m'eson >ede'iqjh t`o AK ka'i >estin >'ison to~ic >ap`o
t~wn LO, ON, t`o >'ara sugke'imenon >ek t~wn >ap`o
t~wn LO, ON m'eson >est'in. p'alin, >epe`i <rht'on >esti t`o DK ka'i
>estin >'ison t~w| d`ic <up`o t~wn LO, ON, ka`i a<ut`o <rht'on
>estin. ka`i >epe`i >as'ummetr'on >esti t`o AI t~w| ZK, >as'ummetron
>'ara >est`i ka`i t`o >ap`o t~hc LO t~w| >ap`o t~hc ON; a<i
LO, ON >'ara dun'amei e>is`in >as'ummetroi poio~usai t`o  m`en sugke'imenon >ek t~wn >ap> a>ut~wn tetrag'wnwn m'eson, t`o d`e
d`ic <up> a>ut~wn <rht'on. <h loip`h >'ara <h LN >'alog'oc >estin <h
kaloum'enh met`a <rhto~u m'eson t`o <'olon poio~usa; ka`i d'unatai
t`o AB qwr'ion.}

\gr{<H t`o AB  >'ara qwr'ion dunam'enh met`a <rhto~u m'eson t`o <'olon
poio~us'a >estin; <'oper >'edei de~ixai.}}

\ParallelRText{
\begin{center}
{\large Proposition 95}
\end{center}

If an area is contained by a rational (straight-line)
and a fifth apotome then the square-root of the area is that (straight-line) which with a rational (area) makes a medial whole.

\epsfysize=1.3in
\centerline{\epsffile{Book10/fig093e.eps}}

For let the area $AB$ have been contained by the rational (straight-line)
$AC$ and the fifth apotome $AD$. I say that the square-root of area $AB$
is  that (straight-line) which with a rational (area) makes a medial whole.

For let $DG$ be an attachment to $AD$. Thus, $AG$ and $DG$ are rational (straight-lines which are) commensurable in square only [Prop. 10.73], and the attachment $GD$ is
commensurable in length the the (previously) laid down rational (straight-line) $AC$, and the square on the whole, $AG$, is greater than (the square on)
the attachment, $DG$, by the (square) on (some straight-line)
incommensurable (in length) with ($AG$) [Def. 10.15]. 
Thus, if (some area), equal to the fourth part of the (square) on $DG$,
is applied to $AG$, falling short by a square figure, then it divides ($AG$)
into (parts which are) incommensurable (in length) [Prop. 10.18]. Therefore, let $DG$
have been divided in half at point $E$, and let (some area), equal to the
(square) on $EG$, have been applied to $AG$, falling short by a
square figure, and let it be the (rectangle contained) by $AF$ and $FG$.
Thus, $AF$ is incommensurable in length with $FG$. And since
$AG$ is incommensurable in length with $CA$, and both are
rational (straight-lines), $AK$ is thus a medial (area) [Prop. 10.21]. Again, since $DG$ is rational,
and commensurable in length with $AC$, $DK$ is a rational (area)
[Prop. 10.19].

Therefore, let the square $LM$, equal to $AI$, have been constructed. 
And let the square $NO$, equal to $FK$, (and) about the same angle, $LPM$,
have been subtracted (from $NO$). Thus, the squares $LM$ and
$NO$ are about the same diagonal [Prop. 6.26]. 
Let $PR$ be their (common) diagonal, and let (the rest of) the figure have been drawn.
So, similarly (to the previous propositions), we can show that $LN$ is the square-root of area $AB$.
I say that $LN$ is that (straight-line) which with a rational (area) makes a medial whole.

For since $AK$ was shown (to be) a medial (area), and is equal to (the sum of) the squares on $LP$ and $PN$, the sum of the (squares) on $LP$
and $PN$ is thus medial. Again, since $DK$ is rational, and
is equal to twice the (rectangle contained) by $LP$ and $PN$,  (the latter) is also
rational. And since $AI$ is incommensurable with $FK$, the (square) on
$LP$ is thus also incommensurable with the (square) on $PN$. Thus,
$LP$ and $PN$ are (straight-lines which are) incommensurable in square, making the
sum of the squares on them medial, and twice the (rectangle contained) by
them rational. Thus, the remainder $LN$ is the irrational (straight-line)
called that which with a rational (area) makes a medial whole [Prop. 10.77].
And it is the square-root of area $AB$.

Thus, the square-root of area $AB$ is that (straight-line) which with a rational (area) makes a medial whole. (Which is) the very thing it was required to show.}
\end{Parallel}

%%%%
%10.96
%%%%
\pdfbookmark[1]{Proposition 10.96}{pdf10.96}
\begin{Parallel}{}{}
\ParallelLText{
\begin{center}
{\large \ggn{96}.}
\end{center}\vspace*{-7pt}

\gr{>E`an qwr'ion peri'eqhtai <up`o <rht~hc ka`i >apotom~hc <'ekthc, <h t`o
qwr'ion dunam'enh met`a m'esou m'eson t`o <'olon poio~us'a >estin.}\\

\epsfysize=1.3in
\centerline{\epsffile{Book10/fig093g.eps}}

\gr{Qwr'ion g`ar t`o AB perieq'esjw <up`o <rht~hc t~hc AG ka`i >apotom~hc 
<'ekthc t~hc AD; l'egw, <'oti <h t`o AB qwr'ion dunam'enh [<h]
met`a m'esou m'eson t`o <'olon poio~us'a >estin.}

\gr{>'Estw g`ar  t~h| AD prosarm'ozousa <h DH; a<i >'ara AH, HD <rhta'i
e>isi dun'amei m'onon s'ummetroi, ka`i o>udet'era a>ut~wn s'ummetr'oc
>esti t~h| >ekkeim'enh| <rht~h| t~h| AG m'hkei, <h d`e <'olh <h AH
t~hc prosarmozo'ushc t~hc DH me~izon d'unatai t~w| >ap`o >asumm'etrou
<eaut~h| m'hkei. >epe`i o>~un <h AH t~hc HD me~izon d'unatai t~w| >ap`o
<asumm'etrou >eaut~h| m'hkei, <e`an >'ara t~w| tet'artw| m'erei
to~u >ap`o t~hc DH >'ison par`a t`hn AH parablhj~h| >elle~ipon
e>'idei tetrag'wnw|, e>ic >as'ummetra a>ut`hn diele~i. tetm'hsjw
o>~un <h DH d'iqa kat`a t`o E [shme~ion], ka`i t~w| >ap`o t~hc
EH >'ison par`a t`hn AH parabebl'hsjw >elle~ipon e>'idei tetrag'wnw|,
ka`i >'estw t`o <up`o t~wn AZ, ZH; >as'ummetroc >'ara >est`in
<h AZ t~h| ZH m'hkei. <wc d`e <h AZ pr`oc t`hn ZH, o<'utwc >est`i t`o AI pr`oc t`o ZK; >as'ummetron
>'ara >est`i t`o AI t~w| ZK. ka`i >epe`i a<i AH, AG <rhta'i
e>isi dun'amei m'onon s'ummetroi, m'eson >est`i t`o AK. p'alin, >epe`i a<i AG, DH <rhta'i e>isi ka`i >as'ummetroi m'hkei, m'eson >est`i ka`i t`o
DK. >epe`i o>~un a<i AH, HD dun'amei m'onon s'ummetro'i e>isin,
>as'ummetroc >'ara >est`in <h AH t~h| HD m'hkei. <wc d`e <h AH pr`oc t`hn HD, o<'utwc >est`i t`o AK pr`oc t`o KD; >as'ummetron >'ara >est`i
t`o AK t~w| KD.}

\gr{Sunest'atw o>~un t~w| m`en AI >'ison tetr'agwnon t`o LM, t~w| d`e
ZK >'ison >afh|r'hsjw per`i t`hn a>ut`hn gwn'ian t`o NX; per`i t`hn
a>ut`hn >'ara di'ametr'on >esti t`a LM, NX tetr'agwna. >'estw
a>ut~wn di'ametroc <h OR, ka`i katagegr'afjw t`o sq~hma.
<omo'iwc d`h to~ic >ep'anw de'ixomen, <'oti <h LN d'unatai t`o AB
qwr'ion. l'egw, <'oti <h LN [<h] met`a m'esou m'eson t`o <'olon poio~us'a
>estin.}

\gr{>Epe`i g`ar m'eson >ede'iqjh t`o AK ka'i >estin >'ison to~ic >ap`o t~wn
LO, ON, t`o >'ara sugke'imenon >ek t~wn >ap`o t~wn LO, ON
m'eson >est'in. p'alin, >epe`i m'eson >ede'iqjh t`o DK ka'i >estin  >'ison
t~w| d`ic <up`o  t~wn LO, ON, ka`i t`o d`ic <up`o t~wn LO, ON
m'eson >est'in. ka`i >epe`i >as'ummetron >ede'iqjh t`o AK t~w| DK, >as'ummetra [>'ara] >est`i ka`i t`a >ap`o t~wn LO, ON tetr'agwna
t~w| d`ic <up`o t~wn LO, ON. ka`i >epe`i >as'ummetr'on >esti t`o AI t~w|
ZK, >as'ummetron >'ara ka`i t`o >ap`o t~hc LO t~w|
>ap`o t~hc ON; a<i LO, ON >'ara dun'amei e>is`in
>as'ummetroi poio~usai t'o te sugke'imenon >ek t~wn >ap>
a>ut~wn tetrag'wnwn m'eson ka`i t`o d`ic <up> a>ut~wn
m'eson >'eti te t`a >ap> a>ut~wn tetr'agwna >as'ummetra t~w| d`ic <up>
a>ut~wn. <h >'ara LN >'alog'oc >estin <h kaloum'emh met`a m'esou
m'eson t`o <'olon poio~usa; ka`i d'unatai t`o AB qwr'ion.}

\gr{<H >'ara t`o qwr'ion dunam'enh met`a m'esou m'eson t`o <'olon
poio~us'a >estin; <'oper >'edei de~ixai.}}

\ParallelRText{
\begin{center}
{\large Proposition 96}
\end{center}

If an area is contained by a rational (straight-line)
and a sixth apotome then the square-root of the area is that (straight-line) which with a medial (area) makes a medial whole.

\epsfysize=1.3in
\centerline{\epsffile{Book10/fig093e.eps}}

For let the area $AB$ have been contained by the rational (straight-line)
$AC$ and the sixth apotome $AD$. I say that the square-root of area $AB$
is  that (straight-line) which with a medial (area) makes a medial whole.

For let $DG$ be an attachment to $AD$. Thus, $AG$ and $GD$ are rational (straight-lines which are) commensurable in square only [Prop. 10.73], and neither of them is
commensurable in length with the (previously) laid down rational (straight-line) $AC$, and the square on the whole, $AG$, is greater than (the square on)
the attachment, $DG$, by the (square) on (some straight-line)
incommensurable in length with ($AG$) [Def. 10.16]. Therefore, since the square on $AG$
is greater than (the square on) $GD$ by the (square) on (some straight-line)
incommensurable in length with ($AG$), thus if (some area), equal to
the fourth part of square on $DG$, is applied to $AG$, falling short
by a square figure, then it divides ($AG$) into (parts which are) incommensurable
(in length) [Prop. 10.18]. Therefore, let $DG$ have been cut in  half at [point] $E$. And let (some area), equal to the
(square) on $EG$, have been applied to $AG$, falling short by a square figure.
And let it be the (rectangle contained) by $AF$ and $FG$. $AF$ is
thus incommensurable in length with $FG$. And as $AF$ (is) to $FG$,
so $AI$ is to $FK$ [Prop. 6.1].  Thus, $AI$
is incommensurable with $FK$ [Prop. 10.11]. 
And since $AG$ and $AC$ are rational (straight-lines which are)
commensurable in square only, $AK$ is a medial (area) [Prop. 10.21]. Again, since $AC$ and $DG$
are rational (straight-lines which are) incommensurable in length, $DK$
is also a medial (area) [Prop. 10.21]. Therefore,
since $AG$ and $GD$ are commensurable in square only,  $AG$
is thus incommensurable in length with $GD$. And as $AG$ (is) to $GD$,
so $AK$ is to $KD$ [Prop. 6.1]. Thus, $AK$
is incommensurable with $KD$ [Prop. 10.11].

Therefore, let the square $LM$, equal to $AI$, have been constructed.
And let $NO$, equal to $FK$, (and) about the same angle, have been subtracted (from $LM$). Thus, the squares $LM$ and $NO$ are about the
same diagonal [Prop. 6.26]. 
Let $PR$ be their (common) diagonal, and let (the rest of) the figure have been drawn.
So, similarly to the above, we can show that $LN$ is the square-root of
area $AB$. I say that $LN$ is  that (straight-line) which with a medial (area) makes a medial whole.

For since $AK$ was shown (to be) a medial (area), and is equal to the
(sum of the) squares on $LP$ and $PN$, the sum of the (squares) on
$LP$ and $PN$ is medial. Again, since $DK$ was shown (to
be) a medial (area), and is equal to twice the (rectangle contained)
by $LP$ and $PN$, twice the (rectangle contained) by $LP$ and
$PN$ is also medial. And since $AK$ was shown (to be) incommensurable
with $DK$, [thus] the (sum of the) squares on $LP$ and $PN$ is also
incommensurable with twice the (rectangle contained) by $LP$ and $PN$.
And since $AI$ is incommensurable with $FK$, the (square) on
$LP$ (is) thus also incommensurable with the (square) on $PN$. Thus, $LP$ and $PN$
are (straight-lines which are) incommensurable in square, making the sum of the squares on them
medial, and twice the (rectangle contained) by  medial, and, furthermore,
the (sum of the) squares on them incommensurable with twice the
(rectangle contained) by them. Thus, $LN$ is the irrational (straight-line) called  that  which with a medial (area) makes a medial whole [Prop. 10.78]. And it is the square-root of area $AB$.

Thus, the square-root of area ($AB$) is that (straight-line) which with a medial (area) makes a medial whole. (Which is) the very thing it was required to show.}
\end{Parallel}

%%%%
%10.97
%%%%
\pdfbookmark[1]{Proposition 10.97}{pdf10.97}
\begin{Parallel}{}{}
\ParallelLText{
\begin{center}
{\large \ggn{97}.}
\end{center}\vspace*{-7pt}

\gr{T`o >ap`o >apotom~hc par`a <rht`hn paraball'omenon pl'atoc poie~i
>apotom`hn pr'wthn.}

\epsfysize=1.6in
\centerline{\epsffile{Book10/fig097g.eps}}

\gr{>'Estw >apotom`h `h AB, <rht`h d`e <h GD, ka`i t~w| >ap`o t~hc AB >'ison 
par`a t`hn GD parabebl'hsjw t`o GE pl'atoc poio~un t`hn GZ;
l'egw, <'oti <h GZ >apotom'h >esti pr'wth.}

\gr{>'Estw g`ar t~h| AB prosarm'ozousa <h BH; a<i >'ara AH, HB <rhta'i
e>isi dun'amei m'onon s'ummetroi. ka`i t~w| m`en >ap`o t~hc AH >'ison
par`a t`hn GD parabebl'hsjw t`o GJ, t~w| d`e >ap`o t~hc BH t`o KL. <'olon
>'ara t`o GL >'ison >est`i to~ic >ap`o t~wn AH, HB; <~wn t`o GE >'ison
>est`i t~w| >ap`o t~hc AB; loip`on >'ara t`o ZL >'ison >est`i t~w| d`ic
<up`o t~wn AH, HB. tetm'hsjw <h ZM d'iqa kat`a t`o N shme~ion,
ka`i >'hqjw di`a to~u N t~h| GD par'allhloc <h NX; <ek'ateron >'ara
t~wn ZX, LN >'ison >est`i t~w| <up`o t~wn AH, HB. ka`i >epe`i t`a  >ap`o
t~wn AH, HB <rht'a >estin, ka'i >esti to~ic >ap`o t~wn AH, HB >'ison t`o
DM, <rht`on >'ara >est`i t`o DM. ka`i par`a <rht`hn t`hn GD parab'eblhtai
pl'atoc poio~un t`hn GM; <rht`h >'ara >est`in <h GM ka`i s'ummetroc
t~h| GD m'hkei. p'alin, >epe`i m'eson >est`i t`o d`ic <up`o t~wn AH, HB,
ka`i t~w| d`ic <up`o t~wn AH, HB >'ison t`o ZL, m'eson >'ara t`o
ZL. ka`i par`a <rht`hn t`hn GD par'akeitai pl'atoc poio~un t`hn ZM;
<rht`h >'ara >est`in <h ZM ka`i >as'ummetroc t~h| GD m'hkei.
ka`i >epe`i t`a m`en >ap`o t~wn AH, HB <rht'a >estin, t`o d`e d`ic <up`o
t~wn AH, HB m'eson, >as'ummetra >'ara >est`i t`a >ap`o t~wn AH, HB
t~w|  d`ic <up`o t~wn AH, HB. ka`i to~ic m`en >ap`o t~wn AH, HB
>'ison >est`i t`o GL, t~w| d`e d`ic <up`o t~wn AH, HB t`o ZL; >as'ummetron
>'ara >est`i t`o DM t~w| ZL. <wc d`e t`o DM pr`oc t`o ZL,
o<'utwc >est`in <h GM pr`oc t`hn ZM. >as'ummetroc >'ara >est`in
<h GM
 t~h| ZM m'hkei. ka'i e>isin >amf'oterai
<rhta'i; a<i >'ara GM, MZ <rhta'i e>isi dun'amei m'onon s'ummetroi;
<h GZ >'ara >apotom'h >estin. l'egw d'h, <'oti ka`i pr'wth.}

\gr{>Epe`i g`ar t~wn >ap`o t~wn AH, HB m'eson >an'alog'on >esti t`o
<up`o t~wn AH, HB, ka'i >esti t~w| m`en >ap`o t~hc AH >'ison
t`o GJ, t~w| d`e >ap`o t~hc BH >'ison t`o KL, t~w| d`e >up`o
t~wn AH, HB t`o NL, ka`i t~wn GJ, KL >'ara m'eson >an'alog'on
>esti t`o NL; >'estin >'ara <wc t`o GJ pr`oc t`o NL, o<'utwc
t`o NL pr`oc t`o KL. >all> <wc m`en t`o GJ pr`oc t`o NL, o<'utwc
>est`in <h GK pr`oc t`hn NM; <wc d`e t`o NL pr`oc t`o KL,
o<'utwc >est`in  <h NM pr`oc t`hn KM; t`o >'ara <up`o t~wn GK, KM
>'ison >est`i t~w| >ap`o t~hc NM, tout'esti t~w| tet'artw| m'erei
to~u >ap`o t~hc ZM. ka`i epe`i s'ummetr'on >esti t`o >ap`o t~hc
AH t~w| >ap`o t~hc HB, s'ummetr'on [>esti] ka`i t`o GJ t~w| KL.
<wc d`e t`o GJ pr`oc t`o KL, o<'utwc <h GK
pr`oc t`hn KM; s'ummetroc >'ara >est`in <h GK t~h| KM.
>epe`i o>~un d'uo e>uje~iai >'aniso'i e>isin a<i GM, MZ, ka`i
t~w| tet'artw| m'erei to~u >ap`o t~hc ZM >'ison par`a t`hn GM
parab'eblhtai >elle~ipon e>'idei tetrag'wnw| t`o <up`o t~wn
GK, KM, ka'i >esti s'ummetroc <h GK t~h| KM, <h >'ara GM t~hc
MZ me~izon d'unatai t~w| >ap`o summ'etrou <eaut~h| m'hkei.
ka'i >estin <h GM s'ummetroc t~h| >ekkeim'enh| <rht~h| t~h| GD
m'hkei; <h >'ara GZ >apotom'h >esti pr'wth.}

\gr{T`o >'ara >ap`o >apotom~hc par`a <rht`hn paraball'omenon
pl'atoc poie~i >apotom`hn pr'wthn; <'oper >'edei de~ixai.}}

\ParallelRText{
\begin{center}
{\large Proposition 97}
\end{center}

The (square) on an apotome, applied
to a rational (straight-line), produces  a first apotome as breadth.

\epsfysize=1.6in
\centerline{\epsffile{Book10/fig097e.eps}}

Let $AB$ be an apotome, and $CD$ a rational (straight-line). And
let $CE$, equal to the (square) on $AB$, have been applied to
$CD$, producing $CF$ as breadth. I say that $CF$ is a first apotome.

For let $BG$ be an attachment to $AB$. Thus, $AG$ and $GB$
are rational (straight-lines which are) commensurable in square only
[Prop. 10.73]. And let $CH$, equal to the
(square) on $AG$, and $KL$, (equal) to the (square) on $BG$, have been applied to $CD$. Thus, the whole of $CL$ is equal to the (sum of the
squares) on $AG$ and $GB$, of which $CE$ is equal to the (square) on
$AB$. The remainder $FL$ is thus equal to twice the (rectangle contained)
by $AG$ and $GB$ [Prop. 2.7]. Let $FM$
have been cut in half at point $N$. And let $NO$ have been drawn
through $N$, parallel to $CD$. Thus, $FO$ and $LN$ are each equal to
the (rectangle contained) by $AG$ and $GB$. And since the (sum of the
squares) on $AG$ and $GB$ is rational, and $DM$ is equal to the (sum of the squares) on
$AG$ and $GB$, $DM$ is thus rational. And it has
been applied to the rational (straight-line) $CD$, producing $CM$ as breadth. Thus, $CM$
is rational, and commensurable in length with $CD$ [Prop. 10.20]. Again, since twice the
(rectangle contained) by $AG$ and $GB$ is medial, and $FL$
(is) equal to twice the (rectangle contained) by $AG$ and $GB$, $FL$
(is) thus a medial (area). And it is applied to the rational (straight-line)
$CD$, producing $FM$ as breadth. $FM$ is thus rational, and
incommensurable in  length with $CD$ [Prop. 10.22]. And since the (sum of the
squares) on $AG$ and $GB$ is rational, and twice the (rectangle contained)
by $AG$ and $GB$ medial, the (sum of the squares) on
$AG$ and $GB$ is thus incommensurable with twice the (rectangle
contained) by $AG$ and $GB$. And $CL$ is equal to the (sum of the
squares) on $AG$ and $GB$, and $FL$ to twice the (rectangle contained)
by $AG$ and $GB$. $DM$ is thus incommensurable with $FL$.
And as $DM$ (is) to $FL$, so $CM$ is to $FM$ [Prop. 6.1]. $CM$ is thus incommensurable in length
with $FM$ [Prop. 10.11]. And both are rational 
(straight-lines). Thus, $CM$ and $MF$ are rational (straight-lines
which are) commensurable in square only. $CF$ is thus an apotome
[Prop. 10.73]. So, I say that
(it is) also a first (apotome).

For since the (rectangle contained) by $AG$ and $GB$ is the
mean proportional to the (squares) on $AG$ and $GB$ [Prop. 10.21~lem.], and $CH$ is equal
to the (square) on $AG$, and $KL$ equal to the (square) on 
$BG$, and $NL$ to the (rectangle contained) by $AG$ and $GB$,
$NL$ is thus also the mean proportional to $CH$ and $KL$. Thus,
as $CH$ is to $NL$, so $NL$ (is) to $KL$. But, as $CH$ (is) to
$NL$, so $CK$ is to $NM$, and as $NL$ (is) to $KL$, so
$NM$ is to $KM$ [Prop. 6.1]. 
Thus, the (rectangle contained) by $CK$ and $KM$
is equal to the (square) on $NM$---that is to say, to the fourth
part of the (square) on $FM$ [Prop. 6.17]. 
And since the (square) on $AG$ is commensurable with the
(square) on $GB$, $CH$ [is] also commensurable with
$KL$. And as $CH$ (is) to $KL$, so $CK$ (is) to $KM$ [Prop. 6.1]. $CK$ is thus
commensurable (in length) with $KM$ [Prop. 10.11].  Therefore, since $CM$ and
$MF$ are two unequal straight-lines, and the (rectangle
contained) by $CK$ and $KM$, equal to the fourth part of the
(square) on $FM$, has been applied to $CM$, falling short by a square figure, and $CK$ is commensurable
(in length) with $KM$,  the square on $CM$ is thus greater than
(the square on) $MF$ by the (square) on (some straight-line)
commensurable in length with ($CM$) [Prop. 10.17]. 
And $CM$ is commensurable in length with the (previously) laid down rational (straight-line) $CD$. Thus, $CF$ is a first apotome [Def. 10.15].

Thus, the (square) on an apotome, applied to a rational (straight-line),
produces  a first apotome as breadth. (Which is) the very thing it was required to show.}
\end{Parallel}

%%%%
%10.98
%%%%
\pdfbookmark[1]{Proposition 10.98}{pdf10.98}
\begin{Parallel}{}{}
\ParallelLText{
\begin{center}
{\large \ggn{98}.}
\end{center}\vspace*{-7pt}

\gr{T`o >ap`o m'eshc >apotom~hc pr'wthc par`a <rht`hn paraball'omenon
pl'atoc poie~i >apotom`hn deut'eran.}

\gr{>'Estw m'eshc >apotom`h pr'wth <h AB, <rht`h d`e <h GD,
ka`i t~w| >ap`o t~hc AB >'ison par`a t`hn GD parabebl'hsjw
t`o GE pl'atoc poio~un t`hn GZ; l'egw, <'oti <h GZ >apotom'h
>esti deut'era.}

\gr{>'Estw g`ar t~h| AB prosarm'ozousa <h BH; a<i >'ara AH, HB m'esai
e>is`i dun'amei m'onon s'ummetroi <rht`on peri'eqousai. ka`i t~w|
m`en >ap`o t~hc AH >'ison par`a t`hn GD parabebl'hsjw t`o GJ
pl'atoc poio~un t`hn GK, t~w| d`e >ap`o t~hc HB >'ison t`o KL
pl'atoc poio~un t`hn KM; <'olon >'ara t`o GL >'ison >est`i to~ic
>ap`o t~wn AH, HB; m'eson >'ara ka`i t`o GL. ka`i par`a <rht`hn
t`hn GD par'akeitai pl'atoc poio~un t`hn GM; <rht`h >'ara >est`in
<h GM ka`i >as'ummetroc t~h| GD m'hkei. ka`i >epe`i t`o GL
>'ison >est`i to~ic >ap`o t~wn AH, HB, <~wn t`o >ap`o t~hc
AB >'ison >est`i t~w| GE, loip`on >'ara t`o d`ic <up`o t~wn
AH, HB >'ison >est`i t~w| ZL. <rht`on d'e [>esti] t`o d`ic <up`o
t~wn AH, HB; <rht`on >'ara t`o ZL. ka`i par`a <rht`hn t`hn ZE
par'akeitai pl'atoc poio~un t`hn ZM; <rht`h >'ara >est`i ka`i <h ZM
ka`i s'ummetroc t~h| GD m'hkei. >epe`i o>~un t`a m`en >ap`o
t~wn AH, HB, tout'esti t`o GL, m'eson >est'in, t`o d`e d`ic <up`o
t~wn AH, HB, tout'esti t`o ZL, <rht'on >as'ummetron >'ara >est`i
t`o GL t~w| ZL. <wc d`e t`o GL pr`oc t`o ZL, o<'utwc
>est`in <h GM pr`oc t`hn ZM; >as'ummetroc >'ara <h GM t~h| ZM
m'hkei. ka'i e>isin >amf'oterai <rhta'i; a<i >'ara GM, MZ <rhta'i
e>isi dun'amei m'onon s'ummetroi; <h GZ >'ara >apotom'h >estin.
l'egw d'h, <'oti ka`i deut'era.}\\~\\~\\~\\~\\~\\~\\~\\~\\~\\~\\~\\

\epsfysize=1.6in
\centerline{\epsffile{Book10/fig097g.eps}}

\gr{Tetm'hsjw g`ar <h ZM d'iqa kat`a t`o N, ka`i >'hqjw di`a to~u
N t~h| GD par'allhloc <h NX; <ek'ateron >'ara t~wn
ZX, NL  >'ison >est`i t~w| <up`o t~wn AH, HB. ka`i >epe`i
t~wn >ap`o t~wn AH, HB tetrag'wnwn m'eson >an'alog'on >esti t`o
<up`o t~wn AH, HB, ka'i >estin >'ison t`o m`en >ap`o t~hc AH t~w|
GJ, t`o d`e <up`o t~wn AH, HB t~w| NL, t`o d`e >ap`o t~hc BH t~w|
KL, ka`i t~wn GJ, KL >'ara m'eson >an'alog'on >esti t`o NL; >'estin
>'ara <wc t`o GJ pr`oc t`o NL, o<'utwc t`o NL pr`oc t`o KL.
>all> <wc m`en t`o GJ pr`oc t`o NL, o<'utwc >est`in <h GK pr`oc t`hn
NM, <wc d`e t`o NL pr`oc t`o KL, o<'utwc
>est`in <h NM pr`oc t`hn MK; <wc >'ara <h GK pr`oc t`hn NM, o<'utwc
>est`in <h NM pr`oc t`hn KM; t`o >'ara <up`o t~wn GK, KM
>'ison >est`i t~w| >ap`o t~hc NM, tout'esti t~w| tet'artw| m'erei to~u
>ap`o t~hc ZM [ka`i >epe`i s'ummetr'on >esti t`o >ap`o t~hc AH
t~w| >ap`o t~hc BH, s'ummetr'on >esti ka`i t`o GJ t~w| KL, tout'estin
<h GK t~h| KM]. >epe`i o>~un d'uo e>uje~iai >'aniso'i e>isin a<i GM, MZ,
ka`i t~w| tet'atrw| m'erei to~u >ap`o t~hc MZ >'ison par`a t`hn me'izona
t`hn GM parab'eblhtai >elle~ipon e>'idei tetrag'wnw| t`o <up`o t~wn
GK, KM ka`i e>ic s'ummetra a>ut`hn diaire~i, <h >'ara GM t~hc
MZ me~izon d'unatai t~w| >ap`o summ'etrou <eaut~h| m'hkei. ka'i
>estin <h prosarm'ozousa <h ZM s'ummetroc m'hkei t~h| >ekkeim'enh|
<rht~h| t~h| GD; <h >'ara GZ >apotom'h >esti deut'era.}

\gr{T`o >'ara >ap`o m'eshc >apotom~hc pr'wthc par`a <rht`hn paraball'omenon pl'atoc poie~i >apotom`hn deut'eran; <'oper >'edei de~ixai.}}

\ParallelRText{
\begin{center}
{\large Proposition 98}
\end{center}

The (square) on a first apotome of
a medial (straight-line), applied to a rational (straight-line), produces
 a second apotome as breadth.
  
Let $AB$ be a first apotome of a medial (straight-line), and
$CD$ a rational (straight-line). And let $CE$, equal to the (square) on
$AB$, have been applied to $CD$, producing
$CF$ as breadth. I say that $CF$ is a second apotome.

For let $BG$ be an attachment to $AB$. Thus,
$AG$ and $GB$ are medial (straight-lines which are)
commensurable in square only, containing a rational (area)
[Prop. 10.74]. And let $CH$, equal to
the (square) on $AG$, have been applied to $CD$, producing $CK$
as breadth, and $KL$, equal to the (square) on $GB$, producing $KM$ as
breadth. Thus, the whole of $CL$ is equal to the (sum of the squares)
on $AG$ and $GB$. Thus, $CL$ (is) also a medial (area) [Props.~10.15, 10.23~corr.]. And it is
applied to the rational (straight-line) $CD$, producing $CM$ as breadth.
$CM$ is thus rational, and incommensurable in length with
$CD$ [Prop. 10.22]. And since $CL$ is equal to the
(sum of the squares) on $AG$ and $GB$, of which the (square)
on $AB$ is equal to $CE$, 
the remainder, twice  the (rectangle contained) by $AG$ and $GB$, is thus equal to $FL$ [Prop. 2.7]. And twice the (rectangle contained)
by $AG$ and $GB$ [is] rational. Thus, $FL$ (is) rational. And it
is applied to the rational (straight-line) $FE$, producing $FM$ as breadth.
$FM$ is thus also rational, and commensurable in length with $CD$
[Prop. 10.20]. Therefore, since the (sum of the squares) on $AG$ and $GB$---that is to say, $CL$---is medial, 
and twice the (rectangle contained) by $AG$ and $GB$---that is to say,
$FL$---(is) rational, $CL$ is thus incommensurable with $FL$. 
And as $CL$ (is) to $FL$, so $CM$ is to $FM$ [Prop. 6.1]. Thus, $CM$ (is) incommensurable
in length with $FM$ [Prop. 10.11]. And they are both rational (straight-lines). Thus, $CM$ and $MF$ are rational (straight-lines which are) commensurable in square only. $CF$ is thus an
apotome [Prop. 10.73]. So, I say that (it is)
also a second (apotome).

\epsfysize=1.6in
\centerline{\epsffile{Book10/fig097e.eps}}

For let $FM$ have been cut in half at $N$. And let $NO$ have been drawn
through (point) $N$, parallel to $CD$. Thus, $FO$ and
$NL$ are each equal to the (rectangle contained) by $AG$ and $GB$.
And since
the (rectangle contained) by $AG$ and $GB$ is the mean
proportional to the squares on $AG$ and $GB$ [Prop. 10.21~lem.], and the (square) on $AG$
is equal to $CH$, and the (rectangle contained) by $AG$ and
$GB$ to $NL$, and the (square) on $BG$ to $KL$,  $NL$ is thus also the
mean proportional to $CH$ and $KL$. Thus, as $CH$ is to $NL$, so
$NL$ (is) to $KL$ [Prop. 5.11]. But, as $CH$ (is) to $NL$, so $CK$ is to
$NM$, and as $NL$ (is) to $KL$, so $NM$ is to $MK$
[Prop. 6.1]. Thus, as $CK$ (is) to $NM$, so $NM$
is to $KM$ [Prop. 5.11]. The (rectangle contained) by  $CK$ and $KM$ is thus
equal to the (square) on $NM$ [Prop. 6.17]---that is to say, to the fourth part of the
(square) on $FM$ [and since the (square) on $AG$ is commensurable
with the (square) on $BG$, $CH$ is also commensurable with $KL$---that is to say, $CK$ with $KM$].  Therefore, since $CM$ and $MF$ are two
unequal straight-lines, and the (rectangle contained) by $CK$ and
$KM$, equal to the fourth part
of the (square) on $MF$,  has been applied to the greater $CM$,
falling short by a square figure, and divides it into commensurable
(parts), the square on $CM$ is thus greater than (the square on)
$MF$ by the (square) on (some straight-line) commensurable
in length with ($CM$) [Prop. 10.17]. The
attachment $FM$ is also commensurable in length with the (previously)
laid down rational (straight-line) $CD$. $CF$ is thus a second apotome [Def. 10.16].

Thus, the (square) on a first apotome of
a medial (straight-line), applied to a rational (straight-line), produces
 a second apotome as breadth. (Which is) the very thing it was required to
show.}
\end{Parallel}

%%%%
%10.99
%%%%
\pdfbookmark[1]{Proposition 10.99}{pdf10.99}
\begin{Parallel}{}{}
\ParallelLText{
\begin{center}
{\large \ggn{99}.}
\end{center}\vspace*{-7pt}

\gr{T`o >ap`o m'eshc >apotom~hc deut'erac par`a <rht`hn paraball'omenon
pl'atoc poie~i >apotom`hn tr'ithn.}\\

\epsfysize=1.6in
\centerline{\epsffile{Book10/fig097g.eps}}

\gr{>'Estw m'eshc >apotom`h deut'era <h AB, <rht`h d`e <h GD, ka`i t~w|
>ap`o t~hc AB >'ison par`a t`hn GD parabebl'hsjw t`o GE pl'atoc
poio~un t`hn GZ; l'egw, <'oti <h GZ >apotom'h >esti tr'ith.}

\gr{>'Estw g`ar t~h| AB prosarm'ozousa <h BH; a<i >'ara AH, HB m'esai
e>is`i dun'amei m'onon s'ummetroi m'eson peri'eqousai. ka`i t~w| m`en
>ap`o t~hc AH >'ison par`a t`hn GD parabebl'hsjw t`o GJ pl'atoc poio~un
t`hn GK, t~w| d`e >ap`o t~hc BH >'ison par`a t`hn KJ parabebl'hsjw
t`o KL pl'atoc poio~un t`hn KM; <'olon >'ara t`o GL >'ison >est`i
to~ic >ap`o t~wn AH, HB [ka'i >esti m'esa t`a >ap`o t~wn AH, HB];
m'eson >'ara ka`i t`o GL. ka`i par`a <rht`hn t`hn GD parab'eblhtai
pl'atoc poio~un t`hn GM; <rht`h >'ara >est`in <h GM ka`i >as'ummetroc
t~h| GD m'hkei. ka`i >epe`i <'olon t`o GL >'ison >est`i to~ic >ap`o
t~wn AH, HB, <~wn t`o GE >'ison >est`i t~w| >ap`o t~hc AB,
loip`on >'ara t`o LZ >'ison >est`i t~w| d`ic <up`o t~wn AH, HB. tetm'hsjw
o>~un <h ZM d'iqa kat`a t`o N shme~ion, ka`i t~h| GD par'allhloc
>'hqjw <h NX; <ek'ateron >'ara t~wn ZX, NL >'ison >est`i t~w|
<up`o t~wn AH, HB. m'eson d`e t`o <up`o t~wn AH, HB; m'eson
>'ara >est`i ka`i t`o ZL. ka`i par`a <rht`hn t`hn EZ par'akeitai pl'atoc
poio~un t`hn ZM; <rht`h >'ara ka`i <h ZM ka`i >as'ummetroc t~h| GD
m'hkei. ka`i >epe`i a<i AH, HB dun'amei m'onon e>is`i s'ummetroi,
>as'ummetroc >'ara [>est`i] m'hkei <h AH t~h| HB; >as'ummetron
>'ara >est`i ka`i t`o >ap`o t~hc AH t~w| <up`o t~wn AH, HB.
>all`a t~w| m`en >ap`o t~hc AH s'ummetr'a >esti t`a >ap`o t~wn
AH, HB, t~w| d`e <up`o t~wn AH, HB t`o d`ic <up`o t~wn
AH, HB; >as'ummetra >'ara >est`i t`a >ap`o t~wn
AH, HB t~w| d`ic <up`o t~wn AH, HB. >all`a to~ic m`en >ap`o
t~wn AH, HB >'ison >est`i t`o GL, t~w| d`e d`ic <up`o t~wn AH, HB
>'ison >est`i t`o ZL; >as'ummetron >'ara >est`i t`o GL t~w| ZL.
<wc d`e t`o GL pr`oc t`o ZL, o<'utwc >est`in <h GM pr`oc t`hn ZM;
>as'ummetroc >'ara >est`in <h GM t~h| ZM m'hkei. ka'i e>isin
>amf'oterai <rhta'i; a<i >'ara GM, MZ <rhta'i e>isi dun'amei m'onon
s'ummetroi; >apotom`h >'ara >est`in <h GZ. l'egw d'h, <'oti ka`i tr'ith.}

\gr{>Epe`i g`ar s'ummetr'on >esti t`o >ap`o t~hc AH t~w| >ap`o t~hc HB, s'ummetron >'ara ka`i t`o GJ t~w| KL; <'wste ka`i <h GK t~h| KM. ka`i
>epe`i t~wn >ap`o t~wn AH, HB m'eson >an'alog'on >esti t`o <up`o
t~wn AH, HB, ka'i >esti t~w| m`en >ap`o t~hc AH >'ison t`o GJ, t~w|
d`e >ap`o t~hc HB >'ison t`o KL,  t~w| d`e <up`o t~wn AH, HB >'ison
t`o NL, ka`i t~wn GJ, KL >'ara m'eson >an'alog'on >esti t`o NL; >'estin
>'ara <wc t`o GJ pr`oc t`o NL, o<'utwc t`o NL pr`oc t`o KL. >all>
<wc m`en t`o GJ pr`oc t`o NL, o<'utwc >est`in <h  GK pr`oc t`hn NM,
<wc d`e t`o NL pr`oc t`o KL, o<'utwc >est`in <h NM pr`oc t`hn KM; <wc
>'ara <h GK pr`oc t`hn MN, o<'utwc >est`in <h MN pr`oc t`hn
KM; t`o >'ara <up`o t~wn GK, KM >'ison >est`i t~w| [>ap`o
t~hc MN, tout'esti t~w|] tet'artw| m'erei to~u >ap`o t~hc ZM. >epe`i
o>~un d'uo e>uje~iai >'aniso'i e>isin a<i GM, MZ, ka`i t~w| tet'artw|
m'erei to~u >ap`o t~hc ZM >'ison par`a t`hn GM parab'eblhtai >elle~ipon
e>'idei tetrag'wnw| ka`i e>ic s'ummetra a>ut`hn diaire~i, <h GM >'ara
t~hc MZ me~izon d'unatai t~w| >ap`o summ'etrou <eaut~h|. ka`i
o>udet'era t~wn GM, MZ s'ummetr'oc >esti m'hkei t~h| >ekkeim'enh|
<rht~h| t~h| GD; <h >'ara GZ >apotom'h >esti tr'ith.}

\gr{T`o >'ara >ap`o m'eshc >apotom~hc deut'erac par`a <rht`hn paraball'omenon pl'atoc poie~i >apotom`hn tr'ithn; <'oper >'edei de~ixai.}}

\ParallelRText{
\begin{center}
{\large Proposition 99}
\end{center}

The (square) on a second apotome of a medial
(straight-line), applied to a rational (straight-line), produces a
third apotome  as breadth.

\epsfysize=1.6in
\centerline{\epsffile{Book10/fig097e.eps}}

Let $AB$ be the second apotome of a medial (straight-line), and
$CD$ a rational (straight-line). And let $CE$, equal to the (square) on  $AB$,
have been applied to $CD$, producing $CF$ as breadth. I say that $CF$
is a third apotome.

For let $BG$ be an attachment to $AB$. Thus, $AG$ and $GB$ are medial
(straight-lines which are) commensurable in square only, containing
a medial (area) [Prop. 10.75]. And let $CH$,
equal to the (square) on $AG$, have been applied to $CD$, producing
$CK$ as breadth. And let $KL$, equal to the (square) on $BG$, have
been applied to $KH$, producing $KM$ as breadth. Thus, the whole
of $CL$ is equal to the (sum of the squares) on $AG$ and $GB$ [and
the (sum of the squares) on $AG$ and $GB$ is medial]. $CL$ (is) thus
also medial [Props.~10.15, 10.23~corr.]. And it has been applied to the rational (straight-line)
$CD$, producing $CM$ as breadth. Thus, $CM$ is rational, and
incommensurable in length with $CD$ [Prop. 10.22]. And since the whole of $CL$ is equal
to the (sum of the squares) on $AG$ and $GB$, of which $CE$ is
equal to the (square) on $AB$, the remainder $LF$ is thus equal to
twice the (rectangle contained) by $AG$ and $GB$ [Prop. 2.7]. Therefore, let $FM$ have been
cut in half at point $N$. And let $NO$ have been drawn parallel to $CD$.
Thus, $FO$ and $NL$ are each equal to the (rectangle contained) by
$AG$ and $GB$. And the (rectangle contained) by $AG$ and $GB$ (is)
medial. Thus, $FL$ is also medial. And it is applied to the
rational (straight-line) $EF$, producing $FM$ as breadth.
$FM$ is thus rational, and incommensurable in length with $CD$ [Prop. 10.22]. And since $AG$ and $GB$
are commensurable in square only, $AG$ [is] thus incommensurable
in length with $GB$. Thus, the (square) on $AG$ is also incommensurable
with the (rectangle contained) by $AG$ and $GB$
[Props.~6.1, 10.11]. But,  the (sum of the squares) on $AG$
and $GB$ is commensurable with the (square) on $AG$, and twice the
(rectangle contained) by $AG$ and $GB$ with the (rectangle contained)
by $AG$ and $GB$. The (sum of the squares) on $AG$ and $GB$
is thus incommensurable with twice the (rectangle contained)
by $AG$ and $GB$ [Prop. 10.13]. But,
$CL$ is equal to the (sum of the squares) on $AG$ and $GB$, 
and $FL$ is equal to twice the (rectangle contained) by $AG$ and $GB$. Thus,
$CL$ is incommensurable with $FL$. And as $CL$ (is) to $FL$, so
$CM$ is to $FM$ [Prop. 6.1].  $CM$ is thus
incommensurable in length with $FM$ [Prop. 10.11]. And they are both rational (straight-lines). Thus, $CM$ and $MF$ are rational (straight-lines which are)
commensurable in square only. $CF$ is thus an apotome [Prop. 10.73]. So, I say that (it is) also a
third (apotome).

For since the (square) on $AG$ is commensurable with the (square) on 
$GB$, $CH$ (is) thus also commensurable with $KL$. Hence,
$CK$ (is) also  (commensurable in length) with $KM$ [Props.~6.1, 10.11]. 
And since the (rectangle contained) by $AG$ and $GB$ is the mean
proportional to the (squares) on $AG$ and $GB$ [Prop. 10.21~lem.], and $CH$
is equal to the (square) on $AG$, and $KL$ equal to the (square) on  $GB$,
and $NL$ equal to the (rectangle contained) by $AG$ and $GB$, $NL$ is
thus also the mean proportional to $CH$ and $KL$. Thus, as $CH$ is to
$NL$, so $NL$ (is) to $KL$. But, as $CH$ (is) to $NL$, so $CK$
is to $NM$, and as $NL$ (is) to $KL$, so $NM$ (is) to $KM$
[Prop. 6.1]. Thus, as $CK$ (is) to $MN$, so $MN$
is to $KM$ [Prop. 5.11]. Thus, the (rectangle contained) by $CK$ and $KM$
is equal to the [(square) on $MN$---that is to say, to the] fourth part of the
(square) on $FM$ [Prop. 6.17]. Therefore,
since $CM$ and $MF$ are two unequal straight-lines, and
(some area), equal to the fourth part of the (square) on  $FM$, has been
applied to $CM$, falling short by a square figure, and divides it into
commensurable (parts),  the square on $CM$ is thus greater than (the square on) $MF$ by the (square) on (some straight-line) commensurable
(in length) with ($CM$) [Prop. 10.17]. 
And neither of $CM$ and $MF$ is commensurable in length with the
(previously) laid down rational (straight-line) $CD$. $CF$ is thus a
third apotome [Def. 10.13].

Thus, the (square) on a second apotome of a medial
(straight-line), applied to a rational (straight-line), produces  a
third apotome as breadth. (Which is) the very thing it was required to show.}
\end{Parallel}

%%%%%
%10.100
%%%%%
\pdfbookmark[1]{Proposition 10.100}{pdf10.100}
\begin{Parallel}{}{}
\ParallelLText{
\begin{center}
{\large \ggn{100}.}
\end{center}\vspace*{-7pt}

\gr{T`o >ap`o >el'assonoc par`a <rht`hn paraball'omenon pl'atoc poie~i
>apotom`hn tet'arthn.}\\

\epsfysize=1.6in
\centerline{\epsffile{Book10/fig097g.eps}}

\gr{>'Estw >el'asswn <h AB, <rht`h d`e <h GD, ka`i t~w| >ap`o
t~hc AB >'ison par`a <rht`hn t`hn GD parabebl'hsjw t`o GE
pl'atoc poio~un t`hn GZ; l'egw, <'oti <h GZ >apotom'h >esti
tet'arth.}

\gr{>'Estw g`ar t~h| AB prosarm'ozousa <h BH; a<i >'ara AH, HB dun'amei
e>is`in >as'ummetroi poio~usai t`o m`en sugke'imenon >ek t~wn >ap`o
t~wn AH, HB tetrag'wnwn <rht'on, t`o d`e d`ic <up`o t~wn AH, HB
m'eson. ka`i t~w| m`en >ap`o t~hc AH >'ison par`a t`hn GD parabebl'hsjw
t`o GJ pl'atoc poio~un t`hn GK, t~w| d`e >ap`o t~hc BH
>'ison t`o KL pl'atoc poio~un t`hn KM; <'olon >'ara t`o GL >'ison
>est`i to~ic >ap`o t~wn AH, HB. ka'i >esti t`o sugke'imenon >ek t~wn
>ap`o t~wn AH, HB <rht'on; <rht`on >'ara >est`i ka`i t`o GL.
ka`i par`a <rht`hn t`hn GD par'akeitai pl'atoc poio~un t`hn GM; <rht`h >'ara
ka`i <h GM ka`i s'ummetroc t~h| GD m'hkei. ka`i >epe`i <'olon
t`o GL >'ison >est`i to~ic >ap`o t~wn AH, HB, <~wn t`o GE >'ison
>est`i t~w| >ap`o t~hc AB, loip`on >'ara t`o ZL >'ison >est`i t~w| d`ic
<up`o t~wn AH, HB. tetm'hsjw o>~un <h ZM d'iqa kat`a t`o N shme~ion,
ka`i >'hqjw d`ia to~u N <opot'era| t~wn GD, ML par'allhloc <h NX;
<ek'ateron >'ara t~wn ZX, NL >'ison >est`i t~w| <up`o t~wn AH, HB. ka`i
>epe`i t`o d`ic <up`o t~wn AH, HB m'eson >est`i ka'i >estin >'ison t~w|
ZL, ka`i t`o ZL >'ara m'eson >est'in. ka`i par`a <rht`hn t`hn ZE par'akeitai
pl'atoc poio~un t`hn ZM; <rht`h >'ara >est`in <h ZM ka`i >as'ummetroc
t~h| GD m'hkei. ka`i >epe`i t`o m`en sugke'imenon >ek t~wn >ap`o
t~wn AH, HB <rht'on >estin, t`o d`e d`ic <up`o t~wn AH, HB m'eson,
>as'ummetra [>'ara] >est`i t`a >ap`o t~wn AH, HB 
 t~w| d`ic <up`o t~wn AH, HB.
>'ison d'e [>esti] t`o GL to~ic >ap`o t~wn AH, HB, t~w| d`e d`ic <up`o
t~wn AH, HB >'ison t`o ZL; >as'ummetron >'ara [>est`i] t`o GL t~w| ZL.
<wc d`e t`o GL pr`oc t`o ZL, o<'utwc >est`in
  <h GM pr`oc t`hn MZ; >as'ummetroc >'ara
  >est`in <h GM t~h| MZ m'hkei.
ka'i e>isin >amf'oterai <rhta'i; a<i >'ara GM, MZ <rhta'i
e>isi dun'amei m'onon s'ummetroi; >apotom`h >'ara >est`in <h GZ. l'egw
[d'h], <'oti ka`i tet'arth.}

\gr{>Epe`i g`ar a<i AH, HB dun'amei e>is`in >as'ummetroi, >as'ummet\-ron
>'ara ka`i t`o >ap`o t~hc AH t~w| >ap`o t~hc HB.  ka'i >esti t~w| m`en
>ap`o t~hc AH >'ison t`o GJ, t~w| d`e >ap`o t~hc HB >'ison
t`o KL; >as'ummetron >'ara >est`i t`o GJ t~w| KL. <wc d`e t`o GJ pr`oc
t`o KL, o<'utwc >est`in <h GK pr`oc t`hn KM; >as'ummetroc >'ara >est`in
<h GK t~h| KM m'hkei. ka`i >epe`i t~wn >ap`o t~wn AH, HB m'eson >an'alog'on >esti t`o <up`o t~wn AH, HB, ka'i >estin >'ison t`o
m`en >ap`o t~hc AH t~w| GJ, t`o d`e >ap`o t~hc HB t~w| KL, t`o d`e <up`o
t~wn AH, HB t~w| NL, t~wn >'ara GJ, KL m'eson >an'alog'on >esti
t`o NL; >'estin >'ara <wc t`o GJ pr`oc t`o NL, o<'utwc t`o NL pr`oc
t`o KL. >all> <wc m`en t`o GJ pr`oc t`o NL, o<'utwc >est'in <h GK
pr`oc t`hn NM, <wc d`e t`o NL pr`oc t`o KL, o<'utwc >est`in <h NM
pr`oc t`hn KM; <wc >'ara <h GK pr`oc t`hn MN, o<'utwc >est`in
<h MN pr`oc t`hn KM; t`o >'ara <up`o t~wn GK, KM >'ison >est`i
t~w| >ap`o t~hc MN, tout'esti t~w| tet'artw| m'erei to~u >ap`o t~hc
ZM. >epe`i o>~un d'uo e>uje~iai >'aniso'i e>isin a<i GM, MZ, ka`i
t~w| tetr'artw| m'erei to~u >ap`o t~hc MZ >'ison par`a t`hn GM parab'eblhtai
>elle~ipon e>'idei tetrag'wnw| t`o <up`o t~wn GK, KM ka`i
e>ic >as'ummetra a>ut`hn diaire~i, <h >'ara GM t~hc MZ me~izon
d'unatai t~w| >ap`o >asumm'etrou <eaut~h|. ka'i >estin <'olh
<h GM s'ummetroc m'hkei t~h| >ekkeim'enh| <rht~h| t~h| GD;
<h >'ara GZ >apotom'h >esti tet'arth.}

\gr{T`o >'ara >ap`o >el'assonoc ka`i t`a <ex~hc.}}

\ParallelRText{
\begin{center}
{\large Proposition 100}
\end{center}

The (square) on a minor
(straight-line), applied to a rational (straight-line), produces  a
fourth apotome as breadth.

\epsfysize=1.6in
\centerline{\epsffile{Book10/fig097e.eps}}

Let $AB$ be a minor (straight-line), and $CD$ a rational (straight-line).
And let $CE$, equal to the (square) on $AB$, have been applied to the rational (straight-line) $CD$, producing $CF$ as breadth. I say that
$CF$ is a fourth apotome.

For let $BG$ be an attachment to $AB$. Thus, $AG$ and $GB$ are incommensurable in square, making the sum of the squares on $AG$ and $GB$ rational, and twice the (rectangle contained) by $AG$ and $GB$ medial [Prop. 10.76]. And let $CH$,
equal to the (square) on $AG$, have been applied to $CD$, producing
$CK$ as breadth, and $KL$, equal to the (square) on $BG$, producing
$KM$ as breadth. Thus, the whole of $CL$ is equal to the (sum of the
squares) on $AG$ and $GB$. And the sum of the (squares) on $AG$ and
$GB$ is rational. $CL$ is thus also rational. And it is applied to the
rational (straight-line) $CD$, producing $CM$ as breadth. Thus, $CM$
(is) also rational, and commensurable in length with $CD$ [Prop. 10.20]. And since the whole of $CL$
is equal to the (sum of the squares) on $AG$ and $GB$, of which
$CE$ is equal to the (square) on $AB$, the remainder $FL$ is thus
equal to twice the (rectangle contained) by $AG$ and $GB$ [Prop. 2.7]. Therefore, let $FM$ have been cut in
half at point $N$. And let $NO$ have been drawn through $N$, parallel
to either of $CD$ or $ML$. Thus, $FO$ and $NL$ are each
equal to the (rectangle contained) by $AG$ and $GB$. And since
twice the (rectangle contained) by $AG$ and $GB$ is medial, and is equal
to $FL$, $FL$ is thus also medial. And it is applied to the rational (straight-line) $FE$, producing $FM$ as breadth. Thus, $FM$ is  rational, and incommensurable in length with $CD$ [Prop. 10.22]. And since the sum of the (squares)
on $AG$ and $GB$ is rational, and twice the (rectangle contained) by $AG$
and $GB$ medial, the (sum of the squares) on $AG$ and $GB$
is [thus] incommensurable with twice the (rectangle contained)
by $AG$ and $GB$. And $CL$ (is) equal to the (sum of the squares)
on $AG$ and $GB$, and $FL$ equal to twice the (rectangle contained) by $AG$ and $GB$. $CL$ [is] thus incommensurable with $FL$. And as
$CL$ (is) to $FL$, so $CM$ is to $MF$ [Prop. 6.1]. 
$CM$ is thus incommensurable in length with $MF$ [Prop. 10.11]. And both are rational (straight-lines). 
Thus, $CM$ and $MF$ are rational (straight-lines which are) commensurable
in square only. $CF$ is thus an apotome [Prop. 10.73]. [So], I say that (it is) also
a fourth (apotome).

For since $AG$ and $GB$ are incommensurable in square, the (square)
on $AG$ (is) thus also incommensurable with the (square) on $GB$.
And $CH$ is equal to the (square) on $AG$, and $KL$ equal to the (square) on $GB$. Thus, $CH$ is incommensurable with $KL$. And as $CH$ (is) to
$KL$, so $CK$ is to $KM$ [Prop. 6.1]. 
$CK$ is thus incommensurable in length with $KM$ [Prop. 10.11]. And since the (rectangle contained)
by $AG$ and $GB$ is the mean proportional to the (squares) on $AG$
and $GB$ [Prop. 10.21~lem.], and the
(square) on $AG$ is equal to $CH$, and the (square) on $GB$ to $KL$,
and the (rectangle contained) by $AG$ and $GB$ to $NL$, $NL$
is thus the mean proportional to $CH$ and $KL$. Thus, as $CH$
is to $NL$, so $NL$ (is) to $KL$. But, as $CH$ (is) to $NL$, so
$CK$ is to $NM$, and as $NL$ (is) to $KL$, so $NM$ is to $KM$ [Prop. 6.1]. Thus, as $CK$ (is) to $MN$, so
$MN$ is to $KM$ [Prop. 5.11]. The (rectangle
contained) by $CK$ and $KM$ is thus equal to the (square) on $MN$---that is to say, to the fourth part of the (square) on $FM$ [Prop. 6.17]. Therefore, since $CM$ and $MF$ are two unequal straight-lines, and the (rectangle contained) by $CK$ and
$KM$, equal to the fourth part of the (square) on $MF$, has been applied
to $CM$, falling short by a square figure, and divides it into incommensurable (parts), the square on $CM$ is thus greater than (the
square on) $MF$ by the (square) on (some straight-line) incommensurable
(in length) with ($CM$) [Prop. 10.18]. 
And the whole of $CM$ is commensurable in length with the (previously)
laid down rational (straight-line) $CD$. Thus, $CF$ is a fourth apotome
[Def. 10.14].

Thus, the (square) on a minor, and so on \ldots}
\end{Parallel}

%%%%%
%10.101
%%%%%
\pdfbookmark[1]{Proposition 10.101}{pdf10.101}
\begin{Parallel}{}{}
\ParallelLText{
\begin{center}
{\large \ggn{101}.}
\end{center}\vspace*{-7pt}

\gr{T`o >ap`o t~hc met`a <rhto~u m'eson t`o <'olon poio'ushc
par`a <rht`hn paraball'omenon pl'atoc poie~i >apotom`hn p'empthn.}\\

\epsfysize=1.6in
\centerline{\epsffile{Book10/fig097g.eps}}

\gr{>'Estw <h met`a <rhto~u m'eson t`o <'olon poio~usa <h AB,
<rht`h d`e <h GD, ka`i t~w| >ap`o t~hc AB >'ison par`a t`hn GD
parabebl'hsjw t`o GE pl'atoc poio~un t`hn GZ; l'egw, <'oti <h GZ
>apotom'h >esti p'empth.}

\gr{>'Estw g`ar t~h| AB prosarm'ozousa <h BH; a<i >'ara AH, HB e>uje~iai
dun'amei e>is`in >as'ummetroi poio~usai t`o m`en sugke'imenon
>ek t~wn >ap> a>ut~wn tetrag'wnwn m'eson, t`o d`e d`ic <up>
a>ut~wn <rht'on, ka`i t~w| m`en >ap`o t~hc AH >'ison par`a t`hn GD
parabebl'hsjw t`o GJ, t~w| d`e >ap`o t~hc HB <'ison t`o KL;
<'olon >'ara t`o GL >'ison >est`i to~ic >ap`o t~wn AH, HB.
t`o d`e sugke'imenon >ek
t~wn >ap`o t~wn AH, HB <'ama m'eson >est'in; m'eson >'ara >est`i
t`o GL. ka`i par`a <rht`hn t`hn GD par'akeitai pl'atoc poio~un
t`hn GM; <rht`h >'ara >est`in <h GM ka`i >as'ummetroc t~h| GD. ka`i
>epe`i <'olon t`o GL >'ison >est`i to~ic >ap`o t~wn AH, HB, <~wn
t`o GE >'ison >est`i t~w| >ap`o t~hc AB, loip`on >'ara t`o ZL >'ison
>est`i t~w| d`ic <up`o t~wn AH, HB. tetm'hsjw o>~un <h ZM d'iqa
kat`a t`o N, ka`i >'hqjw di`a to~u N <opot'era| t~wn GD, ML par'allhloc
<h NX; <ek'ateron >'ara t~wn ZX, NL >'ison >est`i t~w| <up`o t~wn
AH, HB, ka`i >epe`i t`o d`ic <up`o t~wn AH, HB <rht'on >esti ka'i
[>estin] >'ison t~w| ZL, <rht`on >'ara >est`i t`o ZL. ka`i par`a <rht`hn
t`hn EZ par'akeitai pl'atoc poio~un t`hn ZM; <rht`h >'ara >est`in <h ZM
ka`i s'ummetroc t~h| GD m'hkei. ka`i >epe`i t`o m`en GL m'eson
>est'in, t`o d`e ZL <rht'on, >as'ummetron >'ara >est`i t`o GL t~w| ZL. <wc d`e t`o GL pr`oc t`o ZL, o<'utwc <h GM pr`oc t`hn MZ; >as'ummetroc
>'ara >est`in <h GM t~h| MZ m'hkei. ka'i e>isin >amf'oterai <rhta'i;
a<i >'ara GM, MZ <rhta'i e>isi dun'amei m'onon s'ummetroi; >apotom`h
>'ara >est`in <h GZ. l`egw d'h, <'oti ka`i p'empth.}

\gr{<Omo'iwc g`ar de'ixomen, <'oti t`o <up`o t~wn GKM >'ison
>est`i t~w| >ap`o t~hc NM, tout'esti t~w| tet'artw| m'erei to~u
>ap`o t~hc ZM. ka`i >epe`i >as'ummetr'on >esti t`o >ap`o t~hc
AH t~w| >ap`o t~hc HB, >'ison d`e t`o m`en >ap`o t~hc AH t~w|
GJ, t`o d`e >ap`o t~hc HB t~w| KL, >as'ummetron >'ara
t`o GJ t~w| KL. <wc d`e t`o GJ pr`oc t`o KL, o<'utwc
<h GK pr`oc t`hn KM; >as'ummetroc >'ara <h GK t~h| KM m'hkei.
>epe`i o>~un d'uo e>uje~iai >'aniso'i e>isin a<i GM, MZ, ka`i t~w|
tet'artw| m'erei to~u >ap`o t~hc ZM >'ison par`a t`hn GM parab'eblhtai
>elle~ipon e>'idei tetrag'wnw| ka`i e>ic >as'ummetra a>ut`hn diaire~i,
<h >'ara GM t~hc MZ me~izon d'unatai t~w| >ap`o >as'umm'etrou
<eaut~h|. ka'i >estin <h prosarm'ozousa <h ZM s'ummetroc t~h|
>ekkeim'enh| <rht~h| t~h| GD; <h >'ara GZ >apotom'h >esti
p'empth; <'oper >'edei de~ixai.}}

\ParallelRText{
\begin{center}
{\large Proposition 101}
\end{center}\vspace*{-7pt}

The (square) on that (straight-line) which
with a rational (area) makes a medial whole, applied to a rational (straight-line), produces a fifth apotome as breadth.

\epsfysize=1.6in
\centerline{\epsffile{Book10/fig097e.eps}}

Let $AB$ be that (straight-line) which with a rational (area) makes a
medial whole, and $CD$ a rational (straight-line). And let $CE$, equal
to the (square) on $AB$, have been applied to $CD$, producing $CF$
as breadth. I say that $CF$ is a fifth apotome.

Let $BG$ be an attachment to $AB$. Thus, the straight-lines $AG$
and $GB$ are incommensurable in square, making the sum of the
squares on them medial, and twice the (rectangle contained) by them
rational [Prop. 10.77]. And let $CH$, equal 
to the (square) on $AG$, have been applied to $CD$, and $KL$, equal
to the (square) on $GB$. The whole of $CL$
is thus equal to the (sum of the squares) on $AG$ and $GB$. 
And the sum of the (squares) on $AG$ and $GB$ together is medial. Thus, $CL$
is medial. And it has been applied to the rational (straight-line) $CD$,
producing $CM$ as breadth. $CM$ is thus rational, and incommensurable
(in length) with $CD$ [Prop. 10.22]. And since the whole
of $CL$ is equal to the (sum of the squares) on $AG$ and $GB$, of
which $CE$ is equal to the (square) on $AB$, the remainder $FL$ is
thus equal to twice the (rectangle contained) by $AG$ and $GB$ [Prop. 2.7]. Therefore, let $FM$ have been
cut in half at $N$. And let $NO$ have been drawn through $N$, parallel
to either of $CD$ or $ML$. Thus, $FO$ and $NL$ are each equal
to the (rectangle contained) by $AG$ and $GB$. And since twice the
(rectangle contained) by $AG$ and $GB$ is rational, and [is]
equal to $FL$, $FL$ is thus rational. And it is applied to the rational
(straight-line) $EF$, producing $FM$ as breadth. Thus, $FM$ is rational,
and commensurable in length with $CD$ [Prop. 10.20]. And since $CL$ is medial, and
$FL$ rational, $CL$ is thus incommensurable with $FL$. 
And as $CL$ (is) to $FL$, so $CM$ (is) to $MF$ [Prop. 6.1]. $CM$ is thus incommensurable in length
with $MF$ [Prop. 10.11]. And both are
rational. Thus, $CM$ and $MF$ are rational (straight-lines which are)
commensurable in square only. $CF$ is thus an apotome [Prop. 10.73].  So, I say that (it is) also
a fifth (apotome).

For, similarly (to the previous propositions), we can show that the
(rectangle contained) by $CKM$ is equal to the (square) on $NM$---that is
to say, to the fourth part of the (square) on $FM$.  And since the (square)
on $AG$ is incommensurable with the (square) on $GB$, and  the
(square) on $AG$ (is)
equal to $CH$, and the (square) on $GB$ to $KL$, $CH$ (is) thus
incommensurable with $KL$. And as $CH$ (is) to $KL$, so
$CK$ (is) to $KM$ [Prop. 6.1]. 
Thus, $CK$ (is) incommensurable in length with $KM$ [Prop. 10.11]. Therefore, since $CM$ and $MF$
are two unequal straight-lines, and (some area), equal to the
fourth part of the (square) on $FM$, has been applied to $CM$, falling
short by a square figure, and divides it into incommensurable
(parts), the square on $CM$ is thus greater than (the square on) $MF$
by the (square) on (some straight-line) incommensurable (in length)
with ($CM$) [Prop. 10.18]. And the attachment
$FM$ is commensurable with the (previously) laid down
rational (straight-line) $CD$. Thus, $CF$ is a fifth apotome [Def. 10.15]. (Which is) the very thing it was required to show.}
\end{Parallel}

%%%%%
%10.102
%%%%%
\pdfbookmark[1]{Proposition 10.102}{pdf10.102}
\begin{Parallel}{}{}
\ParallelLText{
\begin{center}
{\large \ggn{102}.}
\end{center}\vspace*{-7pt}

\gr{T`o >ap`o t~hc met`a m'esou m'eson t`o <'olon poio'ushc par`a <rht`hn
paraball'omenon pl'atoc poie~i >apotom`hn <'ekthn.}\\

\epsfysize=1.6in
\centerline{\epsffile{Book10/fig097g.eps}}

\gr{>'Estw <h met`a m'esou m'eson t`o <'olon poio~usa <h AB, <rht`h d`e <h
GD, ka`i t~w| >ap`o t~hc AB >'ison par`a t`hn GD parabebl'hsjw
t`o GE pl'atoc poio~un t`hn GZ; l'egw, <'oti <h GZ >apotom'h
>estin <'ekth.}

\gr{>'Estw g`ar t~h| AB prosarm'ozousa <h BH; a<i >'ara AH, HB
dun'amei e>is`in >as'ummetroi poio~usai t'o te sugke'imenon
>ek t~wn >ap> a>ut~wn tetrag'wnwn m'eson ka`i t`o d`ic <up`o
t~wn AH, HB m'eson ka`i >as'ummetron t`a >ap`o t~wn AH, HB
t~w| d`ic <up`o t~wn AH, HB. parabebl'hsjw o>~un par`a t`hn GD
t~w| m`en >ap`o t~hc AH >'ison t`o GJ pl'atoc poio~un t`hn GK,
t~w| d`e >ap`o t~hc BH t`o KL; <'olon >'ara t`o GL >'ison
>est`i to~ic >ap`o t~wn AH, HB; m'eson >'ara [>est`i] ka`i t`o GL.
ka`i par`a <rht`hn t`hn GD par'akeitai pl'atoc poio~un t`hn GM;
<rht`h >'ara >est`in <h GM ka`i >as'ummetroc t~h| GD m'hkei.
>epe`i o>~un t`o GL >'ison >est`i to~ic >ap`o t~wn
AH, HB, <~wn t`o GE >'ison t~w| >ap`o t~hc AB, loip`on
>'ara t`o ZL >'ison >est`i t~w| d`ic <up`o t~wn AH, HB. ka'i
>esti t`o d`ic <up`o t~wn AH, HB m'eson; ka`i t`o ZL >'ara
m'eson >est'in. ka`i par`a <rht`hn t`hn ZE par'akeitai pl'atoc
poio~un t`hn ZM; <rht`h >'ara >est`in <h ZM ka`i >as'ummetroc
t~h| GD m'hkei. ka`i >epe`i t`a >ap`o t~wn AH, HB >as'ummetr'a
>esti t~w| d`ic <up`o t~wn AH, HB, ka'i >esti to~ic m`en >ap`o
t~wn AH, HB >'ison t`o GL, t~w| d`e d`ic <up`o t~wn AH,
HB >'ison t`o ZL, >as'ummetroc >'ara [>est`i] t`o GL t~w| ZL.
<wc d`e t`o GL pr`oc t`o 
ZL, o<'utwc >est`in <h GM pr`oc t`hn
MZ; >as'ummetroc >'ara
>est`in <h GM t~h| MZ
m'hkei. ka'i e>isin >amf'oterai <rhta'i. a<i GM, MZ >'ara <rhta'i
e>isi dun'amei m'onon s'ummetroi; >apotom`h >'ara >est`in <h GZ.
l'egw d'h, <'oti ka`i <'ekth.}

\gr{>Epe`i g`ar t`o ZL >'ison >est`i t~w| d`ic <up`o t~wn AH, HB, tetm'hsjw
d'iqa <h ZM kat`a t`o N, ka`i >'hqjw di`a to~u N t~h| GD par'allhloc
<h NX; <ek'ateron >'ara t~wn ZX, NL >'ison >est`i t~w| <up`o
t~wn AH, HB. ka`i >epe`i a<i AH, HB dun'amei e>is`in >as'ummetroi,
>as'ummetron >'ara >est`i t`o >ap`o t~hc AH t~w| >ap`o t~hc
HB. >all`a t~w| m`en >ap`o t~hc AH >'ison >est`i t`o GJ, t~w| d`e
>ap`o t~hc HB >'ison >est`i t`o KL; >as'ummetron >'ara >est`i
t`o GJ t~w| KL. <wc d`e t`o GJ pr`oc t`o KL, o<'utwc >est`in
<h GK pr`oc t`hn KM; >as'ummetroc >'ara >est`in <h GK t~h| KM.
ka`i >epe`i t~wn >ap`o t~wn AH, HB m'eson >an'alog'on >esti
t`o <up`o t~wn AH, HB, ka`i >esti t~w| m`en >ap`o
t~hc AH >'ison t`o GJ, t~w| d`e >ap`o t~hc HB >'ison
t`o KL, t~w| d`e <up`o t~wn AH, HB >'ison t`o NL, ka`i t~wn >'ara
GJ, KL m'eson >an'alog'on >esti t`o NL; >'estin >'ara <wc t`o GJ
pr`oc t`o NL, o<'utwc t`o NL pr`oc t`o KL. ka`i di`a t`a a>ut`a <h GM
t~hc MZ me~izon d'unatai t~w| >ap`o >asumm'etrou <eaut~h|. ka`i
o>udet'era a>ut~wn s'ummetr'oc
>esti t~h| >ekkeim'enh| <rht~h| t~h| GD; <h GZ >'ara >apotom'h
>estin <'ekth; <'oper >'edei de~ixai.}}

\ParallelRText{
\begin{center}
{\large Proposition 102}
\end{center}

The (square) on that (straight-line)
which with a medial (area) makes a medial whole, applied to
a rational (straight-line), produces a sixth apotome as breadth.

\epsfysize=1.6in
\centerline{\epsffile{Book10/fig097e.eps}}

Let $AB$ be that (straight-line) which with a medial (area) makes a
medial whole, and $CD$ a rational (straight-line). And let $CE$,
equal to the (square) on $AB$, have been applied to $CD$, producing
$CF$ as breadth. I say that $CF$ is a sixth apotome.

For let $BG$ be an attachment to $AB$. Thus, $AG$ and $GB$
are incommensurable in square, making the sum of the squares on them
medial, and twice the (rectangle contained) by $AG$ and $GB$ medial,
and the (sum of the squares) on $AG$ and $GB$ incommensurable
with twice the (rectangle contained) by $AG$ and $GB$ [Prop. 10.78]. Therefore, let $CH$, equal to
the (square) on $AG$, have been applied to $CD$, producing $CK$
as breadth, and $KL$, equal to the (square) on $BG$. Thus, the whole of
$CL$ is equal to the (sum of the squares) on $AG$ and $GB$. $CL$ [is]
thus also medial. And it is applied to the rational (straight-line) $CD$,
producing $CM$ as breadth. Thus,  $CM$ is rational, and incommensurable
in length with $CD$ [Prop. 10.22]. Therefore,
since $CL$ is equal to the (sum of the squares) on $AG$ and $GB$,
of which $CE$ (is) equal to the (square) on $AB$, the remainder $FL$
is thus equal to twice the (rectangle contained) by $AG$ and $GB$ [Prop. 2.7]. And twice the (rectangle contained)
by  $AG$ and $GB$ (is) medial. Thus, $FL$ is also medial.
And it is applied to the rational (straight-line) $FE$, producing $FM$
as breadth. $FM$ is thus rational, and incommensurable
in length with $CD$ [Prop. 10.22]. 
And since the (sum of the squares) on $AG$ and $GB$ is incommensurable
with twice the (rectangle contained) by $AG$ and $GB$, and $CL$ equal to the
(sum of the squares) on $AG$ and $GB$, and $FL$ equal to twice the
(rectangle contained) by $AG$ and $GB$, $CL$ [is] thus
incommensurable with $FL$. And as $CL$ (is) to $FL$, so $CM$ is
to $MF$ [Prop. 6.1]. Thus, $CM$ is incommensurable
in length with $MF$ [Prop. 10.11]. And they
are both rational. Thus, $CM$ and $MF$ are rational (straight-lines which are) commensurable in square only. $CF$ is thus an apotome [Prop. 10.73]. So, I say that (it is) also a
sixth (apotome).

For since $FL$ is equal to twice the (rectangle contained) by $AG$
and $GB$, let $FM$ have been cut in half at $N$, and let $NO$
have been drawn through $N$, parallel to $CD$. Thus, $FO$ and $NL$
are each equal to the (rectangle contained) by $AG$ and $GB$. And since
$AG$ and $GB$ are incommensurable in square, the (square) on $AG$
is thus incommensurable with the (square) on $GB$. But, $CH$ is equal to
the (square) on $AG$, and $KL$ is equal to the (square) on $GB$. 
Thus, $CH$ is incommensurable with $KL$. And as $CH$ (is) to
$KL$, so $CK$ is to $KM$ [Prop. 6.1].  Thus,
$CK$ is incommensurable (in length) with $KM$ [Prop. 10.11]. And since the (rectangle contained)
by $AG$ and $GB$ is the mean proportional to the (squares) on 
$AG$ and $GB$ [Prop. 10.21~lem.], and $CH$ is equal to the (square) on $AG$, and $KL$ 
equal to the (square) on $GB$, and $NL$ equal to the (rectangle contained)
by $AG$ and $GB$, $NL$ is thus also the mean proportional to
$CH$ and $KL$. Thus, as $CH$ is to $NL$, so $NL$ (is) to $KL$.
And for the same (reasons as the preceding propositions), the square on 
$CM$ is greater than (the square on) $MF$ by the (square) on (some
straight-line) incommensurable (in length) with ($CM$) [Prop. 10.18]. And neither
of them is commensurable with the (previously) laid down rational 
(straight-line) $CD$. Thus, $CF$ is a sixth apotome 
[Def. 10.16]. (Which is) the very thing it was required to show.}
\end{Parallel}

%%%%%
%10.103
%%%%%
\pdfbookmark[1]{Proposition 10.103}{pdf10.103}
\begin{Parallel}{}{}
\ParallelLText{
\begin{center}
{\large\ggn{103}.}
\end{center}\vspace*{-7pt}

\gr{<H t~h| >apotom~h| m'hkei s'ummetroc >apotom'h >esti ka`i t~h| t'axei <h
a>ut'h.}

\epsfysize=0.7in
\centerline{\epsffile{Book10/fig103g.eps}}

\gr{>'Estw >apotom`h <h AB, ka`i t~h| AB m'hkei s'ummetroc >'estw <h GD;
l'egw, <'oti ka`i <h GD >apotom'h >esti ka`i t~h| t'axei <h a>ut`h t~h|
AB.}

\gr{>Epe`i g`ar >apotom'h >estin <h AB, >'estw a>ut~h| prosarm'ozousa
<h BE; a<i AE, EB >'ara <rhta'i e>isi dun'amei m'onon s'ummetroi.
ka`i t~w| t~hc AB pr`oc t`hn GD l'ogw| <o a>ut`oc gegon'etw <o t~hc
BE pr`oc t`hn
 DZ; ka`i <wc <`en >'ara pr`oc <'en, p'anta [>est`i] pr`oc p'anta; >'estin >'ara ka`i <wc <'olh <h AE pr`oc <'olhn t`hn
GZ, o<'utwc <h AB pr`oc t`hn GD. s'ummetroc d`e <h AB t~h|
GD m'hkei; s'ummetroc >'ara ka`i <h AE m`en t~h| GZ, <h d`e BE t~h|
DZ. ka`i a<i AE, EB <rhta'i e>isi dun'amei m'onon s'ummetroi; ka`i
a<i
GZ, ZD >'ara <rhta'i e>isi dun'amei m'onon s'ummetroi
[>apotom`h >'ara >est`in <h GD. l'egw d'h, <'oti ka`i t~h| t'axei
<h a>ut`h t~h| AB].}

\gr{>Epe`i o>~un >estin <wc <h AE pr`oc t`hn GZ, o<'utwc <h BE pr`oc t`hn
DZ, >enall`ax >'ara >est`in <wc <h AE pr`oc t`hn EB, o<'utwc <h GZ
pr`oc t`hn ZD. >'htoi d`h <h AE t~hc EB me~izon d'unatai t~w| >ap`o
summ'etrou <eaut~h| >`h t~w| >ap`o >asumm'etrou. e>i m`en o>~un
<h AE t~hc EB me~izon d'unatai t~w| >ap`o summ'etrou <eaut~h|,
ka`i <h GZ t~hc ZD me~izon dun'hsetai t~w| >ap`o summ'etrou
<eaut~h|. ka`i e>i m`en s'ummetr'oc >estin <h AE t~h| >ekkeim'enh| <rht~h|
m'hkei, ka`i <h GZ, e>i d`e <h BE, ka`i <h DZ, e>i d`e o>udet'era
t~wn AE, EB, ka`i o>udet'era t~wn GZ, ZD. e>i d`e <h AE [t~hc EB]
me~izon d'unatai t~w| >ap`o >asumm'etrou <eaut~h|, ka`i <h GZ t~hc
ZD me~izon dun'hsetai t~w| >ap`o >asumm'etrou <eaut~h|. ka`i e>i m`en
s'ummetr'oc >estin <h AE t~h| >ekkeim'enh| <rht~h| m'hkei, ka`i <h GZ,
e>i d`e <h BE, ka`i <h DZ, e>i d`e o>udet'era t~wn AE, EB, o>udet'era
t~wn GZ, ZD.}

\gr{>Apotom`h >'ara >est`in <h GD ka`i t~h| t'axei <h a>ut`h
t~h| AB; <'oper >'edei de~ixai.}}

\ParallelRText{
\begin{center}
{\large Proposition 103}
\end{center}

A (straight-line) commensurable in length
with an apotome is an apotome, and  (is) the same in order.

\epsfysize=0.7in
\centerline{\epsffile{Book10/fig103e.eps}}

Let $AB$ be an apotome, and let $CD$ be commensurable in length with $AB$. I say that $CD$ is also an apotome, and (is) the same in order as $AB$.

For since $AB$ is an apotome, let $BE$ be an attachment to it. Thus,
$AE$ and $EB$ are rational (straight-lines which are) commensurable in
square only [Prop. 10.73]. And let it have
been contrived that the (ratio)
of $BE$ to $DF$ is the same as the ratio of $AB$ to $CD$
[Prop. 6.12]. Thus, also, as one is to one, (so)
all [are] to all [Prop. 5.12]. And thus as the whole $AE$ is to the whole $CF$, so $AB$
(is) to $CD$. And $AB$ (is) commensurable in length with $CD$.
$AE$ (is) thus also commensurable (in length) with $CF$, and $BE$
with $DF$ [Prop. 10.11]. And $AE$ and $BE$
are rational (straight-lines which are) commensurable in square only.
Thus, $CF$ and $FD$ are also rational (straight-lines which are) commensurable in square only [Prop. 10.13].
[$CD$ is thus an apotome. So, I say that (it is) also the same in order as $AB$.]

Therefore, since as $AE$ is to $CF$, so $BE$ (is) to $DF$, thus, alternately,
as $AE$ is to $EB$, so $CF$ (is) to $FD$ [Prop. 5.16]. So,  the square on $AE$
is greater than (the square on) $EB$ either by the (square) on (some straight-line)
commensurable,  or by the (square) on (some straight-line)
incommensurable, (in length) with ($AE$). Therefore, if the
(square) on $AE$ is greater than (the square on) $EB$ by the (square)
on (some straight-line) commensurable (in length) with ($AE$) then the
square on $CF$ will also be greater than (the square on) $FD$ by the (square)
on (some straight-line) commensurable (in length) with ($CF$)
[Prop. 10.14]. And if $AE$ is commensurable
in length with a (previously) laid down rational (straight-line) then so (is)
$CF$ [Prop. 10.12], and if $BE$ (is commensurable), so (is) $DF$, and if neither of
$AE$ or $EB$ (are commensurable), neither  (are) either of $CF$ or $FD$
[Prop. 10.13]. And if the (square) on
$AE$ is greater [than (the square on) $EB$] by the (square) on
(some straight-line) incommensurable (in length) with ($AE$) then the
(square) on $CF$ will also be greater than (the square on) $FD$ by the (square)
on (some straight-line) incommensurable (in length) with ($CF$)
[Prop. 10.14].  And if $AE$ is commensurable
in length with a (previously) laid down rational (straight-line), so (is)
$CF$ [Prop. 10.12], and if $BE$ (is commensurable), so (is) $DF$, and if neither of
$AE$ or $EB$ (are commensurable),  neither (are)  either of $CF$ or $FD$
[Prop. 10.13].

Thus, $CD$ is an apotome, and (is) the same in order as $AB$ [Defs.~10.11---10.16].
(Which is) the very thing it was required to show.}
\end{Parallel}

%%%%%
%10.104
%%%%%
\pdfbookmark[1]{Proposition 10.104}{pdf10.104}
\begin{Parallel}{}{}
\ParallelLText{
\begin{center}
{\large \ggn{104}.}
\end{center}\vspace*{-7pt}

\gr{<H t~h| m'eshc >apotom~h| s'ummetroc m'eshc >apotom'h >esti
ka`i t~h| t'axei <h a>ut'h.}\\

\epsfysize=0.7in
\centerline{\epsffile{Book10/fig103g.eps}}

\gr{>'Estw m'eshc >apotom`h <h AB, ka`i t~h| AB m'hkei s'ummetroc 
>'estw <h GD; l'egw, <'oti ka`i <h GD m'eshc >apotom'h >esti
ka`i t~h| t'axei <h a>ut`h t~h| AB.}

\gr{>Epe`i g`ar m'eshc >apotom'h >estin <h AB, >'estw a>ut~h| prosarm'ozousa
<h EB. a<i AE, EB >'ara m'esai e>is`i dun'amei m'onon s'ummetroi.
ka`i gegon'etw <wc <h AB pr`oc t`hn GD, o<'utwc <h BE pr`oc
t`hn DZ; s'ummetroc >'ara [>est`i] ka`i <h AE t~h| GZ, <h d`e BE t~h| DZ.
a<i d`e AE, EB m'esai e>is`i dun'amei m'onon s'ummetroi; ka`i a<i
GZ, ZD >'ara m'esai e>is`i dun'amei m'onon s'ummetroi;
m'eshc >'ara >apotom'h >estin <h GD. l'egw d'h, <'oti ka`i
t~h| t'axei >est`in <h a>ut`h t~h| AB.}

\gr{>Epe`i [g'ar] >estin <wc <h AE pr`oc t`hn EB, o<'utwc
<h GZ pr`oc t`hn ZD [>all> <wc m`en <h AE pr`oc t`hn EB, o<'utwc
t`o >ap`o t~hc AE pr`oc t`o <up`o t~wn AE, EB, <wc d`e <h GZ pr`oc
t`hn ZD, o<'utwc t`o >ap`o t~hc GZ pr`oc t`o <up`o t~wn GZ, ZD], >'estin
>'ara ka`i <wc t`o >ap`o t~hc AE pr`oc t`o <up`o t~wn AE, EB, o<'utwc
t`o >ap`o t~hc GZ pr`oc t`o <up`o t~wn GZ, ZD [ka`i >enall`ax
<wc t`o >ap`o t~hc AE pr`oc t`o >ap`o t~hc GZ, o<'utwc t`o <up`o
t~wn AE, EB pr`oc t`o <up`o t~wn GZ, ZD]. s'ummetron d`e t`o
>ap`o t~hc AE t~w| >ap`o t~hc GZ; s'ummetron >'ara >est`i ka`i t`o
<up`o t~wn AE, EB t~w| <up`o t~wn GZ, ZD.
e>'ite o>~un <rht'on >esti t`o <up`o t~wn AE, EB, <rht`on >'estai
ka`i t`o <up`o t~wn GZ, ZD, e>'ite m'eson [>est`i] t`o <up`o
t~wn AE, EB, m'eson [>est`i] ka`i t`o <up`o t~wn GZ, ZD.}

\gr{M'eshc >'ara >apotom'h >estin <h GD ka`i t~h| t'axei <h
a>ut`h t~h| AB; <'oper >'edei de~ixai.}}

\ParallelRText{
\begin{center}
{\large Proposition 104}
\end{center}

A (straight-line) commensurable (in length)
with an apotome of a medial (straight-line) is an apotome of a
medial (straight-line), and (is) the same in order.

\epsfysize=0.7in
\centerline{\epsffile{Book10/fig103e.eps}}

Let $AB$ be an apotome of a medial (straight-line), and let $CD$
be commensurable in length with $AB$. I say that $CD$ is also an
apotome of a medial (straight-line), and (is) the same in order as $AB$.

For since $AB$ is an apotome of a medial (straight-line), let $EB$ be
an attachment to it. Thus, $AE$ and $EB$ are medial (straight-lines which are) commensurable in square only [Props.~10.74, 10.75]. And let it have been contrived that as
$AB$ is to $CD$, so $BE$ (is) to $DF$ [Prop. 6.12]. 
Thus, $AE$ [is] also commensurable (in length) with $CF$, and $BE$
with $DF$ [Props.~5.12, 10.11]. And $AE$ and $EB$ are medial (straight-lines which are) commensurable in square only. $CF$ and $FD$
are thus also medial (straight-lines which are) commensurable in square only
[Props.~10.23, 10.13]. 
Thus, $CD$ is an apotome of a medial (straight-line) [Props.~10.74, 10.75].
So, I say that it is also the same in order as $AB$.

\mbox{[}For] since as $AE$ is to $EB$, so $CF$ (is) to $FD$ [Props.~5.12, 5.16] [but
as $AE$ (is) to $EB$, so the (square) on $AE$ (is) to the (rectangle
contained) by $AE$ and $EB$, and as $CF$ (is) to $FD$, so the
(square) on $CF$ (is) to the (rectangle contained) by $CF$ and $FD$],
thus as  the (square) on $AE$ is to the (rectangle contained) by $AE$
and $EB$, so the (square) on $CF$ also (is) to the (rectangle contained)
by $CF$ and $FD$ [Prop. 10.21~lem.] [and,
alternately, as the (square) on $AE$ (is) to the (square) on $CF$, so
the (rectangle contained) by $AE$ and $EB$ (is) to the (rectangle contained)
by $CF$ and $FD$]. And the (square) on $AE$ (is) commensurable
with the (square) on $CF$. Thus, the (rectangle contained) by $AE$ and
$EB$ is also commensurable with the (rectangle contained) by $CF$ and
$FD$ [Props.~5.16, 10.11].
Therefore, either the (rectangle contained) by $AE$ and $EB$ is rational, and
the (rectangle contained) by $CF$ and $FD$ will also be rational [Def. 10.4], or
the (rectangle contained) by $AE$ and $EB$ [is] medial, and the
(rectangle contained) by $CF$ and $FD$ [is] also medial [Prop. 10.23~corr.].

Therefore, $CD$ is the apotome of a medial (straight-line), and is
the same in order as $AB$ [Props.~10.74, 10.75]. (Which is) the very thing it was required to show.}
\end{Parallel}

%%%%%
%10.105
%%%%%
\pdfbookmark[1]{Proposition 10.105}{pdf10.105}
\begin{Parallel}{}{}
\ParallelLText{
\begin{center}
{\large \ggn{105}.}
\end{center}\vspace*{-7pt}

\gr{<H t~h| >el'assoni s'ummetroc >el'asswn >est'in.}\\

\epsfysize=0.7in
\centerline{\epsffile{Book10/fig103g.eps}}

\gr{>'Estw g`ar >el'asswn <h AB ka`i t~h| AB s'ummetroc <h GD; l'egw,
<'oti ka`i <h GD >el'asswn >est'in.}

\gr{Gegon'etw g`ar t`a a>ut'a; ka`i >epe`i a<i AE, EB dun'amei e>is`in
>as'ummetroi, ka`i a<i GZ, ZD >'ara dun'amei e>is`in >as'ummetroi.
>epe`i o>~un >estin <wc <h AE pr`oc t`hn
EB, o<'utwc <h GZ pr`oc t`hn ZD, >'estin >'ara ka`i <wc t`o >ap`o
t~hc AE pr`oc t`o >ap`o t~hc EB, o<'utwc t`o >ap`o t~hc GZ pr`oc t`o
>ap`o t~hc ZD. sunj'enti >'ara >est`in <wc t`a >ap`o t~wn AE, EB pr`oc
t`o >ap`o t~hc EB, o<'utwc t`a >ap`o t~wn GZ, ZD pr`oc t`o >ap`o t~hc
ZD [ka`i >enall'ax];  s'ummetron d'e >esti t`o >ap`o t~hc BE t~w| >ap`o
t~hc DZ; s'ummetron >'ara ka`i t`o sugke'imenon >ek t~wn >ap`o
t~wn AE, EB tetrag'wnwn t~w| sugkeim'enw| >ek t~wn >ap`o t~wn
GZ, ZD tetrag'wnwn. <rht`on d'e >esti t`o sugke'imenon >ek t~wn
>ap`o t~wn AE, EB tetrag'wnwn; <rht`on >'ara >est`i ka`i t`o sugke'imenon
>ek t~wn >ap`o t~wn GZ, ZD tetrag'wnwn. p'alin, >epe'i >estin <wc t`o
>ap`o t~hc AE pr`oc t`o <up`o t~wn AE, EB, o<'utwc t`o >ap`o t~hc
GZ pr`oc t`o <up`o t~wn GZ, ZD, s'ummetron d`e t`o >ap`o t~hc AE
tetr'agwnon t~w| >ap`o t~hc GZ tetrag'wnw|, s'ummetron >'ara >est`i
ka`i t`o <up`o t~wn AE, EB t~w| <up`o t~wn GZ, ZD. m'eson d`e t`o
<up`o t~wn AE, EB; m'eson >'ara ka`i t`o <up`o t~wn GZ, ZD; a<i
GZ, ZD >'ara dun'amei e>is`in >as'ummetroi poio~usai t`o m`en sugke'imenon >ek t~wn >ap> a>ut~wn tetrag'wnwn <rht'on, t`o d>
<up> a>ut~wn m'eson.}

\gr{>El'asswn >'ara >est`in <h GD; <'oper >'edei de~ixai.}}

\ParallelRText{
\begin{center}
{\large Proposition 105}
\end{center}

A (straight-line) commensurable (in length) with a minor (straight-line) is a minor (straight-line).

\epsfysize=0.7in
\centerline{\epsffile{Book10/fig103e.eps}}

For let $AB$ be a minor (straight-line), and (let) $CD$ (be) commensurable
(in length) with $AB$. I say that $CD$ is also a minor (straight-line).

For let the same things have been contrived (as in the former proposition).
And since $AE$ and $EB$ are (straight-lines which are) incommensurable in square [Prop. 10.76], $CF$ and $FD$ are thus also
(straight-lines which are)  incommensurable in square [Prop. 10.13].
Therefore, since as $AE$ is to $EB$, so $CF$ (is) to $FD$ [Props.~5.12, 5.16], thus also
as the (square) on $AE$ is to the (square) on $EB$, so the (square)
on $CF$ (is) to the (square) on $FD$ [Prop. 6.22]. Thus, via composition,
as the (sum of the squares) on $AE$ and $EB$ is to the (square) on $EB$,
so the (sum of the squares) on $CF$ and $FD$  (is) to the (square) on $FD$  [Prop. 5.18], [also alternately]. And the
(square) on $BE$ is commensurable with the (square) on $DF$ [Prop. 10.104]. The sum of the squares on $AE$ and $EB$ (is) thus also commensurable with the sum of the squares
on $CF$ and $FD$ [Prop. 5.16, 10.11].  And the sum of the (squares) on
$AE$ and $EB$ is rational [Prop. 10.76].
Thus, the sum of the (squares) on $CF$ and $FD$ is also rational
[Def. 10.4]. Again, since as the (square) on $AE$
is to the (rectangle contained) by $AE$ and $EB$, so
the (square) on $CF$ (is) to the (rectangle contained) by
$CF$ and $FD$ [Prop. 10.21~lem.],
and the square on $AE$ (is) commensurable with the square on
$CF$, the (rectangle contained) by $AE$ and $EB$ is thus also commensurable
with the (rectangle contained) by $CF$ and $FD$. And the
(rectangle contained) by $AE$ and $EB$ (is) medial [Prop. 10.76].  Thus, the (rectangle contained) by
$CF$ and $FD$ (is) also medial [Prop. 10.23~corr.]. $CF$ and $FD$
are thus (straight-lines which are) incommensurable in square, making the sum of the
squares on them rational, and the (rectangle contained) by them
medial.

Thus, $CD$ is a minor (straight-line) [Prop. 10.76]. (Which is) the very thing it
was required to show.}
\end{Parallel}

%%%%%
%10.106
%%%%%
\pdfbookmark[1]{Proposition 10.106}{pdf10.106}
\begin{Parallel}{}{}
\ParallelLText{
\begin{center}
{\large \ggn{106}.}
\end{center}\vspace*{-7pt}

\gr{<H t~h| met`a <rhto~u m'eson t`o <'olon poio'ush| s'ummetroc met`a
<rhto~u m'eson t`o <'olon poio~us'a >estin.}\\~\\

\epsfysize=0.7in
\centerline{\epsffile{Book10/fig103g.eps}}

\gr{>'Estw met`a <rhto~u m'eson t`o <'olon poio~usa <h AB ka`i t~h|
AB s'ummetroc <h GD; l'egw, <'oti ka`i <h GD met`a <rhto~u
m'eson t`o <'olon poio~us'a >estin.}

\gr{>'Estw g`ar t~h| AB prosarm'ozousa <h BE; a<i AE, EB >'ara dun'amei
e>is`in >as'ummetroi poio~usai t`o m`en sugke'imenon >ek t~wn >ap`o
t~wn AE, EB tetrag'wnwn m'eson, t`o d> <up> a>ut~wn <rht'on. ka`i
t`a a>ut`a kateskeu'asjw. <omo'iwc d`h de'ixomen to~ic pr'oteron,
<'oti a<i GZ, ZD >en t~w| a>ut~w| l'ogw| e>is`i ta~ic AE, EB,
ka`i s'ummetr'on >esti t`o sugke'imenon >ek t~wn >ap`o t~wn AE,
EB tetrag'wnwn t~w| sugkeim'enw| >ek t~wn >ap`o t~wn
GZ, ZD tetrag'wnwn, t`o d`e <up`o t~wn AE, EB t~w| <up`o
t~wn GZ, ZD; <'wste ka`i a<i GZ, ZD dun'amei e>is`in
>as'ummetroi poio~usai t`o m`en sugke'imenon >ek t~wn
>ap`o t~wn GZ, ZD tetrag'wnwn m'eson, t`o d> <up> a>ut~wn
<rht'on.}

\gr{<H GD >'ara met`a <rhto~u m'eson t`o  <'olon poio~us'a
>estin; <'oper >'edei de~ixai.}}

\ParallelRText{
\begin{center}
{\large Proposition 106}
\end{center}

A (straight-line) commensurable (in length)
with a (straight-line) which with a rational (area) makes a medial
whole is a (straight-line) which with a rational (area) makes a medial
whole.

\epsfysize=0.7in
\centerline{\epsffile{Book10/fig103e.eps}}

Let $AB$ be a (straight-line) which with a rational (area) makes
a medial whole, and (let) $CD$ (be) commensurable
(in length) with $AB$. I say that $CD$ is also a (straight-line)
which with a rational (area) makes a medial (whole).

For let $BE$ be an attachment to $AB$. Thus, $AE$ and
$EB$ are (straight-lines which are) incommensurable in square, making the sum of the
squares on $AE$ and $EB$ medial,  and the (rectangle contained)
by them rational [Prop. 10.77]. 
And let the same construction have been made (as in the previous propositions).
So, similarly
to the previous  (propositions), we can  show that $CF$ and $FD$
are in the same ratio as $AE$ and $EB$, and the sum of the squares
on $AE$ and $EB$ is commensurable with the sum of the squares on
$CF$ and $FD$, and the (rectangle contained) by $AE$ and $EB$
with the (rectangle contained) by $CF$ and $FD$. Hence, $CF$ and
$FD$ are also (straight-lines which are) incommensurable in square, making the sum of the
squares on $CF$ and $FD$ medial, and the (rectangle contained) by them
rational.

$CD$ is thus a (straight-line) which with a rational (area) makes a
medial whole [Prop. 10.77]. (Which is)
the very thing it was required to show.}
\end{Parallel}

%%%%%
%10.107
%%%%%
\pdfbookmark[1]{Proposition 10.107}{pdf10.107}
\begin{Parallel}{}{}
\ParallelLText{
\begin{center}
{\large \ggn{107}.}
\end{center}\vspace*{-7pt}

\gr{<H t~h| met`a m'esou m'eson t`o <'olon poio'ush| s'ummetroc ka`i a>ut`h
met`a m'esou m'eson t`o <'olon poio~us'a >estin.}\\~\\

\epsfysize=0.7in
\centerline{\epsffile{Book10/fig103g.eps}}

\gr{>'Estw met`a m'esou m'eson t`o <'olon poio~usa <h AB, ka`i t~h|
AB >'estw s'ummetroc <h GD; l'egw, <'oti ka`i <h GD met`a m'esou
m'eson t`o <'olon poio~us'a >estin.}

\gr{>'Estw g`ar t~h| AB prosarm'ozousa <h BE, ka`i t`a a>ut`a
kateskeu'asjw; a<i AE, EB >'ara dun'amei e<is`in >as'ummetroi
poio~usai t'o te sugke'imenon >ek t~wn >ap> a>ut~wn tetrag'wnwn
m'eson ka`i t`o <up> a>ut~wn m'eson ka`i >'eti >as'ummetron t`o
sugk'eimenon >ek t~wn >ap> a>ut~wn tetrag'wnwn t~w| <up>
a>ut~wn. ka'i e>isin, <wc >ede'iqjh, a<i AE, EB s'ummetroi ta~ic
GZ, ZD, ka`i t`o sugke'imenon >ek t~wn >ap`o t~wn AE, EB
tetrag'wnwn t~w| sugkeim'enw| >ek t~wn >ap`o t~wn GZ, ZD, t`o d`e
<up`o t~wn AE, EB t~w| <up`o t~wn GZ, ZD; ka`i a<i GZ, ZD >'ara
dun'amei e>is`in >as'ummetroi poio~usai t'o te sugke'imenon
>ek t~wn >ap> a>ut~wn tetrag'wnwn m'eson ka`i t`o <up>
>a>ut~wn m'eson ka`i >'eti >as'ummetron t`o sugke'imenon >ek t~wn
>ap> a>ut~wn [tetrag'wnwn] t~w| <up> a>ut~wn.}

\gr{<H GD >'ara met`a m'esou m'eson t`o <'olon poio~us'a
>estin; <'oper >'edei de~ixai.}}

\ParallelRText{
\begin{center}
{\large Proposition 107}
\end{center}

A (straight-line) commensurable
(in length) with a (straight-line) which with a medial (area) makes a
medial whole is  itself also a (straight-line) which with a medial (area) makes
a medial whole.

\epsfysize=0.7in
\centerline{\epsffile{Book10/fig103e.eps}}

Let $AB$ be a (straight-line) which with a medial (area) makes a medial
whole, and let $CD$ be commensurable (in length) with $AB$.
I say that $CD$ is also a (straight-line) which with a medial (area) makes
a medial whole.

For let $BE$ be an attachment to $AB$. And let the same
construction have been made (as in the previous propositions). Thus,
$AE$ and $EB$ are (straight-lines which are) incommensurable in square, making
the sum of the squares on them medial, and the (rectangle contained)
by them medial, and, further, the sum of the squares on them incommensurable with the (rectangle contained) by them [Prop. 10.78].  And, as was shown (previously),
$AE$ and $EB$ are commensurable (in length) with $CF$ and $FD$ (respectively),
and the sum of the squares on $AE$ and $EB$ with the sum of the squares on
$CF$ and $FD$, and the (rectangle contained) by $AE$ and $EB$ with
the (rectangle contained) by $CF$ and $FD$. Thus, $CF$ and
$FD$ are also (straight-lines which are) incommensurable in square, making the sum of the squares on
them medial, and the (rectangle contained) by them medial, and, further,
the sum of the [squares] on them incommensurable with the (rectangle
contained) by them.

Thus, $CD$ is a (straight-line) which with a medial (area) makes a medial
whole [Prop. 10.78]. (Which is) the very thing
it was required to show.}
\end{Parallel}

%%%%%
%10.108
%%%%%
\pdfbookmark[1]{Proposition 10.108}{pdf10.108}
\begin{Parallel}{}{}
\ParallelLText{
\begin{center}
{\large \ggn{108}.}
\end{center}\vspace*{-7pt}

\gr{>Ap`o <rhto~u m'esou >afairoum'enou <h t`o loip`on qwr'ion dunam'enh 
m'ia d'uo >al'ogwn g'inetai >'htoi >apotom`h >`h >el'asswn.}\\~\\

\epsfysize=1.5in
\centerline{\epsffile{Book10/fig108g.eps}}

\gr{>Ap`o g`ar <rhto~u to~u BG m'eson >afh|r'hsjw t`o BD; l'egw,
<'oti <h t`o loip`on dunam'enh t`o EG m'ia d'uo >al'ogwn g'inetai
>'htoi >apotom`h >`h >el'asswn.}

\gr{>Ekke'isjw g`ar <rht`h <h ZH, ka`i t~w| m`en BG >'ison
par`a t`hn ZH parabebl'hsjw >orjog'wnion parallhl'ogrammon
t`o HJ, t~w| d`e DB >'ison >afh|r'hsjw t`o HK;
loip`on >'ara t`o EG >'ison >est`i t~w| LJ. >epe`i o>~un
<rht`on m'en >esti t`o BG, m'eson d`e t`o BD, >'ison d`e t`o m`en
BG t~w| HJ, t`o d`e BD t~w| HK, <rht`on m`en >'ara >est`i t`o HJ,
m'eson d`e t`o HK. ka`i par`a <rht`hn t`hn ZH par'akeitai;  <rht`h
m`en >'ara <h ZJ ka`i s'ummetroc t~h| ZH m'hkei, <rht`h d`e <h
ZK ka`i >as'ummetroc t~h| ZH m'hkei; >as'ummetroc >'ara >est`in
<h ZJ t~h| ZK m'hkei. a<i ZJ, ZK >'ara <rhta'i e>isi dun'amei
m'onon s'ummetroi; >apotom`h >'ara >est`in <h KJ, prosarm'ozousa
d`e a>ut~h| <h KZ. >'htoi d`h <h JZ t~hc ZK me~izon d'unatai
t~w| >ap`o summ'etrou >`h o>'u.}

\gr{Dun'asjw pr'oteron t~w| >ap`o summ'etrou. ka'i >estin <'olh <h JZ
s'ummetroc t~h| >ekkeim'enh| <rht~h| m'hkei t~h| ZH;
>apotom`h >'ara pr'wth >est`in <h KJ. t`o d> <up`o <rht~hc
ka`i >apotom~hc pr'wthc perieq'omenon <h dunam'enh
>apotom'h >estin. <h >'ara t`o LJ, tout'esti t`o EG, dunam'enh
>apotom'h >estin.}

\gr{E>i d`e <h JZ t~hc ZK me~izon d'unatai t~w| >ap`o >asumm'etrou
<eaut~h|, ka'i >estin <'olh <h ZJ s'ummetroc t~h| >ekkeim'enh| <rht~h|
m'hkei t~h| ZH, >apotom`h tet'arth >est`in <h KJ. t`o d> <up`o
<rht~hc ka`i
 >apotom~hc tet'arthc perieq'omenon <h dunam'enh
>el'asswn >est'in; <'oper >'edei de~ixai.}}

\ParallelRText{
\begin{center}
{\large Proposition 108}
\end{center}

A
medial (area) being subtracted from a rational (area), one of two irrational (straight-lines)
arise (as) the square-root of the remaining area---either an apotome, or a minor (straight-line).

\epsfysize=1.5in
\centerline{\epsffile{Book10/fig108e.eps}}

For let the medial (area) $BD$ have been subtracted from the
rational (area) $BC$. I say that  one of two irrational (straight-lines) arise
(as) the square-root of the
remaining (area), $EC$---either
an apotome, or a minor (straight-line).

For let the rational (straight-line) $FG$ have been laid out,
and let the right-angled parallelogram $GH$, equal to $BC$, have
been applied to $FG$, and let $GK$, equal to $DB$, have
been subtracted (from $GH$).  Thus, the remainder $EC$
is equal to $LH$. Therefore, since $BC$
is a rational (area), and $BD$ a medial (area), and $BC$
(is) equal to $GH$, and $BD$ to $GK$, $GH$ is thus a rational (area),
and $GK$ a medial (area). And they are applied to the rational (straight-line)
$FG$. Thus, $FH$ (is) rational, and commensurable in length with $FG$
[Prop. 10.20], and $FK$ (is) also rational, and
incommensurable in length with $FG$ [Prop. 10.22]. Thus, $FH$ is incommensurable
in length with $FK$ [Prop. 10.13]. 
$FH$ and $FK$ are thus rational (straight-lines which are) commensurable
in square only. Thus, $KH$ is an apotome [Prop. 10.73], and $KF$ an attachment to it.
So, the square on $HF$ is  greater than (the square on) $FK$ by
the (square) on (some straight-line which is) either commensurable,
or not (commensurable), (in length with $HF$).

First, let the square (on it) be (greater) by the (square) on  (some straight-line
which is) commensurable
(in length with $HF$). And the whole of $HF$ is commensurable
in length with the (previously) laid down rational (straight-line) $FG$. 
Thus, $KH$ is a first apotome [Def. 10.1]. And
the square-root of an (area) contained by a rational (straight-line)
and a first apotome is an apotome [Prop. 10.91]. 
Thus, the square-root of $LH$---that is to say, (of) $EC$---is an apotome.

And if the square on  $HF$ is greater than (the square on) $FK$ by the
(square) on (some straight-line which is) incommensurable (in length) with ($HF$),
and (since) the whole of $FH$ is commensurable in length with
the (previously) laid down rational (straight-line) $FG$, $KH$ is a fourth
apotome [Prop. 10.14]. And the square-root
of an (area) contained by a rational (straight-line) and a fourth apotome
is a minor (straight-line)
[Prop. 10.94]. (Which is) the very thing it was required to show.}
\end{Parallel}

%%%%%
%10.109
%%%%%
\pdfbookmark[1]{Proposition 10.109}{pdf10.109}
\begin{Parallel}{}{}
\ParallelLText{
\begin{center}
{\large \ggn{109}.}
\end{center}\vspace*{-7pt}

\gr{>Ap`o m'esou <rhto~u >afairoum'enou >'allai d'uo >'alogoi g'inontai
>'htoi m'eshc >apotom`h pr'wth >`h met`a <rhto~u m'eson t`o <'olon poio~usa.}

\gr{>Ap`o g`ar m'esou to~u BG <rht`on >afh|r'hsjw t`o BD. l'egw, <'oti
<h t`o loip`on t`o EG dunam'enh m'ia d'uo >al'ogwn g'inetai >'htoi
m'eshc >apotom`h pr'wth >`h met`a <rhto~u m'eson t`o <'olon
poio~usa.}

\gr{>Ekke'isjw g`ar <rht`h <h ZH, ka`i parabebl'hsjw <omo'iwc
t`a qwr'ia. >'esti d`h >akolo'ujwc <rht`h m`en <h ZJ ka`i >as'ummetroc
t~h| ZH m'hkei, <rht`h d`e <h KZ ka`i s'ummetroc t~h| ZH m'hkei; a<i
ZJ, ZK >'ara <rhta'i e>isi dun'amei m'onon s'ummetroi; >apotom`h
>'ara >est`in <h KJ, prosarm'ozousa d`e ta'uth| <h ZK. >'htoi d`h <h JZ
t~hc ZK me~izon d'unatai t~w| >ap`o summ'etrou <eaut~h| >`h
t~w| >ap`o >asumm'etrou.}\\~\\~\\~\\~\\~\\~\\~\\~\\

\epsfysize=2.in
\centerline{\epsffile{Book10/fig109g.eps}}

\gr{E>i m`en o>~un <h JZ t~hc ZK me~izon d'unatai t~w| >ap`o summ'etrou
<eaut~h|, ka'i >estin <h prosarm'ozousa <h ZK s'ummetroc t~h| >ekkeim'enh|
<rht~h| m'hkei t~h| ZH, >apotom`h deut'era >est`in <h KJ. <rht`h d`e
<h ZH; <'wste <h t`o LJ, tout'esti t`o EG, dunam'enh m'eshc >apotom`h
pr'wth >est'in.}

\gr{E>i d`e <h JZ t~hc ZK me~izon d'unatai t~w| >ap`o >asumm'etrou, ka'i
>estin <h prosarm'ozousa <h ZK s'ummet\-roc t~h| >ekkeim'enh| <rht~h|
m'hkei t~h| ZH, >apotom`h p'empth >est`in <h KJ; <'wste <h t`o
EG dunam'enh met`a <rhto~u m'eson t`o <'olon poio~us'a
>estin; <'oper >'edei de~ixai.}}

\ParallelRText{
\begin{center}
{\large Proposition 109}
\end{center}

A rational (area) being subtracted from
a medial (area), two other irrational (straight-lines) arise (as the square-root
of the remaining area)---either a first apotome of a medial (straight-line),
or that (straight-line) which with a rational (area) makes a medial whole.

For let the rational (area) $BD$ have been subtracted from the medial
(area) $BC$. I say that one of two irrational (straight-lines) arise (as) the square-root of the remaining (area), $EC$---either a first apotome of a medial
(straight-line), or that (straight-line) which with a rational (area) makes
a medial whole.

For let the rational (straight-line) $FG$ be laid down, and let
similar areas (to the preceding proposition) have  been applied (to it). So,
accordingly, $FH$ is rational, and incommensurable in length with
$FG$, and $KF$ (is) also rational, and commensurable
in length with $FG$. Thus, $FH$ and $FK$ are rational (straight-lines which
are) commensurable in square only [Prop. 10.13].
$KH$ is thus an apotome [Prop. 10.73], and $FK$ an attachment to it. So, the square on $HF$ is greater than (the square on)
$FK$ either by the (square) on (some straight-line) commensurable (in length)
with ($HF$), or by the (square) on (some straight-line) incommensurable
(in length with $HF$).

\epsfysize=2.in
\centerline{\epsffile{Book10/fig109e.eps}}

Therefore, if the square on $HF$ is greater than (the square on) $FK$
by the (square) on (some straight-line) commensurable (in length) with ($HF$),
and (since) the attachment $FK$ is commensurable
in length with the (previously) laid down rational (straight-line) $FG$,
$KH$ is a second apotome [Def. 10.12]. 
And $FG$ (is) rational. Hence, the square-root of $LH$---that is to say, (of)
$EC$---is a first apotome of a medial (straight-line) [Prop. 10.92].

And if the square on $HF$ is greater than (the square on) $FK$
by the (square) on (some straight-line) incommensurable (in length with $HF$),
and (since) the attachment $FK$ is commensurable
in length with the (previously) laid down rational (straight-line) $FG$,
$KH$ is a fifth apotome [Def. 10.15]. Hence,
the square-root of $EC$ is that (straight-line) which with a rational (area)
makes a medial whole [Prop. 10.95]. 
(Which is) the very thing it was required to show.}
\end{Parallel}

%%%%%
%10.110
%%%%%
\pdfbookmark[1]{Proposition 10.110}{pdf10.110}
\begin{Parallel}{}{}
\ParallelLText{
\begin{center}
{\large \ggn{110}.}
\end{center}\vspace*{-7pt}

\gr{>Ap`o m'esou m'esou >afairoum'enou >asumm'etrou t~w| <'olw| a<i loipa`i
d'uo >'alogoi g'inontai >'htoi m'eshc >apotom`h deut'era >`h met`a
m'esou m'eson t`o <'olon poio~usa.}

\gr{>Afh|r'hsjw g`ar <wc >ep`i t~wn prokeim'enwn katagraf~wn >ap`o m'esou
to~u BG m'eson t`o BD >as'ummetron t~w| <'olw|; l'egw, <'oti
<h t`o EG dunam'enh m'ia >est`i d'uo >al'ogwn >'htoi m'eshc
>apotom`h deut'era >`h met`a m'esou m'eson t`o <'olon
poio~usa.}\\~\\~\\~\\~\\~\\

\epsfysize=2.2in
\centerline{\epsffile{Book10/fig109g.eps}}

\gr{>Epe`i g`ar m'eson >est`in <ek'ateron t~wn BG, BD, ka`i >as'ummetron
t`o BG t~w| BD, >'estai >akolo'ujwc <rht`h <ekat'era t~wn ZJ, ZK
ka`i >as'ummetroc t~h| ZH m'hkei. ka`i >epe`i >as'ummetr'on >esti
t`o BG t~w| BD, tout'esti t`o HJ t~w| HK, >as'ummetroc ka`i <h JZ
t~h| ZK; a<i ZJ, ZK >'ara <rhta'i e>isi dun'amei m'onon s'ummetroi;
>apotom`h >'ara >est`in <h KJ [prosarm'ozousa d`e <h ZK. >'htoi
d`h <h ZJ t~hc ZK me~izon d'unatai t~w| >ap`o summ'etrou
>`h t~w| >ap`o >asumm'etrou <eaut~h|].}

\gr{E>i m`en d`h <h ZJ t~hc ZK me~izon d'unatai t~w| >ap`o summ'etrou
<eaut~h|, ka`i o>ujet'era t~wn ZJ, ZK s'ummetr'oc >esti t~h|
>ekkeim'emnh| <rht~h| m'hkei t~h| ZH, >apotom`h tr'ith >est`in
<h KJ. <rht`h d`e <h KL, t`o d> <up`o <rht~hc
ka`i >apotom~hc tr'ithc perieq'omenon >orjog'wnion >'alog'on
>estin, ka`i <h dunam'enh a>ut`o >'alog'oc >estin, kale~itai d`e m'eshc
>apotom`h deut'era; <'wste <h t`o LJ, tout'esti t`o EG, dunam'enh
m'eshc >apotom'h >esti deuter'a.}

\gr{E>i d`e <h ZJ t~hc ZK me~izon d'unatai t~w| >ap`o
>asumm'etrou <eaut~h| [m'hkei], ka`i o>ujet'era t~wn JZ, ZK
s'ummetr'oc >esti t~h| ZH m'hkei, >apotom`h <'ekth >est`in <h KJ.
t`o d> <up`o <rht~hc ka`i >apotom~hc <'ekthc
<h dunam'enh >est`i met`a m'esou m'eson t`o <'olon poio~usa.
<h t`o LJ >'ara, tout'esti t`o  EG, dunam'enh met`a m'esou m'eson
t`o <'olon poio~us'a >estin; <'oper >'edei de~ixai.}}

\ParallelRText{
\begin{center}
{\large Proposition 110}
\end{center}

A medial (area), incommensurable with the whole, being subtracted
from a medial (area),  the two remaining
irrational (straight-lines) arise (as) the (square-root of the area)---either a second apotome of a medial (straight-line), 
or that (straight-line) which with a medial (area) makes a medial whole.

For, as in the previous figures, let the medial (area) $BD$, incommensurable with the whole, have been subtracted from the medial (area) $BC$. I say that
the square-root of $EC$ is one of two irrational (straight-lines)---either
a second apotome of a medial (straight-line), or that (straight-line) which
with a medial (area) makes a medial whole.

\epsfysize=2.2in
\centerline{\epsffile{Book10/fig109e.eps}}

For since $BC$ and $BD$ are each medial (areas), and $BC$ (is) incommensurable with $BD$, accordingly,  $FH$ and $FK$ will each be rational (straight-lines), and incommensurable
in length with $FG$ [Prop. 10.22]. And
since $BC$ is incommensurable with $BD$---that is to say, $GH$ with
$GK$---$HF$ (is) also incommensurable (in length) with $FK$ [Props.~6.1, 10.11]. Thus,
$FH$ and $FK$ are rational (straight-lines which are) commensurable
in square only. $KH$ is thus as apotome [Prop. 10.73], [and $FK$ an attachment (to it).
So, the square on $FH$ is  greater than (the square on)
$FK$ either by the (square) on (some straight-line) commensurable, or by the (square) on (some straight-line) incommensurable, (in length)
with ($FH$).]

So, if the square on $FH$ is greater than (the square on) $FK$ by the
(square) on (some straight-line) commensurable (in length) with ($FH$),
and (since) neither of $FH$ and $FK$ is commensurable in length with
the (previously) laid down rational (straight-line) $FG$, $KH$ is a
third apotome [Def. 10.3]. And $KL$ (is) rational.
And the rectangle contained by a rational (straight-line) and a third
apotome is irrational, and the square-root of it is that irrational (straight-line)
called a second apotome of a medial (straight-line) [Prop. 10.93]. Hence, the square-root of $LH$---that is to say, (of) $EC$---is a second apotome of a medial (straight-line).

And if the square on $FH$ is greater than (the square on) $FK$ by the
(square) on (some straight-line) incommensurable [in length] with ($FH$), 
and (since) neither of $HF$ and $FK$ is commensurable in length
with $FG$, $KH$ is a sixth apotome [Def. 10.16].
And the square-root of the (rectangle contained) by a rational (straight-line) and a sixth apotome is that (straight-line) which with a medial (area)
makes a medial whole [Prop. 10.96]. Thus,
the square-root of $LH$---that is to say, (of) $EC$---is 
that (straight-line) which with a medial (area) makes a medial whole.
(Which is) the very thing it was required to show.}
\end{Parallel}

%%%%%
%10.111
%%%%%
\pdfbookmark[1]{Proposition 10.111}{pdf10.111}
\begin{Parallel}{}{}
\ParallelLText{
\begin{center}
{\large \ggn{111}.}
\end{center}\vspace*{-7pt}

\gr{<H >apotom`h o>uk >'estin <h a>ut`h t~h| >ek d'uo >onom'atwn.}

\epsfysize=2.4in
\centerline{\epsffile{Book10/fig111g.eps}}

\gr{>'Estw >apotom`h <h AB; l'egw, <'oti <h AB o>uk >'estin <h a>ut`h t~h|
>ek d'uo >onom'atwn.}

\gr{E>i g`ar dunat'on, >'estw; ka`i >ekke'isjw <rht`h <h DG, ka`i t~w|
>ap`o t~hc AB >'ison par`a t`hn GD parabebl'hsjw >orjog'wnion
t`o GE pl'atoc poio~un t`hn DE. >epe`i o>~un >apotom'h >estin <h AB, >apotom`h
pr'wth >est`in <h DE. >'estw a>ut~h| prosarm'ozousa <h EZ; a<i
DZ, ZE >'ara <rhta'i e>isi dun'amei m'onon s'ummetroi, ka`i
<h DZ t~hc ZE me~izon d'unatai t~w| >ap`o summ'etrou <eaut~h|,
ka`i <h DZ s'ummetr'oc >esti t~h| >ekkeim'enh| <rht~h| m'hkei
t~h| DG. p'alin, >epe`i >ek d'uo >onom'atwn >est`in <h AB, >ek
d'uo >'ara  >onom'atwn pr'wth >est`in <h DE. dih|r'hsjw e>ic
t`a >on'omata kat`a t`o H, ka`i >'estw me~izon >'onoma t`o DH;
a<i DH, HE >'ara <rhta'i e>isi dun'amei m'onon s'ummetroi,
ka`i <h DH t~hc HE me~izon d'unatai t~w| >ap`o
summ'etrou <eaut~h|, ka`i t`o me~izon <h DH s'ummetr'oc >esti
t~h| >ekkeim'enh| <rht~h| m'hkei t~h| DG. ka`i <h DZ >'ara t~h|
DH s'ummetr'oc >esti m'hkei; ka`i loip`h >'ara <h HZ s'ummetr'oc
>esti t~h| DZ m'hkei. [>epe`i o~un s'ummetr'oc >estin <h DZ
t~h| HZ, <rht`h d'e >estin <h DZ, <rht`h >'ara >est`i ka`i <h
HZ. >epe`i o>~un s'ummetr'oc >estin 
 <h DZ
t~h| HZ m'hkei] >as'ummetroc d`e <h DZ t~h| EZ m'hkei. 
>as'ummetroc >'ara >est`i ka`i <h ZH t~h| EZ m'hkei.
a<i
HZ, ZE >'ara  <rhta'i [e>isi] dun'amei m'onon s'ummetroi; >apotom`h
>'ara >est`in <h EH. >all`a ka`i <rht'h; <'oper >est`in >ad'unaton.}

\gr{<H >'ara >apotom`h o>uk >'estin <h a>ut`h t~h| >ek d'uo
>onom'atwn; <'oper >'edei de~ixai.}}

\ParallelRText{
\begin{center}
{\large Proposition 111}
\end{center}

An apotome is not the same as a binomial.

\epsfysize=2.4in
\centerline{\epsffile{Book10/fig111e.eps}}

Let $AB$ be an apotome. I say that $AB$ is not the same as a binomial.

For, if possible, let it be (the same). And let a rational (straight-line)
$DC$ be laid down. And let the rectangle $CE$, equal to the (square) on $AB$, have been applied to $CD$, producing $DE$ as breadth. Therefore,
since $AB$ is an apotome, $DE$ is a first apotome [Prop. 10.97]. Let $EF$ be an attachment to it.
Thus, $DF$ and $FE$ are rational (straight-lines which are) commensurable
in square only, and the square on $DF$ is greater than (the square on)
$FE$ by the (square) on (some straight-line) commensurable (in length)
with ($DF$), and $DF$ is commensurable in length with the (previously)
laid down rational (straight-line) $DC$ [Def. 10.10]. 
Again, since $AB$ is a binomial, $DE$ is thus a first binomial [Prop. 10.60]. Let ($DE$) have been divided
into its (component) terms at $G$, and let $DG$ be the greater term.
Thus, $DG$ and $GE$ are rational (straight-lines which are)
commensurable in square only, and the square on $DG$ is greater
than (the square on) $GE$ by the (square) on (some straight-line)
commensurable (in length) with ($DG$), and the greater (term)
$DG$ is commensurable in length with the (previously)
laid down rational (straight-line) $DC$ [Def. 10.5].
Thus, $DF$ is also commensurable in length with $DG$ [Prop. 10.12].
The remainder $GF$ is thus commensurable in length with $DF$
[Prop. 10.15]. [Therefore, since
$DF$ is commensurable with $GF$, and $DF$ is rational, $GF$
is thus also rational. Therefore, since $DF$ is commensurable
in length with $GF$,] $DF$ (is) incommensurable in length with $EF$.
Thus, $FG$ is also incommensurable in length with $EF$ [Prop. 10.13]. $GF$ and $FE$ [are] thus
rational (straight-lines which are) commensurable in square only.
Thus, $EG$ is an apotome [Prop. 10.73].
But, (it is) also rational. The very thing is impossible.

Thus, an apotome is not the same as a binomial. (Which is) the very thing
it was required to show.}
\end{Parallel}

\begin{Parallel}{}{}
\ParallelLText{
\begin{center}
{\large \gr{[P'orisma.]}}
\end{center}\vspace*{-7pt}

\gr{<H >apotom`h ka`i a<i met> a>ut`hn >'alogoi o>'ute
t~h| m'esh| o>'ute >all'hlaic e>is`in a<i a>uta'i.}

\gr{T`o m`en g`ar >ap`o m'eshc par`a <rht`hn paraball'omenon
pl'atoc poie~i <rht`hn ka`i >as'ummetron t~h|, par>
<`hn par'akeitai, m'hkei, t`o d`e >ap`o >apotom~hc par`a <rht`hn
paraball'omenon pl'atoc poie~i >apotom`hn pr'wthn, t`o
d`e >ap`o m'eshc >apotom~hc pr'wthc par`a <rht`hn paraball'omenon
pl'atoc poie~i >apotom`hn deut'eran, t`o d`e >ap`o
m'eshc >apotom~hc deut'erac par`a <rht`hn paraball'omenon
pl'atoc poie~i >apotom`hn tr'ithn, t`o d`e >ap`o >el'assonoc
par`a <rht`hn paraball'omenon pl'atoc poie~i
>apotom`hn tet'arthn, t`o d`e >ap`o t~hc met`a <rhto~u m'eson
t`o <'olon poio'ushc par`a <rht`hn paraball'omenon pl'atoc
poie~i >apotom`hn p'empthn, t`o d`e >ap`o t~hc met`a m'esou
m'eson t`o <'olon poio'ushc par`a <rht`hn paraball'omenon
pl'atoc poie~i >apotom`hn <'ekthn. >epe`i o>~un t`a e>irhm'ena
pl'ath diaf'erei to~u te pr'wtou ka`i >all'hlwn, to~u m`en pr'wtou,
<'oti <rht'h >estin, >all'hlwn d`e, >epe`i t~h| t'axei
o>uk e>is`in a<i a>uta'i, d~hlon, <wc ka`i a>uta`i a<i >'alogoi
diaf'erousin >all'hlwn. ka`i >epe`i d'edeiktai <h >apotom`h
o>uk o>~usa <h a>ut`h t~h| >ek d'uo >onom'atwn, poio~usi
d`e pl'ath par`a <rht`hn paraball'omenai a<i met`a t`hn >apotom`hn
>apotom`ac
>akolo'ujwc <ek'asth t~h| t'axei t~h| kaj> a<ut'hn, a<i d`e met`a t`hn
>ek d'uo >onom'atwn t`ac >ek d'uo >onom'atwn ka`i a>uta`i t~h|
t'axei
 >akolo'ujwc, <'eterai >'ara e>is`in a<i met`a
t`hn >apotom`hn ka`i <'eterai a<i met`a t`hn >ek d'uo
>onom'atwn, <wc e>~inai t~h| t'axei p'asac >al'ogouc \ov{ig},}}

\ParallelRText{
\begin{center}
{\large [Corollary]}
\end{center}\vspace*{-7pt}

The apotome and the  irrational (straight-lines) after it are neither the
same as a medial (straight-line) nor (the same) as one another.

For the (square) on a medial (straight-line), applied to a rational (straight-line), produces as breadth a rational (straight-line which is) incommensurable in length with the
(straight-line) to which  (the area) is applied [Prop. 10.22]. And the (square) on an apotome, applied to a rational (straight-line), produces as breadth a first apotome [Prop. 10.97]. 
And the (square) on a first
apotome of a medial (straight-line), applied to a rational (straight-line),
produces as breadth a second apotome [Prop. 10.98].
And the (square) on a second
apotome of a medial (straight-line), applied to a rational (straight-line),
produces as breadth a third apotome [Prop. 10.99].
And (square) on a minor (straight-line), applied to a rational (straight-line),
produces as breadth a fourth apotome [Prop. 10.100]. And (square) on that (straight-line) which with a rational (area) produces a medial whole, applied to a rational (straight-line),
produces as breadth a fifth apotome [Prop. 10.101]. And (square) on that (straight-line) which with a medial (area) produces a medial whole, applied to a rational
 (straight-line), produces as breadth a sixth apotome [Prop. 10.102]. Therefore, since the aforementioned breadths differ from the first (breadth), and from one another---from the
first, because it is rational, and from one another since they are not the
same in order---clearly,  the  irrational (straight-lines)
themselves also differ from one another. And since it has been shown that an apotome
is not the same as a binomial [Prop. 10.111],
and (that) the (irrational straight-lines) after the apotome, being applied to
a rational (straight-line), produce as breadth, each according to its own (order), apotomes, and (that) the (irrational straight-lines) after
the binomial  themselves also (produce as breadth), according (to their) order, binomials, 
the (irrational straight-lines) after the apotome are thus different, and the
(irrational straight-lines) after the binomial (are also) different, so that
there are, in order, 13 irrational (straight-lines) in all:}
\end{Parallel}

\begin{Parallel}{}{}
\ParallelLText{

\gr{M'eshn,}\vspace*{-7pt}

\gr{>Ek d'uo >onom'atwn,}\vspace*{-7pt}

\gr{>Ek d'uo m'eswn pr'wthn,}\vspace*{-7pt}

\gr{>Ek d'uo m'eswn deut'eran,}\vspace*{-7pt}

\gr{Me'izona,}\vspace*{-7pt}

\gr{<Rht`on ka`i m'eson dunam'enhn,}\vspace*{-7pt}

\gr{D'uo m'esa dunam'enhn,}\vspace*{-7pt}

\gr{>Apotom'hn,}\vspace*{-7pt}

\gr{M<eshc >apotom`hn pr'wthn,}\vspace*{-7pt}

\gr{M<eshc >apotom`hn deut'eran,}\vspace*{-7pt}

\gr{>El'assona},\vspace*{-7pt}

\gr{Met`a <rhto~u m'eson t`o <'olon poio~usan,}~\\\vspace*{-7pt}

\gr{Met`a m'esou m'eson t`o <'olon poio~usan.}}

\ParallelRText{Medial,

Binomial,

First bimedial,

Second bimedial,

Major,

Square-root of a rational plus a medial (area),

Square-root of (the sum of) two medial (areas),

Apotome,

First apotome of a medial,

Second apotome of a medial,

Minor,

That which with a rational (area) produces a medial whole,

That which with a medial (area) produces a medial whole.}
\end{Parallel}

%%%%%
%10.112
%%%%%
\pdfbookmark[1]{Proposition 10.112}{pdf10.112}
\begin{Parallel}{}{}
\ParallelLText{
\begin{center}
{\large \ggn{112}.}
\end{center}\vspace*{-7pt}

\gr{T`o >ap`o <rht~hc par`a t`hn >ek d'uo >onom'atwn paraball'omenon
pl'atoc poie~i >apotom'hn, <~hc t`a >on'omata s'ummetr'a >esti to~ic
t~hc >ek d'uo >onom'atwn >on'omasi ka`i >'eti >en t~w| a>ut~w| l'ogw|,
ka`i >'eti <h ginom'enh >apotom`h t`hn a>ut`hn <'exei t'axin t~h| >ek
d'uo >onom'atwn.}\\

\epsfysize=0.9in
\centerline{\epsffile{Book10/fig112g.eps}}

\gr{>'Estw <rht`h m`en <h A, >ek d'uo >onom'atwn d`e <h BG, <~hc me~izon
>'onoma >'estw <h DG, ka`i t~w| >ap`o t~hc A >'ison >'estw t`o
<up`o t~wn BG, EZ; l'egw, <'oti <h EZ >apotom'h >estin, <~hc t`a >on'omata
s'ummetr'a >esti to~ic GD, DB, ka`i >en t~w| a>ut~w| l'ogw|,
ka`i >'eti <h EZ t`hn a>ut`hn <'exei t'axin t~h| BG.}

\gr{>'Estw g`ar p'alin t~w| >ap`o t~hc A >'ison t`o <up`o t~wn BD, H.
>epe`i o>~un t`o <up`o t~wn BG, EZ >'ison >est`i t~w| <up`o
t~wn BD, H, >'estin >'ara <wc <h GB pr`oc t`hn BD,  o<'utwc <h H
pr`oc t`hn EZ. me'izwn d`e <h GB t~hc BD; me'izwn >'ara >est`i ka`i
<h H t~hc EZ. >'estw t~h| H >'ish <h EJ; >'estin >'ara <wc <h GB pr`oc
t`hn BD, o<'utwc <h JE pr`oc t`hn EZ; diel'onti >'ara >est`in <wc <h GD
pr`oc t`hn BD, o<'utwc <h JZ pr`oc t`hn ZE. gegon'etw <wc
<h JZ pr`oc t`hn ZE, o<'utwc <h ZE pr`oc t`hn KE;
ka`i <'olh >'ara <h JK pr`oc <'olhn t`hn KZ >estin, <wc <h ZK
pr`oc KE; <wc g`ar <`en t~wn <hgoum'enwn pr`oc <`en t~wn
<epom'enwn, o<'utwc <'apanta t`a <hgo'umena pr`oc <'apanta
t`a <ep'omena. <wc d`e <h ZK pr`oc KE, o<'utwc >est`in <h GD
pr`oc t`hn DB; ka`i <wc >'ara <h JK pr`oc KZ, o<'utwc
<h GD pr`oc t`hn DB. s'ummetron d`e t`o >ap`o t~hc GD t~w|
>ap`o t~hc DB; s'ummetron >'ara >est`i ka`i t`o >ap`o t~hc JK
t~w| >ap`o t~hc KZ. ka'i >estin <wc t`o >ap`o t~hc JK
pr`oc t`o >ap`o t~hc KZ,
o<'utwc <h JK pr`oc t`hn KE, >epe`i a<i
tre~ic
a<i JK, KZ, KE >an'alog'on e>isin. s'ummetroc
>'ara <h JK t~h| KE m'hkei. <'wste ka`i <h JE t~h| EK
s'ummetr'oc >esti m'hkei.
ka`i >epe`i t`o >ap`o t~hc A >'ison
>est`i t~w| <up`o t~wn EJ, BD, <rht`on d'e >esti t`o >ap`o
t~hc A, <rht`on >'ara >est`i ka`i t`o <up`o t~wn EJ, BD. ka`i par`a
<rht`hn t`hn BD par'akeitai; <rht`h >'ara >est`in <h EJ ka`i s'ummetroc
t~h| BD m'hkei; <'wste ka`i <h s'ummetroc a>ut~h| <h EK <rht'h >esti
ka`i s'ummetroc t~h| BD m'hkei. >epe`i o>~un >estin <wc <h GD
pr`oc DB, o<'utwc <h ZK pr`oc KE, a<i d`e GD, DB dun'amei
m'onon e>is`i s'ummetroi, ka`i a<i ZK, KE dun'amei
m'onon e>is`i s'ummetroi. <rht`h d'e >estin <h KE; <rht`h >'ara
>est`i ka`i <h ZK. a<i ZK, KE >'ara <rhta`i dun'amei m'onon e>is`i
s'ummetroi; >apotom`h >'ara >est`in <h EZ.}

\gr{>'Htoi d`e <h GD t~hc DB me~izon d'unatai t~w| >ap`o summ'etrou
<eaut~h| >`h t~w| >ap`o >asumm'etrou.}

\gr{E>i m`en o>~un <h GD t~hc DB me~izon d'unatai t~w| >ap`o
summ'etrou [<eaut~h|], ka`i <h ZK t~hc KE me~izon
dun'hsetai t~w| >ap`o summ'etrou <eaut~h|. ka`i e>i m`en
s'ummetr'oc >estin <h GD t~h| >ekkeim'enh| <rht~h| m'hkei,
ka`i <h ZK; e>i d`e <h BD, ka`i <h KE; e>i d`e o>udet'era t~wn GD,
DB, ka`i o>udet'era t~wn ZK, KE.}

\gr{E>i d`e <h GD t~hc DB me~izon d'unatai t~w| >ap`o >asumm'etrou
<eaut~h|, ka`i <h ZK t~hc  KE me~izon dun'hsetai t~w| >ap`o
>asumm'etrou <eaut~h|. ka`i e>i m`en <h GD s'ummetr'oc >esti
t~h| >ekkeim'enh| <rht~h| m'hkei, ka`i <h ZK; e>i d`e <h BD, ka`i <h KE;
e>i d`e o>udet'era t~wn GD, DB, ka`i o>udet'era t~wn ZK, KE;
<'wste >apotom'h >estin <h ZE, <~hc t`a >on'omata t`a ZK, KE s'ummetr'a
>esti to~ic t~hc >ek d'uo >onom'atwn >on'omasi to~ic GD, DB
ka`i >en t~w| a>ut~w| l'ogw|, ka`i t`hn a>ut~hn t'axin >'eqei t~h|
BG; <'oper >'edei de~ixai.}}

\ParallelRText{
\begin{center}
{\large Proposition 112}$^\dag$
\end{center}

The (square) on a rational (straight-line), applied
to a binomial (straight-line), produces  as breadth an apotome whose terms are commensurable (in length) with the terms of the
binomial, and, furthermore, in the same ratio. Moreover,
the created apotome will have the same order as the binomial.

\epsfysize=0.9in
\centerline{\epsffile{Book10/fig112e.eps}}

Let $A$ be a rational (straight-line), and $BC$ a binomial
(straight-line), of which let $DC$ be the greater term. And let the
(rectangle contained) by $BC$ and $EF$ be equal to the (square) on $A$.
I say that $EF$ is an apotome whose terms are commensurable (in length)
with $CD$ and $DB$, and in the same ratio, and, moreover, that $EF$
will have the same order as $BC$.

For, again, let the (rectangle contained) by $BD$ and $G$  be equal
to the (square) on $A$. Therefore, since the (rectangle contained) by $BC$
and $EF$ is equal to the (rectangle contained) by $BD$ and $G$, thus as
$CB$ is to $BD$, so $G$ (is) to $EF$ [Prop. 6.16].
And $CB$ (is) greater than $BD$. Thus, $G$ is also greater than $EF$
[Props.~5.16, 5.14].
Let $EH$ be equal to $G$. Thus, as $CB$ is to $BD$, so $HE$
(is) to $EF$. Thus, via separation, as $CD$ is to $BD$, so $HF$ (is) to $FE$
[Prop. 5.17]. Let it have been contrived that as
$HF$ (is) to $FE$, so $FK$ (is) to $KE$. And, thus, the
whole $HK$ is to the whole $KF$, as $FK$ (is) to $KE$.
For as one of the leading (proportional magnitudes is) to one of
the following, so all of the leading (magnitudes) are to all of the
following [Prop. 5.12]. And as $FK$
(is) to $KE$, so $CD$ is to $DB$ [Prop. 5.11].  And, thus, as $HK$ (is) to $KF$, so
$CD$ is to $DB$ [Prop. 5.11].
And the (square) on  $CD$ (is) 
 commensurable with the (square) on $DB$ [Prop. 10.36]. The (square) on $HK$ is
 thus also commensurable with the (square) on $KF$ [Props.~6.22, 10.11].
  And as the (square) on $HK$ is to the (square) on $KF$, so $HK$ (is)
  to $KE$, since the three (straight-lines) $HK$, $KF$, and $KE$ are
  proportional [Def. 5.9]. $HK$ is thus commensurable
  in length with $KE$ [Prop. 10.11]. 
  Hence, $HE$ is also commensurable in length with $EK$
  [Prop. 10.15]. And since the (square) on $A$
  is equal to the (rectangle contained) by $EH$ and $BD$, and the (square)
  on $A$ is rational, the (rectangle contained) by $EH$ and $BD$ is thus
  also rational. And it is applied to the rational (straight-line) $BD$.
  Thus, $EH$ is  rational, and commensurable in length with $BD$ [Prop. 10.20]. And, hence, the (straight-line)
  commensurable (in length) with it, $EK$, is also rational [Def. 10.3], and commensurable
  in length with $BD$ [Prop. 10.12].  Therefore,
  since as $CD$  is to $DB$, so $FK$ (is) to $KE$, and $CD$ and $DB$
  are (straight-lines which are) commensurable in square only, $FK$ and $KE$ are also
  commensurable in square only [Prop. 10.11]. 
  And $KE$ is rational. Thus, $FK$ is also rational. $FK$ and
  $KE$ are thus rational (straight-lines which are) commensurable in square
  only. Thus, $EF$ is an apotome [Prop. 10.73].
  
  And the square on $CD$ is greater than (the square on) $DB$ either
  by the (square) on (some straight-line) commensurable, or by the
  (square) on (some straight-line) incommensurable, (in length) with ($CD$).
  
  Therefore, if the square on $CD$ is greater than (the square on) $DB$
  by the (square) on (some straight-line) commensurable  (in length) with [$CD$]  then the square on $FK$ will also be greater than (the square on) $KE$
  by the (square) on (some straight-line) commensurable  (in length) with ($FK$) [Prop. 10.14]. And if $CD$ is commensurable in length with a (previously) laid down rational (straight-line), (so) also (is) $FK$ [Props.~10.11, 10.12]. And if $BD$ (is commensurable), (so) also (is) $KE$ [Prop. 10.12]. And if neither of
  $CD$ or $DB$ (is commensurable), neither also (are) either of $FK$ or $KE$.
  
And if the square on $CD$ is greater than (the square on) $DB$
  by the (square) on (some straight-line) incommensurable  (in length) with ($CD$)  then the square on $FK$ will also be greater than (the square on) $KE$
  by the (square) on (some straight-line) incommensurable  (in length) with ($FK$) [Prop. 10.14]. And if $CD$ is commensurable in length with a (previously) laid down rational (straight-line), (so) also  (is) $FK$ [Props.~10.11, 10.12]. And if $BD$ (is commensurable), (so) also (is) $KE$ [Prop. 10.12]. And if neither of
  $CD$ or $DB$ (is commensurable), neither also (are) either of $FK$ or $KE$.
  Hence, $FE$ is an apotome whose terms, $FK$ and $KE$, are
  commensurable (in length) with the terms, $CD$ and $DB$, of the binomial,
  and in the same ratio. And ($FE$) has the same order as $BC$ [Defs.~10.5---10.10]. 
  (Which is) the very thing it was required to show.}
  \end{Parallel}


\vspace{7pt}{\footnotesize\noindent$^\dag$ Heiberg considers this proposition, and
the succeeding ones, to be  relatively early interpolations into the original
text.}

%%%%%
%10.113
%%%%%
\pdfbookmark[1]{Proposition 10.113}{pdf10.113}
\begin{Parallel}{}{}
\ParallelLText{
\begin{center}
{\large \ggn{113}.}
\end{center}\vspace*{-7pt}

\gr{T`o >ap`o <rht~hc par`a >apotom`hn paraball'omenon pl'atoc poie~i
t`hn >ek d'uo >onom'atwn, <~hc t`a >on'omata s'ummetr'a >esti
to~ic t~hc >apotom~hc >on'omasi ka`i >en t~w| a>ut~w| l'ogw|,
>'eti d`e <h ginom'enh >ek d'uo >onom'atwn t`hn a>ut`hn t'axin
>'eqei t~h| >apotom~h|.}\\

\epsfysize=0.9in
\centerline{\epsffile{Book10/fig113g.eps}}

\gr{>'Estw <rht`h m`en <h A, >apotom`h d`e <h BD, ka`i t~w| >ap`o
t~hc A >'ison >'estw t`o <up`o t~wn BD, KJ, <'wste t`o >ap`o
t~hc A <rht~hc par`a t`hn BD >apotom`hn paraball'omenon
pl'atoc poie~i t`hn KJ; l'egw, <'oti >ek d'uo >onom'atwn >est`in
<h KJ, <~hc t`a >on'omata s'ummetr'a >esti to~ic t~hc BD
>on'omasi ka`i >en t~w| a>ut~w| l'ogw|, ka`i >'eti <h KJ t`hn
a>ut`hn >'eqei t'axin t~h| BD.}

\gr{>'Estw g`ar t~h| BD prosarm'ozousa <h DG; a<i BG, GD >'ara
<rhta'i e>isi dun'amei m'onon s'ummetroi. ka`i t~w| >ap`o t~hc
A >'ison >'estw ka`i t`o <up`o t~wn BG, H. 
<rht`on d`e t`o >ap`o t~hc A; <rht`on >'ara ka`i t`o <up`o
t~wn BG, H.
ka`i par`a <rht`hn
t`hn BG parab'eblhtai; <rht`h >'ara >est`in <h H ka`i s'ummetroc
t~h| BG m'hkei. >epe`i o>~un t`o <up`o t~wn BG, H >'ison
>est`i t~w| <up`o t~wn BD, KJ, >an'alogon >'ara >est`in <wc <h
GB pr`oc BD, o<'utwc <h KJ pr`oc H. me'izwn d`e <h BG
t~hc BD; me'izwn >'ara ka`i <h KJ t~hc H. ke'isjw t~h| H >'ish
<h KE; s'ummetroc >'ara >est`in <h KE t~h| BG m'hkei. ka`i >epe'i
>estin <wc <h GB pr`oc BD, o<'utwc <h JK pr`oc KE, >anastr'eyanti
>'ara >est`in <wc <h BG pr`oc t`hn GD, o<'utwc <h KJ pr`oc JE.
gegon'etw <wc <h KJ pr`oc JE, o<'utwc <h JZ pr`oc ZE; ka`i loip`h >'ara
<h KZ pr`oc ZJ >estin, <wc <h KJ pr`oc JE, tout'estin [<wc] <h BG
pr`oc GD. a<i d`e BG, GD dun'amei m'onon [e>is`i] s'ummetroi;
ka`i a<i KZ, ZJ >'ara dun'amei m'onon e>is`i s'ummetroi; ka`i
>epe'i >estin <wc <h KJ pr`oc JE, <h KZ pr`oc ZJ, >all> <wc <h
KJ pr`oc JE, <h JZ pr`oc ZE, ka`i <wc >'ara <h KZ pr`oc ZJ, <h JZ pr`oc
ZE; <'wste ka`i <wc <h pr'wth pr`oc t`hn tr'ithn, t`o >ap`o t~hc
pr'wthc pr`oc t`o >ap`o t~hc deut'erac; ka`i <wc >'ara <h KZ pr`oc
ZE, o<'utwc t`o >ap`o t~hc KZ pr`oc t`o >ap`o t~hc ZJ. s'ummetron
d'e >esti t`o >ap`o t~hc KZ t~w| >ap`o t~hc ZJ; a<i g`ar KZ, ZJ
dun'amei e>is`i s'ummetroi; s'ummetroc >'ara >est`i ka`i <h KZ
t~h| ZE m'hkei; <'wste <h KZ ka`i t~h| KE s'ummetr'oc [>esti]
m'hkei. <rht`h d'e >estin <h KE ka`i s'ummetroc t~h| BG m'hkei.
<rht`h >'ara ka`i <h KZ ka`i s'ummetroc t~h| BG m'hkei. ka`i >epe'i
>estin <wc <h BG pr`oc GD, o<'utwc <h KZ pr`oc ZJ, >enall`ax
<wc <h BG pr`oc KZ, o<'utwc <h DG pr`oc ZJ. s'ummetroc
d`e <h BG t~h| KZ; s'ummetroc >'ara ka`i <h ZJ t~h| GD
m'hkei. a<i BG, GD d`e <rhta'i e>isi dun'amei m'onon s'ummetroi;
ka`i a<i KZ, ZJ >'ara <rhta'i e>isi dun'amei m'onon s'ummetroi;
>ek d'uo >onom'atwn >est`in >'ara <h KJ.}

\gr{E>i m`en o>~un <h BG t~hc GD me~izon d'unatai t~w| >ap`o
summ'etrou <eaut~h|, ka`i <h KZ t~hc ZJ me~izon dun'hsetai
t~w| >ap`o summ'etr\-ou <eaut~h|. ka`i e>i m`en s'ummetr'oc
>estin <h BG t~h| >ekkeim'enh| <rht~h| m'hkei, ka`i <h KZ, e>i d`e
<h GD s'ummetr'oc >esti t~h| >ekkeim'enh| <rht~h| m'hkei,
ka`i <h ZJ, e>i d`e o>udet'era t~wn BG, GD, o>udet'era
t~wn KZ, ZJ.}

\gr{E>i d`e <h BG t~hc GD me~izon d'unatai t~w| >ap`o >asumm'etrou
<eaut~h|, ka`i <h KZ t~hc ZJ me~izon dun'hsetai t~w| >ap`o
>asumm'etr\-ou <eaut~h|. ka`i e>i m`en s'ummetr'oc >estin
<h BG t~h| >ekkeim'enh| <rht~h| m'hkei, ka`i <h KZ,
e>i d`e <h GD, ka`i <h ZJ, e>i d`e o>udet'era t~wn BG, GD, o>udet'era
t~wn KZ, ZJ.}

\gr{>Ek d'uo >'ara >onom'atwn >est`in <h KJ, <~hc t`a >on'omata
t`a KZ, ZJ s'ummetr'a [>esti] to~ic t~hc >apotom~hc >on'omasi
to~ic BG, GD ka`i >en t~w| a>ut~w| l'ogw|, ka`i >'eti <h KJ
t~h| BG t`hn a>ut`hn <'exei t'axin; <'oper >'edei de~ixai.}}

\ParallelRText{
\begin{center}
{\large Proposition 113}
\end{center}

The (square) on a rational (straight-line),
applied to an apotome, produces as breadth a binomial whose terms
are commensurable with the terms of the apotome, and in  the same ratio.
Moreover, the created binomial has the same order as the apotome.

\epsfysize=0.9in
\centerline{\epsffile{Book10/fig113e.eps}}

Let $A$ be a rational (straight-line), and $BD$ an apotome. And let the
(rectangle contained) by $BD$ and $KH$ be equal to the (square) on $A$,
such that the square on the rational (straight-line) $A$, applied to the apotome
$BD$, produces $KH$ as breadth. I say that $KH$ is a binomial whose
terms are commensurable with the terms of $BD$, and in the same ratio, and,
moreover, that $KH$ has the same order as $BD$.

For let $DC$ be an attachment to $BD$. Thus, $BC$ and $CD$
are rational (straight-lines which are) commensurable in square only [Prop. 10.73]. And let the (rectangle contained)
by $BC$ and $G$ also be equal to the (square) on $A$. And the (square)
on $A$ (is) rational. The (rectangle contained) by $BC$ and $G$ (is) thus also rational. And it has been applied to the rational (straight-line) $BC$.
Thus, $G$ is rational, and commensurable in length with $BC$ [Prop. 10.20]. Therefore, since the (rectangle contained) by $BC$ and $G$ is equal to the (rectangle contained) by
$BD$ and $KH$, thus, proportionally, as $CB$ is to $BD$, so $KH$ (is) to
$G$ [Prop. 6.16]. And $BC$ (is) greater than $BD$.
Thus, $KH$ (is) also greater than $G$ [Prop. 5.16,
5.14]. Let $KE$ be made equal to $G$.
$KE$ is thus commensurable in length with $BC$. And since as
$CB$ is to $BD$, so $HK$ (is) to $KE$, thus, via conversion, as
$BC$ (is) to $CD$, so $KH$ (is) to $HE$ [Prop. 5.19~corr.]. 
Let it have been contrived that as
$KH$ (is) to $HE$, so $HF$ (is) to $FE$. And thus the remainder $KF$ is to
$FH$, as $KH$ (is) to $HE$---that is to say, [as] $BC$ (is) to $CD$
[Prop. 5.19]. And $BC$ and $CD$
[are] commensurable in square only. $KF$ and $FH$ are thus also commensurable in square only [Prop. 10.11].
And since as $KH$ is to $HE$, (so) $KF$ (is) to $FH$, but as $KH$
(is) to $HE$, (so) $HF$ (is) to $FE$, thus, also as $KF$ (is) to $FH$, (so)
$HF$ (is) to $FE$ [Prop. 5.11].
And hence as the first (is) to the third,
so the (square) on the first (is) to the (square) on the second [Def. 5.9].  And thus as $KF$ (is) to $FE$, so the (square) on $KF$ (is) to the (square) on $FH$. And the (square) on $KF$
is commensurable with the (square) on $FH$. For $KF$ and $FH$
are commensurable in square. Thus, $KF$ is also commensurable
in length with  $FE$ [Prop. 10.11]. 
 Hence, $KF$ [is] also commensurable in length with $KE$
  [Prop. 10.15]. And $KE$ is rational,
and commensurable in length with $BC$.  Thus, $KF$ (is) also rational,
and commensurable in length with $BC$ [Prop. 10.12].  And since as $BC$ is to
$CD$, (so)
$KF$ (is) to $FH$, alternately, as $BC$ (is) to $KF$, so $DC$
(is) to $FH$ [Prop. 5.16]. And $BC$ (is) commensurable (in length) with $KF$. Thus, $FH$ (is) also commensurable in length with $CD$ [Prop. 10.11]. And
$BC$ and $CD$ are rational (straight-lines which are) commensurable
in square only. $KF$ and $FH$ are thus also rational (straight-lines which
are) commensurable in square only [Def. 10.3, Prop.~10.13]. Thus, $KH$ is a binomial [Prop. 10.36].

Therefore,  if the square on $BC$ is greater than (the square on) $CD$
by the (square) on (some straight-line) commensurable (in length) with ($BC$), 
then the square on $KF$ will also be greater than (the square on) $FH$
by the (square) on (some straight-line) commensurable (in length) with ($KF$) [Prop. 10.14]. And if $BC$ is commensurable
in length with a (previously) laid down rational (straight-line), (so) also (is) $KF$
[Prop. 10.12].
And if $CD$ is commensurable in length with a (previously) laid
down rational (straight-line), (so) also  (is) $FH$ [Prop. 10.12]. And if neither of $BC$ or $CD$
(are commensurable), neither also (are) either of $KF$ or $FH$ [Prop. 10.13].

And if the square on $BC$ is greater than (the square on) $CD$
by the (square) on (some straight-line) incommensurable (in length) with ($BC$) then
the square on $KF$ will also be greater than (the square on) $FH$
by the (square) on (some straight-line) incommensurable (in length) with ($KF$) [Prop. 10.14]. And if $BC$ is commensurable
in length with a (previously) laid down rational (straight-line), (so) also (is) $KF$
[Prop. 10.12].
And if $CD$ is commensurable, (so) also (is) $FH$ [Prop. 10.12]. And if neither of $BC$ or $CD$
(are commensurable), neither also (are) either of $KF$ or $FH$ [Prop. 10.13].

$KH$ is thus a binomial whose terms, $KF$ and $FH$, [are] commensurable
(in length) with the terms, $BC$ and $CD$,  of the apotome, and in the same ratio. 
Moreover, $KH$ will have the same order as $BC$ [Defs.~10.5---10.10]. 
  (Which is) the very thing it was required to show.}
\end{Parallel}

%%%%%
%10.114
%%%%%
\pdfbookmark[1]{Proposition 10.114}{pdf10.114}
\begin{Parallel}{}{}
\ParallelLText{
\begin{center}
{\large \ggn{114}.}
\end{center}\vspace*{-7pt}

\gr{>E`an qwr'ion peri'eqhtai <up`o >apotom~hc ka`i t~hc >ek d'uo
>onom'atwn, <~hc t`a >on'omata s'ummetr'a t'e >esti to~ic t~hc
>apotom~hc >on'omasi ka`i >en t~w| a>ut~w| l'ogw|, <h t`o qwr'ion
dunam'enh <rht'h >estin.}

\epsfysize=1.8in
\centerline{\epsffile{Book10/fig114g.eps}}

\gr{Perieq'esjw g`ar qwr'ion t`o <up`o t~wn AB, GD <up`o >apotom~hc
t~hc AB ka`i t~hc >ek d'uo >onom'atwn t~hc GD, <~hc me~izon >'onoma
>'estw t`o GE, ka`i >'estw t`a >on'omata t~hc >ek d'uo >onom'atwn
t`a GE, ED s'ummetr'a te to~ic t~hc >apotom~hc >on'omasi to~ic
AZ, ZB ka`i >en t~w| a>ut~w| l'ogw|, ka`i >'estw <h t`o <up`o t~wn
AB, GD dunam'enh <h H; l'egw, <'oti <rht'h >estin <h H.}

\gr{>Ekke'isjw g`ar <rht`h <h J, ka`i t~w| >ap`o t~hc J >'ison par`a
t`hn GD parabebl'hsjw pl'atoc poio~un t`hn KL; >apotom`h >'ara
>est`in <h KL, <~hc t`a >on'omata >'estw t`a KM, ML
s'ummetra to~ic t~hc >ek d'uo >onom'atwn >on'omasi to~ic
GE, ED ka`i >en t~w| a>ut~w| l'ogw|. >all`a ka`i a<i GE, ED
s'ummetro'i t'e e>isi ta~ic AZ, ZB ka`i >en t~w| a>ut~w| l'ogw|;
>'estin >'ara <wc <h AZ pr`oc t`hn ZB, o<'utwc <h KM pr`oc ML.
>enall`ax >'ara >est`in <wc <h AZ pr`oc t`hn KM, o<'utwc
<h BZ pr`oc t`hn LM; ka`i loip`h >'ara <h AB pr`oc
loip`hn t`hn KL >estin <wc <h AZ pr`oc KM. s'ummetroc
d`e <h AZ t~h| KM; s'ummetroc >'ara >est`i ka`i <h AB t~h| KL.
ka'i >estin <wc <h AB pr`oc KL, o<'utwc t`o <up`o t~wn GD, AB
pr`oc t`o <up`o t~wn GD, KL; s'ummetron >'ara >est`i ka`i t`o <up`o
t~wn GD, AB t~w| <up`o t~wn GD, KL. >'ison d`e t`o <up`o
t~wn GD, KL
t~w| >ap`o t~hc J; s'ummetron >'ara
>est`i t`o <up`o t~wn GD, AB t~w| >ap`o t~hc J. t~w| d`e <up`o t~wn
GD, AB >'ison >est`i t`o >ap`o t~hc H; s'ummetron >'ara
>est`i t`o >ap`o t~hc H t~w| >ap`o t~hc J. <rht`on d`e t`o >ap`o
t~hc J; <rht`on >'ara >est`i ka`i t`o >ap`o t~hc H; <rht`h
>'ara >est`in <h H. ka`i d'unatai t`o <up`o t~wn GD, AB.}

\gr{>E`an >'ara qwr'ion peri'eqhtai <up`o >apotom~hc ka`i t~hc
>ek d'uo >onom'atwn, <~hc t`a >on'omata s'ummetr'a >esti to~ic
t~hc >apotom~hc >on'omasi ka`i >en t~w| a>ut~w| l'ogw|,
<h t`o qwr'ion dunam'enh <rht'h >estin.}\\~\\~\\~\\~\\~\\~\\~\\~\\~\\~\\~\\

\begin{center}
{\large \gr{P'orisma}.}
\end{center}\vspace*{-7pt}

\gr{Ka`i g'egonen <hm~in ka`i di`a to'utou faner'on, <'oti dunat'on >esti
<rht`on qwr'ion <up`o >al'ogwn e>ujei~wn peri'eqe\-sjai.
<'oper >'edei de~ixai.}}

\ParallelRText{
\begin{center}
{\large Proposition 114}
\end{center}

If an area is contained by an apotome, and
a binomial whose terms are commensurable with, and in the same ratio as, the 
terms of the apotome then the square-root of the area is a rational (straight-line).

\epsfysize=1.8in
\centerline{\epsffile{Book10/fig114e.eps}}

For let an area, the  (rectangle contained) by $AB$ and $CD$, have
been contained by the
apotome $AB$, and the binomial $CD$, of which let the greater term be $CE$. And let the terms of the binomial, $CE$ and $ED$, be commensurable with the terms of the apotome, $AF$ and $FB$ (respectively), and in the same ratio. And
let the square-root of the (rectangle contained) by $AB$ and $CD$ be $G$.
I say that $G$ is a rational (straight-line).

For let the rational (straight-line) $H$ be laid down. And let (some rectangle),
equal to the (square) on $H$, have been applied to $CD$, producing
$KL$ as breadth. Thus, $KL$ is an apotome, of which let the
terms, $KM$ and $ML$, be  commensurable with the terms of the
binomial, $CE$ and $ED$ (respectively), and in the same ratio [Prop. 10.112]. But, $CE$ and $ED$ are also
commensurable with $AF$ and $FB$ (respectively), and in the same ratio. Thus,
as $AF$ is to $FB$, so $KM$ (is) to $ML$. Thus, alternately, as 
$AF$ is to $KM$, so $BF$ (is) to $LM$ [Prop. 5.16]. Thus, the remainder $AB$ is
also to the remainder $KL$ as $AF$ (is) to $KM$ [Prop. 5.19]. And $AF$ (is) commensurable with
$KM$ [Prop. 10.12]. $AB$ is thus also commensurable with $KL$ [Prop. 10.11]. 
And as $AB$ is to $KL$, so the (rectangle contained) by $CD$ and
$AB$ (is) to the (rectangle contained) by $CD$ and $KL$ [Prop. 6.1].  Thus, the (rectangle contained)
by $CD$ and $AB$ is also commensurable with the
(rectangle contained) by $CD$ and $KL$ [Prop. 10.11]. And the (rectangle contained) by
$CD$ and $KL$ (is) equal to the (square) on $H$. Thus, the
(rectangle contained) by $CD$ and $AB$ is  commensurable with the
(square) on $H$. And the (square) on $G$ is equal to the
(rectangle contained) by $CD$ and $AB$. The (square) on $G$ is thus commensurable with the (square) on $H$. And the (square) on $H$ (is)
rational. Thus, the (square) on $G$ is also rational. $G$ is thus rational.
And it is the square-root of the (rectangle contained) by $CD$ and $AB$.

Thus, if an area is contained by an apotome, and
a binomial whose terms are commensurable with, and in the same ratio as, the 
terms of the apotome, then the square-root of the area is a rational (straight-line).\\

\begin{center}
{\large Corollary}
\end{center}\vspace*{-7pt}

And it has also been made clear to us,  through this,  that it is
possible for  a rational
area to be contained by irrational straight-lines. (Which is) the very thing it was required to show.}
\end{Parallel}

%%%%%
%10.115
%%%%%
\pdfbookmark[1]{Proposition 10.115}{pdf10.115}
\begin{Parallel}{}{}
\ParallelLText{
\begin{center}
{\large \ggn{115}.}
\end{center}\vspace*{-7pt}

\gr{>Ap`o m'eshc >'apeiroi >'alogoi g'inontai, ka`i o>udem'ia o>udemi~a|
t~wn pr'oteron <h a>ut'h.}\\

\epsfysize=1.in
\centerline{\epsffile{Book10/fig115g.eps}}

\gr{>'Estw m'esh <h A; l'egw, <'oti >ap`o t~hc A >'apeiroi
>'alogoi g'inontai, ka`i o>udem'ia o>udemi~a| t~wn pr'oteron
<h a>ut'h.}

\gr{>Ekke'isjw <rht`h <h B, ka`i t~w| <up`o t~wn B, A >'ison
>'estw t`o >ap`o t~hc G; >'alogoc >'ara >est`in <h G;
t`o g`ar <up`o >al'ogou ka`i <rht~hc >'alog'on >estin. ka`i
o>udemi~a| t~wn pr'oteron <h a>ut'h; t`o g`ar >ap> o>udemi~ac
t~wn pr'oteron par`a <rht`hn paraball'omenon pl'atoc poie~i
m'eshn. p'alin d`h t~w| <up`o t~wn B, G >'ison >'estw t`o >ap`o
t~hc D; >'alogon >'ara >est`i t`o >ap`o t~hc D. >'alogoc >'ara
>est`in <h D; ka`i o>udemi~a| t~wn pr'oteron <h a>ut'h; t`o g`ar
>ap> o>udemi~ac t~wn pr'oteron par`a <rht`hn paraball'omenon
pl'atoc poie~i t`hn G. <omo'iwc d`h t~hc toia'uthc t'axewc >ep>
>'apeiron probaino'ushc faner'on, <'oti >ap`o t~hc m'eshc 
>'apeiroi >'alogoi g'inontai, ka`i o>udem'ia o>udemi~a| t~wn pr'oteron
<h a>ut'h; <'oper >'edei de~ixai.}}

\ParallelRText{
\begin{center}
{\large Proposition 115}
\end{center}

An infinite (series) of irrational (straight-lines) can be created from a medial
(straight-line), and none of them is the same as any of the preceding
(straight-lines).

\epsfysize=1.in
\centerline{\epsffile{Book10/fig115e.eps}}

Let $A$ be a medial (straight-line). I say that an infinite (series) of irrational (straight-lines) can be created from $A$, and that none of them is the same as any of the preceding
(straight-lines).

Let the rational (straight-line) $B$ be laid down. And let the (square) on $C$
be equal to the (rectangle contained) by $B$ and $A$. Thus, $C$ is
irrational [Def. 10.4]. For an (area contained) by
an irrational and a rational (straight-line) is irrational [Prop. 10.20]. And ($C$ is) not the same as
any of the preceding (straight-lines). For the (square) on none of
the preceding (straight-lines), applied to a rational (straight-line),
produces a medial (straight-line) as breadth. So, again, let the (square)
on $D$ be equal to the (rectangle contained) by $B$ and $C$. Thus,
the (square) on $D$ is irrational [Prop. 10.20]. 
$D$ is thus irrational [Def. 10.4]. And ($D$ is)
not the same as any of the preceding (straight-lines). For the (square) on
none of the preceding (straight-lines), applied to a rational (straight-line),
produces $C$ as breadth. So, similarly, this arrangement being
advanced to infinity, it is clear that an infinite (series) of irrational (straight-lines) can be created from a medial
(straight-line), and that none of them is the same as any of the preceding
(straight-lines). (Which is) the very thing it was required to show.}
\end{Parallel}
