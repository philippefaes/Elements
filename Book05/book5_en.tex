%%%%%%
% BOOK 5
%%%%%%
\cleardoublepage 
\pdfbookmark[0]{Book 5}{book5}
\addcontentsline{toc}{chapter}{Book 5}
\pagestyle{plain}
\begin{center}
{\Huge ELEMENTS BOOK 5}\\
\spa\spa\spa
{\huge\it Proportion\symbolfootnote[2]{The theory of proportion set out in this
book is generally attributed to Eudoxus of Cnidus. The novel feature of this theory is its ability to deal with irrational magnitudes, which had hitherto been
a major stumbling block for Greek mathematicians. Throughout the footnotes in this
book, $\alpha$, $\beta$, $\gamma$, {\em etc.}, denote general (possibly irrational) magnitudes, whereas $m$, $n$, $l$,  {\em etc.}, denote
positive integers.}}
\end{center}\newpage

%%%%%%%
% Definitions
%%%%%%%
\pdfbookmark[1]{Definitions}{def5}
\pagestyle{fancy}
\cfoot{\gr{\thepage}}
\chead{\large ELEMENTS BOOK 5}
\begin{Parallel}{}{} 
\ParallelLText{
\begin{center}
\large{\gr{<'Oroi}.}
\end{center}\vspace*{-7pt}

\ggn{1}.~\gr{M'eroc >est`i m'egejoc meg'ejouc t`o >'elasson to~u me'izonoc,
<'otan katametr~h| t`o me~izon.}

\ggn{2}.~\gr{Pollapl'asion d`e t`o me~izon to~u >el'attonoc, <'otan katametr~htai <up`o to~u >el'attonoc.}

\ggn{3}.~\gr{L'ogoc >est`i d'uo megej~wn <omogen~wn <h kat`a phlik'otht'a
poia sq'esic.}

\ggn{4}.~\gr{L'ogon >'eqein pr`oc >'allhla meg'ejh l'egetai, <`a d'unatai
pollaplasiaz'omena >all'hlwn <uper'eqein.}

\ggn{5}.~\gr{>En t~w| a>ut~w| l'ogw| meg'ejh l'egetai e>~inai pr~wton pr`oc
de'uteron ka`i tr'iton pr`oc t'etarton, <'otan t`a to~u pr'wtou ka'i
tr'itou >is'akic pollapl'asia t~wn to~u deut'erou ka`i tet'artou
>is'akic pollaplas'iwn kaj> <opoiono~un pollaplasiasm`on
<ek'ateron <ekat'erou >`h <'ama <uper'eqh| >`h <'ama >'isa >~h| >`h
<'ama >elle'ip~h| lhfj'enta kat'allhla.}

\ggn{6}.~\gr{T`a d`e t`on a>ut`on >'eqonta l'ogon meg'ejh >an'alogon
kale'isjw.}

\ggn{7}.~\gr{<'Otan d`e t~wn >is'akic pollaplas'iwn t`o m`en to~u pr'wtou
pollapl'asion <uper'eqh| to~u to~u deut'erou pollaplas'iou, t`o d`e
to~u tr'itou pollapl'asion m`h <uper'eqh| to~u to~u tet'artou
pollaplas'iou, t'ote t`o pr~wton pr`oc t`o de'uteron me'izona l'ogon >'eqein
l'egetai, >'hper t`o tr'iton pr`oc t`o t'etarton.}

\ggn{8}.~\gr{>Analog'ia d`e >en tris`in <'oroic >elaq'isth >est'in.}

\ggn{9}.~\gr{<'Otan d`e tr'ia meg'ejh >an'alogon >~h|, t`o pr~wton pr`oc t`o
tr'iton diplas'iona l'ogon >'eqein l'egetai >'hper pr`oc t`o
de'uteron.}

\ggn{10}.~\gr{<'Otan d`e t'essara meg'ejh >an'alogon >~h|, t`o pr~wton
pr`oc t`o t'etarton triplas'iona l'ogon >'eqein l'egetai >'hper
pr`oc t`o de'uteron, ka`i >ae`i <ex~hc <omo'iwc, <wc
>`an <h >analog'ia <up'arqh|.}

\ggn{11}.~\gr{<Om'ologa meg'ejh l'egetai t`a m`en <hgo'umena to~ic
<hgoum'enoic t`a d`e <ep'omena to~ic <epom'enoic.}

\ggn{12}.~\gr{>Enall`ax l'ogoc >est`i l~hyic to~u <hgoum'enou pr`oc t`o
<hgo'umenon ka`i to~u <epom'enou pr`oc t`o <ep'omenon.}

\ggn{13}.~\gr{>An'apalin l'ogoc >est`i l~hyic to~u <epom'enou <wc
<hgoum'enou pr`oc t`o <hgo'umenon <wc <ep'omenon.}

\ggn{14}.~\gr{S'unjesic l'ogou >est`i l~hyic to~u <hgoum'enou met`a to~u
<epom'enou <wc <en`oc pr`oc a>ut`o t`o <ep'omenon.}

\ggn{15}.~\gr{Dia'iresic l'ogou >est`i l~hyic t~hc <uperoq~hc, <~h| <uper'eqei
t`o <hgo'umenon to~u <epom'enou, pr`oc a>ut`o t`o <ep'omenon.}

\ggn{16}.~\gr{>Anastrof`h l'ogou >est`i l~hyic to~u <hgoum'enou pr`oc
t`hn <uperoq'hn, ~<h| <uper'eqei t`o <hgo'umenon to~u
<epom'enou.}

\ggn{17}.~\gr{Di> >'isou l'ogoc >est`i plei'onwn >'ontwn megej~wn ka`i
>'allwn a>uto~ic >'iswn t`o pl~hjoc s'unduo lambanom'enwn
ka`i >en t~w| a>ut~w| l'ogw|, <'otan >~h| <wc >en to~ic pr'wtoic
meg'ejesi t`o pr~wton pr`oc t`o >'esqaton, o<'utwc >en to~ic
deut'eroic meg'ejesi t`o pr~wton pr`oc t`o >'esqaton; >`h >'allwc;
l~hyic t~wn >'akrwn kaj> <upexa'iresin t~wn  m'eswn.}

\ggn{18}.~\gr{Tetaragm'enh d`e >analog'ia >est'in, <'otan tri~wn >'ontwn
megej~wn ka`i >'allwn a>uto~ic >'iswn t`o pl~hjoc
g'inhtai <wc m`en >en to~ic pr'wtoic meg'ejesin <hgo'umenon pr`oc
>ep'omenon, o<'utwc  >en to~ic deut'eroic meg'ejesin
<hgo'umenon pr`oc <ep'omenon, <wc d`e >en to~ic pr'wtoic
meg'ejesin <ep'omenon pr`oc >'allo ti, o<'utwc >en to~ic
deut'eroic >'allo ti pr`oc <hgo'umenon.}}

\ParallelRText{
\begin{center}
{\large Definitions}
\end{center}

1.~A magnitude is a part of a(nother) magnitude, the lesser of the greater, when
it measures the greater.$^\dag$

2.~And the greater (magnitude is) a multiple of the lesser when it is measured by
the lesser.

3.~A ratio is a certain type of condition  with respect to size of two magnitudes of the same kind.$^\ddag$

4.~(Those) magnitudes are said to have a ratio with respect to one another which,
being multiplied, are capable of exceeding one another.$^\S$

5.~Magnitudes are said to be in the same ratio, the first to the second, and
the third to the fourth, when  equal multiples of the first and the third 
either both exceed, are both equal to, or are both less than, 
equal multiples of the second and the fourth, respectively,  being taken in corresponding order, according to any kind of multiplication whatever.$^\P$

6.~And let magnitudes having the same ratio be called proportional.$^\ast$

7.~And when for equal multiples (as in Def.~5), the multiple of the first (magnitude) exceeds 
the multiple of the second, and the multiple of the third (magnitude) does not
exceed the multiple of the fourth, then the first (magnitude) is said to have a greater ratio
to the second than the third (magnitude has) to the fourth.

8.~And a  proportion in three terms is the smallest (possible).$^\$$

9.~And when three magnitudes are proportional, the first is said to
have to the third the squared$^\|$ ratio  of that (it has) to the second.$^{\dag\dag}$

10.~And when four magnitudes are (continuously) proportional, the first is said to have to the fourth the cubed$^{\ddag\ddag}$ ratio  of that (it has) to the second.$^{\S\S}$ And  so on,   similarly, in successive order, whatever the (continuous) proportion
might be.

11.~These magnitudes are said to be corresponding (magnitudes):
the leading to the leading (of two ratios), and
the following to the following.

12.~An alternate ratio is a taking of the (ratio of the) leading  (magnitude) to the leading (of two equal ratios), and (setting it equal to) the (ratio of the)
following   (magnitude) to the following.$^{\P\P}$

13.~An inverse  ratio is a taking of the (ratio of the) following (magnitude) as the leading and the leading (magnitude) as the
following.$^{\ast\ast}$

14.~A composition of a  ratio is a taking of the (ratio of the) leading plus the following (magnitudes),
as one, to the  following (magnitude) by itself.$^{\$\$}$

 15.~A separation of a  ratio is a taking of the (ratio of the) excess by which the
 leading (magnitude) exceeds the following to the following (magnitude)
 by itself.$^{\|\|}$
 
16.~A conversion  of a ratio is a taking of the (ratio of the) leading (magnitude) 
 to the excess by which the leading (magnitude) exceeds the following.$^{\dag\dag\dag}$
 
17.~There being several magnitudes,
   and  other (magnitudes)
 of equal  number to them, (which are)  also in
 the same ratio taken two by two,
  a  ratio via equality (or {\em ex aequali}) occurs when  as the first is to the last  in the first (set of) magnitudes, so  the first (is) to the last   in the second (set of) magnitudes. Or alternately, (it is) a taking of the (ratio of the) outer (magnitudes) by the removal of the 
 inner (magnitudes).$^{\ddag\ddag\ddag}$
 
18.~There being three magnitudes,
  and  other (magnitudes)
 of equal  number to them,  a  perturbed proportion occurs  when  as 
 the  leading is to the following
 in the first (set of) magnitudes, so the  leading (is) to the following  in the second (set of) magnitudes, and as   the following (is) to
 some other ({\em i.e.}, the remaining magnitude) in the first (set of) magnitudes, so some other (is)  to the leading  in the second (set of) magnitudes.$^{\S\S\S}$}
\end{Parallel}


\vspace{7pt}{\footnotesize \noindent$^\dag$ In other words, $\alpha$ is said to be a part of $\beta$ if $\beta = m\,\alpha$.\\[0.5ex]
$^\ddag$ In modern notation, the ratio of two 
magnitudes, $\alpha$ and $\beta$, is denoted $\alpha:\beta$.\\[0.5ex]
$^\S$ In other words, $\alpha$ has a ratio with respect to $\beta$ if $m\,\alpha >\beta$
and $n\,\beta>\alpha$, for some $m$ and $n$.\\[0.5ex]
$^\P$ In other words, $\alpha:\beta::\gamma:\delta$ if and only if $m\,\alpha>n\,\beta$ whenever $m\,\gamma>n\,\delta$, and $m\,\alpha=n\,\beta$
whenever $m\,\gamma=n\,\delta$, and $m\,\alpha< n\,\beta$ whenever $m\,\gamma<
n\,\delta$, for all $m$ and $n$. This definition  is the kernel of Eudoxus' theory of proportion, and is valid even if $\alpha$,  $\beta$, {\em etc.}, are irrational.\\[0.5ex]
$^\ast$ Thus if $\alpha$ and $\beta$ have the same ratio
as $\gamma$ and $\delta$ then they are proportional. In modern notation,
$\alpha:\beta::\gamma:\delta$.\\[0.5ex]
$^{\$}$ In modern notation, a proportion in three terms---$\alpha$, $\beta$, and $\gamma$---is written: $\alpha:\beta::\beta:\gamma$.\\
$^{\|}$ Literally, ``double''.\\[0.5ex]
$^{\dag\dag}$ In other words, if $\alpha:\beta::\beta:\gamma$ then $\alpha:\gamma::\alpha^{\,2}:\beta^{\,2}$.\\[0.5ex]
$^{\ddag\ddag}$ Literally, ``triple''.\\[0.5ex]
$^{\S\S}$ In other words, if $\alpha:\beta::\beta:\gamma::\gamma:\delta$ then $\alpha:\delta::\alpha^{\,3}:\beta^{\,3}$.\\[0.5ex]
$^{\P\P}$ In other words, if $\alpha:\beta::\gamma:\delta$ then the alternate ratio
corresponds to  $\alpha:\gamma::\beta:\delta$.\\[0.5ex]
$^{\ast\ast}$ In other words, if  $\alpha:\beta$ then the inverse ratio
corresponds to $\beta:\alpha$.\\[0.5ex]
$^{\$\$}$ In other words, if
 $\alpha:\beta$ then the composed ratio
 corresponds to  $\alpha+\beta:\beta$.\\[0.5ex]
 $^{\|\|}$ In other words, if
 $\alpha:\beta$ then the separated ratio
 corresponds to  $\alpha-\beta:\beta$.\\[0.5ex]
 $^{\dag\dag\dag}$ In other words, if
 $\alpha:\beta$ then the converted ratio
 corresponds to  $\alpha:\alpha-\beta$.\\[0.5ex]
 $^{\ddag\ddag\ddag}$ In other words, if $\alpha, \beta, \gamma$ are the first set of
 magnitudes, and $\delta, \epsilon, \zeta$ the second set, and $\alpha:
 \beta:\gamma::\delta:\epsilon:\zeta$,  then
 the ratio via equality (or {\em ex aequali}) corresponds to
 $\alpha:\gamma::\delta:\zeta$.\\[0.5ex]
 $^{\S\S\S}$ In other words, if $\alpha, \beta, \gamma$ are the first set of
 magnitudes, and $\delta, \epsilon, \zeta$ the second set, and $\alpha:\beta::\delta:\epsilon$ as well as $\beta:\gamma::\zeta:\delta$, then
 the proportion is said to be perturbed.}

%%%%%%
% Prop 5.1
%%%%%%
\pdfbookmark[1]{Proposition 5.1}{pdf5.1}
\begin{Parallel}{}{} 
\ParallelLText{
\begin{center}
{\large \ggn{1}.}
\end{center}\vspace*{-7pt}

\gr{>E`an >~h| <oposao~un meg'ejh <oposwno~un megej~wn >'iswn t`o pl~hjoc
<'ekaston <ek'astou >is'akic pollapl'asion, <osapl'asi'on >estin <`en
t~wn megej~wn <en'oc, tosautapl'asia >'estai ka`i t`a p'anta t~wn p'antwn.}\\~
\\

\epsfysize=0.6in
\centerline{\epsffile{Book05/fig01g.eps}}

\gr{>'Estw <oposao~un meg'ejh t`a AB, GD <oposwno~un megej~wn t~wn
E, Z >'iswn t`o pl~hjoc <'ekaston <ek'astou >is'akic pollapl'asion; l'egw, <'oti
<osapl'asi'on >esti t`o AB to~u E, tosautapl'asia >'estai ka`i t`a AB, GD
t~wn E, Z.}

\gr{>Epe`i g`ar >is'akic >est`i pollapl'asion t`o AB to~u E ka`i t`o GD to~u
Z, <'osa >'ara >est`in >en t~w| AB meg'ejh >'isa t~w| E, tosa~uta ka`i
>en t~w| GD >'isa t~w| Z. dih|r'hsjw t`o m`en AB e>ic t`a t~w| E meg'ejh >'isa t`a AH, HB, t`o d`e GD e>ic t`a t~w| Z >'isa t`a GJ, JD;
>'estai d`h >'ison t`o pl~hjoc t~wn AH, HB t~w| pl'hjei
t~wn GJ, JD. ka`i >epe`i >'ison >est`i t`o m`en AH t~w| E, t`o d`e 
GJ t~w| Z, >'ison >'ara t`o AH t~w| E, ka`i t`a AH, GJ to~ic E, Z. di`a
t`a a>ut`a d`h >'ison >est`i t`o HB t~w| E, ka`i t`a HB, JD to~ic
E, Z; <'osa >'ara >est`in >en t~w| AB >'isa t~w| E, tosa~uta ka`i >en to~ic
AB, GD >'isa to~ic E, Z; <osapl'asion >'ara >est`i t`o AB to~u E,
tosautapl'asia >'estai ka`i t`a AB, GD t~wn E, Z.}

\gr{>E`an >'ara >~h| <oposao~un meg'ejh <oposwno~un megej~wn >'iswn t`o pl~hjoc
<'ekaston <ek'astou >is'akic pollapl'asion, <osapl'asi'on >estin <`en
t~wn megej~wn <en'oc, tosautapl'asia >'estai ka`i t`a p'anta t~wn p'antwn;
<'oper >'edei de~ixai.}}

\ParallelRText{
\begin{center}
{\large Proposition 1}$^\dag$
\end{center}

If there are any number of magnitudes whatsoever (which are)  equal multiples, respectively, of
some (other) magnitudes,  of equal number (to them), then as many times as one of the (first) magnitudes is (divisible) by one (of the second), so many times
will all (of the first magnitudes) also (be divisible) by all (of the second).

\epsfysize=0.6in
\centerline{\epsffile{Book05/fig01e.eps}}

Let there be any number of magnitudes whatsoever, $AB$, $CD$, (which are) equal multiples,
respectively, of some (other) magnitudes, $E$, $F$,  of equal  number
(to them). I say that as many times as $AB$ is (divisible) by $E$, so many times will
$AB$, $CD$ also be (divisible) by $E$, $F$.

For since $AB$, $CD$ are equal multiples of $E$, $F$, thus as many
magnitudes as (there) are in  $AB$ equal to $E$, so many (are there) also in $CD$ equal to $F$. Let $AB$ have been divided into magnitudes $AG$, $GB$, equal to $E$,
and $CD$ into (magnitudes) $CH$, $HD$, equal to $F$. So, the number of (divisions)
$AG$, $GB$ will be equal to the number of (divisions) $CH$, $HD$. And since
$AG$ is equal to $E$, and $CH$ to $F$,   $AG$ (is)  thus equal to $E$, and $AG$, $CH$ to
$E$, $F$. So, for the same (reasons), $GB$ is equal to $E$, and $GB$, $HD$ to $E$, $F$.
Thus, as many (magnitudes) as (there) are in $AB$ equal to $E$, so many (are there)
also in $AB$, $CD$ equal to $E$, $F$. Thus, as many times as $AB$ is (divisible) by $E$, so many
times will $AB$, $CD$ also be (divisible)  by $E$, $F$.

Thus, if there are any number of magnitudes whatsoever (which are) equal multiples, respectively, of
some (other) magnitudes,  of  equal  number (to them), then as many times as one of the (first) magnitudes is (divisible) by one (of the second), so many times
will all (of the first magnitudes) also (be divisible) by all (of the second). (Which is)
the very thing it was required to show.}
\end{Parallel}


\vspace{7pt}{\footnotesize \noindent$^\dag$ In modern notation, this proposition
reads $m\,\alpha + m\,\beta +\cdots= m\,(\alpha+\beta+\cdots)$.}

%%%%%%
% Prop 5.2
%%%%%%
\pdfbookmark[1]{Proposition 5.2}{pdf5.2}
\begin{Parallel}{}{} 
\ParallelLText{
\begin{center}
{\large \ggn{2}.}
\end{center}\vspace*{-7pt}

\gr{>E`an pr~wton deut'erou >is'akic >~h| pollapl'asion ka`i tr'iton tet'artou, >~h|
d`e ka`i p'empton deut'erou >is'akic pollapl'asion ka`i <'ekton tet'artou, ka`i
suntej`en pr~wton ka`i p'empton deut'erou >is'akic >'estai pollapl'asion
ka`i tr'iton ka`i <'ekton tet'artou.}

\gr{Pr~wton g`ar t`o AB deut'erou to~u G >is'akic >'estw pollapl'asion ka`i
tr'iton t`o DE tet'artou to~u Z, >'estw d`e ka`i p'empton t`o BH deut'erou to~u
G >is'akic pollapl'asion ka`i <'ekton t`o EJ tet'artou to~u Z; l'egw,
<'oti ka`i suntej`en pr~wton ka`i p'empton t`o AH deut'erou to~u 
G >is'akic >'estai pollapl'asion ka`i tr'iton ka`i <'ekton
t`o DJ tet'artou to~u Z.}\\~\\~\\~\\~\\

\epsfysize=1.5in
\centerline{\epsffile{Book05/fig02g.eps}}

\gr{>Epe`i g`ar >is'akic >est`i pollapl'asion t`o AB to~u G ka`i t`o 
DE to~u Z, <'osa >'ara >est`in >en t~w| AB >'isa t~w| G, tosa~uta ka`i >en
t~w| DE >'isa t~w| Z. di`a t`a a>ut`a d`h ka`i <'osa >est`in >en t~w| BH
>'isa t~w| G, tosa~uta ka`i >en t~w| EJ >'isa t~w| Z; <'osa >'ara >est`in
>en <'olw| t~w| AH >'isa t~w| G, tosa~uta ka`i >en <'olw| t~w| DJ >'isa t~w|
Z; <osapl'asion >'ara >est`i t`o AH to~u G, tosautapl'asion >'estai ka`i t`o
DJ to~u Z. ka`i suntej`en >'ara pr~wton ka`i p'empton t`o AH deut'erou
to~u G >is'akic >'estai pollapl'asion ka`i tr'iton ka`i <'ekton t`o
DJ tet'artou to~u Z.}

\gr{>E`an >'ara pr~wton deut'erou >is'akic >~h| pollapl'asion ka`i tr'iton tet'artou, >~h|
d`e ka`i p'empton deut'erou >is'akic pollapl'asion ka`i <'ekton tet'artou, ka`i
suntej`en pr~wton ka`i p'empton deut'erou >is'akic >'estai pollapl'asion
ka`i tr'iton ka`i <'ekton tet'artou; <'oper >'edei de~ixai.}}

\ParallelRText{
\begin{center}
{\large Proposition 2}$^{\dag}$
\end{center}

If a first (magnitude) and a third are equal multiples of a second
and a fourth (respectively), and a fifth (magnitude) and a sixth (are) also equal multiples of the
second and fourth (respectively), then the first (magnitude) and the fifth, being added together, and the third and the sixth,  (being added together), will also be equal multiples of the
second (magnitude) and the fourth (respectively).

For let a first (magnitude) $AB$ and a third $DE$ be equal multiples of a
second $C$ and a fourth $F$ (respectively). And let a fifth (magnitude) $BG$
and a sixth $EH$ also be (other) equal multiples of the second $C$ and the fourth $F$
(respectively). I say that the first (magnitude) and the fifth, being added together, (to give) $AG$,
and the third (magnitude) and the sixth, (being added together, to give) $DH$, will also be
equal multiples of the second (magnitude) $C$ and the fourth $F$ (respectively).

\epsfysize=1.5in
\centerline{\epsffile{Book05/fig02e.eps}}

For since $AB$ and $DE$ are equal multiples of $C$ and $F$ (respectively), 
thus as many (magnitudes) as (there) are in $AB$ equal to $C$, so many (are there) also
in $DE$ equal to $F$.  And so, for the same (reasons),   as many (magnitudes) as (there) are in $BG$ equal to $C$, so many (are there) also in $EH$ equal to $F$.
Thus, as many (magnitudes) as (there) are in the whole of $AG$ equal to $C$,
so many (are there) also in the whole of $DH$ equal to $F$.
Thus, as many times as $AG$ is (divisible) by $C$, so many times will $DH$ also
be divisible by $F$. Thus, the first (magnitude) and the fifth, being
added together, (to give) $AG$, and the third (magnitude) and the sixth, (being
added together, to give) $DH$, will also be equal multiples of the second (magnitude)
$C$ and the fourth $F$ (respectively).

Thus, if a first (magnitude) and a third are equal multiples of a second
and a fourth (respectively), and a fifth (magnitude) and a sixth (are) also equal multiples of the
second and fourth (respectively), then the  first (magnitude) and the fifth, being
added together, and the third  and sixth, (being added together), will also be equal multiples of the
second (magnitude) and the fourth (respectively). (Which is) the very thing it was required to
show.}
\end{Parallel}


\vspace{7pt}{\footnotesize \noindent$^\dag$ In modern notation, this propostion reads
$m\,\alpha+n\,\alpha = (m+n)\,\alpha$.}

%%%%%%
% Prop 5.3
%%%%%%
\pdfbookmark[1]{Proposition 5.3}{pdf5.3}
\begin{Parallel}{}{} 
\ParallelLText{
\begin{center}
{\large \ggn{3}.}
\end{center}\vspace*{-7pt}

\gr{>E`an pr~wton deut'erou >is'akic >~h| pollapl'asion ka`i tr'iton tet'artou,
lhfj~h| d`e >is'akic pollapl'asia to~u te pr'wtou ka`i tr'itou, ka`i di> 
>'isou t~wn  lhfj'entwn <ek'ateron <ekat'erou >is'akic >'estai pollapl'asion
t`o m`en to~u deut'erou t`o d`e to~u tet'artou.}

\gr{Pr~wton g`ar t`o A deut'erou to~u B >is'akic >'estw pollapl'asion ka`i
tr'iton t`o G tet'artou to~u D, ka`i e>il'hfjw t~wn A, G >is'akic pollapl'asia
t`a EZ, HJ; l'egw, <'oti >is'akic >est`i pollapl'asion t`o EZ to~u B ka`i
t`o HJ to~u D.}

\gr{>Epe`i g`ar >is'akic >est`i pollapl'asion t`o EZ to~u A ka`i t`o HJ to~u
G, <'osa >'ara >est`in >en t~w| EZ >'isa t~w| A, tosa~uta ka`i >en t~w|
HJ >'isa t~w| G. dih|r'hsjw t`o m`en EZ e>ic t`a t~w| A meg'ejh >'isa
t`a EK, KZ, t`o d`e HJ e>ic t`a t~w| G >'isa t`a HL, LJ; >'estai d`h >'ison
t`o pl~hjoc t~wn EK, KZ t~w| pl'hjei t~wn HL, LJ. ka`i >epe`i
>is'akic >est`i pollapl'asion t`o A to~u B ka`i t`o G to~u D,  >'ison d`e t`o
m`en EK t~w| A, t`o d`e HL t~w| G, >is'akic >'ara >est`i pollapl'asion
t`o EK to~u B ka`i t`o HL to~u D. di`a t`a a>ut`a d`h >is'akic >est`i
pollapl'asion t`o KZ to~u B ka`i t`o LJ to~u D. >epe`i o~>un
pr~wton t`o EK deut'erou to~u B >'is'akic >est`i pollapl'asion ka`i
tr'iton t`o HL tet'artou to~u D, >'esti d`e ka`i p'empton t`o KZ deut'erou to~u
B >is'akic pollapl'asion ka`i <'ekton t`o LJ tet'artou to~u D,
ka`i suntej`en  >'ara pr~wton ka`i p'empton t`o EZ deut'erou to~u B
>is'akic >est`i pollapl'asion ka`i tr'iton ka`i <'ekton t`o HJ
tet'artou to~u D.}\\~\\~\\~\\~\\~\\~\\

\epsfysize=2in
\centerline{\epsffile{Book05/fig03g.eps}}

\gr{>E`an >'ara pr~wton deut'erou >is'akic >~h| pollapl'asion ka`i tr'iton tet'artou,
lhfj~h| d`e to~u pr'wtou ka`i tr'itou >is'akic pollapl'asia, ka`i di> 
>'isou t~wn  lhfj'entwn <ek'ateron <ekat'erou >is'akic >'estai pollapl'asion
t`o m`en to~u deut'erou t`o d`e to~u tet'artou; <'oper >'edei de~ixai.}}

\ParallelRText{
\begin{center}
{\large Proposition 3}$^\dag$
\end{center}

If a first (magnitude) and a third are equal multiples
of a second and a fourth (respectively), and equal multiples are taken of the
first and the third, then, via equality,  the 
(magnitudes) taken will also be equal multiples of the second (magnitude) and the fourth, respectively.

For let a first (magnitude) $A$ and a third $C$ be equal multiples of a
second $B$ and a fourth $D$ (respectively), and let the equal multiples
$EF$ and $GH$ have been taken of $A$ and $C$ (respectively). I say that
$EF$ and $GH$ are equal multiples of $B$ and $D$ (respectively).

For since $EF$ and $GH$ are equal multiples of $A$ and $C$ (respectively), thus as many (magnitudes) as (there) are in $EF$ equal to $A$, so many (are there) also in
$GH$ equal to $C$. Let $EF$ have been divided into magnitudes $EK$, $KF$ equal
to $A$, and $GH$ into (magnitudes) $GL$, $LH$ equal to $C$. So, the number of (magnitudes)
$EK$, $KF$ will be equal to the number of (magnitudes) $GL$, $LH$.
And since $A$ and $C$ are equal multiples of $B$ and $D$ (respectively), and $EK$ (is)
equal to $A$, and $GL$ to $C$,  $EK$ and $GL$ are thus equal multiples of $B$ and $D$
(respectively). So, for the same (reasons), $KF$ and $LH$ are equal multiples
of $B$ and $D$ (respectively). Therefore, since the first (magnitude) $EK$ and the
third $GL$ are equal multiples of the second $B$ and the fourth $D$ (respectively), and the fifth (magnitude) $KF$ and
the sixth $LH$ are  also  equal multiples of the second $B$ and the fourth $D$ (respectively),  then  the first (magnitude) and fifth, being added together,
(to give) $EF$, and the third (magnitude) and sixth, (being added together, to give) $GH$, are thus also equal multiples of the second (magnitude) $B$ and the fourth $D$ (respectively) [Prop. 5.2].

\epsfysize=2in
\centerline{\epsffile{Book05/fig03e.eps}}

Thus, if a first (magnitude) and a third are equal multiples
of a second and a fourth (respectively), and equal multiples are taken of the
first and the third, then, via equality,  the 
(magnitudes) taken will also be equal multiples of the second (magnitude) and the fourth, respectively. (Which is) the very thing it was required to show.}
\end{Parallel}


\vspace{7pt}{\footnotesize \noindent$^\dag$ In modern notation, this proposition reads
$m(n\,\alpha) = (m\,n)\,\alpha$.}


%%%%%%
% Prop 5.4
%%%%%%
\pdfbookmark[1]{Proposition 5.4}{pdf5.4}
\begin{Parallel}{}{} 
\ParallelLText{
\begin{center}
{\large \ggn{4}.}
\end{center}\vspace*{-7pt}

\gr{>E`an pr~wton pr`oc de'uteron t`on a>ut`on >'eqh| l'ogon ka`i tr'iton
pr`oc t'etarton, ka`i t`a >is'akic pollapl'asia to~u te pr'wtou ka`i
tr'itou pr`oc t`a >is'akic pollapl'asia to~u deut'erou ka`i tet'artou kaj>
<opoiono~un pollaplasiasm`on t`on a>ut`on <'exei l'ogon lhfj'enta
kat'allhla.}

\gr{Pr~wton g`ar t`o A pr`oc de'uteron t`o B t`on a>ut`on >eq'etw l'ogon
ka`i tr'iton t`o G pr`oc t'etarton t`o D, ka`i e>il'hfjw t~wn m`en A, G
>is'akic pollapl'asia t`a E, Z, t~wn d`e B, D >'alla, <`a >'etuqen, >is'akic
pollapl'asia t`a H, J; l'egw, <'oti >est`in <wc t`o E pr`oc t`o H, o<'utwc 
t`o Z pr`oc t`o J.}\\~\\

\epsfysize=3in
\centerline{\epsffile{Book05/fig04g.eps}}

\gr{E>il'hfjw g`ar t~wn m`en E, Z >is'akic pollapl'asia t`a K, L, t~wn d`e H,
J >'alla, <`a >'etuqen, >is'akic pollapl'asia t`a M, N.}

\gr{\mbox{[}Ka`i] >epe`i >is'akic >est`i pollapl'asion t`o m`en E to~u A, t`o d`e Z to~u
G, ka`i e>'ilhptai t~wn E, Z >'is'akic pollapl'asia t`a K, L, >'is'akic
>'ara >est`i pollapl'asion t`o K to~u A ka`i t`o L to~u G. di`a t`a a>ut`a d`h
>is'akic >est`i pollapl'asion t`o M to~u B ka`i t`o N to~u D. ka`i >epe'i
>estin <wc t`o A pr`oc t`o B, o<'utwc t`o G pr`oc t`o D, ka`i e>'ilhptai
t~wn m`en A, G >is'akic pollapl'asia t`a K, L, t~wn d`e B, D >'alla, <`a
>'etuqen, >is'akic pollapl'asia t`a M, N, e>i >'ara <uper'eqei t`o K
to~u M, <uper'eqei ka`i t`o L to~u N, ka`i e>i >'ison, >'ison, ka`i
e>i >'elatton, >'elatton. ka'i >esti t`a m`en K, L t~wn E, Z >is'akic 
pollapl'asia, t`a d`e M, N t~wn H, J >'alla, <`a >'etuqen, >is'akic
pollapl'asia;
 >'estin >'ara <wc t`o E pr`oc t`o H, o<'utwc t`o Z
pr`oc t`o J.}

\gr{>E`an >'ara pr~wton pr`oc de'uteron t`on a>ut`on >'eqh| l'ogon ka`i tr'iton
pr`oc t'etarton, ka`i t`a >is'akic pollapl'asia to~u te pr'wtou ka`i
tr'itou pr`oc t`a >is'akic pollapl'asia to~u deut'erou ka`i tet'artou t`on a>ut`on <'exei l'ogon kaj> <opoiono~un pollaplasiasm`on  lhfj'enta
kat'allhla; <'oper >'edei de~ixai.}}

\ParallelRText{
\begin{center}
{\large Proposition 4}$^\dag$
\end{center}

If  a first (magnitude) has the same ratio to a second 
that a third  (has) to a fourth then equal multiples of the first (magnitude) and the third will also have the same ratio to equal multiples of the second and the fourth,  being taken in corresponding order, according to any kind of
multiplication whatsoever.

For let a first (magnitude) $A$ have the same ratio to a second $B$ that
a third $C$  (has) to a fourth $D$. And let equal multiples $E$ and $F$ have
been taken of $A$ and $C$ (respectively), and  other random equal multiples
$G$ and $H$  of $B$ and $D$ (respectively). I say that as $E$ (is)
to $G$, so $F$ (is) to $H$.

\epsfysize=3in
\centerline{\epsffile{Book05/fig04e.eps}}

For let equal multiples $K$ and $L$ have been taken of $E$ and $F$ (respectively),
and other random equal multiples $M$ and $N$ of $G$ and $H$ (respectively).

\mbox{[}And] since $E$ and $F$ are equal multiples of $A$ and $C$ (respectively), and
the equal multiples $K$ and $L$ have been taken of $E$ and $F$ (respectively),
 $K$ and $L$ are thus equal multiples of $A$ and $C$ (respectively) [Prop. 5.3]. So, for the same (reasons), $M$ and $N$
 are equal multiples of $B$ and $D$ (respectively). And since as $A$ is to
 $B$, so $C$ (is) to $D$, and the equal multiples $K$ and $L$ have been taken of $A$ and
 $C$ (respectively), and the other random equal multiples  $M$ and $N$ of $B$ and $D$ (respectively), then if $K$ exceeds $M$ then $L$ also
 exceeds $N$, and if ($K$ is) equal (to $M$ then $L$ is also) equal (to $N$), and if ($K$ is) less (than $M$ then $L$ is also) less (than $N$) [Def. 5.5]. And $K$ and $L$ are equal multiples of 
 $E$ and $F$ (respectively), and $M$ and $N$ other random equal multiples of $G$ and $H$ (respectively). Thus, as $E$ (is) to $G$, so $F$ (is) to $H$ 
 [Def. 5.5].
 
Thus, if a first (magnitude) has the same ratio to a second 
that a third  (has) to a fourth then equal multiples of the first (magnitude) and the third will also have the same ratio to equal multiples of the second and the fourth,  being taken in corresponding order, according to any kind of
multiplication whatsoever. (Which is) the very thing
it was required to show.}
\end{Parallel}


\vspace{7pt}{\footnotesize \noindent$^\dag$ In modern notation, this
proposition reads that if $\alpha:\beta :: \gamma:\delta$ then
$m\,\alpha:n\,\beta::m\,\gamma:n\,\delta$, for all $m$ and $n$.}\\~\\~\\

%%%%%%
% Prop 5.5
%%%%%%
\pdfbookmark[1]{Proposition 5.5}{pdf5.5}
\begin{Parallel}{}{} 
\ParallelLText{
\begin{center}
{\large \ggn{5}.}
\end{center}\vspace*{-7pt}

\gr{>E`an m'egejoc meg'ejouc >is'akic >~h| pollapl'asion, <'oper >afairej`en
>afairej'entoc, ka`i t`o loip`on to~u loipo~u >is'akic >'estai pollapl'asion,
<osapl'asi'on >esti t`o <'olon to~u <'olou.}\\~\\

\epsfysize=0.7in
\centerline{\epsffile{Book05/fig05g.eps}}

\gr{M'egejoc g`ar t`o AB meg'ejouc to~u GD >is'akic >'estw pollapl'asion,
<'oper >afairej`en t`o AE >afairej'entoc to~u GZ; l'egw, <'oti ka`i loip`on
t`o EB loipo~u to~u ZD >is'akic >'estai pollapl'asion, <osapl'asi'on 
>estin <'olon t`o AB <'olou to~u GD.}

\gr{<Osapl'asion g'ar >esti t`o AE to~u GZ, tosautapl'asion gegon'etw ka`i
t`o EB to~u GH.}

\gr{Ka`i >epe`i >is'akic >est`i pollapl'asion t`o AE to~u GZ ka`i t`o EB to~u
HG, >is'akic >'ara >est`i pollapl'asion t`o AE to~u GZ ka`i t`o AB to~u
HZ. ke~itai d`e >is'akic pollapl'asion t`o AE to~u GZ ka`i t`o AB to~u
GD. >is'akic >'ara >est`i pollapl'asion t`o AB <ekat'erou t~wn HZ, GD;
>'ison >'ara t`o HZ t~w| GD. koin`on >afh|r'hsjw t`o GZ; loip`on
>'ara t`o HG loip~w| t~w| ZD >'ison >est'in. ka`i >epe`i >is'akic
>est`i pollapl'asion t`o AE to~u GZ ka`i t`o EB to~u HG, >'ison
d`e t`o HG t~w|  DZ, 
>is'akic >'ara  >est`i pollapl'asion t`o AE to~u GZ ka`i t`o EB to~u ZD.
>is'akic d`e <up'okeitai pollapl'asion t`o AE to~u GZ ka`i t`o AB to~u
GD; >is'akic >'ara 
>est`i pollapl'asion t`o EB to~u ZD ka`i t`o AB to~u GD. ka`i loip`on 
>'ara t`o EB loipo~u to~u ZD >is'akic >'estai pollapl'asion, <osapl'asi'on
>estin <'olon t`o AB <'olou to~u GD.}

\gr{>E`an >'ara m'egejoc meg'ejouc >is'akic >~h| pollapl'asion, <'oper >afairej`en
>afairej'entoc, ka`i t`o loip`on to~u loipo~u >is'akic >'estai pollapl'asion,
<osapl'asi'on >esti ka`i t`o <'olon to~u <'olou; <'oper >'edei de~ixai.}}

\ParallelRText{
\begin{center}
{\large Proposition 5}$^\dag$
\end{center}

If a magnitude is the same multiple of a magnitude
that a (part) taken away (is) of a (part) taken away (respectively) then the remainder will
also be the same multiple of the remainder as that  which the whole (is) of the whole (respectively).

\epsfysize=0.7in
\centerline{\epsffile{Book05/fig05e.eps}}

For let the magnitude $AB$ be the same multiple of the magnitude $CD$ that
the (part)  taken away $AE$ (is) of the (part)  taken away $CF$ (respectively).
I say that the remainder $EB$ will also be the same multiple
of the remainder $FD$ as that which the whole $AB$ (is) of the whole
$CD$ (respectively).

For as many times as $AE$ is (divisible) by $CF$, so many times let $EB$
also have been made (divisible) by $CG$.

And since $AE$ and $EB$ are equal multiples of $CF$ and $GC$ (respectively),
$AE$ and $AB$ are thus equal multiples of $CF$ and $GF$ (respectively) 
 [Prop. 5.1]. And  $AE$ and $AB$ are assumed (to be) equal multiples of $CF$ and $CD$ (respectively). Thus,
 $AB$ is an equal multiple of each of $GF$ and $CD$. Thus, $GF$ (is) equal to
 $CD$. Let $CF$ have been subtracted from both. Thus, the remainder $GC$ is
 equal to the remainder $FD$. And since $AE$ and $EB$ are equal multiples
 of $CF$ and $GC$ (respectively), and $GC$ (is) equal to $DF$, $AE$  and $EB$ are thus
 equal multiples of $CF$ and $FD$ (respectively). And $AE$ and $AB$ are assumed
 (to be) equal multiples of $CF$ and $CD$ (respectively). Thus, $EB$ and $AB$ are
 equal multiples of $FD$ and $CD$ (respectively). Thus, the remainder
 $EB$ will also be the same multiple of the remainder $FD$
 as that which the whole $AB$ (is) of the whole $CD$ (respectively).
 
 Thus, if a magnitude is the same multiple of a magnitude
that a (part) taken away (is) of a (part) taken away (respectively) then the remainder will
also be the same multiple of the remainder as that  which the whole (is) of the whole (respectively).
  (Which is) the very thing it was required to show.}
\end{Parallel}


\vspace{7pt}{\footnotesize \noindent$^\dag$ In modern notation, this
proposition reads $m\,\alpha-m\,\beta=m\,(\alpha-\beta)$.}

%%%%%%
% Prop 5.6
%%%%%%
\pdfbookmark[1]{Proposition 5.6}{pdf5.6}
\begin{Parallel}{}{} 
\ParallelLText{
\begin{center}
{\large \ggn{6}.}
\end{center}\vspace*{-7pt}

\gr{>E`an d'uo meg'ejh d'uo megej~wn >is'akic >~h| pollapl'asia, ka`i >afairej'enta
tin`a t~wn a>ut~wn >is'akic >~h| pollapl'asia, ka`i t`a loip`a to~ic a>uto~ic
>'htoi >'isa >est`in >`h >is'akic a>ut~wn pollapl'asia.}

\gr{D'uo g`ar meg'ejh t`a AB, GD d'uo megej~wn t~wn E, Z >is'akic >'estw
pollapl'asia, ka`i >afairej'enta t`a AH, GJ t~wn a>ut~wn t~wn E, Z >is'akic
>'estw pollapl'asia;  l'egw, <'oti ka`i loip`a t`a HB, JD to~ic E, Z
>'htoi >'isa >est`in >`h >is'akic a>ut~wn pollapl'asia.}\\~\\~\\

\epsfysize=1.8in
\centerline{\epsffile{Book05/fig06g.eps}}

\gr{>'Estw g`ar pr'oteron t`o HB t~w| E >'ison; l'egw, <'oti ka`i t`o JD t~w| Z
>'ison >est'in.}

\gr{Ke'isjw g`ar t~w| Z >'ison t`o GK. >epe`i >is'akic >est`i pollapl'asion
t`o AH to~u E ka`i t`o GJ to~u Z,  >'ison d`e t`o m`en HB t~w| E,
t`o d`e KG t~w| Z, >is'akic >'ara >est`i pollapl'asion t`o AB to~u E ka`i
t`o KJ to~u Z. >is'akic d`e <up'okeitai pollapl'asion t`o AB to~u E ka`i
t`o GD to~u Z; >'is'akic >'ara >est`i pollapl'asion t`o KJ to~u Z ka`i t`o GD to~u Z. >epe`i o>~un <ek'ateron t~wn KJ, GD
to~u Z >is'akic >est`i pollapl'asion, >'ison >'ara >est`i t`o KJ t~w| GD.
koin`on >afh|r'hsjw t`o GJ; loip`on >'ara t`o KG loip~w| t~w| JD
>'ison >est'in. >all`a t`o Z t~w| KG >estin >'ison; ka`i t`o JD >'ara
t~w| Z >'ison >est'in. <'wste e>i t`o HB t~w| E >'ison >est'in, ka`i t`o JD
>'ison >'estai t~w| Z.}

\gr{<Omo'iwc d`h de'ixomen, <'oti, k>`a|n pollapl'asion >~h| t`o HB to~u E,
tosautapl'asion >'estai ka`i t`o JD to~u Z.}

\gr{>E`an >'ara d'uo meg'ejh d'uo megej~wn >is'akic >~h| pollapl'asia, ka`i >afairej'enta
tin`a t~wn a>ut~wn >is'akic >~h| pollapl'asia, ka`i t`a loip`a to~ic a>uto~ic
>'htoi >'isa >est`in >`h >is'akic a>ut~wn pollapl'asia; <'oper >'edei
de~ixai.}}

\ParallelRText{
\begin{center}
{\large Proposition 6}$^\dag$
\end{center}

If two magnitudes are equal multiples of two (other) magnitudes, and some (parts) taken away (from the former magnitudes) are equal multiples of the latter (magnitudes, respectively),
then the remainders  are also either equal to the latter (magnitudes), or (are) equal multiples of them (respectively).

For let two magnitudes $AB$ and $CD$ be equal multiples of two magnitudes
$E$ and $F$ (respectively). And let the (parts)  taken away (from the former) $AG$ and $CH$ be equal multiples of $E$ and $F$ (respectively). I say that the remainders $GB$ and
$HD$ are also either equal to $E$ and $F$ (respectively), or (are) equal multiples of them.

\epsfysize=1.8in
\centerline{\epsffile{Book05/fig06e.eps}}

For let $GB$ be, first of all, equal to $E$. I say that $HD$ is also equal to $F$.

For let  $CK$ be made equal to $F$. Since $AG$ and $CH$ are equal multiples of $E$ and $F$ (respectively), and $GB$ (is) equal to $E$, and $KC$ to $F$,
$AB$ and $KH$ are thus equal multiples of $E$ and $F$ (respectively)
[Prop. 5.2].
And $AB$ and $CD$ are assumed (to be) equal multiples of $E$ and $F$ (respectively). Thus, $KH$ and $CD$ are equal multiples of $F$ and $F$ (respectively). Therefore, $KH$ and $CD$ are each equal multiples of $F$. Thus, $KH$
is equal to $CD$. Let $CH$ have be taken away from both. Thus, the remainder
$KC$ is equal to the remainder $HD$. But, $F$ is equal to $KC$. Thus, $HD$ is also equal to $F$. Hence, if $GB$ is equal to $E$ then $HD$ will also be equal to $F$.

So, similarly, we can show that even if $GB$ is a multiple of $E$ then $HD$
will also be the same multiple of $F$.

Thus, if two magnitudes are equal multiples of two (other) magnitudes, and some (parts) taken away (from the former magnitudes) are equal multiples of the latter (magnitudes, respectively),
then the remainders  are also either equal to the latter (magnitudes), or (are) equal multiples of them (respectively). (Which is) the very thing it was required to show.}
\end{Parallel}


\vspace{7pt}{\footnotesize \noindent$^\dag$ In modern notation, this
proposition reads $m\,\alpha-n\,\alpha = (m-n)\,\alpha$.}

%%%%%%
% Prop 5.7
%%%%%%
\pdfbookmark[1]{Proposition 5.7}{pdf5.7}
\begin{Parallel}{}{} 
\ParallelLText{
\begin{center}
{\large \ggn{7}.}
\end{center}\vspace*{-7pt}

\gr{T`a >'isa pr`oc t`o a>ut`o t`on a>ut`on >'eqei l'ogon ka`i t`o a>ut`o pr`oc
t`a >'isa.}

\gr{>'Estw >'isa meg'ejh t`a A, B, >'allo d'e ti, <`o >'etuqen,
m'egejoc t`o G; l'egw, <'oti <ek'ateron t~wn A, B pr`oc t`o G t`on
a>ut`on >'eqei l'ogon, ka`i t`o G pr`oc <ek'ateron t~wn A, B.}\\~\\

\epsfysize=0.75in
\centerline{\epsffile{Book05/fig07g.eps}}

\gr{E>il'hfjw g`ar t~wn m`en A, B >is'akic pollapl'asia t`a D, E, to~u d`e G
>'allo, <`o >'etuqen, pollapl'asion t`o Z.}

\gr{>Epe`i o>~un >is'akic >est`i pollapl'asion t`o D to~u A ka`i t`o E to~u B,
>'ison d`e t`o A t~w| B, >'ison >'ara ka`i t`o D t~w| E. >'allo d'e, <'o >'etuqen,
t`o Z. E>i >'ara <uper'eqei t`o D to~u Z, <uper'eqei ka`i t`o E to~u Z, ka`i
e>i >'ison, >'ison, ka`i e>i >'elatton, >'elatton. ka'i >esti t`a m`en D, E
t~wn A, B >is'akic pollapl'asia, t`o d`e Z to~u G >'allo, <`o >'etuqen,
pollapl'asion; >'estin >'ara <wc t`o A pr`oc t`o G, o<'utwc t`o B pr`oc t`o
G.}

\gr{L'egw [d'h], <'oti ka`i t`o G pr`oc <ek'ateron t~wn A, B t`on a>ut`on
>'eqei l'ogon.}

\gr{T~wn g`ar a>ut~wn kataskeuasj'entwn <omo'iwc de'ixomen, <'oti
>'ison >est`i t`o D t~w| E; >'allo d'e ti t`o Z;
e>i >'ara <uper'eqei t`o Z to~u D, <uper'eqei ka`i to~u E, ka`i e>i >'ison,
>'ison, ka`i e>i >'elatton, >'elatton. ka'i >esti t`o m`en Z to~u G
pollapl'asion, t`a d`e D, E t~wn A, B >'alla, <`a >'etuqen, >is'akic
pollapl'asia; >'estin >'ara <wc t`o G pr`oc t`o A, o<'utwc t`o G pr`oc
t`o B.}

\gr{T`a >'isa >'ara pr`oc t`o a>ut`o t`on a>ut`on >'eqei l'ogon ka`i t`o a>ut`o pr`oc
t`a >'isa.}\\~\\~\\~\\~\\~\\

\begin{center}
{\large \gr{P'orisma}.}
\end{center}\vspace*{-7pt}

\gr{>Ek d`h to'utou faner'on, <'oti >e`an meg'ejh tin`a >an'alogon >~h|, ka`i
>an'apalin >an'alogon >'estai. <'oper >'edei de~ixai.}}

\ParallelRText{
\begin{center}
{\large Proposition 7}
\end{center}

Equal (magnitudes) have the same ratio to the same 
(magnitude), and the latter (magnitude has the same ratio) to the equal
(magnitudes).

Let $A$ and $B$ be equal magnitudes, and $C$ some other random magnitude.
I say that $A$ and $B$ each have the same ratio to $C$, and (that) $C$ (has the same
ratio) to each of $A$ and $B$.

\epsfysize=0.75in
\centerline{\epsffile{Book05/fig07e.eps}}

For let the equal multiples $D$ and $E$ have been taken of $A$ and $B$ (respectively), and the other
random multiple $F$ of $C$.

Therefore, since $D$ and $E$ are equal multiples of $A$ and $B$ (respectively), and
$A$ (is) equal to $B$, $D$ (is) thus also equal to $E$. And $F$ (is) different, at random.
Thus, if $D$ exceeds $F$ then $E$ also exceeds $F$, and if ($D$ is) equal (to $F$ then
$E$ is also) equal (to $F$), and if ($D$ is) less (than $F$ then
$E$ is also) less (than $F$). And $D$ and $E$ are equal multiples of $A$ and $B$ (respectively), and $F$ another random multiple of $C$. Thus, as $A$ (is) to
$C$, so $B$ (is) to $C$ [Def. 5.5].

\mbox{[}So] I say that $C$$^\dag$ also has the same ratio to each of $A$ and $B$.

For, similarly,  we can show, by the same construction, that $D$ is equal to $E$.
And $F$ (has) some other (value). Thus, if $F$ exceeds $D$ then it also exceeds $E$,
and if ($F$ is) equal (to $D$  then it is also) equal (to $E$), and if ($F$ is) less (than $D$ then it is also) less (than $E$). And $F$ is a multiple of $C$, and $D$ and $E$ other random equal
multiples of $A$ and $B$. Thus, as $C$ (is) to $A$, so $C$ (is)
to $B$  [Def. 5.5]. 

Thus, equal (magnitudes) have the same ratio to the same 
(magnitude), and the latter (magnitude has the same ratio) to the equal
(magnitudes).\\

\begin{center}
\large{Corollary}$^\ddag$
\end{center}\vspace*{-7pt}

So (it is) clear, from this, that if some magnitudes are proportional
then they will also be proportional inversely. (Which is) the very thing it was
required to show.} 
\end{Parallel}


\vspace{7pt}{\footnotesize \noindent$^\dag$ The Greek text has ``$E$'', which is obviously a mistake.\\[0.5ex]
$^\ddag$ In modern notation, this corollary reads that if $\alpha:\beta::\gamma:\delta$ then
$\beta:\alpha::\delta:\gamma$.}

%%%%%%
% Prop 5.8
%%%%%%
\pdfbookmark[1]{Proposition 5.8}{pdf5.8}
\begin{Parallel}{}{} 
\ParallelLText{
\begin{center}
{\large \ggn{8}.}
\end{center}\vspace*{-7pt}

\gr{T~wn >an'iswn megej~wn t`o me~izon pr`oc t`o a>ut`o me'izona l'ogon >'eqei >'hper t`o >'elatton. ka`i t`o a>ut`o pr`oc t`o >'elatton me'izona
l'ogon >'eqei >'hper pr`oc t`o me~izon.}

\gr{>'Estw >'anisa meg'ejh t`a AB, G, ka`i >'estw me~izon t`o AB, >'allo d'e,
<`o >'etuqen, t`o D; l'egw, <'oti t`o AB pr`oc t`o D me'izona l'ogon
>'eqei >'hper t`o G pr`oc t`o D, ka`i t`o D pr`oc t`o G me'izona l'ogon
>'eqei >'hper pr`oc t`o AB.}\\~\\

\epsfysize=1.8in
\centerline{\epsffile{Book05/fig08g.eps}}

\gr{>Epe`i g`ar me~iz'on >esti t`o AB to~u G, ke'isjw t~w| G >'ison t`o BE;
t`o d`h >'elasson t~wn AE, EB pollaplasiaz'omenon >'estai pot`e to~u
D me~izon. >'estw pr'oteron t`o AE >'elatton to~u EB, ka`i pepollaplasi'asjw
t`o AE, ka`i >'estw a>uto~u pollapl'asion t`o ZH me~izon >`on to~u
D, ka`i <osapl'asi'on >esti t`o ZH to~u AE, tosautapl'asion gegon'etw
ka`i t`o m`en HJ to~u EB t`o d`e K to~u G; ka`i e>il'hfjw to~u D 
dipl'asion m`en t`o L, tripl'asion d`e t`o M, ka`i <ex~hc <en`i ple~ion,
<'ewc >`an t`o lamban'omenon pollapl'asion m`en g'enhtai to~u D, 
pr'wtwc d`e me~izon to~u K. e>il'hfjw, ka`i >'estw t`o N tetrapl'asion
m`en to~u D, pr'wtwc d`e me~izon to~u K.}

\gr{>Epe`i o>~un t`o K to~u N  pr'wtwc >est`in >'elatton, t`o K >'ara to~u 
M o>'uk >estin >'elatton. ka`i >epe`i >is'akic >est`i pollapl'asion
t`o ZH to~u AE ka`i t`o HJ to~u EB, >is'akic >'ara >est`i pollapl'asion
t`o ZH to~u AE ka`i t`o ZJ to~u AB. >is'akic d'e >esti pollapl'asion
t`o ZH to~u AE ka`i t`o K to~u G; >is'akic >'ara >est`i pollapl'asion
t`o ZJ to~u AB ka`i t`o K to~u G. t`a ZJ, K >'ara t~wn AB, G >is'akic
>est`i pollapl'asia. p'alin, >epe`i >is'akic >est`i pollapl'asion t`o HJ 
to~u EB ka`i t`o K to~u G, >'ison d`e t`o EB t~w| G, >'ison >'ara
ka`i t`o HJ t~w| K. t`o d`e K to~u M o>'uk >estin >'elatton; o>ud>
>'ara t`o HJ to~u M >'elatt'on >estin. me~izon d`e t`o ZH to~u D; <'olon
>'ara t`o ZJ sunamfot'erwn t~wn D, M me~iz'on >estin. >all`a sunamf'otera
t`a D, M t~w| N >estin >'isa, >epeid'hper t`o M to~u D tripl'asi'on
>estin, sunamf'otera d`e t`a M, D to~u D >esti tetrapl'asia, >'esti
d`e ka`i t`o N to~u D tetrapl'asion; sunamf'otera >'ara t`a M, D t~w| N >'isa
>est'in. >all`a t`o ZJ t~wn M, D me~iz'on >estin; t`o ZJ >'ara to~u N
<uper'eqei; t`o d`e K to~u N o>uq <uper'eqei. ka'i >esti t`a m`en ZJ, K t~wn AB, G >is'akic pollapl'asia, t`o d`e N to~u D >'allo, <`o
>'etuqen, pollapl'asion; t`o AB >'ara pr`oc t`o D me'izona l'ogon
>'eqei >'hper t`o G pr`oc t`o D.}

\gr{L'egw d'h, <'oti ka`i t`o D pr`oc t`o G me'izona l'ogon >'eqei >'hper
t`o D pr`oc t`o AB.}

\gr{T~wn g`ar a>ut~wn kataskeuasj'entwn <omo'iwc de'ixomen, <'oti
t`o m`en N to~u K <uper'eqei, t`o d`e N to~u ZJ o>uq <uper'eqei.
ka'i >esti t`o m`en N to~u D pollapl'asion, t`a d`e ZJ, K t~wn AB, G
>'alla, <`a >'etuqen, >is'akic pollapl'asia; t`o D >'ara pr`oc t`o G
me'izona l'ogon >'eqei >'hper t`o D pr`oc t`o AB.}

\gr{>All`a d`h t`o AE to~u EB me~izon >'estw. t`o d`h >'elatton t`o EB
pollaplasiaz'omenon >'estai pot`e to~u D me~izon. pepollaplasi'asjw, ka`i
>'estw t`o HJ pollapl'asion m`en to~u EB, me~izon d`e to~u D; ka`i
<osapl'asi'on >esti t`o HJ to~u EB, tosautapl'asion gegon'etw
ka`i t`o m`en ZH to~u AE, t`o d`e K to~u G. <omo'iwc d`h de'ixomen,
<'oti t`a ZJ, K t~wn AB, G >is'akic >est`i pollapl'asia; ka`i e>il'hfjw
<omo'iwc t`o N pollapl'asion m`en to~u D, pr'wtwc d`e me~izon
to~u ZH; <'wste p'alin t`o ZH to~u M o>'uk >estin >'elasson. me~izon
d`e t`o HJ to~u D; <'olon >'ara t`o ZJ t~wn D, M, tout'esti to~u N,
<uper'eqei. t`o d`e K to~u N o>uq <uper'eqei, >epeid'hper ka`i t`o
ZH me~izon >`on to~u HJ, tout'esti to~u K, to~u N o>uq <uper'eqei.
ka`i <wsa'utwc katakoloujo~untec to~ic >ep'anw pera'inomen
t`hn >ap'odeixin.}

\gr{T~wn >'ara >an'iswn megej~wn t`o me~izon pr`oc t`o a>ut`o me'izona l'ogon >'eqei >'hper t`o >'elatton; ka`i t`o a>ut`o pr`oc t`o >'elatton me'izona
l'ogon >'eqei >'hper pr`oc t`o me~izon; <'oper >'edei de~ixai.}}

\ParallelRText{
\begin{center}
{\large Proposition 8}
\end{center}

For unequal magnitudes, the greater (magnitude) has a greater
ratio  than the lesser to the same (magnitude). And the latter (magnitude)
has a greater ratio to the lesser (magnitude) than to the greater.

Let $AB$ and $C$ be unequal magnitudes, and let $AB$ be the greater (of the two),
and $D$ another random magnitude. I say that $AB$ has a greater ratio to $D$
than  $C$ (has) to $D$, and (that) $D$ has a greater ratio to $C$ than (it has) to $AB$.

\epsfysize=1.8in
\centerline{\epsffile{Book05/fig08e.eps}}

For since $AB$ is greater than $C$, let $BE$ be made equal to $C$.
So, the lesser of $AE$ and $EB$, being multiplied, will sometimes be greater than $D$ [Def. 5.4]. First of all, let $AE$ be less than $EB$, and let $AE$ have been multiplied, 
and let $FG$ be a multiple of it which (is) greater than $D$.
And as many times as $FG$ is (divisible) by $AE$, so many times
let $GH$ also have become (divisible) by $EB$, and $K$ by $C$. And
let the double multiple $L$ of $D$ have been taken, and the triple
multiple $M$, and several more, (each increasing) in order by one,
until the (multiple) taken  becomes the first multiple of $D$ (which is) greater than $K$. Let it have been taken, and let it also be the quadruple multiple
$N$ of $D$---the first (multiple) greater than $K$.

Therefore, since $K$ is less than $N$ first, $K$ is thus not less than $M$.
And since $FG$ and $GH$ are equal multiples of $AE$ and $EB$ (respectively),
$FG$ and $FH$ are thus equal multiples of $AE$ and $AB$ (respectively) [Prop. 5.1].
And $FG$ and $K$ are equal multiples of $AE$ and $C$ (respectively).
Thus, $FH$ and $K$ are equal multiples of $AB$ and $C$ (respectively).
Thus, $FH$, $K$ are equal multiples of $AB$, $C$. Again, since $GH$ and 
$K$ are equal
multiples of $EB$ and $C$, and $EB$ (is) equal to $C$, $GH$ (is) thus also equal to
$K$. And $K$ is not less than $M$. Thus, $GH$ not less than $M$ either. And $FG$
(is) greater than $D$. Thus, the whole of $FH$ is greater than $D$
and $M$ (added) together. But, $D$ and $M$ (added) together is equal to $N$, 
inasmuch as $M$ is three times $D$, and $M$ and $D$ (added) together is four
times $D$, and $N$ is also four times $D$. Thus, $M$ and $D$ (added) together is
equal to $N$. But, $FH$ is greater than $M$ and $D$. Thus, $FH$ exceeds $N$. And $K$ does not exceed $N$. 
And $FH$, $K$  are equal multiples of $AB$, $C$, and $N$ another random multiple
of $D$. Thus, $AB$ has a greater ratio to $D$  than $C$ (has) to $D$ [Def. 5.7].

So, I say that $D$ also has a greater ratio to $C$ than $D$ (has) to $AB$.

For, similarly, by the same construction, we can show that 
$N$ exceeds $K$, and $N$ does not exceed $FH$. And $N$ is a multiple of $D$,
and $FH$, $K$ other  random  equal multiples of $AB$, $C$ (respectively). Thus, $D$ has a greater
ratio to  $C$ than $D$ (has) to $AB$ [Def. 5.5].

And so let $AE$ be greater than $EB$. So, the lesser, $EB$, being multiplied, will
sometimes be greater than $D$. Let it have been multiplied, and let $GH$ be
a multiple of $EB$ (which is) greater than $D$. And as many times as $GH$ is
(divisible) by $EB$, so many times let $FG$ also have become (divisible) by $AE$,
and $K$ by $C$. So, similarly (to the above), we can show that $FH$ and $K$ are equal multiples of
$AB$ and $C$ (respectively). And, similarly (to the above), let the multiple $N$ of $D$,  (which is)
the first (multiple) greater than $FG$, have been taken. So,  $FG$ is again not less than $M$.
And $GH$ (is) greater than $D$. Thus, the whole of $FH$ exceeds $D$ and $M$, that is to say $N$.  And $K$ does not exceed $N$, inasmuch as $FG$, which (is) greater than
$GH$---that is to say, $K$---also does not exceed $N$. And, following the above (arguments), we  (can) complete the proof in the same manner.

Thus, for unequal magnitudes, the greater (magnitude) has a greater
ratio  than the lesser to the same (magnitude). And the latter (magnitude)
has a greater ratio to the lesser (magnitude) than to the greater. (Which is)
the very thing it was required to show.}
\end{Parallel}

%%%%%%
% Prop 5.9
%%%%%%
\pdfbookmark[1]{Proposition 5.9}{pdf5.9}
\begin{Parallel}{}{} 
\ParallelLText{
\begin{center}
{\large \ggn{9}.}
\end{center}\vspace*{-7pt}

\gr{T`a pr`oc t`o a>ut`o t`on a>ut`on >'eqonta l`ogon >'isa >all'hloic >est'in;
ka`i pr`oc <`a t`o a>ut`o t`on a>ut`on <'eqei l'ogon, >eke~ina >'isa >est'in.}\\

\epsfysize=0.5in
\centerline{\epsffile{Book05/fig09g.eps}}

\gr{>Eq'etw g`ar <ek'ateron t~wn A, B pr`oc t`o G t`on a>ut`on l'ogon;
l'egw, <'oti >'ison >est`i t`o A t~w| B.}

\gr{E>i g`ar m'h, o>uk >`an <ek'ateron t~wn A, B pr`oc t`o G t`on a>ut`on
e>~iqe l'ogon; >'eqei d'e; >'ison >'ara >est`i t`o A t~w| B.}

\gr{>Eq'etw d`h p'alin t`o G pr`oc <ek'ateron t~wn A, B t`on a>ut`on
l'ogon; l'egw, <'oti >'ison >est`i t`o A t~w| B.}

\gr{E>i g`ar m'h, o>uk >`an t`o G pr`oc <ek'ateron t~wn A, B t`on a>ut`on e>~iqe l'ogon; >'eqei d'e; >'ison >'ara >est`i t`o A t~w| B.}

\gr{T`a >'ara pr`oc t`o a>ut`o t`on a>ut`on >'eqonta l'ogon >'isa >all'hloic
>est'in; ka`i pr`oc <`a t`o a>ut`o t`on a>ut`on >'eqei l'ogon, >eke~ina
>'isa >est'in; <'oper >'edei de~ixai.}}

\ParallelRText{
\begin{center}
{\large Proposition 9}
\end{center}

(Magnitudes) having the same ratio to the same (magnitude) are equal to one another. And those (magnitudes) to which the same (magnitude)
has the same ratio are equal.

\epsfysize=0.5in
\centerline{\epsffile{Book05/fig09e.eps}}

For let $A$ and $B$ each have the same ratio to $C$. I say that $A$ is equal to $B$.

For if not, $A$ and $B$ would not each have the same ratio to $C$ [Prop. 5.8].
But they do. Thus, $A$ is equal to $B$.

So, again, let $C$ have the same ratio to each of $A$ and $B$. I say that $A$ is equal to $B$.

For if not, $C$ would not have the same ratio to each of $A$ and $B$ [Prop. 5.8]. But it does. Thus, $A$ is equal to $B$.

Thus, (magnitudes) having the same ratio to the same (magnitude) are equal to one another. And those (magnitudes) to which the same (magnitude)
has the same ratio are equal. (Which is) the very thing it was required to show.}
\end{Parallel}

%%%%%%
% Prop 5.10
%%%%%%
\pdfbookmark[1]{Proposition 5.10}{pdf5.10}
\begin{Parallel}{}{} 
\ParallelLText{
\begin{center}
{\large \ggn{10}.}
\end{center}\vspace*{-7pt}

\gr{T~wn pr`oc t`o a>ut`o l'ogon >eq'ontwn t`o me'izona l'ogon >'eqon >eke~ino me~iz'on >estin; pr`oc <`o d`e t`o a>ut`o me'izona l'ogon >'eqei, >eke~ino
>'elatt'on >estin.}\\

\epsfysize=0.5in
\centerline{\epsffile{Book05/fig10g.eps}}

\gr{>Eq'etw g`ar t`o A pr`oc t`o G me'izona l'ogon >'hper t`o B pr`oc t`o G;
l'egw, <'oti me~iz'on >esti t`o A  to~u B.}

\gr{E>i g`ar m'h, >'htoi >'ison >est`i t`o A t~w| B >`h >'elasson. >'ison
m`en o>~un o>'uk >est`i t`o A t~w| B; <ek'ateron g`ar  >`an t~wn A, B
pr`oc t`o G t`on a>ut`on e>~iqe l'ogon. o>uk >'eqei d'e; o>uk >'ara >'ison
>est`i t`o A t~w| B. o>ud`e m`hn >'elass'on >esti t`o A to~u B; t`o A g`ar
>`an pr`oc t`o G  >el'assona l'ogon e>~iqen >'hper t`o B pr`oc t`o G. o>uk
>'eqei d'e; o>uk >'ara >'elass'on >esti t`o A to~u B. >ede'iqjh d`e o>ud`e >'ison; me~izon >'ara >est`i t`o A to~u B.}

\gr{>Eq'etw d`h p'alin t`o G pr`oc t`o B me'izona l'ogon >'hper t`o G pr`oc
t`o A; l'egw, <'oti >'elass'on >esti t`o B to~u A.}

\gr{E>i g`ar m'h, >'htoi >'ison >est`in >`h me~izon. >'ison m`en o>~un o>'uk
>esti t`o B t~w| A; t`o G g`ar >`an pr`oc <ek'ateron t~wn A, B t`on a>ut`on
e>~iqe l'ogon. o>uk >'eqei d'e; o>uk >'ara >'ison >est`i t`o A t~w| B. o>ud`e
m`hn me~iz'on >esti t`o B to~u A; t`o G g`ar >`an pr`oc t`o B >el'assona
l'ogon e>~iqen >'hper pr`oc t`o A. o>uk >'eqei d'e; o>uk >'ara
me~iz'on >esti t`o B to~u A. >ede'iqjh d'e, <'oti o>ud`e >'ison; >'elatton
>'ara >est`i t`o B to~u A.}

\gr{T~wn >'ara  pr`oc t`o a>ut`o l'ogon >eq'ontwn t`o me'izona l'ogon >'eqon me~iz'on >estin; ka`i pr`oc <`o  t`o a>ut`o me'izona l'ogon >'eqei, >eke~ino
>'elatt'on >estin; <'oper >'edei de~ixai.}}

\ParallelRText{
\begin{center}
{\large Proposition 10}
\end{center}

For (magnitudes) having a ratio to the same (magnitude), that (magnitude which) has the
greater ratio is (the) greater. And that (magnitude) to which the latter (magnitude) has a greater ratio is (the)
lesser.

\epsfysize=0.5in
\centerline{\epsffile{Book05/fig10e.eps}}

For let $A$ have a greater ratio to $C$ than $B$ (has) to $C$. I say that $A$ is greater than $B$.

For if not, $A$ is surely either equal to or less than $B$. In fact, $A$ is not
equal to $B$. For (then) $A$ and $B$ would each have the same ratio to $C$ 
 [Prop. 5.7]. But they do not. Thus, $A$
 is not equal to $B$. Neither, indeed, is $A$ less than $B$. For (then)
 $A$ would have a lesser ratio to $C$ than $B$ (has) to $C$ [Prop. 5.8]. But it does not. Thus, $A$ is not less
 than $B$. And it was shown not (to be) equal either. Thus, $A$ is greater than $B$.
 
 So, again, let $C$ have a greater ratio to $B$ than $C$ (has) to $A$. I say that
 $B$ is less than $A$.
 
 For if not, (it is) surely either equal or greater. In fact, $B$ is not equal to 
 $A$. For (then) $C$ would have the same ratio to each of $A$ and $B$  [Prop. 5.7]. But it does not. Thus, $A$ is not equal to $B$. 
 Neither, indeed, is $B$ greater than $A$. For (then) $C$ would have a lesser
 ratio to $B$ than (it has) to $A$ [Prop. 5.8].
 But it does not. Thus, $B$ is not greater than $A$. And it was shown that
  (it is) not equal (to $A$) either. Thus, $B$ is less than $A$.
  
 Thus, for (magnitudes) having a ratio to the same (magnitude), that (magnitude which)
 has the greater  ratio is (the) greater. And that (magnitude) to which the latter (magnitude) has a greater ratio is (the)
lesser. (Which is) the very thing it was required to show.}
\end{Parallel}

%%%%%%
% Prop 5.11
%%%%%%
\pdfbookmark[1]{Proposition 5.11}{pdf5.11}
\begin{Parallel}{}{} 
\ParallelLText{
\begin{center}
{\large \ggn{11}.}
\end{center}\vspace*{-7pt}

\gr{O<i t~w| a>ut~w| l'ogw| o<i a>uto`i ka`i >all'hloic e>is`in o<i a>uto'i.}\\

\epsfysize=0.75in
\centerline{\epsffile{Book05/fig11g.eps}}

\gr{>'Estwsan g`ar <wc m`en t`o A pr`oc t`o B, o<'utwc t`o G pr`oc t`o D, <wc
d`e t`o G pr`oc t`o D, o<'utwc t`o E pr`oc t`o Z; l'egw, <'oti >est`in <wc
t`o A pr`oc t`o B, o<'utwc t`o E pr`oc t`o Z.}

\gr{E>il'hfjw g`ar t~wn A, G, E >is'akic pollapl'asia t`a H, J, K, t~wn d`e B, D, Z 
>'alla, <`a >'etuqen, >is'akic pollapl'asia t`a L, M, N.}

\gr{Ka`i >epe'i >estin <wc t`o A pr`oc t`o B, o<'utwc t`o G
pr`oc t`o D, ka`i e>'ilhptai t~wn m`en A, G >is'akic pollapl'asia t`a H, J,
t~wn d`e B, D >'alla, <`a >'etuqen, >is'akic pollapl'asia t`a L, M, e>i
>'ara <uper'eqei t`o H to~u L, <uper'eqei ka`i t`o J to~u M, ka`i e>i
>'ison >est'in, >'ison, ka`i e>i >elle'ipei, >elle'ipei. p'alin, >epe'i
>estin <wc t`o G pr`oc t`o D, o<'utwc t`o E pr`oc t`o Z, ka`i e>'ilhptai
t~wn  G, E >is'akic pollapl'asia t`a J, K, t~wn d`e D, Z >'alla, <`a >'etuqen,
>is'akic pollapl'asia t`a M, N, e>i >'ara <uper'eqei t`o J to~u M,
<uper'eqei ka`i t`o K to~u N, ka`i e>i >'ison, >'ison, ka`i e>i
>'ellaton, >'elatton. >all`a e>i <upere~iqe t`o J to~u M, <upere~iqe
ka`i t`o H to~u L, ka`i e>i >'ison, >'ison, ka`i e>i >'elatton, 
>'elatton; <'wste ka`i e>i <uper'eqei t`o H to~u L, <uper'eqei ka`i t`o
K to~u N, ka`i e>i >'ison, >'ison, ka`i e>i >'elatton, >'elatton. ka'i
>esti t`a  m`en H, K t~wn A, E >is'akic pollapl'asia, t`a d`e L, N t~wn
B, Z >'alla, <`a >'etuqen, >is'akic pollapl'asia; >'estin >'ara <wc t`o A pr`oc
t`o B, o<'utwc t`o E pr`oc t`o Z.}

\gr{O<i >'ara t~w| a>ut~w| l'ogw| o<i a>uto`i ka`i >all'hloic e>is`in o<i a>uto'i;
<'oper >'edei de~ixai.}}

\ParallelRText{
\begin{center}
{\large Proposition 11}$^\dag$
\end{center}

(Ratios which are) the same with the same ratio  are also the same with one another.

\epsfysize=0.75in
\centerline{\epsffile{Book05/fig11e.eps}}

For let it be that as $A$ (is) to $B$, so $C$ (is) to $D$, and as $C$ (is) to $D$, so $E$ (is) to $F$. I say
that as  $A$ is to $B$, so $E$ (is) to $F$.

For let the equal multiples $G$, $H$,  $K$ have been taken of $A$, $C$,  $E$ (respectively), and
the other random equal multiples $L$, $M$,  $N$ of $B$, $D$, $F$ (respectively).

And since as $A$ is to $B$, so $C$ (is) to $D$, and the equal multiples $G$ and $H$
have been taken of $A$ and $C$ (respectively), and the other random equal
multiples $L$ and $M$ of $B$ and $D$ (respectively),  thus if $G$ exceeds $L$ then
$H$ also exceeds $M$, and if ($G$ is) equal (to $L$ then $H$ is also)
equal (to $M$), and if ($G$ is) less (than $L$ then $H$ is also) less (than $M$)  [Def. 5.5]. Again, since as $C$ is to $D$, so $E$
(is) to $F$, and the equal multiples $H$ and $K$ have been taken of $C$ and $E$
(respectively), and the other random equal multiples $M$ and $N$ of
$D$ and $F$ (respectively), thus if $H$ exceeds $M$ then
$K$ also exceeds $N$, and if ($H$ is) equal  (to $M$ then $K$ is also)
equal (to $N$), and if ($H$ is) less (than $M$  then $K$ is also) less (than $N$) [Def. 5.5]. But (we saw that) if $H$ was exceeding $M$ then $G$ was also
exceeding $L$, and if ($H$ was) equal (to $M$ then $G$ was also)
equal (to $L$), and if ($H$ was) less (than $M$ then $G$ was also) less (than $L$).
And, hence, if $G$ exceeds $L$ then
$K$ also exceeds $N$, and if ($G$ is) equal (to $L$ then $K$ is also)  equal (to $N$), and if ($G$ is) less (than $L$ then $K$ is also) less (than $N$). And $G$ and $K$ are equal multiples of $A$ and $E$ (respectively), and $L$ and $N$
other random equal multiples of $B$ and $F$ (respectively). Thus, as $A$ is to $B$,
so $E$ (is) to $F$ [Def. 5.5].

Thus, (ratios which are) the same with the same ratio  are also the same with one another. (Which is) the very thing it was required to show.}
\end{Parallel}


\vspace{7pt}{\footnotesize \noindent$^\dag$ In modern notation, this proposition
reads that if $\alpha:\beta::\gamma:\delta$ and $\gamma:\delta::\epsilon:\zeta$ then $\alpha:\beta::\epsilon:\zeta$.}

%%%%%%
% Prop 5.12
%%%%%%
\pdfbookmark[1]{Proposition 5.12}{pdf5.12}
\begin{Parallel}{}{} 
\ParallelLText{
\begin{center}
{\large \ggn{12}.}
\end{center}\vspace*{-7pt}

\gr{>E`an >~h| <oposao~un meg'ejh  >an'alogon, >'estai <wc <`en t~wn <hgoum'enwn pr`oc <`en t~wn <epom'enwn, o<'utwc <'apanta t`a <hgo'umena pr`oc <'apanta t`a <ep'omena.}\\

\epsfysize=1.3in
\centerline{\epsffile{Book05/fig12g.eps}}

\gr{>'Estwsan <oposao~un meg'ejh >an'alogon t`a A, B, G, D, E, Z, <wc t`o
A pr`oc t`o B, o<'utwc t`o G pr`oc t`o D, ka`i t`o E pr`oc to Z; l'egw, <'oti
>est`in <wc t`o A pr`oc t`o B, o<'utwc t`a A, G, E pr`oc t`a B, D, Z.}

\gr{E>il'hfjw g`ar t~wn m`en A, G, E >is'akic pollapl'asia t`a H, J, K, t~wn
d`e B, D, Z >'alla, <`a >'etuqen, >is'akic pollapl'asia t`a L, M, N.}

\gr{Ka`i >epe'i >estin <wc t`o A pr`oc t`o B, o<'utwc t`o G pr`oc t`o D, ka`i
t`o E pr`oc t`o Z, ka`i e>'ilhptai t~wn m`en A, G, E >is'akic pollapl'asia
t`a H, J, K t~wn d`e B, D, Z >'alla, <`a >'etuqen, >is'akic pollapl'asia
t`a L, M, N, e>i >'ara <uper'eqei t`o H to~u L, <uper'eqei ka`i t`o J
to~u M, ka`i t`o K to~u N, ka`i e>i >'ison, >'ison, ka`i e>i >'elatton, >'elatton.
<'wste ka`i e>i <uper'eqei t`o H to~u L, <uper'eqei ka`i t`a H, J, K t~wn
L, M, N,  ka`i e>i >'ison, >'isa, ka`i e>i >'elatton, >'elattona. ka'i >esti
t`o m`en H ka`i t`a H, J, K to~u A ka`i t~wn A, G, E >is'akic
pollapl'asia, >epeid'hper >e`an >~h|  <oposao~un meg'ejh <oposwno~un
megej~wn >'iswn t`o pl~hjoc <'ekaston <ek'astou >is'akic pollapl'asion,
<osapl'asi'on >estin <`en t~wn megej~wn <en'oc, tosautapl'asia >'estai
ka`i t`a p'anta t~wn p'antwn. di`a t`a a>ut`a d`h ka`i t`o L ka`i t`a L, M, N
to~u B ka`i t~wn B, D, Z >is'akic >est`i pollapl'asia;
>'estin >'ara <wc t`o A pr`oc t`o B, o<'utwc t`a A, G, E pr`oc t`a B, D, Z.}

\gr{>E`an >'ara >~h| <oposao~un meg'ejh  >an'alogon, >'estai <wc <`en t~wn <hgoum'enwn pr`oc <`en t~wn <epom'enwn, o<'utwc <'apanta t`a <hgo'umena pr`oc <'apanta t`a <ep'omena; <'oper >'edei de~ixai.}}

\ParallelRText{
\begin{center}
{\large Proposition 12}$^\dag$
\end{center}

If there are any number  of magnitudes whatsoever
(which are) proportional then as one of the leading (magnitudes is) to one of the following, so
will 
all of the leading (magnitudes) be to all of the following.

\epsfysize=1.3in
\centerline{\epsffile{Book05/fig12e.eps}}

Let there be any number of magnitudes whatsoever, 
$A$, $B$, $C$, $D$, $E$, $F$, (which are) proportional, (so that) as $A$ (is) to $B$, so $C$ (is) to $D$, and 
$E$ to $F$. I say that as $A$ is to $B$, so  $A$, $C$, $E$ (are) to  $B$, $D$, $F$.

For let the equal multiples $G$, $H$,  $K$ have been taken of $A$, $C$, $E$
(respectively), and the other random equal multiples $L$, $M$,  $N$ of
$B$, $D$, $F$ (respectively).

And since as $A$ is to $B$, so $C$ (is) to $D$, and $E$ to $F$, and the equal multiples
$G$, $H$, $K$ have been taken of $A$, $C$,  $E$ (respectively), and the other
random equal multiples $L$, $M$,  $N$ of $B$, $D$,  $F$ (respectively), thus
if $G$ exceeds $L$ then $H$ also exceeds $M$, and $K$ (exceeds) $N$, and if ($G$ is)
equal (to $L$ then $H$ is also) equal (to $M$, and $K$  to $N$), and if ($G$ is) less (than $L$ then $H$ is also) less (than $M$, and $K$  than $N$)  [Def. 5.5].  And, hence, if $G$ exceeds $L$ then
$G$, $H$, $K$ also exceed $L$, $M$, $N$, and if ($G$ is) equal
(to $L$ then $G$, $H$, $K$ are also) equal (to  $L$, $M$, $N$) and if ($G$ is) less (than $L$ then
$G$, $H$, $K$ are also) less (than $L$, $M$, $N$). And $G$ and $G$, $H$, $K$ are equal multiples
of $A$ and $A$, $C$, $E$ (respectively), inasmuch as if there are any number of magnitudes whatsoever (which are)  equal multiples, respectively, of
some  (other) magnitudes,  of equal  number (to them), then as many times as one of the (first) magnitudes is (divisible) by one (of the second), so many times
will all (of the first magnitudes) also (be divisible) by all (of the second)
 [Prop. 5.1]. So, for the same (reasons), $L$
 and $L$, $M$, $N$ are also equal multiples of $B$ and $B$, $D$, $F$ (respectively). Thus,
 as $A$ is to $B$, so $A$, $C$, $E$ (are) to $B$, $D$, $F$ (respectively).
 
Thus, if there are any number  of magnitudes whatsoever
(which are) proportional then as one of the leading (magnitudes is) to one of the following, so
will 
all of the leading (magnitudes) be to all of the following. (Which is) the
very thing it was required to show.}
\end{Parallel}


\vspace{7pt}{\footnotesize \noindent$^\dag$ In modern notation, this proposition
reads that if $\alpha:\alpha'::\beta:\beta'::\gamma:\gamma'$ {\em etc.}\ then
$\alpha:\alpha'::(\alpha+\beta+\gamma+\cdots):(\alpha'+\beta'+\gamma'+\cdots)$.}

%%%%%%
% Prop 5.13
%%%%%%
\pdfbookmark[1]{Proposition 5.13}{pdf5.13}
\begin{Parallel}{}{} 
\ParallelLText{
\begin{center}
{\large \ggn{13}.}
\end{center}\vspace*{-7pt}

\gr{>E`an pr~wton pr`oc de'uteron t`on a>ut`on >`eqh|  l'ogon ka`i tr'iton
pr`oc t'etarton, tr'iton d`e pr`oc t'etarton me'izona l'ogon >'eqh| >`h p'empton
pr`oc <'ekton, ka`i pr~wton pr`oc de'uteron me'izona l'ogon <'exei >`h
p'empton pr`oc <'ekton.}\\

\epsfysize=0.65in
\centerline{\epsffile{Book05/fig13g.eps}}

\gr{Pr~wton g`ar t`o A pr`oc de'uteron t`o B t`on a>ut`on >eq'etw l'ogon ka`i
tr'iton t`o G pr`oc t'etarton t`o D, tr'iton d`e t`o G pr`oc t'etarton t`o D
me'izona l'ogon >eq'etw >`h p'empton t`o E pr`oc <'ekton t`o Z. l'egw,
<'oti ka`i pr~wton t`o A pr`oc de'uteron t`o B me'izona l'ogon <'exei
>'hper p'empton t`o E pr`oc <'ekton t`o Z.}

\gr{>Epe`i g`ar >'esti tin`a t~wn m`en G, E >is'akic pollapl'asia, t~wn d`e D, Z
>'alla, <`a >'etuqen, >is'akic pollapl'asia, ka`i t`o m`en to~u G pollapl'asion
to~u to~u D pollaplas'iou <uper'eqei, t`o d`e to~u E pollapl'asion to~u to~u Z
pollaplas'iou o>uq <uper'eqei, e>il'hfjw, ka`i >'estw t~wn m`en G, E >is'akic
pollapl'asia t`a H, J, t~wn d`e D, Z >'alla, <`a >'etuqen, >is'akic pollapl'asia
t`a K, L, <'wste t`o m`en H to~u K <uper'eqein, t`o d`e J to~u L m`h
<uper'eqein; ka`i <osapl'asion m'en >esti t`o H to~u G, tosautapl'asion
>'estw ka`i t`o M to~u A, <osapl'asion d`e t`o K to~u D, tosautapl'asion
>'estw ka`i t`o N to~u B.}

\gr{Ka`i >epe'i >estin <wc t`o A pr`oc t`o B, o<'utwc t`o G pr`oc t`o D, ka`i
e>'ilhptai t~wn m`en A, G >is'akic pollapl'asia t`a M, H, t~wn d`e B, D
>'alla, <`a >'etuqen, >is'akic pollapl'asia t`a N, K, e>i >'ara <uper'eqei
t`o M to~u N, <uper'eqei ka`i t`o H to~u K, ka`i e>i >'ison, >'ison,
ka`i e>i >'elatton, >'ellaton. <uper'eqei d`e t`o H to~u K; <uper'eqei
>'ara ka`i t`o M to~u N. t`o d`e J to~u L o>uq <uper'eqei; ka'i
>esti t`a m`en M, J t~wn A, E >is'akic pollapl'asia, t`a d`e N, L t~wn
B, Z >'alla, <`a >'etuqen, >is'akic pollapl'asia; t`o >'ara A pr`oc
t`o B me'izona l'ogon >'eqei >'hper t`o E pr`oc t`o Z.}

\gr{>E`an >'ara pr~wton pr`oc de'uteron t`on a>ut`on >`eqh|  l'ogon ka`i tr'iton
pr`oc t'etarton, tr'iton d`e pr`oc t'etarton me'izona l'ogon >'eqh| >`h p'empton
pr`oc <'ekton, ka`i pr~wton pr`oc de'uteron me'izona l'ogon <'exei >`h
p'empton pr`oc <'ekton; <'oper >'edei de~ixai.}}

\ParallelRText{
\begin{center}
{\large Proposition 13}$^\dag$
\end{center}

If a first (magnitude) has the same ratio to a second that a third (has) to a fourth, and the third (magnitude) has a greater ratio to the fourth than a fifth (has) to a sixth, then the first (magnitude) will also have a greater ratio to the
second  than the fifth (has) to the sixth.

\epsfysize=0.65in
\centerline{\epsffile{Book05/fig13e.eps}}

For let a first (magnitude) $A$ have the same ratio to a second $B$ that a
third $C$ (has) to a fourth $D$, and let the third (magnitude) $C$ have a
greater ratio to the fourth $D$  than a fifth $E$ (has) to a sixth $F$. I say that the first (magnitude) $A$
will also have a greater ratio to the second $B$ than the fifth $E$ (has) to the
sixth $F$.

For since there are some equal multiples of $C$ and $E$, and other random equal multiples of $D$ and $F$, (for which) the multiple of $C$ exceeds the (multiple) of $D$, 
and the multiple of $E$ does not exceed the multiple of $F$ [Def. 5.7], let them have been taken. And let
$G$ and $H$ be equal multiples of $C$ and $E$ (respectively), and $K$ and $L$
other random equal multiples of $D$ and $F$ (respectively), such that 
$G$ exceeds $K$, but $H$ does not exceed $L$. And as many times as $G$ is (divisible)
by $C$, so many times let $M$ be (divisible) by $A$. And as many times as
$K$ (is divisible) by $D$, so many times let $N$ be (divisible) by $B$.

And since as $A$ is to $B$, so $C$ (is) to $D$, and the equal multiples $M$ and $G$
have been taken of $A$ and $C$ (respectively), and the other random equal
multiples $N$ and $K$ of $B$ and $D$ (respectively), thus if $M$ exceeds $N$ then
$G$ exceeds $K$, and if ($M$ is) equal (to $N$ then $G$ is also) equal (to $K$),
and if ($M$ is) less (than $N$ then $G$ is also) less (than $K$) [Def. 5.5]. And $G$ exceeds $K$. Thus, $M$ also exceeds $N$.
And $H$ does not exceeds $L$. And $M$ and $H$ are equal multiples of
$A$ and $E$ (respectively), and $N$ and $L$ other random equal multiples of $B$ and $F$
(respectively).  Thus, $A$ has a greater ratio to $B$ than $E$ (has) to $F$  [Def. 5.7].

Thus, if a first (magnitude) has the same ratio to a second  that a third (has) to a fourth, and a third (magnitude) has a greater ratio to a fourth than a fifth (has) to a sixth, then the first (magnitude) will also have a greater ratio to the
second than the fifth (has) to the sixth. (Which is)
the very thing it was required to show.}
\end{Parallel}


\vspace{7pt}{\footnotesize \noindent$^\dag$ In modern notation, this proposition
reads that if $\alpha:\beta::\gamma:\delta$ and $\gamma:\delta > \epsilon:\zeta$ then $\alpha:\beta>\epsilon:\zeta$.}

%%%%%%
% Prop 5.14
%%%%%%
\pdfbookmark[1]{Proposition 5.14}{pdf5.14}
\begin{Parallel}{}{} 
\ParallelLText{
\begin{center}
{\large \ggn{14}.}
\end{center}\vspace*{-7pt}

\gr{>E`an pr~wton pr`oc de'uteron t`on a>ut`on >'eqh| l'ogon ka`i tr'iton  pr`oc t'etarton, t`o d`e
pr~wton to~u tr'itou me~izon >~h|, ka`i t`o de'uteron to~u tet'artou
me~izon >'estai, k>`an >'ison, >'ison, k>`an >'elatton, >'elatton.}\\~\\~\\

\epsfysize=0.5in
\centerline{\epsffile{Book05/fig14g.eps}}

\gr{Pr~wton g`ar t`o A pr`oc de'uteron t`o B a>ut`on >eq'etw l'ogon ka`i
tr'iton t`o G pr`oc t'etarton t`o D, me~izon d`e >'estw t`o A to~u G; l'egw,
<'oti ka`i t`o B to~u D me~iz'on >estin.}

\gr{>Epe`i g`ar t`o A to~u G me~iz'on >estin, >'allo d'e, <`o >'etuqen,
[m'egejoc] t`o B, t`o A >'ara pr`oc t`o B me'izona l'ogon >'eqei
>'hper t`o G pr`oc t`o B. <wc d`e t`o A 	pr`oc t`o B, o<'utwc t`o G pr`oc
t`o
D; ka`i t`o G >'ara pr`oc t`o D me'izona
l'ogon >'eqei >'hper t`o G pr`oc t`o B. pr`oc <`o d`e t`o a>ut`o me'izona
l'ogon >'eqei, >eke~ino >'elass'on >estin; >'elasson >'ara t`o D to~u B; <'wste
me~iz'on >esti t`o B to~u D.}

\gr{<Omo'iwc d`h de~ixomen, <'oti k>`an >'ison >~h| t`o A t~w| G,
>'ison >'estai ka`i t`o B t~w| D, k>'an >'elasson >~h| t`o A
to~u G, >'elasson >'estai ka`i t`o B to~u D.}

\gr{>E`an >'ara pr~wton pr`oc de'uteron t`on a>ut`on >'eqh| l'ogon ka`i tr'iton  pr`oc t'etarton, t`o d`e
pr~wton to~u tr'itou me~izon >~h|, ka`i t`o de'uteron to~u tet'artou
me~izon >'estai, k>`an >'ison, >'ison, k>`an >'elatton, >'elatton;
<'oper >'edei de~ixai.
}}

\ParallelRText{
\begin{center}
{\large Proposition 14}$^\dag$
\end{center}

If a first (magnitude) has the same ratio to a second  that a third (has) to a fourth, and the first (magnitude) is greater than the third,
then the second will also be greater than the fourth. And if (the first magnitude is) equal
(to the third then the second will also be) equal (to the fourth). And
if (the first magnitude is) less (than the third then the second will also be) less (than the fourth).

\epsfysize=0.5in
\centerline{\epsffile{Book05/fig14e.eps}}

For let a first (magnitude) $A$ have the same ratio to a second $B$  that a
third $C$ (has) to a fourth $D$. And let $A$ be greater than $C$. I say that $B$
is also greater than $D$.

For since $A$ is greater than $C$, and $B$ (is) another random [magnitude],
$A$ thus has a greater ratio to $B$ than $C$ (has) to $B$ [Prop. 5.8]. And as $A$ (is) to $B$, so $C$ (is) to
$D$. Thus, $C$ also has a greater ratio to $D$ than $C$ (has) to $B$. And that (magnitude) to
which the same (magnitude) has a greater ratio is the lesser
 [Prop. 5.10]. Thus, $D$
(is) less than $B$. Hence, $B$ is greater than $D$.

So, similarly, we can show that even if $A$ is equal to $C$ then $B$ will also be equal to $D$, and even if $A$ is less than $C$ then $B$ will also be less than $D$.

Thus, if a first (magnitude) has the same ratio to a second  that a third (has) to a fourth, and the first (magnitude) is greater than the third,
then the second will also be greater than the fourth. And if (the first magnitude is) equal
(to the third then the second will also be) equal (to the fourth). And
if (the first magnitude  is) less (than the third then the second will also be) less (than the fourth). (Which is) the very thing it was required to show.}
\end{Parallel}


\vspace{7pt}{\footnotesize \noindent$^\dag$ In modern notation, this proposition
reads that if $\alpha:\beta::\gamma:\delta$ then $\alpha \gtreqqless \gamma$ as
$\beta\gtreqqless\delta$.}

%%%%%%
% Prop 5.15
%%%%%%
\pdfbookmark[1]{Proposition 5.15}{pdf5.15}
\begin{Parallel}{}{} 
\ParallelLText{
\begin{center}
{\large \ggn{15}.}
\end{center}\vspace*{-7pt}

\gr{T`a m'erh to~ic <wsa'utwc pollaplas'ioic t`on a>ut`on >'eqei l'ogon
lhfj'enta kat'allhla.}

\epsfysize=0.8in
\centerline{\epsffile{Book05/fig15g.eps}}

\gr{>'Estw g`ar >is'akic pollapl'asion t`o AB to~u G ka`i to DE to~u Z;
l'egw, <'oti >est`in <wc t`o G pr`oc t`o Z, o<'utwc t`o AB pr`oc t`o DE.}

\gr{>Epe`i g`ar >is'akic >est`i pollapl'asion t`o AB to~u G ka`i t`o DE to~u Z,
<'osa >'ara >est`in >en t~w| AB meg'ejh >'isa t~w| G, tosa~uta ka`i
>en t~w| DE >'isa t~w| Z. dih|r'hsjw t`o m`en AB e>ic t`a t~w| G >'isa
t`a AH, HJ, JB, t`o d`e DE e>ic t`a t~w| Z >'isa t`a DK, KL, LE;
>'estai d`h >'ison t`o pl~hjoc t~wn AH, HJ, JB t~w| pl'hjei t~wn DK, KL, LE. ka`i >epe`i >'isa >est`i t`a AH, HJ, JB  >all'hloic, >'esti d`e ka`i t`a
DK, KL, LE >'isa >all'hloic,
>'estin >'ara
<wc t`o AH pr`oc t`o DK,  o<'utwc t`o HJ pr`oc t`o KL, ka`i t`o JB
pr`oc t`o LE. >'estai >'ara ka`i <wc <`en t~wn <hgoum'enwn 
pr`oc <`en t~wn <epom'enwn, o<'utwc <'apanta t`a <hgoum'ena pr`oc
<'apanta t`a <ep'omena; >'estin >'ara <wc t`o AH pr`oc t`o DK, o<'utwc
t`o AB pr`oc t`o DE. >'ison d`e t`o m`en AH t~w| G, t`o d`e DK
t~w| Z; >'estin >'ara <wc t`o G pr`oc t`o Z o<'utwc t`o AB pr`oc 
t`o DE.}

\gr{T`a >'ara m'erh to~ic <wsa'utwc pollaplas'ioic t`on a>ut`on >'eqei l'ogon
lhfj'enta kat'allhla; <'oper >'edei de~ixai.}}

\ParallelRText{
\begin{center}
{\large Proposition 15}$^\dag$
\end{center}

Parts have the same ratio  as similar multiples,
taken in corresponding order.

\epsfysize=0.8in
\centerline{\epsffile{Book05/fig15e.eps}}

For let $AB$ and $DE$ be equal multiples of $C$ and $F$ (respectively). I say that
as $C$ is to $F$, so $AB$ (is) to $DE$.

For since $AB$ and $DE$ are equal multiples of $C$ and $F$ (respectively), thus
as many magnitudes as there are in $AB$ equal to $C$, so many (are there) also
in $DE$ equal to $F$. Let $AB$ have been divided into (magnitudes) $AG$, $GH$, $HB$, equal to $C$, and $DE$ into (magnitudes) $DK$, $KL$,  $LE$, equal to
$F$. So, the number of (magnitudes) $AG$, $GH$,  $HB$ will equal the number
of (magnitudes) $DK$, $KL$, $LE$.
And since $AG$, $GH$, $HB$ are equal to one another, 
and $DK$, $KL$,  $LE$ are also equal to one another,
thus as $AG$ is to
$DK$, so $GH$ (is) to $KL$, and $HB$ to $LE$ [Prop. 5.7]. And, thus (for proportional magnitudes), as one of the leading
(magnitudes) will be to one of the following, so all of the leading (magnitudes will be)
to all of the following [Prop. 5.12].
Thus, as $AG$ is to $DK$, so $AB$ (is) to $DE$. And $AG$ is equal to $C$, and 
$DK$ to $F$. Thus, as $C$ is to $F$, so $AB$ (is) to $DE$.

Thus, parts have the same ratio  as similar multiples,
taken in corresponding order. (Which is) the very thing it was required to
show.}
\end{Parallel}


\vspace{7pt}{\footnotesize \noindent$^\dag$ In modern notation, this proposition
reads that $\alpha:\beta::m\,\alpha:m\,\beta$.}

%%%%%%
% Prop 5.16
%%%%%%
\pdfbookmark[1]{Proposition 5.16}{pdf5.16}
\begin{Parallel}{}{} 
\ParallelLText{
\begin{center}
{\large \ggn{16}.}
\end{center}\vspace*{-7pt}

\gr{>E`an t'essara meg'ejh >an'alogon >~h|, ka`i >enall`ax >an'alogon >'estai.}

\gr{>'Estw t'essara meg'ejh >an'alogon t`a 
A, B, G, D, <wc t`o A pr`oc t`o B, o<'utwc t`o G pr`oc t`o D; l'egw,
<'oti ka`i >enall`ax [>an'alogon] >'estai, <wc t`o A pr`oc t`o G, 
o<'utwc t`o B pr`oc t`o D.}

\gr{E>il'hfjw g`ar t~wn m`en A, B >is'akic pollapl'asia t`a E, Z, t~wn
d`e G, D >'alla, <`a >'etuqen, >is'akic pollapl'asia t`a H, J.}

\epsfysize=1in
\centerline{\epsffile{Book05/fig16g.eps}}

\gr{Ka`i >epe`i >is'akic >est`i pollapl'asion t`o E to~u A ka`i t`o Z to~u
B, t`a d`e m'erh to~ic <wsa'utwc pollaplas'ioic t`on a>ut`on >'eqei
l'ogon, >'estin >'ara <wc t`o A pr`oc
t`o B, o<'utwc t`o E pr`oc t`o Z. <wc d`e t`o A pr`oc t`o B,
o<'utwc t`o G pr`oc t`o D; ka`i <wc >'ara t`o G pr`oc
t`o D, o<'utwc t`o E pr`oc t`o Z. p'alin, >epe`i t`a H, J t~wn G, D
>is'akic >est`i pollapl'asia, >'estin >'ara <wc t`o G pr`oc t`o D, o<'utwc
t`o H pr`oc t`o J. <wc d`e t`o G pr`oc t`o D, [o<'utwc] t`o E pr`oc
t`o Z; ka`i <wc >'ara t`o E pr`oc t`o Z, o<'utwc t`o H pr`oc t`o J. >e`an
d`e t'essara meg'ejh >an'alogon >~h|, t`o d`e pr~wton to~u tr'itou
me~izon >~h|, ka`i t`o de'uteron to~u tet'artou me~izon >'estai,
k>`an >'ison, >'ison, k>'an >'elatton, >'elatton. e>i >'ara <uper'eqei
t`o E to~u H, <uper'eqei ka`i t`o Z to~u J, ka`i e>i >'ison, >'ison,
ka`i e>i >'elatton, >'elatton. ka'i >esti t`a m`en E, Z t~wn A, B
>is'akic pollapl'asia, t`a d`e H, J t~wn G, D >'alla, <`a >'etuqen, >is'akic
pollapl'asia; >'estin >'ara <wc t`o A pr`oc t`o G, o<'utwc t`o B pr`oc
t`o D.}

\gr{>E`an >'ara t'essara meg'ejh >an'alogon >~h|, ka`i >enall`ax
>an'alogon >'estai; <'oper >'edei de~ixai.}}

\ParallelRText{
\begin{center}
{\large Proposition 16}$^\dag$
\end{center}

If four magnitudes are proportional then they will
also be proportional alternately.

Let $A$, $B$, $C$ and $D$ be four proportional magnitudes, (such that) as $A$
(is) to $B$, so $C$ (is) to $D$. I say that they will also be [proportional]
 alternately, (so that) as $A$ (is) to $C$, so $B$ (is) to $D$.
 
For let the equal multiples $E$ and $F$ have been taken of $A$ and $B$ (respectively),
and the other random equal multiples $G$ and $H$ of $C$ and $D$ (respectively).

\epsfysize=1in
\centerline{\epsffile{Book05/fig16e.eps}}

And since $E$ and $F$ are equal multiples of $A$ and $B$ (respectively),
and parts  have the same ratio as similar multiples [Prop. 5.15], thus as $A$ is to $B$, so $E$ (is) to $F$.
But as $A$ (is) to $B$, so $C$ (is) to $D$. And, thus, as $C$ (is) to $D$, so
$E$ (is) to $F$ [Prop. 5.11].
Again, since $G$ and $H$ are equal multiples of $C$ and $D$ (respectively),  thus
as $C$  is to $D$, so $G$ (is) to $H$ [Prop. 5.15].
But as $C$ (is) to $D$, [so] $E$ (is) to $F$. And, thus, as $E$ (is) to $F$, so
$G$ (is) to $H$ [Prop. 5.11]. And if four magnitudes are proportional, and the first is greater than the third then
the second will also be greater than the fourth, and if (the first is) equal
(to the third then the second will also be) equal (to the fourth), and if
(the first is) less (than the third then the second will also be) less (than the fourth)
[Prop. 5.14]. Thus, if $E$ exceeds $G$ then
$F$ also exceeds $H$, and if ($E$ is) equal (to $G$ then $F$ is also) equal (to $H$), and if
($E$ is) less (than $G$ then $F$ is also) less (than $H$). And $E$ and $F$ are equal multiples of $A$ and $B$ (respectively), and $G$ and $H$ other random equal multiples
of $C$ and $D$ (respectively). Thus, as $A$ is to $C$, so $B$ (is) to $D$ [Def. 5.5].

Thus, if four magnitudes are proportional then they will
also be proportional  alternately. (Which is) the very thing it was required to show.}
\end{Parallel}


\vspace{7pt}{\footnotesize \noindent$^\dag$ In modern notation, this proposition
reads that if $\alpha:\beta::\gamma:\delta$ then $\alpha:\gamma::\beta:\delta$.}

%%%%%%
% Prop 5.17
%%%%%%
\pdfbookmark[1]{Proposition 5.17}{pdf5.17}
\begin{Parallel}{}{} 
\ParallelLText{
\begin{center}
{\large \ggn{17}.}
\end{center}\vspace*{-7pt}

\gr{>E`an sugke'imena meg'ejh >an'alogon >~h|, ka`i diairej'enta >an'alogon
>'estai.}

\epsfysize=1in
\centerline{\epsffile{Book05/fig17g.eps}}

\gr{>'Estw sugke'imena meg'ejh >an'alogon t`a AB, BE, GD, DZ, <wc t`o AB
pr`oc t`o BE, o<'utwc t`o GD pr`oc t`o DZ; l'egw, <'oti ka`i diairej'enta
>an'alogon >'estai, <wc t`o AE pr`oc t`o EB, o<'utwc t`o GZ pr`oc t`o DZ.}

\gr{E>il'hfjw g`ar t~wn m`en AE, EB, GZ, ZD >is'akic pollapl'asia t`a HJ,
JK, LM, MN, t~wn d`e EB, ZD >'alla, <`a >'etuqen, >is'akic
pollapl'asia t`a KX, NP.}

\gr{Ka`i >epe`i >is'akic >est`i pollapl'asion t`o HJ to~u AE ka`i t`o JK
to~u EB, >is'akic >'ara >est`i pollapl'asion t`o HJ to~u AE ka`i t`o HK
to~u AB. >is'akic d'e >esti pollapl'asion t`o HJ to~u AE ka`i t`o LM
to~u GZ; >is'akic >'ara >est`i pollapl'asion t`o HK to~u AB ka`i t`o LM
to~u GZ. p'alin, >epe`i >is'akic >est`i pollapl'asion t`o LM to~u GZ
ka`i t`o MN to~u ZD, >is'akic >'ara >est`i pollapl'asion t`o LM to~u GZ
ka`i t`o LN to~u GD. >is'akic d`e >~hn pollapl'asion t`o LM to~u GZ ka`i t`o
HK to~u AB; >is'akic >'ara >est`i pollapl'asion t`o HK to~u AB ka`i t`o LN
to~u GD. t`a HK, LN >'ara t~wn AB, GD >is'akic >est`i pollapl'asia.
p'alin, >epe`i >is'akic >est`i pollaplas'ion t`o JK to~u EB
ka`i t`o MN to~u ZD, >'esti d`e ka`i t`o KX to~u EB >is'akic pollapl'asion
ka`i t`o NP to~u ZD, ka`i suntej`en t`o JX to~u EB >is'akic >est`i
pollapl'asion ka`i t`o MP to~u ZD. 
ka`i >epe'i >estin <wc t`o AB pr`oc  t`o BE, o<'utwc t`o GD pr`oc t`o DZ, ka`i e>'ilhptai t~wn m`en AB, GD >is'akic pollapl'asia t`a
HK, LN, t~wn d`e EB, ZD >is'akic pollapl'asia t`a JX, MP, e>i >'ara
<uper'eqei t`o HK to~u JX, <uper'eqei ka`i t`o LN to~u MP, ka`i e>i
>'ison, >'ison, ka`i e>i >'elatton, >'elatton. <upereq'etw d`h t`o HK to~u
JX, ka`i koino~u >afairej'entoc to~u JK <uper'eqei >'ara ka`i t`o
HJ to~u KX. >alla e>i <upere~iqe
t`o HK to~u JX
<upere~iqe ka`i t`o LN to~u MP; <uper'eqei >'ara ka`i t`o LN to~u MP,
ka`i koino~u >afairej'entoc to~u MN <uper'eqei
ka`i t`o LM to~u NP; <'wste e>i <uper'eqei t`o HJ to~u KX,
<uper'eqei ka`i t`o LM to~u  NP. <omo'iwc d`h de~ixomen, <'oti k>`an
>'ison >~h| t`o HJ t~w| KX, >'ison >'estai ka`i t`o LM t~w| NP, k>`an
>'elatton, >'elatton. ka'i >esti t`a m`en HJ, LM t~wn AE, GZ
>is'akic pollapl'asia, t`a d`e KX, NP t~wn EB, ZD >'alla, <`a >'etuqen,
>is'akic pollapl'asia; >'estin >'ara <wc t`o AE pr`oc t`o EB, o<'utwc t`o
GZ pr`oc t`o ZD.}

\gr{>E`an >'ara sugke'imena meg'ejh >an'alogon >~h|, ka`i diairej'enta >an'alogon
>'estai; <'oper >'edei de~ixai.}}

\ParallelRText{
\begin{center}
{\large Proposition 17}$^\dag$
\end{center}

If composed  magnitudes are proportional then they will also be proportional (when)
separarted.

\epsfysize=1in
\centerline{\epsffile{Book05/fig17e.eps}}

Let $AB$, $BE$, $CD$, and $DF$ be composed magnitudes (which are) proportional,
(so that) as $AB$ (is) to $BE$, so $CD$ (is) to $DF$. I say that they will also be
proportional (when) separated, (so that) as $AE$ (is) to $EB$, so  $CF$ (is)
to $DF$.

For let the equal multiples $GH$, $HK$, $LM$, and $MN$ have been taken of
$AE$, $EB$, $CF$, and $FD$ (respectively), and the other random equal multiples
$KO$ and $NP$ of $EB$ and $FD$ (respectively).

And since $GH$ and $HK$ are equal multiples of $AE$ and $EB$ (respectively), 
$GH$ and $GK$ are thus equal multiples of $AE$ and $AB$ (respectively) [Prop. 5.1]. But $GH$ and $LM$ are equal
multiples of $AE$ and $CF$ (respectively). Thus, $GK$ and $LM$ are equal
multiples of $AB$ and $CF$ (respectively). Again, since $LM$ and $MN$ are equal
multiples of $CF$ and $FD$ (respectively), $LM$ and $LN$ are thus equal multiples
of $CF$ and $CD$ (respectively) [Prop. 5.1]. 
And $LM$ and $GK$ were equal multiples of $CF$ and $AB$ (respectively).
Thus, $GK$ and $LN$ are equal multiples of $AB$ and $CD$ (respectively).
Thus, $GK$, $LN$ are equal multiples of $AB$, $CD$. Again, since
$HK$ and $MN$ are equal multiples of $EB$ and $FD$ (respectively), and $KO$
and $NP$ are also  equal multiples of $EB$ and $FD$ (respectively),  then, added together, $HO$ and $MP$ are also equal multiples of $EB$ and $FD$ (respectively) [Prop. 5.2]. And since as $AB$ (is) to $BE$, so
$CD$ (is) to $DF$, and the equal multiples $GK$, $LN$ have been
taken of $AB$, $CD$, and the equal multiples $HO$, $MP$ of $EB$, $FD$, thus 
if $GK$ exceeds $HO$ then $LN$ also exceeds $MP$, and if ($GK$ is) equal
(to $HO$ then $LN$ is also) equal (to $MP$), and if ($GK$ is) less (than $HO$ then
$LN$ is also) less (than $MP$) [Def. 5.5]. So let $GK$ exceed $HO$, and thus, $HK$ being taken away from
both, $GH$ exceeds $KO$. But (we saw that) if $GK$ was exceeding $HO$ then
$LN$ was also exceeding $MP$. Thus, $LN$ also exceeds $MP$, and, $MN$ being
taken away from both, $LM$ also exceeds $NP$.  Hence, if $GH$ exceeds $KO$
then $LM$ also exceeds $NP$. So, similarly, we can show that even if
$GH$ is equal to $KO$ then $LM$ will also be equal to $NP$, and
even if ($GH$ is) less (than $KO$ then $LM$ will also be) less (than $NP$).
And $GH$, $LM$ are equal multiples of $AE$, $CF$, and $KO$, $NP$ other random
equal multiples of $EB$, $FD$. Thus, as $AE$ is to $EB$, so  $CF$ (is) to $FD$ [Def. 5.5].

Thus, if composed magnitudes are proportional then they will also be proportional (when)
separarted. (Which is) the very thing it was required to show.}
\end{Parallel}


\vspace{7pt}{\footnotesize \noindent$^\dag$ In modern notation, this proposition
reads that if $\alpha+\beta:\beta::\gamma+\delta:\delta$ then $\alpha:\beta::
\gamma:\delta$.}

%%%%%%
% Prop 5.18
%%%%%%
\pdfbookmark[1]{Proposition 5.18}{pdf5.18}
\begin{Parallel}{}{} 
\ParallelLText{
\begin{center}
{\large \ggn{18}.}
\end{center}\vspace*{-7pt}

\gr{>E`an dih|rhm'ena meg'ejh >an'alogon >~h|, ka`i suntej'enta >an'alogon
>'estai.}

\epsfysize=0.7in
\centerline{\epsffile{Book05/fig18g.eps}}

\gr{>'Estw dih|rhm'ena meg'ejh >an'alogon t`a AE, EB, GZ, ZD, <wc t`o AE
pr`oc t`o EB, o<'utwc t`o GZ pr`oc t`o ZD; l'egw, <'oti ka`i suntej'enta
>an'alogon >'estai, <wc t`o AB pr`oc t`o BE, o<'utwc t`o GD pr`oc t`o ZD.}

\gr{E>i g`ar m'h >est`in <wc t`o AB pr`oc t`o BE, o<'utwc t`o GD pr`oc
t`o DZ, >'estai <wc t`o AB pr`oc t`o BE, o<'utwc t`o GD >'htoi pr`oc >'elass'on ti to~u DZ >`h pr`oc me~izon.}

\gr{>'Estw pr'oteron pr`oc >'elasson t`o DH. ka`i >epe'i >estin <wc t`o AB
pr`oc t`o BE, o<'utwc t`o GD pr`oc t`o DH,  sugke'imena meg'ejh >an'alog'on
>estin; <'wste ka`i diairej'enta >an'alogon >'estai. >'estin >'ara <wc t`o AE
pr`oc t`o EB, o<'utwc t`o GH pr`oc t`o HD. <up'okeitai d`e ka`i <wc t`o
AE pr`oc t`o EB, o<'utwc t`o GZ pr`oc t`o ZD. ka`i <wc >'ara t`o GH pr`oc
t`o HD, o<'utwc t`o GZ pr`oc t`o ZD. me~izon d`e t`o pr~wton t`o GH to~u
tr'itou to~u GZ; me~izon >'ara ka`i t`o de'uteron
t`o HD to~u tet'artou to~u ZD. >all`a ka`i >'elatton; <'oper
>est`in >ad'unaton; o>uk >'ara >est`in <wc t`o AB pr`oc t`o BE, o<'utwc
t`o GD pr`oc >'elasson
 to~u ZD. <omo'iwc d`h de'ixomen, <'oti
o>ud`e pr`oc me~izon; pr`oc a>ut`o >'ara.}

\gr{ >E`an >'ara dih|rhm'ena meg'ejh >an'alogon >~h|, ka`i suntej'enta >an'alogon
>'estai; <'oper >'edei de~ixai.}}

\ParallelRText{
\begin{center}
{\large Proposition 18}$^\dag$
\end{center}

If separated  magnitudes are proportional  then they will also be proportional (when)
composed.

\epsfysize=0.7in
\centerline{\epsffile{Book05/fig18e.eps}}

Let $AE$, $EB$, $CF$, and $FD$ be separated magnitudes (which are) proportional,
(so that) as $AE$ (is) to $EB$, so  $CF$ (is) to $FD$. I say that they will also be proportional (when)  composed, (so that) as $AB$ (is) to $BE$, so  $CD$ (is) to $FD$.

For if (it is) not (the case that) as $AB$ is to $BE$, so $CD$ (is) to $FD$, then it will
surely be (the case that) as $AB$ (is) to $BE$, so
 $CD$ is either to some (magnitude) less than $DF$, or (some magnitude) greater (than $DF$).$^\ddag$
 
 Let it, first of all, be to (some magnitude) less (than $DF$), (namely) $DG$. And since composed magnitudes are proportional, (so that) as $AB$ is
 to $BE$, so $CD$ (is) to $DG$,  they will  thus also be proportional (when) separated 
 [Prop. 5.17]. Thus, as $AE$ is to $EB$, so
 $CG$ (is) to $GD$. But it was also assumed that as $AE$ (is) to $EB$, so $CF$ (is)
 to $FD$. Thus, (it is) also (the case that) as $CG$ (is) to $GD$, so $CF$ (is) to $FD$ [Prop. 5.11]. And the first (magnitude) $CG$ (is)
 greater than the third $CF$. Thus, the second (magnitude) $GD$ (is)
 also greater than the fourth $FD$ [Prop. 5.14]. 
 But (it is) also less. The very thing is impossible. Thus, (it is) not (the case that) as $AB$ is to $BE$, so $CD$ (is) to less than $FD$. Similarly, we can show
 that neither (is it the case) to greater (than $FD$). Thus, (it is the case) to
 the same (as $FD$).
 
 Thus, if separated  magnitudes are proportional  then they will also be 
 proportional (when)
composed. (Which is) the very thing it was required to show.}
\end{Parallel}


\vspace{7pt}{\footnotesize \noindent$^\dag$ In modern notation, this proposition
reads that if $\alpha:\beta::\gamma:\delta$ then $\alpha+\beta:\beta::
\gamma+\delta:\delta$.}\\


\vspace{7pt}{\footnotesize \noindent$^\ddag$ Here, Euclid assumes, without proof, that
a fourth magnitude proportional to three given magnitudes can always
be found.}

%%%%%%
% Prop 5.19
%%%%%%
\pdfbookmark[1]{Proposition 5.19}{pdf5.19}
\begin{Parallel}{}{} 
\ParallelLText{
\begin{center}
{\large \ggn{19}.}
\end{center}\vspace*{-7pt}

\gr{>E`an >~h| <wc <'olon pr`oc <'olon, o<'utwc >afairej`en pr`oc
>afairej'en, ka`i t`o loip`on pr`oc t`o loip`on >'estai <wc <'olon
pr`oc <'olon.}

\epsfysize=0.6in
\centerline{\epsffile{Book05/fig19g.eps}}

\gr{>'Estw g`ar <wc <'olon t`o AB pr`oc <'olon t`o GD, o<'utwc >afairej`en
t`o AE pr`oc >afeirej`en t`o GZ; l'egw, <'oti ka`i loip`on t`o EB pr`oc
loip`on t`o ZD >'estai <wc <'olon t`o AB pr`oc <'olon t`o GD.}

\gr{>Epe`i g'ar >estin <wc t`o AB pr`oc t`o GD, o<'utwc t`o AE pr`oc t`o
GZ, ka`i >enall`ax <wc t`o BA pr`oc t`o AE, o<'utwc t`o DG pr`oc t`o GZ.
ka`i >epe`i sugke'imena meg'ejh >an'alog'on >estin, ka`i diairej'enta
>an'alogon >'estai, <wc t`o BE pr`oc t`o EA, o<'utwc t`o DZ pr`oc
t`o GZ; ka`i >enall'ax, <wc t`o BE pr`oc t`o DZ, o<'utwc t`o EA pr`oc
t`o ZG. <wc d`e t`o AE pr`oc t`o GZ,  o<'utwc <up'okeitai <'olon t`o AB pr`oc <'olon
t`o GD. ka`i loip`on >'ara t`o EB pr`oc loip`on t`o ZD >'estai <wc <'olon
t`o AB pr`oc <'olon t`o GD.}

\gr{>E`an >'ara >~h| <wc <'olon pr`oc <'olon, o<'utwc >afairej`en pr`oc
>afairej'en, ka`i t`o loip`on pr`oc t`o loip`on >'estai <wc <'olon
pr`oc <'olon [<'oper >'edei de~ixai].}

\gr{\mbox{[}Ka`i >epe`i >ede'iqjh <wc t`o AB pr`oc t`o GD, o<'utwc t`o EB
pr`oc t`o ZD, ka`i >enall`ax <wc t`o AB pr`oc t`o BE o<'utwc t`o
GD pr`oc t`o ZD, sugke'imena >'ara meg'ejh >an'alog'on >estin;
>ede'iqjh d`e <wc t`o BA pr`oc t`o AE, o<'utwc t`o DG pr`oc t`o GZ;
ka'i >estin >anastr'eyanti].}\\~\\~\\~\\~\\

\begin{center}
{\large \gr{P'orisma}.}
\end{center}\vspace*{-7pt}

\gr{>Ek d`h to'utou faner'on, <'oti >e`an sugke'imena meg'ejh
>an'alogon >~h|, ka`i >anastr'eyanti >an'alogon >'estai; <'oper >'edei
de~ixai.}}

\ParallelRText{
\begin{center}
{\large Proposition 19}$^\dag$
\end{center}

If as the whole is to the whole so the (part) taken
away is to the (part) taken away then the remainder to the remainder will
also be as the whole (is) to the whole.

\epsfysize=0.6in
\centerline{\epsffile{Book05/fig19e.eps}}

For let the whole $AB$ be to the whole $CD$ as the (part) taken away $AE$
(is) to the (part) taken away $CF$. I say that the remainder $EB$
to the remainder $FD$ will also be as the whole $AB$ (is) to the whole $CD$.

For since as $AB$ is to $CD$, so $AE$ (is) to $CF$, (it is) also (the case), alternately, 
(that) as $BA$ (is)
to $AE$, so $DC$ (is) to $CF$ [Prop. 5.16].
And since composed magnitudes are proportional then they will also
be proportional (when) separated, (so that) as $BE$ (is) to $EA$, so $DF$ (is)
to $CF$ [Prop. 5.17].
Also, alternately, as $BE$ (is) to $DF$, so $EA$ (is) to $FC$ [Prop. 5.16]. And it was assumed that as
$AE$ (is) to $CF$, so the whole $AB$ (is) to the whole $CD$. And, thus,  as the
remainder $EB$ (is) to the remainder $FD$, so the whole $AB$ will be to the
whole $CD$.

Thus, if as the whole is to the whole so the (part) taken
away is to the (part) taken away then the remainder to the remainder will
also be as the whole (is) to the whole. [(Which is) the very thing it was required to show.]

\mbox{[}And since it was shown (that) as $AB$ (is) to $CD$, so $EB$ (is) to $FD$, (it is)
also (the case), alternately, (that) as $AB$ (is) to $BE$, so $CD$ (is) to $FD$. 
Thus, composed magnitudes are proportional. And it was shown
(that) as $BA$ (is) to $AE$, so $DC$ (is) to $CF$. And (the latter) is converted (from the former).]\\

\begin{center}
{\large Corollary}$^\ddag$
\end{center}\vspace*{-7pt}

So (it is) clear, from this, that if composed magnitudes are proportional
then they will also be proportional (when) converted.
(Which is) the very thing it was required to show.}
\end{Parallel}


\vspace{7pt}{\footnotesize \noindent$^\dag$ In modern notation, this proposition
reads that if $\alpha:\beta::\gamma:\delta$ then
$\alpha:\beta::\alpha-\gamma:\beta-\delta$.\\[0.5ex]
$^\ddag$ In modern notation, this corollary reads
that if $\alpha:\beta::\gamma:\delta$ then $\alpha:\alpha-\beta::\gamma:\gamma-\delta$.}

%%%%%%
% Prop 5.20
%%%%%%
\pdfbookmark[1]{Proposition 5.20}{pdf5.20}
\begin{Parallel}{}{} 
\ParallelLText{
\begin{center}
{\large \ggn{20}.}
\end{center}\vspace*{-7pt}

\gr{>E`an >~h| tr'ia meg'ejh ka`i >'alla a>uto~ic >'isa t`o
pl~hjoc, s'unduo lamban'omena ka`i >en t~w| a>ut~w|
l'ogw, di> >'isou d`e t`o pr~wton to~u tr'itou me~izon
>~h|, ka`i t`o t'etarton to~u <'ektou me~izon >'estai,
k>`an >'ison, >'ison, k>`an >'elatton, >'elatton.}\\~\\~\\~\\

\epsfysize=0.7in
\centerline{\epsffile{Book05/fig20g.eps}}

\gr{>'Estw tr'ia meg'ejh t`a A, B, G, ka`i >'alla a>uto~ic
>'isa t`o pl~hjoc t`a D, E, Z, s'unduo lamban'omena
>en t~w| a>ut~w| l'ogw|, <wc m`en t`o A pr`oc t`o B,
o<'utwc t`o D pr`oc t`o E, <wc d`e t`o B pr`oc t`o G,
o<'utwc t`o E pr`oc t`o Z, di> >'isou d`e me~izon >'estw
t`o A to~u G;
 l'egw,
<'oti ka`i t`o D to~u Z me~izon >'estai, k>`an >'ison, >'ison,
k>`an >'elatton, >'elatton.}

\gr{>Epe`i g`ar me~iz'on >esti t`o A to~u G, >'allo d'e ti t`o
B, t`o d`e me~izon pr`oc t`o a>ut`o me'izona l'ogon
>'eqei >'hper t`o >'elatton, t`o A >'ara pr`oc t`o B me'izona l'ogon
>'eqei >'hper t`o G pr`oc t`o B. >all> <wc m`en t`o A pr`oc t`o B
[o<'utwc] t`o D pr`oc t`o E, <wc d`e t`o G pr`oc t`o B, >an'apalin
o<'utwc t`o Z pr`oc t`o E; ka`i t`o D >'ara pr`oc t`o E me'izona l'ogon
>'eqei >'hper t`o Z pr`oc t`o E. t~wn d`e pr`oc t`o a>ut`o l'ogon >eq'ontwn
t`o me'izona l'ogon >'eqon me~iz'on >estin. me~izon >'ara t`o D to~u Z.
<omo'iwc d`h de'ixomen, <'oti k>`an >'ison >~h| t`o A t~w| G, >'ison
>'estai ka`i t`o D t~w| Z, k>`an >'elatton, >'elatton.}

\gr{>E`an >'ara >~h| tr'ia meg'ejh ka`i >'alla a>uto~ic >'isa t`o
pl~hjoc, s'unduo lamban'omena ka`i >en t~w| a>ut~w|
l'ogw, di> >'isou d`e t`o pr~wton to~u tr'itou me~izon
>~h|, ka`i t`o t'etarton to~u <'ektou me~izon >'estai,
k>`an >'ison, >'ison, k>`an >'elatton, >'elatton; <'oper >'edei
de~ixai.}}

\ParallelRText{
\begin{center}
{\large Proposition 20}$^\dag$
\end{center}

If there are three magnitudes, and others of equal
 number to them,   (being) also in the same ratio taken two by two, and (if),
via equality,  the first is greater
than the third then the fourth will also be greater than the sixth. And
 if (the first is) equal (to the third then the fourth will also be) equal
(to the sixth). 
And if (the first is) less (than the third then the fourth will also be) less (than
the sixth).

\epsfysize=0.7in
\centerline{\epsffile{Book05/fig20e.eps}}

Let $A$, $B$, and $C$ be three magnitudes, and $D$, $E$, $F$ other (magnitudes) of equal
number to them, (being) in the same ratio taken two by two, (so that)
as $A$ (is) to $B$, so $D$ (is) to $E$, and as $B$ (is) to $C$, so $E$ (is) to $F$. And let $A$
be greater than $C$, via equality. I say that $D$ will also be greater
than $F$. And if ($A$ is) equal (to $C$ then $D$ will also be) equal (to $F$). And
if ($A$ is) less (than $C$ then $D$ will also be) less (than $F$).

For since $A$ is greater than $C$, and $B$ some other (magnitude), and
the greater (magnitude) has a greater ratio than the lesser to the same (magnitude) [Prop. 5.8], $A$ thus has
a greater ratio to $B$ than $C$ (has) to $B$. But as $A$ (is) to $B$, [so] $D$ (is)
to $E$. And, inversely, as $C$ (is) to $B$,  so  $F$ (is) to $E$ [Prop. 5.7~corr.]. Thus, $D$ also has a greater
ratio to $E$ than $F$ (has) to $E$ [Prop. 5.13]. And for (magnitudes) having a ratio to the
same (magnitude), that having the greater ratio is greater [Prop. 5.10]. Thus, $D$ (is) greater than $F$.
Similarly, we can show that even if $A$ is equal to $C$ then $D$ will also
be equal to $F$, and even if ($A$ is) less (than $C$ then $D$ will also
be) less (than $F$).

Thus, if there are three magnitudes, and others of equal
number to them,   (being) also in the same ratio taken two by two, and (if),
via equality, the first is greater
than the third, then the fourth will also be greater than the sixth. And
 if (the first is) equal (to the third then the fourth will also be) equal
(to the sixth). And (if the first is) less (than the third then the fourth will also be) less (than
the sixth). (Which is) the very thing it was required to show.}
\end{Parallel}


\vspace{7pt}{\footnotesize \noindent$^\dag$ In modern notation, this proposition
reads that if $\alpha:\beta::\delta:\epsilon$ and $\beta:\gamma::\epsilon:\zeta$ then $\alpha\gtreqqless\gamma$ as $\delta\gtreqqless\zeta$.}

%%%%%%
% Prop 5.21
%%%%%%
\pdfbookmark[1]{Proposition 5.21}{pdf5.21}
\begin{Parallel}{}{} 
\ParallelLText{
\begin{center}
{\large \ggn{21}.}
\end{center}\vspace*{-7pt}

\gr{>E`an >~h| tr'ia meg'ejh ka`i >'alla a>uto~ic >'isa t`o pl~hjoc s'unduo lamban'omena ka`i >en t~w| a>ut~w| l'ogw|, >~h| d`e tetaragm'enh a>ut~wn
<h >analog'ia, di> >'isou d`e t`o pr~wton to~u tr'itou me~izon >~h|, ka`i
t`o t'etarton to~u <'ektou me~izon >'estai, k>`an >'ison, >'ison, k>`an
>'elatton, >'elatton.}\\~\\~\\

\epsfysize=0.7in
\centerline{\epsffile{Book05/fig21g.eps}}

\gr{>'Estw tr'ia meg'ejh t`a A, B, G ka`i >'alla a>uto~ic >'isa t`o pl~hjoc
t`a D, E, Z, s'unduo lamban'omena ka`i >en t~w| a>ut~w| l'ogw|, >'estw d`e
tetaragm'enh a>ut~wn <h >analog'ia, <wc m`en t`o A pr`oc t`o B, o<'utwc
t`o E pr`oc t`o Z, <wc d`e t`o B pr`oc t`o G, o<'utwc t`o D pr`oc t`o E,
di> >'isou d`e t`o A to~u G me~izon >'estw; l'egw, <'oti
ka`i t`o D to~u Z me~izon >'estai, k>`an >'ison, >'ison, k>`an >`elatton,
>`elatton.}

\gr{>Epe`i g`ar me~iz'on >esti t`o A to~u G, >'allo d'e ti t`o B, t`o A >'ara
pr`oc t`o B me'izona l'ogon >'eqei >'hper t`o G pr`oc t`o B. >all> <wc
m`en t`o A pr`oc t`o B, o<'utwc t`o E pr`oc t`o Z, <wc d`e t`o G pr`oc
t`o B, >an'apalin o<'utwc t`o E pr`oc t`o D. ka`i t`o E >'ara pr`oc t`o Z
me'izona l'ogon >'eqei >'hper t`o E pr`oc t`o D. pr`oc <`o d`e t`o a>ut`o me'izona l'ogon >'eqei, >eke~ino >'elass'on >estin; >'elasson >'ara
>est`i t`o Z to~u D; me~izon >'ara >est`i t`o D to~u Z.
<omo'iwc d`h de'ixomen, <'oti k>`an >'ison >~h|
t`o A t~w| G, >'ison >'estai ka`i t`o D t~w| Z, k>`an >'elatton, >'elatton.}

\gr{>E`an >'ara >~h| tr'ia meg'ejh ka`i >'alla a>uto~ic >'isa t`o pl~hjoc, s'unduo lamban'omena ka`i >en t~w| a>ut~w| l'ogw|, >~h| d`e tetaragm'enh a>ut~wn
<h >analog'ia, di> >'isou d`e t`o pr~wton to~u tr'itou me~izon >~h|, ka`i
t`o t'etarton to~u <'ektou me~izon >'estai, k>`an >'ison, >'ison, k>`an
>'elatton, >'elatton; <'oper >'edei de~ixai.}}

\ParallelRText{
\begin{center}
{\large Proposition 21}$^\dag$
\end{center}

If there are three magnitudes, and others of equal
 number to them, (being) also in the same ratio taken two by two, and (if)
their proportion (is) perturbed, and (if), via equality,
the first is greater than the third then the fourth will also
be greater than the sixth. And if (the first is) equal (to the third then the
fourth will also be) equal (to the sixth).
And if (the first is) less (than the third then the fourth will also be) less (than
the sixth).

\epsfysize=0.7in
\centerline{\epsffile{Book05/fig21e.eps}}

Let $A$, $B$, and $C$ be three magnitudes, and $D$, $E$, $F$ other (magnitudes)
of equal number to them, (being) in the same ratio taken two by two. And let their proportion be perturbed, (so that) as $A$ (is) to $B$, so $E$ (is) to $F$, and as
$B$ (is) to $C$, so $D$ (is) to $E$. And let $A$  be greater than $C$, via equality. I
say that $D$ will also be greater than $F$.
 And if ($A$ is) equal (to $C$ then $D$ will also be) equal (to $F$). And
if ($A$ is) less (than $C$ then $D$ will also be) less (than $F$).

For since $A$ is greater than $C$, and $B$ some other (magnitude),  $A$ thus
has a greater ratio to $B$ than $C$ (has) to $B$ [Prop. 5.8]. But as $A$ (is) to $B$,
so $E$ (is) to $F$. And, inversely, as $C$ (is) to $B$, so $E$ (is) to $D$ [Prop. 5.7~corr.]. Thus, $E$ also has a greater
ratio to $F$ than $E$ (has) to $D$ [Prop. 5.13]. And that (magnitude) to which the same (magnitude)  has a greater ratio  is (the) lesser (magnitude) [Prop. 5.10].
Thus, $F$ is less than $D$. Thus, $D$ is greater than $F$. Similarly, we can show that
even if $A$ is equal to $C$ then $D$ will also
be equal to $F$, and even if ($A$ is) less (than $C$ then $D$ will also
be) less (than $F$).

Thus, if there are three magnitudes, and others of equal
 number to them, (being) also in the same ratio taken two by two, and (if)
their proportion (is) perturbed, and (if), via equality,
the first is greater than the third then the fourth will also
be greater than the sixth. And if (the first is) equal (to the third then the fourth
will also be) equal (to the sixth).
And if (the first is) less (than the third then the fourth will also be) less (than
the sixth). (Which is) the very thing it was required to show.}
\end{Parallel}


\vspace{7pt}{\footnotesize \noindent$^\dag$ In modern notation, this proposition
reads that if $\alpha:\beta::\epsilon:\zeta$ and $\beta:\gamma::\delta:\epsilon$ then $\alpha\gtreqqless\gamma$ as $\delta\gtreqqless\zeta$.}

%%%%%%
% Prop 5.22
%%%%%%
\pdfbookmark[1]{Proposition 5.22}{pdf5.22}
\begin{Parallel}{}{} 
\ParallelLText{
\begin{center}
{\large \ggn{22}.}
\end{center}\vspace*{-7pt}

\gr{>E`an >~h| <oposao~un meg'ejh ka`i >'alla a>uto~ic >'isa t`o pl~hjoc,
s'unduo lamban'omena ka`i >en t~w| a>ut~w| l'ogw|, ka`i di> >'isou
>en t~w| a>ut~w| l'ogw| >'estai.}\\

\epsfysize=0.58in
\centerline{\epsffile{Book05/fig22g.eps}}

\gr{>'Estw <oposao~un meg'ejh t`a A, B, G ka`i >'alla a>uto~ic >'isa t`o
pl~hjoc t`a D, E, Z, s'unduo lamban'omena >en t~w| a>ut~w| l'ogw|,
<wc m`en t`o A pr`oc t`o B, o<'utwc t`o D pr`oc t`o E, <wc d`e
t`o B pr`oc t`o G, o<'utwc t`o E pr`oc t`o Z; l'egw, <'oti
ka`i di> >'isou >en t~w| a>utw| l'ogw|
>'estai.}

\gr{E>il'hfjw g`ar t~wn m`en A, D >is'akic pollapl'asia
t`a H, J, t~wn d`e B, E >'alla, <`a >'etuqen, >is'akic pollapl'asia
t`a K, L, ka`i >'eti t~wn G, Z >'alla, <`a >'etuqen, >is'akic pollapl'asia
t`a M, N.}

\gr{Ka`i >epe'i  >estin <wc t`o A pr`oc t`o B, o<'utwc t`o D pr`oc t`o E,
ka`i e>'ilhptai t~wn m`en A, D >is'akic pollapl'asia t`a
H, J, t~wn d`e B, E >'alla, <`a >'etuqen, >is'akic pollapl'asia
t`a K, L, >'estin >'ara <wc t`o H pr`oc t`o K, o<'utwc t`o J pr`oc t`o L.
d`ia t`a a>ut`a d`h ka`i <wc t`o K pr`oc t`o M, o<'utwc t`o
L pr`oc
t`o N. >epe`i o>~un tr'ia meg'ejh >est`i t`a H, K, M, ka`i >'alla
a>uto~ic >'isa
t`o pl~hjoc t`a J, L, N, s'unduo lamban'omena ka`i >en t~w| a>ut~w|
l'ogw|, di> >'isou >'ara, e>i <uper'eqei t`o H to~u M,
<uper'eqei ka`i t`o J to~u N, ka`i e>i >'ison, >'ison, ka`i
e>i >'elatton, >'elatton. ka'i >esti t`a m`en H, J t~wn A, D
>is'akic pollapl'asia, t`a d`e M, N t~wn G, Z >'alla, <`a
>'etuqen, >is'akic pollapl'asia. >'estin >'ara <wc t`o A pr`oc
t`o G, o<'utwc t`o D pr`oc t`o Z.}

\gr{>E`an >'ara >~h| <oposao~un meg'ejh ka`i >'alla a>uto~ic >'isa t`o pl~hjoc,
s'unduo lamban'omena >en t~w| a>ut~w| l'ogw|, ka`i di> >'isou
>en t~w| a>ut~w| l'ogw| >'estai; <'oper >'edei de~ixai.}}

\ParallelRText{
\begin{center}
{\large Proposition 22}$^\dag$
\end{center}

If there are any number of magnitudes
whatsoever, and (some) other (magnitudes)  of equal  number to them, (which are)
also in the same ratio taken two by two, then they will also be in the
same ratio via equality.

\epsfysize=0.58in
\centerline{\epsffile{Book05/fig22e.eps}}

Let there be any number of magnitudes whatsoever, $A$, $B$,  $C$, and
(some) other (magnitudes), $D$, $E$, $F$, of equal number to them, (which are)
in the same ratio taken two by two, (so that) as $A$ (is) to $B$, so
$D$ (is) to $E$, and as $B$ (is) to $C$, so $E$ (is) to $F$. I say that they
will also be in the same ratio via equality. (That is, as $A$ is to $C$, so
$D$ is to $F$.)

For let the equal multiples $G$ and $H$ have been taken of $A$ and $D$ (respectively), and the other random equal multiples $K$ and $L$ of $B$ and
$E$ (respectively), and the yet other random equal multiples $M$ and $N$ of
$C$ and $F$ (respectively).

And since as $A$ is to $B$, so $D$ (is) to $E$, and the equal multiples $G$ and $H$
have been taken of $A$ and $D$ (respectively), and the other random
equal multiples $K$ and $L$ of $B$ and $E$ (respectively), thus as $G$ is to $K$,
so $H$ (is) to $L$ [Prop. 5.4]. And, so, for the same
(reasons), as $K$ (is) to $M$, so $L$ (is) to $N$. Therefore, since $G$, $K$, and $M$ are three magnitudes, and $H$, $L$, and $N$ other (magnitudes) of equal number to them,
(which are) also in the same ratio taken two by two, thus, via equality,
if $G$ exceeds $M$ then $H$ also exceeds $N$, and if ($G$ is) equal (to $M$ then $H$
is also) equal (to $N$), and if ($G$ is) less (than $M$ then $H$
is also) less (than $N$) [Prop. 5.20].
And  $G$ and $H$ are equal multiples of $A$ and $D$ (respectively),
and $M$ and $N$ other random equal multiples of $C$ and $F$ (respectively).
Thus, as $A$ is to $C$, so $D$ (is) to $F$ [Def. 5.5].

Thus, if there are any number of magnitudes
whatsoever, and (some) other (magnitudes) of equal number to them, (which are)
also in the same ratio taken two by two, then they will also be in the
same ratio via equality. (Which is) the very thing
it was required to show.}
\end{Parallel}


\vspace{7pt}{\footnotesize \noindent $^\dag$ In modern notation, this proposition
reads that if $\alpha:\beta::\epsilon:\zeta$ and $\beta:\gamma::\zeta:\eta$ 
and $\gamma:\delta::\eta:\theta$ 
then $\alpha:\delta::\epsilon:\theta$.}

%%%%%%
% Prop 5.23
%%%%%%
\pdfbookmark[1]{Proposition 5.23}{pdf5.23}
\begin{Parallel}{}{} 
\ParallelLText{
\begin{center}
{\large\ggn{23}.}
\end{center}\vspace*{-7pt}

\gr{>E`an >~h| tr'ia meg'ejh ka`i >'alla a>uto~ic >'isa t`o pl~hjoc s'unduo
lamban'omena >en t~w| a>ut~w| l'ogw|, >~h| d`e tetaragm'enh a>ut~wn
<h >analog'ia, ka`i di> >'isou >en t~w| a>ut~w| l'ogw| >'estai.}\\

\epsfysize=0.75in
\centerline{\epsffile{Book05/fig23g.eps}}

\gr{>'Estw tr'ia meg'ejh t`a A, B, G ka`i >'alla a>uto~ic >'isa t`o pl~hjoc 
s'unduo lamban'omena >en t~w| a>ut~w| l'ogw| t`a D, E, Z, >'estw
d`e tetaragm'enh a>ut~wn <h >analog'ia, <wc m`en t`o A pr`oc t`o B,
o<'utwc t`o E pr`oc t`o Z, <wc d`e t`o B pr`oc t`o G, o<'utwc t`o D
pr`oc t`o E; l'egw, <'oti >est`in <wc t`o A pr`oc t`o G, o<'utwc
t`o D pr`oc t`o Z.}

\gr{E>il'hfjw t~wn m`en A, B, D >is'akic pollapl'asia t`a H, J, K, t~wn
d`e G, E, Z >'alla, <`a >'etuqen, >is'akic pollapl'asia t`a L, M, N.}

\gr{Ka`i >epe`i >is'akic >est`i pollapl'asia t`a H, J t~wn A, B, t`a d`e m'erh
to`ic <wsa'utwc pollaplas'ioic t`on a>ut`on >'eqei l'ogon,
>'estin >'ara <wc t`o A pr`oc t`o B, o<'utwc t`o H pr`oc
t`o J. di`a t`a a>ut`a d`h ka`i <wc t`o E pr`oc t`o Z, o<'utwc t`o M
pr`oc t`o N; ka'i >estin <wc t`o A pr`oc t`o B, o<'utwc t`o E pr`oc
t`o Z; ka`i <wc >'ara t`o H pr`oc t`o J, o<'utwc t`o M pr`oc t`o N.
ka`i >epe'i >estin <wc t`o B pr`oc t`o G, o<'utwc t`o D pr`oc t`o E,
ka`i >enall`ax <wc t`o B pr`oc t`o D, o<'utwc t`o G pr`oc t`o E. ka`i
>epe`i t`a J, K t~wn B, D >is'akic >est`i pollapl'asia, t`a d`e m'erh 
to~ic >is'akic pollaplas'ioic t`on a>ut`on >'eqei l'ogon, >'estin >'ara
<wc t`o B pr`oc t`o D, o<'utwc t`o J pr`oc t`o K. >all> <wc 
t`o B pr`oc t`o D, o<'utwc t`o G pr`oc t`o E; ka`i <wc >'ara t`o J pr`oc
t`o K, o<'utwc t`o G pr`oc t`o E.
 p'alin, >epe`i t`a L, M
t~wn G, E >is'akic >esti pollapl'asia, >'estin >'ara <wc t`o G
pr`oc t`o E, o<'utwc t`o L pr`oc t`o M. >all> <wc t`o G pr`oc t`o E,
o<'utwc t`o J pr`oc t`o K; ka`i <wc >'ara t`o J pr`oc t`o K, o<'utwc
t`o L pr`oc t`o M, ka`i >enall`ax <wc t`o J pr`oc t`o L, t`o K
pr`oc t`o M. >ede'iqjh d`e ka`i <wc t`o H pr`oc t`o J, o<'utwc
t`o M pr`oc t`o N. >epe`i o>~un tr'ia meg'ejh >est`i t`a H, J, L,
ka`i >'alla a>utoic >'isa t`o pl~hjoc t`a K, M, N s'unduo lamban'omena
>en t~w| a>ut~w| l'ogw|, ka'i >estin a>ut~wn tetaragm'enh <h
>analog'ia, di> >'isou >'ara, e>i <uper'eqei t`o H
to~u L, <uper'eqei ka`i t`o K to~u N, ka`i e>i >'ison, >'ison, ka`i
e>i >'elatton, >'elatton. ka'i >esti t`a m`en H, K t~wn A, D >is'akic
pollapl'asia, t`a d`e L, N t~wn G, Z. >'estin >'ara <wc t`o A pr`oc
t`o G, o<'utwc t`o D pr`oc t`o Z.}

\gr{>E`an >'ara  >~h| tr'ia meg'ejh ka`i >'alla a>uto~ic >'isa t`o pl~hjoc s'unduo
lamban'omena >en t~w| a>ut~w| l'ogw|, >~h| d`e tetaragm'enh a>ut~wn
<h >analog'ia, ka`i di> >'isou >en t~w| a>ut~w| l'ogw| >'estai;
<'oper >'edei de~ixai.}}

\ParallelRText{
\begin{center}
{\large Proposition 23}$^\dag$
\end{center}

If there are three magnitudes, and others of equal number to them, (being) in the same ratio taken two by two, and (if) their
proportion is perturbed, then they will also be in the same ratio via equality.

\epsfysize=0.75in
\centerline{\epsffile{Book05/fig23e.eps}}

Let $A$, $B$, and $C$ be three magnitudes, and $D$, $E$ and $F$ other (magnitudes) of equal  number to them, (being) in the same ratio taken two by two.
And let their proportion be perturbed, (so that) as $A$ (is) to $B$, so $E$ (is) to $F$,
and as $B$ (is) to $C$, so $D$ (is) to $E$. I say that as $A$ is to $C$, so $D$ (is) to $F$.

Let the equal multiples $G$, $H$, and $K$ have been taken of $A$, $B$, and $D$ (respectively), and the other random equal multiples $L$, $M$, and $N$ of
$C$, $E$, and $F$ (respectively).

And since $G$ and $H$ are equal multiples of $A$ and $B$ (respectively), and
parts have the same ratio as similar multiples [Prop. 5.15], thus as $A$ (is) to $B$, so $G$ (is) to $H$. 
And, so, for the same (reasons), as $E$ (is) to $F$, so $M$ (is) to $N$. And as
$A$ is to $B$, so $E$ (is) to $F$.  And, thus, as $G$ (is) to $H$, so $M$ (is) to $N$ [Prop. 5.11]. And since as $B$ is to $C$, so $D$ (is) to $E$,
also, alternately, as $B$ (is) to $D$, so $C$ (is) to $E$ [Prop. 5.16]. And since $H$ and $K$ are equal multiples
of $B$ and $D$ (respectively), and parts have the same ratio as similar multiples
[Prop. 5.15], thus as $B$ is to $D$, so $H$
(is) to $K$. But, as $B$ (is) to $D$, so $C$ (is) to $E$.  And, thus, as $H$ (is) to $K$, so $C$ (is) to $E$ [Prop. 5.11]. Again, since
$L$ and $M$ are  equal multiples of $C$ and $E$ (respectively), thus as
$C$ is to $E$, so $L$ (is) to $M$ [Prop. 5.15].
But, as $C$ (is) to $E$, so $H$ (is) to $K$. And, thus, as $H$ (is) to $K$,
so $L$ (is) to $M$ [Prop. 5.11]. Also, alternately,
as $H$ (is) to $L$, so $K$ (is) to $M$ [Prop. 5.16].
And it was also shown (that) as $G$ (is) to $H$, so $M$ (is) to $N$. Therefore,
since $G$, $H$, and $L$ are three magnitudes, and $K$, $M$, and $N$ other (magnitudes)
of equal number to them, (being) in the same ratio taken two by two, 
and their proportion is perturbed, thus, via equality, if $G$ exceeds
$L$ then $K$ also exceeds $N$, and if ($G$ is) equal (to $L$ then $K$ is also) equal (to $N$),
and if ($G$ is) less (than $L$ then $K$ is also) less (than $N$)  [Prop. 5.21]. And $G$ and $K$ are equal
multiples of $A$ and $D$ (respectively), and $L$ and $N$ of $C$ and $F$ (respectively). 
Thus, as $A$ (is) to $C$, so $D$ (is) to $F$  [Def. 5.5].

Thus, if there are three magnitudes, and others of  equal number to them, (being) in the same ratio taken two by two, and (if) their
proportion is perturbed, then they will also be in the same ratio via equality.
(Which is) the very thing it was required to show.}
\end{Parallel}


\vspace{7pt}{\footnotesize \noindent$^\dag$ In modern notation, this proposition
reads that if $\alpha:\beta::\epsilon:\zeta$ and $\beta:\gamma::\delta:\epsilon$ then $\alpha:\gamma::\delta:\zeta$.}

%%%%%%
% Prop 5.24
%%%%%%
\pdfbookmark[1]{Proposition 5.24}{pdf5.24}
\begin{Parallel}{}{} 
\ParallelLText{
\begin{center}
{\large \ggn{24}.}
\end{center}\vspace*{-7pt}

\gr{>E`an pr~wton pr`oc de'uteron t`on a>ut`on >'eqh| l'ogon ka`i tr'iton
pr`oc t'etarton, >'eqh| d`e ka`i p'empton pr`oc de'uteron t`on a>ut`on
l'ogon ka`i <'ekton pr`oc t'etarton, ka`i suntej`en pr~wton ka`i
p'empton pr`oc de'uteron t`on a>ut`on <'exei l'ogon ka`i tr'iton
ka`i <'ekton pr`oc t'etarton.}\\~\\

\epsfysize=1.4in
\centerline{\epsffile{Book05/fig24g.eps}}

\gr{Pr~wton g`ar t`o AB pr`oc de'ureron t`o G t`on a>ut`on >eq'etw
l'ogon ka`i tr'iton t`o DE pr`oc t'etarton t`o Z, >eq'etw d`e ka`i p'empton
t`o BH pr`oc de'uteron t`o G t`on a>ut`on l'ogon ka`i <'ekton t`o EJ
pr`oc t'etarton t`o Z; l'egw, <'oti ka`i suntej`en pr~wton ka`i p'empton
t`o AH pr`oc de'uteron t`o G t`on a>ut`on <'exei l'ogon, ka`i
tr'iton ka`i <'ekton t`o DJ pr`oc t'etarton t`o Z.}

\gr{>Epe`i g'ar >estin <wc t`o BH pr`oc t`o G, o<'utwc t`o EJ pr`oc t`o Z,
>an'apalin >'ara <wc t`o G pr`oc t`o BH, o<'utwc
t`o Z pr`oc t`o EJ. >epe`i o>~un >estin <wc t`o AB pr`oc t`o G,
o<'utwc t`o DE pr`oc t`o Z, <wc d`e t`o G pr`oc t`o BH, o<'utwc
t`o Z pr`oc t`o EJ, di> >'isou >'ara >est`in <wc t`o AB pr`oc t`o BH, o<'utwc
t`o DE pr`oc t`o EJ. ka`i >epe`i dih|rhm'ena meg'ejh >an'alog'on >estin,
ka`i suntej'enta >an'alogon >'estai; >'estin >'ara <wc t`o AH pr`oc t`o HB,
o<'utwc t`o DJ pr`oc t`o JE. >'esti d`e ka`i <wc t`o BH pr`oc t`o G, o<'utwc
t`o EJ pr`oc t`o Z;
di> >'isou >'ara >est`in <wc t`o AH pr`oc t`o G, o<'utwc t`o
DJ pr`oc t`o Z.}

\gr{>E`an >'ara pr~wton pr`oc de'uteron t`on a>ut`on >'eqh| l'ogon ka`i tr'iton
pr`oc t'etarton, >'eqh| d`e ka`i p'empton pr`oc de'uteron t`on a>ut`on
l'ogon ka`i <'ekton pr`oc t'etarton, ka`i suntej`en pr~wton ka`i
p'empton pr`oc de'uteron t`on a>ut`on <'exei l'ogon ka`i tr'iton
ka`i <'ekton pr`oc t'etarton; <'oper >'edei de`ixai.}}

\ParallelRText{
\begin{center}
{\large Proposition 24}$^\dag$
\end{center}

If a first (magnitude) has to a second the same
ratio  that third (has) to a fourth, and a fifth (magnitude)
also has to the second the same ratio that a sixth  (has) to the fourth, then
  the first (magnitude) and the fifth, added together, will also have the same ratio to the
 second
 that the third (magnitude) and sixth  (added together, have) to the fourth.
 
\epsfysize=1.4in
\centerline{\epsffile{Book05/fig24e.eps}}
 
 For let a first (magnitude) $AB$ have the same ratio to a second $C$
 that a third $DE$ (has) to a fourth $F$. And let a fifth (magnitude) $BG$
 also have the same ratio to the second $C$ that a sixth $EH$ (has) to
 the fourth $F$. I say that the first (magnitude) and the fifth, added together, $AG$, will also  have the same ratio to the second $C$ that the third (magnitude) and the sixth, (added together), $DH$,  (has) to the fourth $F$.
 
 For since as $BG$ is to $C$, so $EH$ (is) to $F$,  thus, inversely, as $C$ (is) to $BG$,
 so $F$ (is) to $EH$  [Prop. 5.7~corr.]. Therefore,
 since as $AB$ is to $C$, so $DE$ (is) to $F$, and as $C$ (is) to $BG$, so
 $F$ (is) to $EH$, thus, via equality, as $AB$ is to $BG$, so $DE$ (is) to $EH$ 
[Prop. 5.22]. And since separated magnitudes
are proportional then they will also be proportional (when) composed [Prop. 5.18]. Thus, as $AG$ is to $GB$, so $DH$
(is) to $HE$. And, also, as $BG$ is to $C$, so $EH$ (is) to $F$. Thus, via equality, 
as $AG$ is to $C$, so $DH$ (is) to $F$ [Prop. 5.22].

Thus, if a first (magnitude) has to a second the same
ratio that a third (has) to a fourth, and a fifth (magnitude)
also has to the second the same ratio that a sixth  (has) to the fourth, then
the first (magnitude) and the fifth, added together, will also have the same ratio to the
 second
 that the third (magnitude) and the sixth  (added together, have) to the fourth. (Which is) the very thing it was required to show.}
\end{Parallel}


\vspace{7pt}{\footnotesize \noindent$^\dag$ In modern notation, this proposition
reads that if $\alpha:\beta::\gamma:\delta$ and $\epsilon:\beta::\zeta:\delta$ then $\alpha+\epsilon:\beta::\gamma+\zeta:\delta$.}

%%%%%%
% Prop 5.25
%%%%%%
\pdfbookmark[1]{Proposition 5.25}{pdf5.25}
\begin{Parallel}{}{} 
\ParallelLText{
\begin{center}
{\large \ggn{25}.}
\end{center}\vspace*{-7pt}

\gr{>E`an t'essara meg'ejh >an'alogon >~h|, t`o m'egiston [a>ut~wn] ka`i
t`o >el'aqiston d'uo t~wn loip~wn me'izon'a >estin.}\\

\epsfysize=1.4in
\centerline{\epsffile{Book05/fig25g.eps}}

\gr{>'Estw t'essara meg'ejh >an'alogon t`a AB, GD, E, Z, <wc t`o AB
pr`oc t`o GD, o<'utwc t`o E pr`oc t`o Z, >'estw d`e m'egiston m`en
a>ut~wn t`o AB, >el'aqiston d`e t`o Z; l'egw, <'oti t`a AB, Z t~wn
GD, E me'izon'a >estin.}

\gr{Ke'isjw g`ar t~w| m`en E >'ison t`o AH, t~w| d`e Z
>'ison t`o GJ.}

\gr{>Epe`i [o>~un] >estin <wc t`o AB pr`oc t`o GD, o<'utwc
t`o E pr`oc t`o Z, >'ison d`e t`o  m`en E t~w| AH, t`o d`e Z t~w| GJ,
>'estin >'ara <wc t`o AB pr`oc t`o GD, o<'utwc
t`o AH pr`oc t`o GJ. ka`i >epe'i >estin <wc <'olon t`o AB pr`oc
<'olon t`o GD, o<'utwc >afairej`en t`o AH pr`oc >afairej`en t`o GJ,
ka`i loip`on >'ara t`o HB pr`oc loip`on t`o JD >'estai <wc <'olon
t`o AB pr`oc <'olon t`o GD. me~izon d`e t`o AB to~u GD; me~izon
>'ara ka`i t`o HB to~u JD. ka`i >epe`i >'ison >est`i t`o m`en AH t~w|
E, t`o d`e GJ t~w| Z, t`a >'ara AH, Z >'isa >est`i to~ic
GJ, E.
ka`i [>epe`i] >e`an [>an'isoic >'isa prostej~h|, t`a <'ola
>'anis'a >estin, >e`an >'ara] t~wn HB, JD >an'iswn >'ontwn ka`i
me'izonoc to~u HB  t~w| m`en HB prostej~h| t`a AH, Z, t~w| d`e JD
prostej~h| t`a GJ, E, sun'agetai t`a AB, Z me'izona t~wn GD, E.}

\gr{>E`an >'ara t'essara meg'ejh >an'alogon >~h|, t`o m'egiston a>ut~wn ka`i
t`o >el'aqiston d'uo t~wn loip~wn me'izon'a >estin. <'oper >'edei de~ixai.}}

\ParallelRText{
\begin{center}
{\large Proposition 25}$^\dag$
\end{center}

If four magnitudes are proportional then the (sum of the)
largest  and the smallest [of them] is greater than the (sum of the)  remaining two (magnitudes).

\epsfysize=1.4in
\centerline{\epsffile{Book05/fig25e.eps}}

Let $AB$, $CD$, $E$, and $F$ be four proportional magnitudes, (such that) as
$AB$ (is) to $CD$, so $E$ (is) to $F$. And let $AB$ be the greatest of them, and $F$ the least. I say that $AB$ and $F$ is greater than $CD$ and $E$.

For let $AG$ be made equal to $E$, and $CH$ equal to $F$.

\mbox{[}In fact,] since as $AB$ is to $CD$, so $E$ (is) to $F$, and $E$ (is) equal to $AG$, and
$F$ to $CH$,  thus as $AB$ is to $CD$, so $AG$ (is) to $CH$. And since the whole
$AB$ is to the whole $CD$ as the (part) taken away $AG$ (is) to the
(part) taken away $CH$, thus the remainder $GB$ will also be to the
remainder $HD$ as the whole $AB$ (is) to the whole $CD$ [Prop. 5.19]. And $AB$ (is) greater than $CD$.
Thus, $GB$ (is) also greater than $HD$. And since $AG$ is equal to $E$, and $CH$
to $F$, thus $AG$ and $F$ is equal to $CH$ and $E$. And [since] if [equal (magnitudes)
are added to unequal (magnitudes) then the wholes are unequal, thus if] $AG$ and $F$ are added to $GB$, and $CH$ and
$E$ to $HD$---$GB$ and
$HD$ being unequal, and $GB$ greater---it is inferred that $AB$ and $F$ (is) greater than $CD$ and $E$.

Thus, if four magnitudes are proportional then the (sum of the)
largest  and the smallest of them is greater than the (sum of the) remaining two (magnitudes). (Which is) the very thing it was required to show.}
\end{Parallel}


\vspace{7pt}{\footnotesize \noindent$^\dag$ In modern notation, this proposition
reads that if $\alpha:\beta::\gamma:\delta$, and $\alpha$ is the greatest and $\delta$ the least, then $\alpha+\delta>\beta+\gamma$.}
